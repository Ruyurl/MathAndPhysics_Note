\chapter{ベクトル空間}
\setcounter{page}{1}



\section{数ベクトル空間}
\subsection{数ベクトル空間}

平面のベクトル全体,空間のベクトル全体の集合をそれぞれ$\mathbb{R}^2,\ \mathbb{R}^3$と表す:
\begin{equation}
	\mathbb{R}^2 =
	\left\{
		\begin{bmatrix}
			x_1\\ x_2
		\end{bmatrix}
		\relmiddle|
		x_1,\ x_2 \in \mathbb{R}
	\right\},\quad
	\mathbb{R}^3 =
	\left\{
		\begin{bmatrix}
			x_1\\ x_2\\ x_3
		\end{bmatrix}
		\relmiddle|
		x_1,\ x_2,\ x_3 \in \mathbb{R}
	\right\}
\end{equation}

これを拡張して,$n$個の実数の組を\textbf{$n$次元数ベクトル}といい,その全体を\textbf{$n$次元数ベクトル空間}$\mathbb{R}^n$という.
\begin{equation}
	\bm{x} =
	\begin{bmatrix}
		x_1\\ x_2\\ \vdots\\ x_n
	\end{bmatrix}
	,\quad \mathbb{R}^n =
	\left\{
		\bm{x} =
		\begin{bmatrix}
			x_1\\ x_2\\ \vdots\\ x_n
		\end{bmatrix}
		\relmiddle|
		x_1,\ x_2,\ \cdots,\ x_n \in \mathbb{R}
	\right\}
\end{equation}

《注》ベクトルを行ベクトルによって書くこともある.

\begin{kousiki}{ベクトルの性質}
	$\bm{x},\ \bm{y},\ \bm{z}$がベクトルで,$\lambda,\ \mu$がスカラーのとき
	\begin{enumerate}[label=\textbf{[\arabic*]}, labelsep=10pt, leftmargin=23pt]
		\item $\bm{x} + \bm{y} = \bm{y} + \bm{x}$
		\item $(\bm{x} + \bm{y}) + \bm{z} = \bm{x} + (\bm{y} + \bm{z})$
		\item $\bm{x} + \bm{0} = \bm{x}$
		\item $\bm{x} + (-\bm{x}) = \bm{0}$
		\item $\lambda(\mu\bm{x}) = (\lambda\mu)\bm{x}$
		\item $(\lambda + \mu)\bm{x} = \lambda\bm{x} + \mu\bm{x}$
		\item $\lambda(\bm{x} + \bm{y}) = \lambda\bm{x} + \lambda\bm{y}$
		\item $1\bm{x} = \bm{x}$
	\end{enumerate}
\end{kousiki}