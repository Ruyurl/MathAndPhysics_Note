\chapter{ベクトル空間}
\setcounter{page}{1}



\section{数ベクトル空間}
\subsection{数ベクトル空間}

平面のベクトル全体,空間のベクトル全体の集合をそれぞれ$\mathbb{R}^2,\ \mathbb{R}^3$と表す:
\begin{equation}
	\mathbb{R}^2 =
	\left\{
		\begin{bmatrix}
			x_1\\ x_2
		\end{bmatrix}
		\relmiddle|
		x_1,\ x_2 \in \mathbb{R}
	\right\},\quad
	\mathbb{R}^3 =
	\left\{
		\begin{bmatrix}
			x_1\\ x_2\\ x_3
		\end{bmatrix}
		\relmiddle|
		x_1,\ x_2,\ x_3 \in \mathbb{R}
	\right\}
\end{equation}

これを拡張して,$n$個の実数の組を\textbf{$n$次元数ベクトル}といい,その全体を\textbf{$n$次元数ベクトル空間}$\mathbb{R}^n$という.
\begin{equation}
	\bm{x} =
	\begin{bmatrix}
		x_1\\ x_2\\ \vdots\\ x_n
	\end{bmatrix}
	,\quad \mathbb{R}^n =
	\left\{
		\bm{x} =
		\begin{bmatrix}
			x_1\\ x_2\\ \vdots\\ x_n
		\end{bmatrix}
		\relmiddle|
		x_1,\ x_2,\ \cdots,\ x_n \in \mathbb{R}
	\right\}
\end{equation}

《注》ベクトルを行ベクトルによって書くこともある.

\begin{kousiki}{ベクトルの性質}
	$\bm{x},\ \bm{y},\ \bm{z}$がベクトルで,$\lambda,\ \mu$がスカラーのとき
	\begin{enumerate}[label=\textbf{[\arabic*]}, labelsep=10pt, leftmargin=23pt]
		\item $\bm{x} + \bm{y} = \bm{y} + \bm{x}$
		\item $(\bm{x} + \bm{y}) + \bm{z} = \bm{x} + (\bm{y} + \bm{z})$
		\item $\bm{x} + \bm{0} = \bm{x}$
		\item $\bm{x} + (-\bm{x}) = \bm{0}$
		\item $\lambda(\mu\bm{x}) = (\lambda\mu)\bm{x}$
		\item $(\lambda + \mu)\bm{x} = \lambda\bm{x} + \mu\bm{x}$
		\item $\lambda(\bm{x} + \bm{y}) = \lambda\bm{x} + \lambda\bm{y}$
		\item $1\bm{x} = \bm{x}$
	\end{enumerate}
\end{kousiki}



\subsection{線形独立}

$m$個の$n$次元数ベクトル$\bm{x}_1,\ \bm{x}_2,\ \cdots,\ \bm{x}_m$について
\begin{equation}
	\lambda_1\bm{x}_1 + \lambda_2\bm{x}_2 + \cdots + \lambda_m\bm{x}_m = \bm{0} \iff \lambda_1 = \lambda_2 = \cdots = \lambda_m = 0 \label{equ13-Senkeidokuritu}
\end{equation}
が成り立つとき,これらのベクトルは\textbf{線形独立}(\textbf{1次独立})であるという.(\ref{equ13-Senkeidokuritu})の否定として
\begin{equation*}
	\lambda_1\bm{x}_1 + \lambda_2\bm{x}_2 + \cdots + \lambda_m\bm{x}_m = \bm{0}
\end{equation*}
を満たす$\lambda_1,\ \lambda_2,\ \cdots,\ \lambda_m$のうち少なくとも1つに$0$でない組が存在するとき,\textbf{線形従属}であるという.

\subsubsection*{線形独立であるかの調べ方}

\begin{tcolorbox}[breakable]
	\f{例} 次の4次元数ベクトル$\bm{x}_1,\ \bm{x}_2,\ \bm{x}_3$は線形独立か線形従属か.
	\begin{equation*}
		\bm{x}_1 =
			\begin{bmatrix*}[r]
				1\\ 1\\ -1\\ 2
			\end{bmatrix*}
		,\ \bm{x}_2 =
			\begin{bmatrix*}[r]
				-2\\ -1\\ 2\\ 1
			\end{bmatrix*}
		,\ \bm{x}_3 =
			\begin{bmatrix*}[r]
				3\\ 2\\ 1\\ 1
			\end{bmatrix*}
	\end{equation*}
	
	\tcblower

	\f{解} $\lambda_1\bm{x}_1 + \lambda_2\bm{x}_2 + \lambda_3\bm{x}_3 =\bm{0}$とおくと
	\begin{equation*}
		\begin{bmatrix}
			\bm{x}_1 & \bm{x}_2 & \bm{x}_3
		\end{bmatrix}
		\begin{bmatrix}
			\lambda_1\\ \lambda_2\\ \lambda_3
		\end{bmatrix}
		= \bm{0} \iff 
		\begin{bmatrix*}[r]
			 1 & -2 & 3\\
			 1 & -1 & 2\\ 
			-1 &  2 & 1\\
			 2 &  1 & 1
		\end{bmatrix*}
		\begin{bmatrix}
			\lambda_1\\ \lambda_2\\ \lambda_3
		\end{bmatrix}
		= \bm{0}
	\end{equation*}
	$\begin{bmatrix}
		\bm{x}_1 & \bm{x}_2 & \bm{x}_3 & \bm{0}
	\end{bmatrix}$に対し掃き出し法を行うと
	\begin{equation*}
		\begin{bmatrix}
			\bm{x}_1 & \bm{x}_2 & \bm{x}_3 & \bm{0}
		\end{bmatrix}
		\longrightarrow
		\begin{bmatrix}
			1 & -2 &  3 & 0\\
			0 &  1 & -1 & 0\\
			0 &  0 &  4 & 0\\
			0 &  5 & -5 & 0
		\end{bmatrix}
		\longrightarrow
		\begin{bmatrix}
			1 & 0 &  1 & 0\\
			0 & 1 & -1 & 0\\
			0 & 0 &  1 & 0\\
			0 & 0 &  0 & 0
		\end{bmatrix}
		\longrightarrow
		\begin{bmatrix}
			1 & 0 & 0 & 0\\
			0 & 1 & 0 & 0\\
			0 & 0 & 1 & 0\\
			0 & 0 & 0 & 0
		\end{bmatrix}
	\end{equation*}
	よって$\lambda_1 = \lambda_2 = \lambda_3 = 0$なので,$\bm{x}_1,\ \bm{x}_2,\ \bm{x}_3$は線形独立.
\end{tcolorbox}



\subsection{基底}

$\mathbb{R}^n$に於いて,その\textbf{基本ベクトル}を次のように定義する.
\begin{equation}
	\bm{e}_1 =
		\begin{bmatrix}
			1\\ 0\\ \vdots\\ 0
		\end{bmatrix}
	,\ \bm{e}_2 =
		\begin{bmatrix}
			0\\ 1\\ \vdots\\ 0
		\end{bmatrix}
	,\ \cdots,\ \bm{e}_n =
		\begin{bmatrix}
			0\\ 0\\ \vdots\\ 1
		\end{bmatrix}
\end{equation}

基本ベクトルは線形独立である.また,$\mathbb{R}^n$の任意のベクトルは,基本ベクトルの線形結合で表される.

\vskip\baselineskip

一般に$\mathbb{R}^n$の$m$個のベクトルの組$\varGamma = \{\bm{a}_1,\ \bm{a}_2,\ \cdots,\ \bm{a}_m\}$が,次の性質(=基本ベクトルの性質)
\begin{enumerate}[labelsep=10pt, leftmargin=28pt]
	\item[(Ⅰ)] それらは線形独立である.
	\item[(Ⅱ)] $\mathbb{R}^n$の任意のベクトルは,それらの線形結合で表される.
\end{enumerate}
を満たすとき,ベクトルの個数$m$は$n$に等しい.このとき,ベクトルの組$\varGamma = \{\bm{a}_1,\ \bm{a}_2,\ \cdots,\ \bm{a}_n\}$を$\mathbb{R}^n$の\textbf{基底}という.

また,
\begin{enumerate}[label=\textbf{[\arabic*]}, labelsep=10pt, leftmargin=23pt]
	\item $m > n$のとき,$\bm{a}_1,\ \bm{a}_2,\ \cdots,\ \bm{a}_m$は線形従属.
	\item $m < n$のとき,$\bm{a}_1,\ \bm{a}_2,\ \cdots,\ \bm{a}_m$の線形結合で表されないベクトルが存在する.
\end{enumerate}



\begin{kousiki}{$\mathbb{R}^n$の基底の条件}
	$\mathbb{R}^n$における$n$個のベクトルの組$\{\bm{a}_1,\ \bm{a}_2,\ \cdots,\ \bm{a}_n\}$について,次の条件は同値である.
	\begin{enumerate}[label=\textbf{[\arabic*]}, labelsep=10pt, leftmargin=23pt]
		\item $\{\bm{a}_1,\ \bm{a}_2,\ \cdots,\ \bm{a}_n\}$は$\mathbb{R}^n$の基底.
		\item $\bm{a}_1,\ \bm{a}_2,\ \cdots,\ \bm{a}_n$は線形独立.
		\item 行列$
			\begin{bmatrix}
				\bm{a}_1 & \bm{a}_2 & \cdots & \bm{a}_n
			\end{bmatrix}
			$は正則.即ち
			$
			\begin{vmatrix}
				\bm{a}_1 & \bm{a}_2 & \cdots & \bm{a}_n
			\end{vmatrix}
			\ne 0$
	\end{enumerate}
\end{kousiki}



\subsection{基底の変換}

$\mathbb{R}^n$の基底$\{\bm{a}_j\} = \{\bm{a}_1,\ \bm{a}_2,\ \cdots,\ \bm{a}_n\}$を単に$\bm{a}$,標準基底$\{\bm{e}_j\}$を$\bm{e}$と表す.$\mathbb{R}^n$の任意のベクトル$\bm{x}$は,$\bm{a}$の線形結合
\begin{equation}
	\bm{x} = y_1\bm{a}_1 + y_2\bm{a}_2 + \cdots + y_n\bm{a}_n
\end{equation}
で一意的に表される.このとき,スカラー$y_1,\ y_2,\ \cdots,\ y_n$を$\bm{x}$の$\bm{a}$に関する\textbf{成分}といい
\begin{equation}
	\begin{bmatrix}
		y_1\\ y_2\\ \vdots\\ y_n
	\end{bmatrix}_{\bm{a}}
\end{equation}
で表すこととする.