\chapter{ベクトル}
\setcounter{page}{1}

\section{ベクトルの成分表示}
\vskip-2\baselineskip
\begin{equation}
	\bm{a} = a_1\bm{i} + a_2\bm{j} + a_3\bm{k} =
	\begin{bmatrix}
		a_1\\ a_2\\ a_3
	\end{bmatrix}
\end{equation}



\section{ベクトルの大きさ(ノルム)}
\vskip-\baselineskip
\begin{equation}
	|\bm{a}| = \sqrt{a_1{}^2 + a_2{}^2 + a_3{}^2}
\end{equation}



\section{内積(スカラー積)\texorpdfstring{$\bm{a} \cdot \bm{b}$}{AdotB}}

$\bm{0}$でない2つのベクトル$\bm{a},\ \bm{b}$のなす角が$\theta\ (0 \le \theta \le \pi)$のとき,
\begin{align}
	\bm{a} \cdot \bm{b} &= |\bm{a}||\bm{b}|\cos\theta\\
						&=
						\begin{bmatrix}
							a_x\\ a_y \\ a_z
						\end{bmatrix}
						\cdot
						\begin{bmatrix}
							b_x\\ b_y \\ b_z
						\end{bmatrix}
						= a_x b_x + a_y b_y + a_z b_z
\end{align}

\begin{kousiki}{内積の性質}
	\begin{enumerate}[label=\textbf{[\arabic*]}, labelsep=10pt, leftmargin=23pt]
		\item $\bm{a} \cdot \bm{b} = \bm{b} \cdot \bm{a}$ \qquad $\bm{a} \cdot \bm{a} = |\bm{a}|^2$
		\item $\bm{a} \cdot (\bm{b} + \bm{c}) = \bm{a} \cdot \bm{b} + \bm{a} \cdot \bm{c}$ \qquad $(\bm{a} + \bm{b}) \cdot \bm{c} = \bm{a} \cdot \bm{c} + \bm{b} \cdot \bm{c}$
		\item $k(\bm{a} \cdot \bm{b}) = (k\bm{a}) \cdot \bm{b} = \bm{a} \cdot (k\bm{b})$ \qquad $(k \in \mathbb{R})$
		\item
			\begin{enumerate}[label=(\roman*), labelsep=10pt, leftmargin=23pt]
				\item $\bm{i} \cdot \bm{i} = \bm{j} \cdot \bm{j} = \bm{k} \cdot \bm{k} = 1$
				\item $\bm{i} \cdot \bm{j} = \bm{j} \cdot \bm{k} = \bm{k} \cdot \bm{i} = 0$
			\end{enumerate}
		\item $\bm{a} \ne \bm{0},\ \bm{b} \ne \bm{0}$のとき,$\bm{a} \perp \bm{b} \iff \bm{a} \cdot \bm{b} = 0$
	\end{enumerate}
	
\end{kousiki}



\section{外積(ベクトル積)\texorpdfstring{$\bm{a} \times \bm{b}$}{AtimesB}}

$\bm{0}$でない2つのベクトル$\bm{a},\ \bm{b}$のなす角が$\theta\ (0 < \theta < \pi)$のとき,
\begin{align}
	\bm{a} \times \bm{b} &= (|\bm{a}||\bm{b}|\sin\theta)\bm{e}\\
	&= \det
	\begin{bmatrix}
		\bm{i} & a_x & b_x\\
		\bm{j} & a_y & b_y\\
		\bm{k} & a_z & b_z
	\end{bmatrix}
\end{align}
$\bm{e}$は,$\bm{a},\ \bm{b},\ \bm{e}$がこの順で右手系を成す向き.$\bm{a} = \bm{0},\ \bm{b} = \bm{0}$のとき,又は$\theta = 0,\ \pi$のときは$\bm{a} \times \bm{b} = \bm{0}$とする.

\begin{kousiki}{外積の性質}
	\begin{enumerate}[label=\textbf{[\arabic*]}, labelsep=10pt, leftmargin=23pt]
		\item $\bm{a} \times \bm{b} = -\bm{b} \times \bm{a}$
		\item 
			\begin{enumerate}[label=(\roman*), labelsep=10pt, leftmargin=23pt]
				\item $\bm{a} \times (\bm{b} + \bm{c}) = \bm{a} \times \bm{b} + \bm{a} \times \bm{c}$
				\item $(\bm{a} + \bm{b}) \times \bm{c} = \bm{a} \times \bm{c}$
			\end{enumerate}
		\item $k(\bm{a} \times \bm{b}) = (k\bm{a}) \times \bm{b} = \bm{a} \times (k\bm{b})$ \qquad $(k \in \mathbb{R})$
		\item
			\begin{enumerate}[label=(\roman*), labelsep=10pt, leftmargin=23pt]
				\item $\bm{i} \times \bm{i} = \bm{j} \times \bm{j} = \bm{k} \times \bm{k} = \bm{0}$
				\item $\bm{i} \times \bm{j} = \bm{k},\quad \bm{j} \times \bm{k} = \bm{i},\quad \bm{k} \times \bm{i} = \bm{j}$
			\end{enumerate}
		\item $\bm{a} \ne \bm{0},\ \bm{b} \ne \bm{0}$のとき,$\bm{a} \heikou \bm{b} \iff \bm{a} \times \bm{b} = \bm{0}$
	\end{enumerate}
	
\end{kousiki}

$\bm{a}$と$\bm{b}$の成す平行四辺形の面積$S$は
\begin{equation}
	S = |\bm{a} \times \bm{b}|
\end{equation}
で求められる.



\section{三重積}
\subsection{スカラー三重積}

3つのベクトル$\bm{a},\ \bm{b},\ \bm{c}$について
\begin{align}
	\bm{a} \cdot (\bm{b} \times \bm{c}) &= |\bm{a}||\bm{b} \times \bm{c}|\cos\varphi \\
	&= \det
	\begin{bmatrix}
		\bm{a} & \bm{b} & \bm{c}
	\end{bmatrix}
	= \det
	\begin{bmatrix}
		a_x & b_x & c_x\\
		a_y & b_y & c_y\\
		a_z & b_z & c_z
	\end{bmatrix}
\end{align}
を\textbf{スカラー三重積}という.$\varphi$は$\bm{a}$と$\bm{b} \times \bm{c}$の成す角である.

\begin{kousiki}{スカラー三重積の性質}
	\begin{enumerate}[label=\textbf{[\arabic*]}, labelsep=10pt, leftmargin=23pt]
		\item $\bm{a} \cdot (\bm{b} \times \bm{c}) = \bm{b} \cdot (\bm{c} \times \bm{a}) = \bm{c} \cdot (\bm{a} \times \bm{b})$
		\item $\bm{a},\ \bm{b},\ \bm{c}$が同一平面上になく,この順で右手系を成すとき,$\bm{a},\ \bm{b},\ \bm{c}$の成す平行六面体の体積$V$は
		\begin{equation}
			V = \bm{a} \cdot (\bm{b} \times \bm{c})
		\end{equation}
		で求められる.
	\end{enumerate}
\end{kousiki}

\textbf{[1]}は,行列式の性質から分かる.\textbf{[2]}について,$\bm{a},\ \bm{b},\ \bm{c}$がこの順で左手系を成すとき$\left(\bunsuu{\pi}{2} < \varphi \le \pi\right)$,$V$は
\begin{equation}
	V = -\bm{a} \cdot (\bm{b} \times \bm{c})
\end{equation}
である.



\subsection{ベクトル三重積}

3つのベクトル$\bm{a},\ \bm{b},\ \bm{c}$について
\begin{equation}
	\bm{a} \times (\bm{b} \times \bm{c})
\end{equation}
を\textbf{ベクトル三重積}という.

\begin{kousiki}{Lagrangeの公式}
	\begin{enumerate}[label=\textbf{[\arabic*]}, labelsep=10pt, leftmargin=23pt]
		\item $\bm{a} \times (\bm{b} \times \bm{c}) = (\bm{a} \cdot \bm{c})\bm{b} - (\bm{a} \cdot \bm{b})\bm{c}$
		\item $(\bm{a} \times \bm{b}) \times \bm{c} = (\bm{a} \cdot \bm{c})\bm{b} - (\bm{b} \cdot \bm{c})\bm{a}$
	\end{enumerate}
\end{kousiki}



\section{ベクトルの平行条件・垂直条件}

$\bm{a} \ne \bm{0}$,$\bm{b} \ne \bm{0}$のとき
\begin{enumerate}[label=\textbf{[\arabic*]}, labelsep=10pt, leftmargin=23pt]
	\item $\bm{a} \heikou \bm{b} \iff \bm{a} \times \bm{b} = \bm{0} \iff \text{$\bm{b} = k\bm{a}$を満たす$k \in \mathbb{R}$が存在}$
	\item $\bm{a} \perp \bm{b} \iff \bm{a} \cdot \bm{b} = 0$
\end{enumerate}



\section{図形への応用}
\subsection{内分点}

$\mathrm{A}(\bm{a})$と$\mathrm{B}(\bm{b})$を結ぶ線分を$m : n$に内分する点の位置ベクトル$\mathrm{P}(\boldsymbol{p})$
\begin{equation}
	\bm{p} = \bunsuu{n\bm{a} + m\bm{b}}{m + n}
\end{equation}



\subsection{外分点}

$\mathrm{A}(\bm{a})$と$\mathrm{B}(\bm{b})$を結ぶ線分を$m : n$に外分する点の位置ベクトル$\mathrm{Q}(\bm{q})$
\begin{equation}
	\bm{q} = \bunsuu{-n\bm{a} + m\bm{b}}{m - n}
\end{equation}



\subsection{中点}

$\mathrm{A}(\bm{a})$と$\mathrm{B}(\bm{b})$を結ぶ線分の中点の位置ベクトル$\mathrm{M}(\bm{m})$
\begin{equation}
	\bm{m} = \bunsuu{\bm{a} + \bm{b}}{2}
\end{equation}



\subsection{重心}

$\mathrm{A}(\bm{a})$,$\mathrm{B}(\bm{b})$,$\mathrm{C}(\bm{c})$を結んでできる三角形の重心の位置ベクトル$\mathrm{G}(\bm{g})$
\begin{equation}
	\bm{g} = \bunsuu{\bm{a} + \bm{b} + \bm{c}}{3}
\end{equation}



\subsection{直線のベクトル方程式}

直線上の1点を$\mathrm{P}(\bm{p})$とする.
\begin{enumerate}[label=\textbf{[\arabic*]}, labelsep=10pt, leftmargin=23pt]
	\item $\mathrm{A}(a)$を通り$\bm{u}$に平行
		\begin{equation}
			\bm{p} = \bm{a} + t\bm{u} \quad (t \in \mathbb{R})
		\end{equation}
	\item $\mathrm{A}(\bm{a})$,$\mathrm{B}(\bm{b})$を通る
		\begin{equation}
			\bm{p} = (1 - t)\bm{a} + t\bm{b} \hspace*{4\zw} \text{($\bm{u} = \bm{b} - \bm{a}$とする)}
		\end{equation}
	\item $\mathrm{A}(\bm{a})$を通り,$\bm{n} (\ne \bm{0})$に垂直な直線
		\begin{equation}
			\bm{n} \cdot (\bm{p} - \bm{a}) = 0
		\end{equation}
		直線が$ax + by + c = 0$のとき,$\bm{n} =
		\begin{bmatrix}
			a\\ b
		\end{bmatrix}
		$
\end{enumerate}



\subsection{直線との距離}

直線$ax + by + c = 0$と$\mathrm{A}(x_0,\ y_0)$との距離
\begin{equation}
	\bunsuu{|ax_0 + by_0 + c|}{\sqrt{a^2 + b^2}}
\end{equation}



\subsection{三角形の面積}

$\triangle\mathrm{OAB}$に於いて,$\vecrm{OA} = \bm{a}$,$\vecrm{OB} = \bm{b}$としたときの三角形の面積$S$
\begin{align}
	&S = \bunsuu{1}{2}\sqrt{|\bm{a}|^2|\bm{b}|^2 - (\bm{a} \cdot \bm{b})^2}
	&
	&S = \bunsuu{1}{2}|a_x b_y - a_y b_x| \quad \text{(平面の場合)}
\end{align}



\subsection{円のベクトル方程式}

中心$\mathrm{C}(\bm{c})$,半径$r$の円のベクトル方程式
\begin{equation}
	(\bm{p} - \bm{c}) \cdot (\bm{p} - \bm{c}) = r^2
\end{equation}



\subsection{平面のベクトル方程式}

$\mathrm{A}(\bm{a})$を通り,$\bm{n}$に垂直な平面のベクトル方程式
\begin{equation}
	\bm{n} \cdot (\bm{p} - \bm{a}) = 0
\end{equation}
平面が$ax + by + cz + d = 0$のとき,$\bm{n} =
\begin{bmatrix}
	a\\ b\\ c
\end{bmatrix}
$


\subsection{球面のベクトル方程式}

中心$\mathrm{C}(\bm{c})$,半径$r$の球面のベクトル方程式
\begin{equation}
	(\bm{p} - \bm{c}) \cdot (\bm{p} - \bm{c}) = r^2
\end{equation}



\subsection{平面との距離}

平面$ax + by + cz + d = 0$と$\mathrm{A}(x_0,\ y_0,\ z_0)$との距離
\begin{equation}
	\bunsuu{|ax_0 + by_0 + cz_0 + d|}{\sqrt{a^2 + b^2 + c^2}}
\end{equation}
\quad