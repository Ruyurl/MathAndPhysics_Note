\documentclass[
	report, paper=a4, head_space=23mm, foot_space=13mm,
	fontsize=8pt, jafontsize=8pt,
	gutter=20mm, line_length=170mm,
	fleqn, twoside
]{jlreq}
\def\pdfpagewidth{\paperwidth}
\def\pdfpageheight{\paperheight}
%%%%%%%%%%%%%%%%%%%%%%%%%%%%%%%%%%%%%%%%%%%%%%%%%%
% report  トップレベル=\chapter
% paper a4に設定
% 余白の設定
%   head_space  天の空き量
%   foot_space  地の空き量
%   gutter   小口(外側)の空き量
% fleqn 別行立て数式を左側に
%%%%%%%%%%%%%%%%%%%%%%%%%%%%%%%%%%%%%%%%%%%%%%%%%%
% todayの再設定
\makeatletter
\newcommand*{\themonth}{\two@digits\month}
\newcommand*{\theday}{\two@digits\day}
\makeatother
\renewcommand{\today}{{\the\year}年{\themonth}月{\theday}日}




%%%%%%%%%%%%%%%%%%%%%%%%%%%%%%%%%%%%%%%%%%%%%%%%%%
%%%%%%%%%%                              %%%%%%%%%%
%%%%%%%%%%           文章関係           %%%%%%%%%%
%%%%%%%%%%                              %%%%%%%%%%
%%%%%%%%%%%%%%%%%%%%%%%%%%%%%%%%%%%%%%%%%%%%%%%%%%
% フォント関係
\usepackage{luatexja-fontspec, luatexja-ruby}
% \setmainjfont{ChikuPori-Regular2.ttf}[BoldFont=LINESeedJP_OTF_Rg.otf]
\setmainjfont{ChikuPori-Regular2.ttf}[BoldFont=ShipporiGothicB2-Bold.ttf]
\setsansjfont{LINESeedJP_OTF_Rg.otf}[BoldFont=LINESeedJP_OTF.otf]
\usepackage{fontspec}
\setsansfont{GenEiMGothic2-Regular.ttf}[BoldFont=GenEiMGothic2-Bold.ttf]
%%%%%%%%%%%%%%%%%%%%%%%%%%%%%%%%%%%%%%%%%%%%%%%%%%
% ノンブル、柱の設定
\ModifyPageStyle{headings}{
	% 柱
	running_head_position=top-center,
	% 柱,ノンブル
	nombre={\thechapter-\thepage},
	nombre_position=top-right,
	nombre_font=\sffamily\bfseries,
}
\pagestyle{headings}
%%%%%%%%%%%%%%%%%%%%%%%%%%%%%%%%%%%%%%%%%%%%%%%%%%
% plain ヘッダーには何も表示せず,フッターにノンブル
%%%%%%%%%%%%%%%%%%%%%%%%%%%%%%%%%%%%%%%%%%%%%%%%%%
% 色関係
% \usepackage{graphicx}
\usepackage[dvipsnames]{xcolor}
%%%%%%%%%%%%%%%%%%%%%%%%%%%%%%%%%%%%%%%%%%%%%%%%%%
% xcolor    dvipsnamesをつけるといろいろな名前が使える:
% https://mathlandscape.com/latex-color/#toc3
%%%%%%%%%%%%%%%%%%%%%%%%%%%%%%%%%%%%%%%%%%%%%%%%%%
\usepackage{multicol, here, multirow, fancybox, array, incgraph, wrapfig, caption, paralist, ascmac}
%%%%%%%%%%%%%%%%%%%%%%%%%%%%%%%%%%%%%%%%%%%%%%%%%%
% 段組み
% here  図や表のH
% multirow  縦結合
% fancybox いい感じのbox
% array 表の罫線を微妙に改善
% incgraph	表紙
% wrapfig	図の回り込み
% paralist	文中のリスト
% ascmac	別行box
%%%%%%%%%%%%%%%%%%%%%%%%%%%%%%%%%%%%%%%%%%%%%%%%%%
% 四角番号
\newcommand{\f}[1]{\fbox{\textbf{#1}}}
% \renewcommand{\labelenumi}{\f{\arabic{enumi}}}
% \renewcommand{\labelenumii}{(\arabic{enumii})}
%%%%%%%%%%%%%%%%%%%%%%%%%%%%%%%%%%%%%%%%%%%%%%%%%%
% enumerateでのデフォルト \f{1},\f{2},...
% enumerateでの第2階層のデフォルト (1),(2),...
%%%%%%%%%%%%%%%%%%%%%%%%%%%%%%%%%%%%%%%%%%%%%%%%%%
\usepackage{tasks}
\settasks{
	label=(\arabic*), label-align=right, 
	item-indent=6pt, label-offset=6pt,
	column-sep=15pt
}
%%%%%%%%%%%%%%%%%%%%%%%%%%%%%%%%%%%%%%%%%%%%%%%%%%
% enumitem  いろいろな設定ができる(emathを先に読み込むので97行目に移動).
% task  横箇条書き
%%%%%%%%%%%%%%%%%%%%%%%%%%%%%%%%%%%%%%%%%%%%%%%%%%
% カウンタの定義
\newcounter{daimoncounter}
% \toiと打つと,カウンタが1つ増えて,その数が表示される.
\newcommand{\daimon}{\refstepcounter{daimoncounter}\thedaimoncounter}
\newcommand{\toi}{\f{\daimon}}
%%%%%%%%%%%%%%%%%%%%%%%%%%%%%%%%%%%%%%%%%%%%%%%%%%
% 選択問題の四角
% 枠の大きさ設定
\newcommand{\senntakuanaume}[1]{\setlength{\fboxsep}{1pt}\setlength{\fboxrule}{0.6pt} \doublebox{\mbox{\textbf{\phantom{ア}#1\phantom{ア}}}} \setlength{\fboxrule}{0.4pt}} %\senntakuanaumeで選択肢問題の穴埋め
%%%%%%%%%%%%%%%%%%%%%%%%%%%%%%%%%%%%%%%%%%%%%%%%%%
%%%%%%%%%%%%%%%%%%%%%%%%%%%%%%%%%%%%%%%%%%%%%%%%%%





%%%%%%%%%%%%%%%%%%%%%%%%%%%%%%%%%%%%%%%%%%%%%%%%%%
%%%%%%%%%%                              %%%%%%%%%%
%%%%%%%%%%           数式関係           %%%%%%%%%%
%%%%%%%%%%                              %%%%%%%%%%
%%%%%%%%%%%%%%%%%%%%%%%%%%%%%%%%%%%%%%%%%%%%%%%%%%
\usepackage{amsmath, emath, mathtools, extarrows, cancel, ulem, enumitem, mleftright, emathMw, siunitx}
\mleftright
%%%%%%%%%%%%%%%%%%%%%%%%%%%%%%%%%%%%%%%%%%%%%%%%%%
% amsmath   いろいろな機能を追加.
% emath     日本に合わせたいろいろな機能を追加.
% mathtools いろいろな機能を追加.
% extarrows 
% cancel    数式に斜線
% mleftright 括弧の空き
% emathMw	enumerate環境下での図の回り込み
% siunitx	角度\ang{}
%%%%%%%%%%%%%%%%%%%%%%%%%%%%%%%%%%%%%%%%%%%%%%%%%%
\renewcommand{\tagform}[1]{(#1)}%
\preEqlabel{}%数式番号デフォルト
%%%%%%%%%%%%%%%%%%%%%%%%%%%%%%%%%%%%%%%%%%%%%%%%%%
% emathを入れたことにより,デフォルトが①②になったので
%%%%%%%%%%%%%%%%%%%%%%%%%%%%%%%%%%%%%%%%%%%%%%%%%%
\renewcommand{\dint}{\displaystyle\int}
\newcommand{\diint}{\displaystyle\iint}
\newcommand{\dsum}{\displaystyle\sum}
\renewcommand{\dlim}{\lim\limits}
\newcommand{\md}{\mathrm{d}}
%%%%%%%%%%%%%%%%%%%%%%%%%%%%%%%%%%%%%%%%%%%%%%%%%%
% 積分記号,和,極限(常にdisplaystyle)
%%%%%%%%%%%%%%%%%%%%%%%%%%%%%%%%%%%%%%%%%%%%%%%%%%
\newcommand{\qlongright}{\quad\longrightarrow\quad}
\newcommand{\qLongright}{\quad\Longrightarrow\quad}
\newcommand{\qLonglr}{\quad\Longleftrightarrow\quad}
\usepackage[e]{esvect}
\renewcommand{\vec}[1]{\hspace*{-0.5pt}\vv{\mathstrut #1}\hspace*{-0.5pt}}
\newcommand{\vecrm}[1]{\vv{\mathrm{\mathstrut #1}}}
%%%%%%%%%%%%%%%%%%%%%%%%%%%%%%%%%%%%%%%%%%%%%%%%%%
% ベクトルの定義しなおし
% 始点と終点を示すベクトルは \vecrm を使用.
%%%%%%%%%%%%%%%%%%%%%%%%%%%%%%%%%%%%%%%%%%%%%%%%%%
\makeatletter

\def\@myfrac@d#1#2{%
	\displaystyle\frac{%
		\raisebox{-.44ex}{$\,#1\,$}%分子
	}{%
		\raisebox{.1ex}{$\,#2\,$}%分母
	}
}
\def\@myfrac@t#1#2{%
	\textstyle\frac{%
		\raisebox{-.04ex}{\scalebox{0.9}{$\,#1\,$}}%分子
	}{%
		\raisebox{-.3ex}{\scalebox{0.9}{$\,#2\,$}}%分母
	}
}
\def\@myfrac@s#1#2{\hspace{-.5pt}%
	\scriptstyle\frac{%
		\raisebox{0.1ex}{\scalebox{0.6}{$\,#1\,$}}%分子
	}{%
		\raisebox{0.1ex}{\scalebox{0.6}{$\,#2\,$}}%分母
	}
\hspace{-.5pt}}
\def\@myfrac@ss#1#2{%
	\scriptscriptstyle#1/#2
}
\def\myfrac#1#2{
	\mathchoice{\@myfrac@d{#1}{#2}}{\@myfrac@t{#1}{#2}}{\@myfrac@s{#1}{#2}}{\@myfrac@ss{#1}{#2}}
}
\makeatother
%%%%%%%%%%%%%%%%%%%%%%%%%%%%%%%%%%%%%%%%%%%%%%%%%%
% \myfrac{分子}{分母}   textstyleとscriptsizeの分数を定義.
% displaystyle ---> displaystyle
% textstyle ---> displaystyleの0.9倍
% scriptstyle ---> displaystyleの0.6倍
% scriptscriptstyle ---> 分子/分母の形
%%%%%%%%%%%%%%%%%%%%%%%%%%%%%%%%%%%%%%%%%%%%%%%%%%
\DeclareMathOperator{\cosec}{cosec}
%%%%%%%%%%%%%%%%%%%%%%%%%%%%%%%%%%%%%%%%%%%%%%%%%%
% 三角関数
%%%%%%%%%%%%%%%%%%%%%%%%%%%%%%%%%%%%%%%%%%%%%%%%%%
\DeclareMathOperator{\csin}{Sin}
\DeclareMathOperator{\Arcsin}{Arcsin}
\DeclareMathOperator{\ccos}{Cos}
\DeclareMathOperator{\Arccos}{Arccos}
\DeclareMathOperator{\ctan}{Tan}
\DeclareMathOperator{\Arctan}{Arctan}
\DeclareMathOperator{\arcsec}{arcsec}
\DeclareMathOperator{\arccsc}{arccsc}
\DeclareMathOperator{\arccot}{arccot}
%%%%%%%%%%%%%%%%%%%%%%%%%%%%%%%%%%%%%%%%%%%%%%%%%%
% 逆三角関数
% cはCapitalの略
%%%%%%%%%%%%%%%%%%%%%%%%%%%%%%%%%%%%%%%%%%%%%%%%%%
\DeclareMathOperator{\sech}{sech}
\DeclareMathOperator{\csch}{csch}
%%%%%%%%%%%%%%%%%%%%%%%%%%%%%%%%%%%%%%%%%%%%%%%%%%
% 双曲線関数
%%%%%%%%%%%%%%%%%%%%%%%%%%%%%%%%%%%%%%%%%%%%%%%%%%
\DeclareMathOperator{\arsinh}{arsinh}
\DeclareMathOperator{\arcosh}{arcosh}
\DeclareMathOperator{\artanh}{artanh}
\DeclareMathOperator{\arsech}{arsech}
\DeclareMathOperator{\arcsch}{arcsch}
\DeclareMathOperator{\arcoth}{arcoth}
%%%%%%%%%%%%%%%%%%%%%%%%%%%%%%%%%%%%%%%%%%%%%%%%%%
% 逆双曲線関数
%%%%%%%%%%%%%%%%%%%%%%%%%%%%%%%%%%%%%%%%%%%%%%%%%%
\DeclareMathOperator{\grad}{grad}
\DeclareMathOperator{\dive}{div}
\DeclareMathOperator{\rot}{rot}
\DeclareMathOperator{\curl}{curl}
%%%%%%%%%%%%%%%%%%%%%%%%%%%%%%%%%%%%%%%%%%%%%%%%%%
% 勾配
%%%%%%%%%%%%%%%%%%%%%%%%%%%%%%%%%%%%%%%%%%%%%%%%%%




%%%%%%%%%%%%%%%%%%%%%%%%%%%%%%%%%%%%%%%%%%%%%%%%%%
%%%%%%%%%%                              %%%%%%%%%%
%%%%%%%%%%           TikZ           %%%%%%%%%%
%%%%%%%%%%                              %%%%%%%%%%
%%%%%%%%%%%%%%%%%%%%%%%%%%%%%%%%%%%%%%%%%%%%%%%%%%
\usepackage{pgfplots, tikz, tikz-3dplot}
\usetikzlibrary{positioning, intersections, calc, arrows.meta, fadings, patterns, lindenmayersystems} %tikzのlibrary
\usepackage{tcolorbox}
\tcbuselibrary{theorems, breakable, raster, skins}

\newtcolorbox{kousiki}[2][]{
	% tikzを用いた記法の処理
	enhanced, %
	% box内左右の余白
	left = 12pt, right = 12pt, %
	% タイトルのフォント指定
	fonttitle = \sffamily\bfseries\large, %
	% タイトルの文字の色
	coltitle = white, %
	% タイトルの背景の色
	colbacktitle = black, %
	% タイトルを左寄せに、少し微調整
	attach boxed title to top left={}, %
	% タイトルボックスの装飾
	boxed title style = {skin = enhancedfirst jigsaw, arc = 1mm, bottom = 0mm, boxrule = 0mm}, %
	% 枠線の太さ
	boxrule = 0.5pt, %
	% 本文の背景色
	colback = black!5!, %
	% 本文の枠の色
	colframe = black, %
	% 左上の角の調整
	sharp corners = northwest, % 
	% 影をつける
	drop fuzzy shadow, %
	% ページマタギOK
	breakable, %
	% タイトルは直接入力
	title = \vspace{3mm}#2, 
	% 弧
	arc = 1mm, %
	#1
}%
\definecolor{tip-green-text}{HTML}{165c26}
\definecolor{tip-green-background}{HTML}{dcffe4}
\definecolor{tip-green-border}{HTML}{96f1b5}
\newenvironment{tip}[1]
	{\tcolorbox[colback=tip-green-background, frame hidden, boxrule=0pt, boxsep=0pt, breakable, enlarge bottom by=0.3cm, enhanced jigsaw, borderline west={3pt}{0pt}{tip-green-border}, title={Tip\quad[#1]\\[1mm]}, colbacktitle={tip-green-border}, coltitle={tip-green-text}, coltext={tip-green-text}, fonttitle={\small\bfseries}, attach title to upper={}]}
	{\endtcolorbox}
% 


%%%%%%%%%%%%%%%%%%%%%%%%%%%%%%%%%%%%%%%%%%%%%%%%%%
%%%%%%%%%%                              %%%%%%%%%%
%%%%%%%%%%           デザイン           %%%%%%%%%%
%%%%%%%%%%                              %%%%%%%%%%
%%%%%%%%%%%%%%%%%%%%%%%%%%%%%%%%%%%%%%%%%%%%%%%%%%
\makeatletter
% 目次の章の体裁カスタマイズ
\newif\ifmaincontents   % メイン内の章かを区別するため
\maincontentsfalse
\renewcommand{\l@chapter}[2]{%
    \ifnum \c@tocdepth >\m@ne
        \setlength\@lnumwidth{0.1\textwidth}
        \ifmaincontents
            \vspace*{0.8\baselineskip}
            \noindent\hrulefill\par%
            \nobreak%
            \noindent {\sffamily\large #1}\par
        \else
            \vspace*{0.8\baselineskip}
            \noindent {\sffamily\large #1}\par
        \fi
    \fi}
	\renewcommand{\l@section}[2]{%
		\setlength\@lnumwidth{0.06\textwidth}
		#1\par
	}
\makeatother
% 章
\ModifyHeading{chapter}{
	font={\huge\sffamily\bfseries},
	format={%
		{\color[gray]{0.5}\rule[-0.5\zw]{2\zw}{1.8\zw}}%
		\if@mainmatter
			\hspace{-2\zw}\raisebox{0.1\zw}{%
				\makebox[2\zw]{\color[gray]{1}\thechapter}}%
		\fi
		\hspace{0.5\zw}#1#2
	}
}
% 節
\ModifyHeading{section}{%
	format={%
		\hrule\par\vspace*{0.25ex}\parbox[b]{\linewidth - 1\zw}{\hspace*{-1\zw}\colorbox{black}{\hspace*{1\zw}\textcolor{white}{#1}}\quad#2}\vspace*{0.25ex}\par\hrule}
}
%
% 小節
\renewcommand{\thesubsection}{\Alph{subsection}}
\ModifyHeading{subsection}{%
	format={%
		{\color{gray}◆}#1{\color{gray}\hspace*{-0.7\zw}◆}\quad#2
	}
}
\renewcommand{\thesubsubsection}{[\arabic{subsubsection}]}
%%%%%%%%%%%%%%%%%%%%%%%%%%%%%%%%%%%%%%%%%%%%%%%%%%
% 章,節のデザインを変更 \MakeUppercase{#1}ですべて大文字   
%%%%%%%%%%%%%%%%%%%%%%%%%%%%%%%%%%%%%%%%%%%%%%%%%%

\usepackage{hyperref}

\hypersetup{
	colorlinks,
	linkcolor=blue,
	urlcolor=blue
}
% hyperref  目次リンク
%%%%%%%%%%%%%%%%%%%%%%%%%%%%%%%%%%%%%%%%%%%%%%%%%%



\begin{document}
\title{\fontsize{40pt}{1\zh} \jfontspec{DelaGothicOne-Regular.ttf}数学・物理 回憶錄}
\date{\LARGE 最終更新日: \today}
\maketitle

\setcounter{tocdepth}{2}
\begin{multicols}{2}
\tableofcontents
	
\end{multicols}



\noindent
\textbf{【注意】}
\begin{enumerate}[leftmargin=15pt]
	\item ベクトルは$[\quad]$で表す.また,原則列ベクトル表記$\bm{a} =
		\begin{bmatrix}
			a_x\\ a_y\\ a_z
		\end{bmatrix}$とする.紙面上の関係で行ベクトル表記で表すときは転置行列を表す$\top$を付けて$\bm{a} =
		{
			\begin{bmatrix}
			a_x & a_y & a_z
		\end{bmatrix}
		}^\top$と表す.
	\item $\bm{i},\ \bm{j},\ \bm{k}$はそれぞれ$x$軸,$y$軸,$z$軸方向の基本ベクトルとする.
\end{enumerate}


% \chapter{微分法}
\setcounter{page}{1}



\section{微分法}

\subsection{関数の連続}

関数$f(x)$の極限値について
\begin{equation}
	\lim_{x \to a} f(x) = \alpha \iff \lim_{x \to a + 0} f(x) = \lim_{x \to a - 0} f(x) = \alpha
\end{equation}
が成り立つ.また,$\dlim_{x \to a} f(x)$が存在して
\begin{equation}
	\lim_{x \to a} f(x) = f(a)
\end{equation}
が成り立つとき,$f(x)$は$x = a$で\textbf{連続}であるという.



\subsection{微分可能性}

\vskip-\baselineskip
\begin{equation}
	f'(a) = \lim_{h \to 0} \bunsuu{f(a + h) - f(a)}{h}
\end{equation}
が存在するとき,関数$f(x)$は$x = a$に於いて\textbf{微分可能}であるという.このとき,次が成り立つ.
\begin{equation}
	\textbf{$f(x)$は$x = a$で微分可能} \qLongright \textbf{$f(x)$は$x = a$で連続}
\end{equation}



\subsection{導関数}

次の式で定義される関数を$f(x)$の\textbf{導関数}という.
\begin{equation}
	\bunsuu{d}{dx}f(x) = \lim_{\varDelta x \to 0} \bunsuu{f(x + \varDelta x) - f(x)}{\varDelta x}
\end{equation}



\begin{kousiki}{導関数の性質と公式}
	$c$を定数とする.
\end{kousiki}

\newpage
% \chapter{積分法}
\setcounter{page}{1}
% \chapter{数列の極限}
\setcounter{page}{1}
% \chapter{関数の展開}
\setcounter{page}{1}
\chapter{偏微分法}
\setcounter{page}{1}
\section{関数の極限}

関数$f(x,\ y)$に於いて,点$(x,\ y)$が点$(a,\ b)$以外の点を取りながら$(a,\ b)$に限りなく近づくとき,関数の値が$C$に限りなく近づくならば,$f(x,\ y)$は$C$に\textbf{収束する}といい,
\begin{equation}
	\lim_{(x,\ y) \to (a,\ b)} f(x,\ y) = C
\end{equation}
と表す.$C$を\textbf{極限値}という.

このとき,$(x,\ y)$がどんな近づき方で$(a,\ b)$に近づいても極限値がある一定の値$C$になることが必要である.

例えば,$f(x,\ y) = \bunsuu{xy}{x^2 + y^2}$について,$\dlim_{(x,\ y) \to (0,\ 0)} f(x,\ y)$を考える.
\begin{enumerate}[label=(\roman*), labelsep=10pt, leftmargin=23pt]
	\item 点を直線$y = x$上で近づけると$f(x,\ y) = \bunsuu{x \cdot x}{x^2 + x^2} = \bunsuu{1}{2}$であるから$\bunsuu{1}{2}$に収束する.
	\item 点を直線$y = 2x$上で近づけると$f(x,\ y) = \bunsuu{x \cdot 2x}{x^2 + (2x)^2} = \bunsuu{2}{5}$であるから$\bunsuu{2}{5}$に収束する.
\end{enumerate}
よって,極限値はない.

関数$f(x,\ y)$の定義域内の点$\mathrm{P}(a,\ b)$について,
\begin{equation}
	\lim_{(x,\ y) \to (a,\ b)} f(x,\ y) = f(a,\ b)
\end{equation}
が成り立つとき,$f(x,\ y)$は点$\mathrm{P}$で\textbf{連続である}という.


\section{偏導関数}

関数$z = f(x,\ y)$に於いて
\begin{equation}
	\bunsuu{\partial z}{\partial x} = \lim_{\varDelta x \to 0} \bunsuu{f(x + \varDelta x,\ y) - f(x,\ y)}{\varDelta x}
\end{equation}
$f(x,\ y)$の\textbf{$x$についての偏導関数}といい,$f_x$とも表す.また,
\begin{equation}
	\bunsuu{\partial z}{\partial y} = \lim_{\varDelta y \to 0} \bunsuu{f(x,\ y + \varDelta y) - f(x,\ y)}{\varDelta y}
\end{equation}
を$f(x,\ y)$の\textbf{$y$についての偏導関数}といい,$f_y$とも表す.

\textbf{$x\ (y)$について偏微分可能}であるとは,点$x = a\ (y = b)$での偏微分係数が存在することである.また,偏微分係数$f_x(a,\ b),\ f_y(a,\ b)$はそれぞれ点$(a,\ b)$の$x$軸方向の傾き,$y$軸方向の傾きを表す.



\subsection{高階偏導関数}

$z = f(x,\ y)$で,2階偏微分可能で全て連続のとき\quad$\bunsuu{\partial^2 z}{\partial y \partial x} = \bunsuu{\partial^2 z}{\partial x \partial y}$

一般に$n$階偏微分可能で全て連続のとき,$n = k + l$とすると$\bunsuu{\partial^n z}{\partial x^k \partial y^l}$は全て等しい.



\subsection{全微分可能性}

\begin{itemize}
	\item $f(x,\ y)$が点$(a,\ b)$で全微分可能 $\qLongright$ $f(x,\ y)$は$(a,\ b)$で連続かつ$(a,\ b)$で偏微分可能
	\item $f(x,\ y)$の$\bunsuu{\partial f}{\partial x},\ \bunsuu{\partial f}{\partial y}$が$(a,\ b)$で存在してそれらが連続である $\qLongright$ $f(x,\ y)$は$(a,\ b)$で全微分可能
	\item $f(x,\ y)$が偏微分可能であっても全微分可能ではない(全微分の方が強い概念).
\end{itemize}

下の式を$z = f(x,\ y)$の\textbf{全微分}という.全微分は,$x,\ y$を微小変化させたときの$z = f(x,\ y)$の変化量,つまり$dx,\ dy$増加したときの$dz$の値である.
\begin{equation}
	dz = \bunsuu{\partial z}{\partial x}dx + \bunsuu{\partial z}{\partial y}dy
\end{equation}

\newpage
\begin{tip}{導出}
	関数$z = f(x,\ y)$の$\varDelta x,\ \varDelta y$に対する増加量$\varDelta z$は
	\begin{equation*}
		\varDelta z = 
		\underbrace{
			\underbrace{
				\bunsuu{\partial z}{\partial x}\varDelta x
			}_{\text{$\varDelta x$についての増加量}}
			+
			\underbrace{
				\bunsuu{\partial z}{\partial y}\varDelta y
			}_{\text{$\varDelta y$についての増加量}}
		}_{接平面の公式}
		{} + \varepsilon
	\end{equation*}
	更に,$\dlim_{\scalebox{0.5}{$(\varDelta x,\ \varDelta y) \to (0,\ 0)$}} \bunsuu{\varepsilon}{\sqrt{(\varDelta x)^2 + (\varDelta y)^2}} = 0$であれば,$z = f(x,\ y)$は全微分可能という.
\end{tip}

ここで,$\dlim_{\scalebox{0.5}{$(\varDelta x,\ \varDelta y) \to (0,\ 0)$}} \bunsuu{\varepsilon}{\sqrt{(\varDelta x)^2 + (\varDelta y)^2}} = 0$を,「$\varepsilon$は$\sqrt{(\varDelta x)^2 + (\varDelta y)^2}$より\textbf{高位の無限小}」という.これは,「$\varepsilon$は$\sqrt{(\varDelta x)^2 + (\varDelta y)^2}$とは比べ物にならないくらい速く$0$に近付く」という意味である.



\subsection{接平面の方程式}

曲面$z = f(x,\ y)$上の点$\bigl(a,\ b,\ f(a,\ b)\bigr)$に於ける接平面の方程式
\begin{equation}
	z - f(a,\ b) = \bunsuu{\partial f(a,\ b)}{\partial x}(x - a) + \bunsuu{\partial f(a,\ b)}{\partial y}(y - b) \label{equ:par_deri-1}
\end{equation}

曲面$f(x,\ y,\ z) = 0$上の点$(a,\ b,\ c)$に於ける接平面の方程式
\begin{equation}
	\begin{bmatrix}
		\bunsuu{\partial f}{\partial x} & \bunsuu{\partial f}{\partial y} & \bunsuu{\partial f}{\partial z} 
	\end{bmatrix}
	\begin{bmatrix}
		x - a\\ y - b\\ z - c
	\end{bmatrix}
	= 0
\end{equation}

これは,陰関数の微分法と式(\ref{equ:par_deri-1})から分かる.また,これより曲面$f(x,\ y,\ z) = 0$の法線ベクトルは以下である.
\begin{equation}
	{
		\begin{bmatrix}
			\bunsuu{\partial f}{\partial x} & \bunsuu{\partial f}{\partial y} & \bunsuu{\partial f}{\partial z}
		\end{bmatrix}
	}^\top
\end{equation}



\subsection{チェーン・ルール(連鎖律)}

\begin{kousiki}{チェーン・ルール(1)}
	$z = f(x,\ y)$が全微分可能で,$x = x(t),\ y = y(t)$が微分可能であるとき,$z$は$t$の関数である.
	\begin{equation}
		\bunsuu{dz}{dt} = \bunsuu{\partial f}{\partial x}\bunsuu{dx}{dt} + \bunsuu{\partial f}{\partial y}\bunsuu{dy}{dt}
	\end{equation}
\end{kousiki}

\begin{kousiki}{チェーン・ルール(2)}
	$z = f(x,\ y)$が全微分可能で,$x = x(u,\ v),\ y = y(u,\ v)$が偏微分可能であるとき
	\begin{gather}
		\bunsuu{\partial z}{\partial u} = \bunsuu{\partial z}{\partial x}\bunsuu{\partial x}{\partial u} + \bunsuu{\partial z}{\partial y}\bunsuu{\partial y}{\partial u}\\
		\bunsuu{\partial z}{\partial v} = \bunsuu{\partial z}{\partial x}\bunsuu{\partial x}{\partial v} + \bunsuu{\partial z}{\partial y}\bunsuu{\partial y}{\partial v}
	\end{gather}
\end{kousiki}



\subsection{陰関数の微分法}

$f(x,\ y) = 0$によって表された$x$の関数$y$の導関数
\begin{equation}
	\bunsuu{dy}{dx} = -\bunsuu{\partial f}{\partial x} \left/ \bunsuu{\partial f}{\partial y}\right.
\end{equation}

\begin{tip}{導出}
	陰関数$y(x)$より,$f\bigl(x,\ y(x)\bigr) = 0$.両辺微分して
	\begin{equation}
		\bunsuu{\partial f}{\partial x}\bunsuu{dx}{dx} + \bunsuu{\partial f}{\partial y}\bunsuu{dy}{dx} = 0 \iff \bunsuu{\partial f}{\partial x} + \bunsuu{\partial f}{\partial y}\bunsuu{dy}{dx} = 0 \quad \left(\because \bunsuu{dx}{dx} = 1\right)
	\end{equation}
\end{tip}

$f(x,\ y,\ z) = 0$によって表された$x,\ y$の関数$z$の導関数
\begin{align}
	&\bunsuu{\partial z}{\partial x} = -\bunsuu{\partial f}{\partial x} \left/ \bunsuu{\partial f}{\partial z}\right.
	&
	&\bunsuu{\partial z}{\partial y} = -\bunsuu{\partial f}{\partial y} \left/ \bunsuu{\partial f}{\partial z}\right.
\end{align}

\begin{tip}{導出}
	陰関数$z(x,\ y)$より,$f\bigl(x,\ y,\ z(x,\ y)\bigr) = 0$.両辺$x$で偏微分して
	\begin{equation}
		\bunsuu{\partial f}{\partial x}\bunsuu{\partial x}{\partial x} + \bunsuu{\partial f}{\partial y}\bunsuu{\partial y}{\partial x} + \bunsuu{\partial f}{\partial z}\bunsuu{\partial z}{\partial x} = 0
		\iff \bunsuu{\partial f}{\partial x} + \bunsuu{\partial f}{\partial z}\bunsuu{\partial z}{\partial x} = 0 \quad \left(\because \bunsuu{\partial x}{\partial x} = 1,\ \bunsuu{\partial y}{\partial x} = 0\right)
	\end{equation}
	両辺$y$で偏微分すると第2式が得られる.
\end{tip}



\subsection{2変数のテイラーの定理}

$f(x,\ y)$が$n$次までの連続な偏導関数を持つとき($C^n$級関数),
\begin{equation*}
	\bunsuu{d^n f}{dt^n} = D^n f = \Bigl(h\bunsuu{\partial}{\partial x} + k\bunsuu{\partial}{\partial y}\Bigr)^n f
\end{equation*}
とおくと
\begin{equation}
	f(a + h,\ b + k) = f(a,\ b) + \bunsuu{1}{1!}Df(a,\ b) + \bunsuu{1}{2!}D^2f(a,\ b) + \cdots + \bunsuu{1}{(n - 1)!}D^{n - 1}f(a,\ b) + 
	\underbrace{
		\bunsuu{1}{n!}D^n f(a + \theta h,\ b + \theta k)
	}_{R_n}
\end{equation}
を満たす$\theta\ (0 < \theta < 1)$が存在する(\textbf{テイラーの定理}).

また,$R_n \xrightarrow{n \to \infty} 0$であるとき
\begin{equation}
	f(a + h,\ b + k) = f(a,\ b) + \bunsuu{1}{1!}Df(a,\ b) + \bunsuu{1}{2!}D^2f(a,\ b) + \cdots + \bunsuu{1}{(n - 1)!}D^{n - 1}f(a,\ b) + \bunsuu{1}{n!}D^n f(a,\ b) + \cdots
\end{equation}
が成り立つ(\textbf{テイラー展開}).

\begin{tip}{補足}
	$z = f(x,\ y)$で,$x = a + ht$,$y = b + kt$とすると,$z = f(a + ht,\ b + kt)$より,1変数$t$の関数になる.微分して
	\begin{equation}
		\bunsuu{df}{dt} = \bunsuu{dx}{dt}\bunsuu{\partial f}{\partial x} + \bunsuu{dy}{dt}\bunsuu{\partial f}{\partial y} = h\bunsuu{\partial f}{\partial x} + k\bunsuu{\partial f}{\partial y} \label{equ:par_deri-2}
	\end{equation}
	ここで,$D = \Bigl(h\bunsuu{\partial}{\partial x} + k\bunsuu{\partial}{\partial y}\Bigr)$とおくと,$\bunsuu{df}{dt}$は$Df = \Bigl(h\bunsuu{\partial}{\partial x} + k\bunsuu{\partial}{\partial y}\Bigr)f$と書ける.
	\vskip\baselineskip
	次に,$\bunsuu{d^2 f}{dt^2}$を考える.式(\ref{equ:par_deri-2})より
	\begin{equation*}
		\bunsuu{d^2 f}{dt^2} = h\bunsuu{d}{dt}\Bigl(\bunsuu{\partial f}{\partial x}\Bigr) + k\bunsuu{d}{dt}\Bigl(\bunsuu{\partial f}{\partial y}\Bigr)
	\end{equation*}
	今,$\bunsuu{\partial f}{\partial x},\ \bunsuu{\partial f}{\partial y}$は2変数$x,\ y$の関数なので,チェーン・ルールより
	\begin{gather*}
		\bunsuu{d}{dt}\Bigl(\bunsuu{\partial f}{\partial x}\Bigr)
		= \bunsuu{\partial}{\partial x}\Bigl(\bunsuu{\partial f}{\partial x}\Bigr)\bunsuu{dx}{dt} + \bunsuu{\partial}{\partial y}\Bigl(\bunsuu{\partial f}{\partial x}\Bigr)\bunsuu{dy}{dt}
		= h\bunsuu{\partial^2 f}{\partial x^2} + k\bunsuu{\partial^2 f}{\partial y \partial x}\\
		% 
		\bunsuu{d}{dt}\Bigl(\bunsuu{\partial f}{\partial y}\Bigr)
		= \bunsuu{\partial}{\partial x}\Bigl(\bunsuu{\partial f}{\partial y}\Bigr)\bunsuu{dx}{dt} + \bunsuu{\partial}{\partial y}\Bigl(\bunsuu{\partial f}{\partial y}\Bigr)\bunsuu{dy}{dt}
		= h\bunsuu{\partial^2 f}{\partial x \partial y} + k\bunsuu{\partial^2 f}{\partial y^2}
	\end{gather*}
	よって,代入して $\bunsuu{d^2 f}{dt^2} = h\Bigl(h\bunsuu{\partial^2 f}{\partial x^2} + k\bunsuu{\partial^2 f}{\partial y \partial x}\Bigr) + k\Bigl(h\bunsuu{\partial^2 f}{\partial x \partial y} + k\bunsuu{\partial^2 f}{\partial y^2}\Bigr)$\\
	第2次偏導関数が存在し,ともに連続とすると$\bunsuu{\partial^2 f}{\partial y \partial x} = \bunsuu{\partial^2 f}{\partial x \partial y}$なので,
	\begin{equation}
		\bunsuu{d^2 f}{dt^2} = h^2\bunsuu{\partial^2 f}{\partial x^2} + 2hk\bunsuu{\partial^2 f}{\partial x \partial y} + k^2\bunsuu{\partial^2 f}{\partial y^2} = \Bigl(h\bunsuu{\partial}{\partial x} + k\bunsuu{\partial}{\partial y}\Bigr)^2 f = D^2 f
	\end{equation}
	このように,$D$を用いると
	\begin{equation}
		\bunsuu{d^n f}{dt^n} = D^n f = \Bigl(h\bunsuu{\partial}{\partial x} + k\bunsuu{\partial}{\partial y}\Bigr)^n f
	\end{equation}
	と略記することができる.
\end{tip}

$a = b = 0$としたときのテイラーの定理を\textbf{マクローリンの定理}という.$h,\ k$の代わりに$x,\ y$でよく表す.
\begin{equation}
	f(x,\ y) = f(0,\ 0) + \bunsuu{1}{1!}Df(0,\ 0) + \bunsuu{1}{2!}D^2 f(0,\ 0) + \cdots + \bunsuu{1}{(n - 1)!}D^{n - 1} f(0,\ 0) + \bunsuu{1}{n!}D^n f(\theta x,\ \theta y)
\end{equation}
を満たす$\theta\ (0 < \theta < 1)$が存在する.



\subsection{包絡線}

変数$x,\ y$の他に任意定数$\alpha$を含んでいる方程式
\begin{equation}
	f(x,\ y,\ \alpha) = 0
\end{equation}
は$\alpha$を変化させて得られる全ての曲線の集合(曲線群)を表している.これを\textbf{曲線群の方程式}という.

曲線群の全ての曲線に接する曲線or直線を曲線群の\textbf{包絡線}という.包絡線上の点$(x,\ y)$は
\begin{equation}
	f(x,\ y,\ \alpha) = 0,\ \bunsuu{\partial}{\partial\alpha}f(x,\ y,\ \alpha) = 0
\end{equation}
を満たす.この2式を求めて,$\alpha$を消去すると包絡線の方程式が求まる.



\section{極値問題}
\subsection{2変数関数の極値}

\textbf{ヘッシアン}$H(a,\ b) =
\begin{vmatrix}
	f_{xx}(a,\ b) & f_{xy}(a,\ b)\\[5pt]
	f_{xy}(a,\ b) & f_{yy}(a,\ b)
\end{vmatrix}
$とおくと,点$(a,\ b)$に於いて
\begin{enumerate}[label=\textbf{[\arabic*]}, labelsep=10pt, leftmargin=23pt]
	\item $H(a,\ b) > 0$のとき
		\begin{align}
			&f_{xx}(a,\ b) > 0 \qLongright \text{点$(a,\ b)$で極小をとる} & &f_{xx}(a,\ b) < 0 \qLongright \text{点$(a,\ b)$で極大をとる}
		\end{align}
	\item $H(a,\ b) < 0$のとき
		\begin{equation}
			\text{点$(a,\ b)$では極値を取らない.}
		\end{equation}
	\item $H(a,\ b) = 0$のとき
		\begin{equation*}
			\text{極値の判定は出来ない.}
		\end{equation*}
\end{enumerate}

\begin{tip}{証明}
	テイラーの定理より
	\begin{equation*}
		f(a + h,\ b + k) - f(a,\ b) = Df(a,\ b) + \bunsuu{1}{2}D^2 f(a + \theta h,\ b + \theta k) \quad (0 < \theta < 1)
	\end{equation*}
	$(a,\ b)$で極値を取るので,$Df(a,\ b) = h \cdot 0 + k \cdot 0 = 0$より
	\begin{equation*}
		f(a + h,\ b + k) - f(a,\ b) = \bunsuu{1}{2}D^2 f(a + \theta h,\ b + \theta k)
	\end{equation*}
	となる.ここで,簡単のため
	\begin{align*}
		A &= f_{xx}(a + \theta h,\ b + \theta k), &
		B &= f_{xy}(a + \theta h,\ b + \theta k), &
		C &= f_{yy}(a + \theta h,\ b + \theta k)
	\end{align*}
	とおくと,$f(a + h,\ b + k) - f(a,\ b) = \bunsuu{1}{2}(Ah^2 + 2Bhk + Ck^2)$\\
	$h,\ k$が十分$0$に近ければ,$AC - B^2,\ A$の符号はそれぞれ$H,\ f_{xx}$の符号に等しくなる.\\
	変形して
	\begin{equation*}
		f(a + h,\ b + k) - f(a,\ b) = \bunsuu{\color{blue}A}{2}\Bigl\{\Bigl(h + \bunsuu{B}{A}k\Bigr)^2 + \bunsuu{\color{red}AC - B^2}{A^2}k^2\Bigr\}
	\end{equation*}
	\begin{enumerate}[label=\textbf{[\arabic*]}, labelsep=10pt, leftmargin=23pt]
		\item $H > 0$(即ち,${\color{red}AC - B^2} > 0$)のとき\\
			$f_{xx} > 0\ ({\color{blue}A} > 0)$ならば,$f(a + h,\ b + k) > f(a,\ b)$(下に凸)より,極小\\
			$f_{xx} < 0\ ({\color{blue}A} < 0)$ならば,$f(a + h,\ b + k) < f(a,\ b)$(上に凸)より,極大
		\item $H < 0$(即ち,${\color{red}AC - B^2} < 0$)のとき
			\begin{enumerate}[labelsep=10pt, leftmargin=23pt]
				\item[(ア)] $A \ne 0$または$C \ne 0$のとき\\
					${\color{red}AC - B^2} < 0$より,$Ah^2 + 2Bhk + Ck^2$は$h,\ k$によって正にも負にもなる.
				\item[(イ)] $A = C = 0$のとき\\
					${\color{red}AC - B^2} < 0$より$B \ne 0$なので,$Ah^2 + 2Bhk + Ck^2 = 2Bhk$は$h,\ k$によって正にも負にもなる.
			\end{enumerate}
	\end{enumerate}
\end{tip}



\subsection{条件付極値問題}

$xy$平面上の点$(x,\ y)$が条件$\varphi(x,\ y) = 0$で表される曲線上を動くとき,平面$x = f(x,\ y)$が極値を取り得る点
\begin{equation}
	\bunsuu{f_x}{\varphi_x} = \bunsuu{f_y}{\varphi_y} \quad (\varphi_x,\ \varphi_y \ne 0) \label{equ:par_deri-3}
\end{equation}

\begin{tip}{導出}
	方程式$\varphi(x,\ y) = 0$の$y$が$x$の関数,即ち$\varphi\bigl(x,\ y(x)\bigr) = 0$とすると,陰関数の微分法より
	\begin{equation}
		\bunsuu{dy}{dx} = -\bunsuu{\partial\varphi}{\partial x} \left/ \bunsuu{\partial \varphi}{\partial y}\right. \label{equ:par_deri-4}
	\end{equation}
	このとき,関数$z = f(x,\ y)$は$x$の関数となるのでチェーン・ルールより,極値をとるとき
	\begin{equation}
		\bunsuu{df}{dx} = \bunsuu{\partial f}{\partial x}\bunsuu{dx}{dx} + \bunsuu{\partial f}{\partial y}\bunsuu{dy}{dx} = 0 \label{equ:par_deri-5}
	\end{equation}
	式(\ref{equ:par_deri-5})に式(\ref{equ:par_deri-4})を代入して
	\begin{equation*}
		\bunsuu{\partial f}{\partial x} - \bunsuu{\partial f}{\partial y} \cdot \bunsuu{\bunsuu{\partial\varphi}{\partial x}}{\bunsuu{\partial\varphi}{\partial y}} = 0
	\end{equation*}
	整理すると得られる.
\end{tip}



\subsubsection*{ラグランジュの未定乗数法}

$f(x,\ y)$は条件$\varphi(x,\ y) = 0$のもとで,$(a,\ b)$で極値をとるとする.$\varphi_x(a,\ b) \ne 0$または$\varphi_y(a,\ b) \ne 0$であれば
\begin{equation}
	\left\{
		\begin{aligned}
			f_x(a,\ b) &= \lambda \cdot \varphi_x(a,\ b)\\
			f_y(a,\ b) &= \lambda \cdot \varphi_y(a,\ b) 
		\end{aligned}
	\right.\quad\text{を満たす$\lambda$が存在}
\end{equation}

\begin{tip}{導出}
	式(\ref{equ:par_deri-3})で,$\bunsuu{f_x}{\varphi_x} = \bunsuu{f_y}{\varphi_y} = \lambda$とおくことによって得られる.
\end{tip}

\vskip\baselineskip

\textbf{【条件付極値問題で,最大値・最小値を問われたとき】}

$\varphi(x,\ y) = 0$が端点をもたない場合,$z = f(x,\ y)$が連続関数であれば,極値を取り得る点が最大値・最小値になる.




\newpage
\chapter{重積分法}
\setcounter{page}{1}

\section{2重積分}
\subsection{定義}
\vskip-\baselineskip
\begin{equation}
	\iint_{D} f(x,\ y)\,dxdy = \lim_{|\varDelta| \to 0} \sum_{i = 1}^{l}\sum_{j = 1}^{m} f(\xi_i,\ \eta_j)\,\varDelta x_i \varDelta y_j
\end{equation}


\subsection{性質}
\vskip-\baselineskip
\begin{equation}
	\left|\iint_{D} f(x,\ y)\,dxdy\right| \le \iint_{D} |f(x,\ y)|\,dxdy
\end{equation}



\subsection{累次積分(逐次積分)}

\begin{kousiki}{累次積分(1)}
	$D = \{(x,\ y) \mid a \le x \le b,\ c \le y \le d\}$とする.
	\begin{equation}
		\iint_{D} f(x,\ y)\,dxdy = \int_{a}^{b} \left\{\int_{c}^{d} f(x,\ y)\,dy\right\}dx = \int_{c}^{d} \left\{\int_{a}^{b} f(x,\ y)\,dx\right\}\,dy
	\end{equation}
\end{kousiki}

範囲に関数が含まれている場合は変数を含んでいる方を先に計算する.
\begin{kousiki}{累次積分(2)}
	\begin{enumerate}[label=\textbf{[\arabic*]}, labelsep=10pt, leftmargin=23pt]
		\item $D = \{(x,\ y) \mid a \le x \le b,\ {\color{cyan}\varphi_1(x)} \le y \le {\color{cyan}\varphi_2(x)}\}$とする.
			\begin{equation}
				\iint_{D} f(x,\ y)\,dxdy = \int_{a}^{b} \left\{\int_{\textcolor{cyan}{\varphi_1(x)}}^{\textcolor{cyan}{\varphi_2(x)}} f(x,\ y)\,dy\right\}dx
			\end{equation}
		\item $D = \{(x,\ y) \mid {\color{cyan}\psi_1(y)} \le x \le {\color{cyan}\psi_2(y)},\ c \le y \le d\}$とする.
			\begin{equation}
				\iint_{D} f(x,\ y)\,dxdy = \int_{c}^{d} \left\{\int_{\textcolor{cyan}{\psi_1(y)}}^{\textcolor{cyan}{\psi_2(y)}} f(x,\ y)\,dx\right\}dy
			\end{equation}
	\end{enumerate}
\end{kousiki}



\subsection{積分順序の交換}

\begin{tip}{例題(1)}
	$y = \sqrt{x}$と$y = \bunsuu{1}{2}x$に囲まれた領域
	\tcblower
	\begin{align*}
		D &= \left\{(x,\ y) \relmiddle| 0 \le x \le 4,\ \bunsuu{1}{2}x \le y \le \sqrt{x}\right\}\\
		&= \{(x,\ y) \mid y^2 \le x \le 2y,\ 0 \le y \le 2\}
	\end{align*}
\end{tip}

\begin{tip}{例題(2)}
	$y = \log x \qLongright x = e^y$より
	\begin{equation*}
		\int_{1}^{e} \left\{\int_{0}^{\log x} f(x,\ y)\,dy\right\}dx = \int_{0}^{1} \left\{\int_{e^y}^{e} f(x,\ y)\,dx\right\}\,dy
	\end{equation*}
\end{tip}



\subsection{変数変換}

\begin{kousiki}{変数変換}
	$x = \varphi(u,\ v),\ y = \psi(u,\ v)$について
	\begin{equation}
		\iint_{D} f(x,\ y)\,dxdy = \iint_{D'} f(\varphi,\ \psi)\left|\det
			\begin{bmatrix}
				x_u & x_v\\ y_u & y_v
			\end{bmatrix}
		\right|\,dudv
	\end{equation}
\end{kousiki}

$\det\begin{bmatrix}
	x_u & x_v\\ y_u & y_v
\end{bmatrix}$を\textbf{ヤコビアン}といい,$\bunsuu{\partial(x,\ y)}{\partial(u,\ v)}$や$J(u,\ v)$で表す.また,$D'$は$D$を$u,\ v$で表しなおした領域である.



\subsection{極座標変換}

$x = r\cos\theta,\ y = r\sin\theta$より
\begin{align}
	\iint_{D} f(x,\ y)\,dxdy &= \iint_{D'} f(r\cos\theta,\ r\sin\theta) \left|\det
	\begin{bmatrix*}[r]
		\cos\theta & -r\sin\theta\\ \sin\theta & r\cos\theta
	\end{bmatrix*}
	\right|\,drd\theta \notag\\
	&= \iint_{D'} f(r\cos\theta,\ r\sin\theta)\,rdrd\theta
\end{align}

\newpage
% \chapter{微分方程式}
\setcounter{page}{1}


\section{1階線形常微分方程式}

以下の形の微分方程式を\textbf{1階線形常微分方程式}という.
\begin{equation}
	\bunsuu{dx}{dt} + P(t)x = Q(t)
\end{equation}

このとき,$Q(t) = 0$の場合,即ち
\begin{equation}
	\bunsuu{dx}{dt} + P(t)x = 0
\end{equation}
の場合を\textbf{斉次}といい,$Q(t) \ne 0$の場合,即ち
\begin{equation}
	\bunsuu{dx}{dt} + P(t)x = Q(t)
\end{equation}
の場合を\textbf{非斉次}という.



\subsection{変数分離形}

1階微分方程式で,
\begin{equation}
	\bunsuu{dx}{dt} = F(x)G(t)
\end{equation}
のように,$x$の関数と$t$の関数の積になる形を\textbf{変数分離形}という.

\noindent
\textbf{【解法】}
\begin{equation*}
	\bunsuu{dx}{dt} = 2tx
\end{equation*}
を例にする.

\begin{enumerate}[label=\textbf{[\arabic*]}, labelsep=10pt, leftmargin=23pt]
	\item 左辺を$x$だけの式,右辺を$t$だけの式にする.
		\begin{equation*}
			\bunsuu{dx}{x} = 2t\,dt
		\end{equation*}
	\item $\int$をつけて両辺積分する.
		\begin{gather*}
			\int \bunsuu{dx}{x} = \int 2t\,dt\\
			\log |x| = t^2 + C_1 \quad \text{($C_1$は任意定数)}
		\end{gather*}
	\item $x$について解く.
		\begin{gather*}
			x = \pm e^{t^2 + C_1} = \pm e^{C_1} e^{t^2}\\
		\intertext{$\pm e^{C_1} = C$とおいて}
			x = Ce^{t^2} \quad \text{($C$は任意定数)}
		\end{gather*}
\end{enumerate}


\subsection{定数変化法}

1階線形常微分方程式
\begin{equation}
	\bunsuu{dx}{dt} + P(t)x = Q(t)
\end{equation}
の一般解を求める.まず,$Q(t) = 0$といた斉次方程式を解く.これは変数分離形で解ける.その結果を用いて非斉次方程式の一般解を導く.

\noindent\textbf{【解法】}
\begin{equation}
	\bunsuu{dx}{dt} + \bunsuu{1}{t}x = 4t^2 + 1 \label{equ:ord_diff-1}
\end{equation}
を例にする.

\begin{enumerate}[label=\textbf{[\arabic*]}, labelsep=10pt, leftmargin=23pt]
	\item まず,斉次方程式
		\begin{equation}
			\bunsuu{dx}{dt} + \bunsuu{1}{t}x = 0
		\end{equation}
		の一般解を求める.
		\begin{gather*}
			\bunsuu{dx}{x} = -\bunsuu{dt}{t}\\
			\int \bunsuu{dx}{x} = -\int \bunsuu{dt}{t}\\
			\log |x| = -\log |t| + C_1 \quad \text{($C_1$は任意定数)}\\
			\log |x| + \log |t| = C_1\\
			\log |xt| = C_1\\
			\pm e^{C_1} = xt\\
			\intertext{$\pm e^{C_1} = C$とおいて}
			x = \bunsuu{C}{t}
		\end{gather*}
	\item これは斉次方程式の一般解である.求めたいのは
\end{enumerate}
% \chapter{ベクトル}
\setcounter{page}{1}

\section{ベクトルの成分表示}
\begin{equation}
	\bm{a} = a_1\bm{i} + a_2\bm{j} + a_3\bm{k} =
	\begin{bmatrix}
		a_1\\ a_2\\ a_3
	\end{bmatrix}
\end{equation}



\section{ベクトルの大きさ(ノルム)}
\vskip-\baselineskip
\begin{equation}
	|\bm{a}| = \sqrt{a_1{}^2 + a_2{}^2 + a_3{}^2}
\end{equation}



\section{内積(スカラー積)\texorpdfstring{$\bm{a} \cdot \bm{b}$}{AdotB}}

$\bm{0}$でない2つのベクトル$\bm{a},\ \bm{b}$のなす角が$\theta\ (0 \le \theta \le \pi)$のとき,
\begin{align}
	\bm{a} \cdot \bm{b} &= |\bm{a}||\bm{b}|\cos\theta\\
						&=
						\begin{bmatrix}
							a_x\\ a_y \\ a_z
						\end{bmatrix}
						\cdot
						\begin{bmatrix}
							b_x\\ b_y \\ b_z
						\end{bmatrix}
						= a_x b_x + a_y b_y + a_z b_z
\end{align}

\begin{kousiki}{内積の性質}
	\begin{enumerate}[label=\textbf{[\arabic*]}, labelsep=10pt, leftmargin=23pt]
		\item $\bm{a} \cdot \bm{b} = \bm{b} \cdot \bm{a}$ \qquad $\bm{a} \cdot \bm{a} = |\bm{a}|^2$
		\item $\bm{a} \cdot (\bm{b} + \bm{c}) = \bm{a} \cdot \bm{b} + \bm{a} \cdot \bm{c}$ \qquad $(\bm{a} + \bm{b}) \cdot \bm{c} = \bm{a} \cdot \bm{c} + \bm{b} \cdot \bm{c}$
		\item $k(\bm{a} \cdot \bm{b}) = (k\bm{a}) \cdot \bm{b} = \bm{a} \cdot (k\bm{b})$ \qquad $(k \in \mathbb{R})$
		\item
			\begin{enumerate}[label=(\roman*), labelsep=10pt, leftmargin=23pt]
				\item $\bm{i} \cdot \bm{i} = \bm{j} \cdot \bm{j} = \bm{k} \cdot \bm{k} = 1$
				\item $\bm{i} \cdot \bm{j} = \bm{j} \cdot \bm{k} = \bm{k} \cdot \bm{i} = 0$
			\end{enumerate}
		\item $\bm{a} \ne \bm{0},\ \bm{b} \ne \bm{0}$のとき,$\bm{a} \perp \bm{b} \iff \bm{a} \cdot \bm{b} = 0$
	\end{enumerate}
	
\end{kousiki}



\section{外積(ベクトル積)\texorpdfstring{$\bm{a} \times \bm{b}$}{AtimesB}}

$\bm{0}$でない2つのベクトル$\bm{a},\ \bm{b}$のなす角が$\theta\ (0 < \theta < \pi)$のとき,
\begin{align}
	\bm{a} \times \bm{b} &= (|\bm{a}||\bm{b}|\sin\theta)\bm{e}\\
	&= \det
	\begin{bmatrix}
		\bm{i} & a_x & b_x\\
		\bm{j} & a_y & b_y\\
		\bm{k} & a_z & b_z
	\end{bmatrix}
\end{align}
$\bm{e}$は,$\bm{a},\ \bm{b},\ \bm{e}$がこの順で右手系を成す向き.$\bm{a} = \bm{0},\ \bm{b} = \bm{0}$のとき,又は$\theta = 0,\ \pi$のときは$\bm{a} \times \bm{b} = \bm{0}$とする.

\begin{kousiki}{外積の性質}
	\begin{enumerate}[label=\textbf{[\arabic*]}, labelsep=10pt, leftmargin=23pt]
		\item $\bm{a} \times \bm{b} = -\bm{b} \times \bm{a}$
		\item 
			\begin{enumerate}[label=(\roman*), labelsep=10pt, leftmargin=23pt]
				\item $\bm{a} \times (\bm{b} + \bm{c}) = \bm{a} \times \bm{b} + \bm{a} \times \bm{c}$
				\item $(\bm{a} + \bm{b}) \times \bm{c} = \bm{a} \times \bm{c}$
			\end{enumerate}
		\item $k(\bm{a} \times \bm{b}) = (k\bm{a}) \times \bm{b} = \bm{a} \times (k\bm{b})$ \qquad $(k \in \mathbb{R})$
		\item
			\begin{enumerate}[label=(\roman*), labelsep=10pt, leftmargin=23pt]
				\item $\bm{i} \times \bm{i} = \bm{j} \times \bm{j} = \bm{k} \times \bm{k} = \bm{0}$
				\item $\bm{i} \times \bm{j} = \bm{k},\quad \bm{j} \times \bm{k} = \bm{i},\quad \bm{k} \times \bm{i} = \bm{j}$
			\end{enumerate}
		\item $\bm{a} \ne \bm{0},\ \bm{b} \ne \bm{0}$のとき,$\bm{a} \heikou \bm{b} \iff \bm{a} \times \bm{b} = \bm{0}$
	\end{enumerate}
	
\end{kousiki}

$\bm{a}$と$\bm{b}$の成す平行四辺形の面積$S$は
\begin{equation}
	S = |\bm{a} \times \bm{b}|
\end{equation}
で求められる.



\section{三重積}
\subsection{スカラー三重積}

3つのベクトル$\bm{a},\ \bm{b},\ \bm{c}$について
\begin{align}
	\bm{a} \cdot (\bm{b} \times \bm{c}) &= |\bm{a}||\bm{b} \times \bm{c}|\cos\varphi \\
	&= \det
	\begin{bmatrix}
		\bm{a} & \bm{b} & \bm{c}
	\end{bmatrix}
	= \det
	\begin{bmatrix}
		a_x & b_x & c_x\\
		a_y & b_y & c_y\\
		a_z & b_z & c_z
	\end{bmatrix}
\end{align}
を\textbf{スカラー三重積}という.$\varphi$は$\bm{a}$と$\bm{b} \times \bm{c}$の成す角である.

\begin{kousiki}{スカラー三重積の性質}
	\begin{enumerate}[label=\textbf{[\arabic*]}, labelsep=10pt, leftmargin=23pt]
		\item $\bm{a} \cdot (\bm{b} \times \bm{c}) = \bm{b} \cdot (\bm{c} \times \bm{a}) = \bm{c} \cdot (\bm{a} \times \bm{b})$
		\item $\bm{a},\ \bm{b},\ \bm{c}$が同一平面上になく,この順で右手系を成すとき,$\bm{a},\ \bm{b},\ \bm{c}$の成す平行六面体の体積$V$は
		\begin{equation}
			V = \bm{a} \cdot (\bm{b} \times \bm{c})
		\end{equation}
		で求められる.
	\end{enumerate}
\end{kousiki}

\textbf{[1]}は,行列式の性質から分かる.\textbf{[2]}について,$\bm{a},\ \bm{b},\ \bm{c}$がこの順で左手系を成すとき$\left(\bunsuu{\pi}{2} < \varphi \le \pi\right)$,$V$は
\begin{equation}
	V = -\bm{a} \cdot (\bm{b} \times \bm{c})
\end{equation}
である.



\subsection{ベクトル三重積}

3つのベクトル$\bm{a},\ \bm{b},\ \bm{c}$について
\begin{equation}
	\bm{a} \times (\bm{b} \times \bm{c})
\end{equation}
を\textbf{ベクトル三重積}という.

\begin{kousiki}{Lagrangeの公式}
	\begin{enumerate}[label=\textbf{[\arabic*]}, labelsep=10pt, leftmargin=23pt]
		\item $\bm{a} \times (\bm{b} \times \bm{c}) = (\bm{a} \cdot \bm{c})\bm{b} - (\bm{a} \cdot \bm{b})\bm{c}$
		\item $(\bm{a} \times \bm{b}) \times \bm{c} = (\bm{a} \cdot \bm{c})\bm{b} - (\bm{b} \cdot \bm{c})\bm{a}$
	\end{enumerate}
\end{kousiki}



\section{ベクトルの平行条件・垂直条件}

$\bm{a} \ne \bm{0}$,$\bm{b} \ne \bm{0}$のとき
\begin{enumerate}[label=\textbf{[\arabic*]}, labelsep=10pt, leftmargin=23pt]
	\item $\bm{a} \heikou \bm{b} \iff \bm{a} \times \bm{b} = \bm{0} \iff \text{$\bm{b} = k\bm{a}$を満たす$k \in \mathbb{R}$が存在}$
	\item $\bm{a} \perp \bm{b} \iff \bm{a} \cdot \bm{b} = 0$
\end{enumerate}



\section{図形への応用}
\subsection{内分点}

$\mathrm{A}(\bm{a})$と$\mathrm{B}(\bm{b})$を結ぶ線分を$m : n$に内分する点の位置ベクトル$\mathrm{P}(\boldsymbol{p})$
\begin{equation}
	\bm{p} = \bunsuu{n\bm{a} + m\bm{b}}{m + n}
\end{equation}



\subsection{外分点}

$\mathrm{A}(\bm{a})$と$\mathrm{B}(\bm{b})$を結ぶ線分を$m : n$に外分する点の位置ベクトル$\mathrm{Q}(\bm{q})$
\begin{equation}
	\bm{q} = \bunsuu{-n\bm{a} + m\bm{b}}{m - n}
\end{equation}



\subsection{中点}

$\mathrm{A}(\bm{a})$と$\mathrm{B}(\bm{b})$を結ぶ線分の中点の位置ベクトル$\mathrm{M}(\bm{m})$
\begin{equation}
	\bm{m} = \bunsuu{\bm{a} + \bm{b}}{2}
\end{equation}



\subsection{重心}

$\mathrm{A}(\bm{a})$,$\mathrm{B}(\bm{b})$,$\mathrm{C}(\bm{c})$を結んでできる三角形の重心の位置ベクトル$\mathrm{G}(\bm{g})$
\begin{equation}
	\bm{g} = \bunsuu{\bm{a} + \bm{b} + \bm{c}}{3}
\end{equation}



\subsection{直線のベクトル方程式}

直線上の1点を$\mathrm{P}(\bm{p})$とする.
\begin{enumerate}[label=\textbf{[\arabic*]}, labelsep=10pt, leftmargin=23pt]
	\item $\mathrm{A}(a)$を通り$\bm{u}$に平行
		\begin{equation}
			\bm{p} = \bm{a} + t\bm{u} \quad (t \in \mathbb{R})
		\end{equation}
	\item $\mathrm{A}(\bm{a})$,$\mathrm{B}(\bm{b})$を通る
		\begin{equation}
			\bm{p} = (1 - t)\bm{a} + t\bm{b} \hspace*{4\zw} \text{($\bm{u} = \bm{b} - \bm{a}$とする)}
		\end{equation}
	\item $\mathrm{A}(\bm{a})$を通り,$\bm{n} (\ne \bm{0})$に垂直な直線
		\begin{equation}
			\bm{n} \cdot (\bm{p} - \bm{a}) = 0
		\end{equation}
		直線が$ax + by + c = 0$のとき,$\bm{n} =
		\begin{bmatrix}
			a\\ b
		\end{bmatrix}
		$
\end{enumerate}



\subsection{直線との距離}

直線$ax + by + c = 0$と$\mathrm{A}(x_0,\ y_0)$との距離
\begin{equation}
	\bunsuu{|ax_0 + by_0 + c|}{\sqrt{a^2 + b^2}}
\end{equation}



\subsection{三角形の面積}

$\triangle\mathrm{OAB}$に於いて,$\vecrm{OA} = \bm{a}$,$\vecrm{OB} = \bm{b}$としたときの三角形の面積$S$
\begin{align}
	&S = \bunsuu{1}{2}\sqrt{|\bm{a}|^2|\bm{b}|^2 - (\bm{a} \cdot \bm{b})^2}
	&
	&S = \bunsuu{1}{2}|a_x b_y - a_y b_x| \quad \text{(平面の場合)}
\end{align}



\subsection{円のベクトル方程式}

中心$\mathrm{C}(\bm{c})$,半径$r$の円のベクトル方程式
\begin{equation}
	(\bm{p} - \bm{c}) \cdot (\bm{p} - \bm{c}) = r^2
\end{equation}



\subsection{平面のベクトル方程式}

$\mathrm{A}(\bm{a})$を通り,$\bm{n}$に垂直な平面のベクトル方程式
\begin{equation}
	\bm{n} \cdot (\bm{p} - \bm{a}) = 0
\end{equation}
平面が$ax + by + cz + d = 0$のとき,$\bm{n} =
\begin{bmatrix}
	a\\ b\\ c
\end{bmatrix}
$


\subsection{球面のベクトル方程式}

中心$\mathrm{C}(\bm{c})$,半径$r$の球面のベクトル方程式
\begin{equation}
	(\bm{p} - \bm{c}) \cdot (\bm{p} - \bm{c}) = r^2
\end{equation}



\subsection{平面との距離}

平面$ax + by + cz + d = 0$と$\mathrm{A}(x_0,\ y_0,\ z_0)$との距離
\begin{equation}
	\bunsuu{|ax_0 + by_0 + cz_0 + d|}{\sqrt{a^2 + b^2 + c^2}}
\end{equation}
\quad
% \chapter{行列}
\setcounter{page}{1}
% \chapter{行列式}
\setcounter{page}{1}
% \chapter{線形変換}
\setcounter{page}{1}
% \chapter{固有値とその応用}
\setcounter{page}{1}
% \chapter{ベクトル空間}
\setcounter{page}{1}



\section{数ベクトル空間}
\subsection{数ベクトル空間}

平面のベクトル全体,空間のベクトル全体の集合をそれぞれ$\mathbb{R}^2,\ \mathbb{R}^3$と表す:
\begin{equation}
	\mathbb{R}^2 =
	\left\{
		\begin{bmatrix}
			x_1\\ x_2
		\end{bmatrix}
		\relmiddle|
		x_1,\ x_2 \in \mathbb{R}
	\right\},\quad
	\mathbb{R}^3 =
	\left\{
		\begin{bmatrix}
			x_1\\ x_2\\ x_3
		\end{bmatrix}
		\relmiddle|
		x_1,\ x_2,\ x_3 \in \mathbb{R}
	\right\}
\end{equation}

これを拡張して,$n$個の実数の組を\textbf{$n$次元数ベクトル}といい,その全体を\textbf{$n$次元数ベクトル空間}$\mathbb{R}^n$という.
\begin{equation}
	\bm{x} =
	\begin{bmatrix}
		x_1\\ x_2\\ \vdots\\ x_n
	\end{bmatrix}
	,\quad \mathbb{R}^n =
	\left\{
		\bm{x} =
		\begin{bmatrix}
			x_1\\ x_2\\ \vdots\\ x_n
		\end{bmatrix}
		\relmiddle|
		x_1,\ x_2,\ \cdots,\ x_n \in \mathbb{R}
	\right\}
\end{equation}

《注》ベクトルを行ベクトルによって書くこともある.

\begin{kousiki}{ベクトルの性質}
	$\bm{x},\ \bm{y},\ \bm{z}$がベクトルで,$\lambda,\ \mu$がスカラーのとき
	\begin{enumerate}[label=\textbf{[\arabic*]}, labelsep=10pt, leftmargin=23pt]
		\item $\bm{x} + \bm{y} = \bm{y} + \bm{x}$
		\item $(\bm{x} + \bm{y}) + \bm{z} = \bm{x} + (\bm{y} + \bm{z})$
		\item $\bm{x} + \bm{0} = \bm{x}$
		\item $\bm{x} + (-\bm{x}) = \bm{0}$
		\item $\lambda(\mu\bm{x}) = (\lambda\mu)\bm{x}$
		\item $(\lambda + \mu)\bm{x} = \lambda\bm{x} + \mu\bm{x}$
		\item $\lambda(\bm{x} + \bm{y}) = \lambda\bm{x} + \lambda\bm{y}$
		\item $1\bm{x} = \bm{x}$
	\end{enumerate}
\end{kousiki}
% \chapter{確率}
\setcounter{page}{1}
% \chapter{データの整理}
\setcounter{page}{1}
% \chapter{確率分布}
\setcounter{page}{1}
% \chapter{推定と検定}
\setcounter{page}{1}
% \chapter{ベクトル解析}
\setcounter{page}{1}

\noindent
\textbf{【注意】}
\begin{enumerate}[leftmargin=15pt]
	\item 特に断りがない限り,ベクトル$\bm{a} =
			\begin{bmatrix}
				a_x\\ a_y\\ a_z
			\end{bmatrix}
			$とする.$\bm{b},\ \bm{c},\ \cdots$も同様.
	\item 行列$A$の転置行列は$A^\top$と表す.ここでは列ベクトルに分数が入って縦に長くなったときに使う.\\
	\f{例} $\bm{a} =
	\begin{bmatrix}
		a_x\\ a_y\\ a_z
	\end{bmatrix}
	=
	{
		\begin{bmatrix}
			a_x & a_y & a_z
		\end{bmatrix}
	}^\top
	$
	\item $\bm{i},\ \bm{j},\ \bm{k}$はそれぞれ$x$軸,$y$軸,$z$軸方向の基本ベクトルとする.
\end{enumerate}



\section{内積(スカラー積)}

$\bm{0}$でない2つのベクトル$\bm{a},\ \bm{b}$のなす角が$\theta\ (0 \le \theta \le \pi)$のとき,
\begin{align}
	\bm{a} \cdot \bm{b} &= |\bm{a}||\bm{b}|\cos\theta\\
						&=
						\begin{bmatrix}
							a_x & a_y & a_z
						\end{bmatrix}
						\cdot
						\begin{bmatrix}
							b_x\\ b_y \\ b_z
						\end{bmatrix}
						= a_x b_x + a_y b_y + a_z b_z
\end{align}
を\textbf{内積}または\textbf{スカラー積}という.

\begin{kousiki}{内積の性質}
	\begin{enumerate}[label=\textbf{[\arabic*]}, labelsep=10pt, leftmargin=23pt]
		\item $\bm{a} \cdot \bm{b} = \bm{b} \cdot \bm{a}$ \qquad $\bm{a} \cdot \bm{a} = |\bm{a}|^2$
		\item $\bm{a} \cdot (\bm{b} + \bm{c}) = \bm{a} \cdot \bm{b} + \bm{a} \cdot \bm{c}$ \qquad $(\bm{a} + \bm{b}) \cdot \bm{c} = \bm{a} \cdot \bm{c}$
		\item $k(\bm{a} \cdot \bm{b}) = (k\bm{a}) \cdot \bm{b} = \bm{a} \cdot (k\bm{b})$ \qquad $(k \in \mathbb{R})$
		\item
			\begin{enumerate}[label=(\roman*), labelsep=10pt, leftmargin=23pt]
				\item $\bm{i} \cdot \bm{i} = \bm{j} \cdot \bm{j} = \bm{k} \cdot \bm{k} = 1$
				\item $\bm{i} \cdot \bm{j} = \bm{j} \cdot \bm{k} = \bm{k} \cdot \bm{i} = 0$
			\end{enumerate}
		\item $\bm{a} \ne \bm{0},\ \bm{b} \ne \bm{0}$のとき,$\bm{a} \perp \bm{b} \iff \bm{a} \cdot \bm{b} = 0$
	\end{enumerate}
	
\end{kousiki}



\section{外積(ベクトル積)}

$\bm{0}$でない2つのベクトル$\bm{a},\ \bm{b}$のなす角が$\theta\ (0 < \theta < \pi)$のとき,
\begin{align}
	\bm{a} \times \bm{b} &= (|\bm{a}||\bm{b}|\sin\theta)\bm{e}\\
	&= 
	\begin{vmatrix}
		\bm{i} & a_x & b_x\\
		\bm{j} & a_y & b_y\\
		\bm{k} & a_z & b_z
	\end{vmatrix}
\end{align}
を\textbf{外積}または\textbf{ベクトル積}という.$\bm{e}$は,$\bm{a},\ \bm{b},\ \bm{e}$がこの順で右手系を成す向き.行列式は形式的な表現である.

$\bm{a} = \bm{0},\ \bm{b} = \bm{0}$のとき,又は$\theta = 0,\ \pi$のときは$\bm{a} \times \bm{b} = \bm{0}$とする.

\begin{kousiki}{外積の性質}
	\begin{enumerate}[label=\textbf{[\arabic*]}, labelsep=10pt, leftmargin=23pt]
		\item $\bm{a} \times \bm{b} = -\bm{b} \times \bm{a}$
		\item 
			\begin{enumerate}[label=(\roman*), labelsep=10pt, leftmargin=23pt]
				\item $\bm{a} \times (\bm{b} + \bm{c}) = \bm{a} \times \bm{b} + \bm{a} \times \bm{c}$
				\item $(\bm{a} + \bm{b}) \times \bm{c} = \bm{a} \times \bm{c}$
			\end{enumerate}
		\item $k(\bm{a} \times \bm{b}) = (k\bm{a}) \times \bm{b} = \bm{a} \times (k\bm{b})$ \qquad $(k \in \mathbb{R})$
		\item
			\begin{enumerate}[label=(\roman*), labelsep=10pt, leftmargin=23pt]
				\item $\bm{i} \times \bm{i} = \bm{j} \times \bm{j} = \bm{k} \times \bm{k} = \bm{0}$
				\item $\bm{i} \times \bm{j} = \bm{k},\quad \bm{j} \times \bm{k} = \bm{i},\quad \bm{k} \times \bm{i} = \bm{j}$
			\end{enumerate}
		\item $\bm{a} \ne \bm{0},\ \bm{b} \ne \bm{0}$のとき,$\bm{a} \heikou \bm{b} \iff \bm{a} \times \bm{b} = \bm{0}$
	\end{enumerate}
	
\end{kousiki}

$\bm{a}$と$\bm{b}$の成す平行四辺形の面積$S$は
\begin{equation}
	S = |\bm{a} \times \bm{b}|
\end{equation}
で求められる.



\section{三重積}
\subsection{スカラー三重積}

3つのベクトル$\bm{a},\ \bm{b},\ \bm{c}$について
\begin{align}
	\bm{a} \cdot (\bm{b} \times \bm{c}) &= |\bm{a}||\bm{b} \times \bm{c}|\cos\varphi \\
	&=
	\begin{vmatrix}
		\bm{a} & \bm{b} & \bm{c}
	\end{vmatrix}
	=
	\begin{vmatrix}
		a_x & b_x & c_x\\
		a_y & b_y & c_y\\
		a_z & b_z & c_z
	\end{vmatrix}
\end{align}
を\textbf{スカラー三重積}という.$\varphi$は$\bm{a}$と$\bm{b} \times \bm{c}$の成す角である.

\begin{kousiki}{スカラー三重積の性質}
	\begin{enumerate}[label=\textbf{[\arabic*]}, labelsep=10pt, leftmargin=23pt]
		\item $\bm{a} \cdot (\bm{b} \times \bm{c}) = \bm{b} \cdot (\bm{c} \times \bm{a}) = \bm{c} \cdot (\bm{a} \times \bm{b})$
		\item $\bm{a},\ \bm{b},\ \bm{c}$が同一平面上になく,この順で右手系を成すとき,$\bm{a},\ \bm{b},\ \bm{c}$の成す平行六面体の体積$V$は
		\begin{equation}
			V = \bm{a} \cdot (\bm{b} \times \bm{c})
		\end{equation}
		で求められる.
	\end{enumerate}
\end{kousiki}

\textbf{[1]}は,行列式の性質から分かる.\textbf{[2]}について,$\bm{a},\ \bm{b},\ \bm{c}$がこの順で左手系を成すとき$\left(\bunsuu{\pi}{2} < \varphi \le \pi\right)$,$V$は
\begin{equation}
	V = -\bm{a} \cdot (\bm{b} \times \bm{c})
\end{equation}
である.



\subsection{ベクトル三重積}

3つのベクトル$\bm{a},\ \bm{b},\ \bm{c}$について
\begin{equation}
	\bm{a} \times (\bm{b} \times \bm{c})
\end{equation}
を\textbf{ベクトル三重積}という.

\begin{kousiki}{Lagrangeの公式}
	\begin{enumerate}[label=\textbf{[\arabic*]}, labelsep=10pt, leftmargin=23pt]
		\item $\bm{a} \times (\bm{b} \times \bm{c}) = (\bm{a} \cdot \bm{c})\bm{b} - (\bm{a} \cdot \bm{b})\bm{c}$
		\item $(\bm{a} \times \bm{b}) \times \bm{c} = (\bm{a} \cdot \bm{c})\bm{b} - (\bm{b} \cdot \bm{c})\bm{a}$
	\end{enumerate}
\end{kousiki}



\section{ベクトル関数の微分法}
\subsection{ベクトル関数の極限と連続}

$t_0$を定数,$\bm{c} = 
\begin{bmatrix}
	c_1\\ c_2\\ c_3
\end{bmatrix}
$を定ベクトルとする.$t$をスカラー変数とするベクトル関数$\bm{f}(t) = 
\begin{bmatrix}
	f_1(t)\\ f_2(t)\\ f_3(t)
\end{bmatrix}
$について,$t$が$t_0$に限りなく近づくときの$\bm{f}(t)$の\textbf{極限}は次で定義される.
\begin{align}
	\lim_{t \to t_0} |\bm{f}(t) - \bm{c}| = 0 &\iff \lim_{t \to t_0} \bm{f}(t) = \bm{c}\\
	&\iff \lim_{t \to t_0} f_1(t) = c_1,\quad \lim_{t \to t_0} f_2(t) = c_2,\quad \lim_{t \to t_0} f_3(t) = c_3
\end{align}

また,$\dlim_{t \to t_0} \bm{f}(t) = \bm{f}(t_0)$が成立するとき,$\bm{f}(t)$は$t = t_0$で\textbf{連続である}という.



\subsection{ベクトル関数の微分法}

ベクトル関数$\bm{f}(t)$に於いて,次式を$\bm{f}(t)$の\textbf{導関数}という.
\begin{align}
	\bunsuu{d\bm{f}(t)}{dt} &= \lim_{\varDelta t \to 0} \bunsuu{\bm{f}(t + \varDelta t) - \bm{f}(t)}{\varDelta t} =
	{
	\begin{bmatrix}
		\bunsuu{df_1(t)}{dt} &
		\bunsuu{df_2(t)}{dt} &
		\bunsuu{df_3(t)}{dt}
	\end{bmatrix}
	}^\top
\end{align}

\begin{kousiki}{微分法の公式}
	$\bm{f}(t),\ \bm{g}(t)$をベクトル関数,$\varphi(t)$をスカラー関数,$\bm{c}$を定ベクトルとする.
	\begin{enumerate}[label=\textbf{[\arabic*]}, labelsep=10pt, leftmargin=23pt, itemsep=6pt]
		\item $\bunsuu{d\bm{c}}{dt} = \bm{0}$
		\item $\bunsuu{d(\bm{f} \pm \bm{g})}{dt} = \bunsuu{d\bm{f}}{dt} \pm \bunsuu{d\bm{g}}{dt}$\hfill(複号同順)
		\item $\bunsuu{d(\varphi\bm{f})}{dt} = \bunsuu{d\varphi}{dt}\bm{f} + \varphi\bunsuu{d\bm{f}}{dt}$
		\item $\bunsuu{d(\bm{f} \cdot \bm{g})}{dt} = \bunsuu{d\bm{f}}{dt} \cdot \bm{g} + \bm{f} \cdot \bunsuu{d\bm{g}}{dt}$
		\item $\bunsuu{d(\bm{f} \times \bm{g})}{dt} = \bunsuu{d\bm{f}}{dt} \times \bm{g} + \bm{f} \times \bunsuu{d\bm{g}}{dt}$
		\item $\bunsuu{d}{dt}\left(\bunsuu{\bm{f}}{\varphi}\right) = \bunsuu{\bunsuu{d\bm{f}}{dt}\varphi - \bm{f}\bunsuu{d\varphi}{dt}}{\varphi^2}$\qquad$(\varphi \ne 0)$
		\item $t = \psi(u)$をスカラー関数とすると \qquad $\bunsuu{d\bm{f}}{du} = \bunsuu{d\bm{f}}{dt}\bunsuu{d\psi}{du}$\hfill(合成関数の微分法)
	\end{enumerate}
\end{kousiki}



\subsection{接線ベクトル}

点$\mathrm{P}$の位置ベクトルが$\bm{r} =
\begin{bmatrix}
	x(t)\\ y(t)\\ z(t)
\end{bmatrix}
$のようにベクトル関数であるとき,$t$が変化するにつれて$\mathrm{P}$はある曲線$C$を描く.この$C$を$\bm{r} = \bm{r}(t)$の表す曲線という.

以下,特に断りがない限り,$\bm{r}(t)$は何度でも微分可能で$\bunsuu{d\bm{r}}{dt} \ne \bm{0}$とする.

$t$の微小変化$\varDelta t$に対応する$\bm{r}(t)$の変化を$\varDelta \bm{r}$とすると$\varDelta \bm{r} = \bm{r}(t + \varDelta t) - \bm{r}(t)$である.
\begin{equation}
	\bunsuu{d\bm{r}}{dt} = \lim_{\varDelta t \to 0} \bunsuu{\varDelta \bm{r}}{\varDelta t} = \lim_{\varDelta t \to 0} \bunsuu{\bm{r}(t + \varDelta t) - \bm{r}(t)}{\varDelta t}
\end{equation}
を曲線$C$の点$\mathrm{P}$における\textbf{接線ベクトル}という.

\begin{enumerate}[leftmargin=18pt, labelsep=10pt, labelsep=10pt, itemindent=9pt]
	\item[\f{例}] $\bm{r}(t) =
		\begin{bmatrix}
			t^2\\ t^3\\ t^4
		\end{bmatrix}
		$ならば,接線ベクトルは$\bunsuu{d}{dt}\bm{r}(t) =
		\begin{bmatrix}
			2t\\ 3t^2\\ 4t^3
		\end{bmatrix}
		$である.特に,$t = 1$の位置$\bm{r}(1) =
		\begin{bmatrix}
			1\\ 1\\ 1
		\end{bmatrix}
		$における接線ベクトルは$\bunsuu{d}{dt}\bm{r}(1) =
		\begin{bmatrix}
			2\\ 3\\ 4
		\end{bmatrix}
		$である.
\end{enumerate}



\subsection{単位接線ベクトル}

接線ベクトルを自身の大きさで割った,大きさ$1$の接線ベクトルを\textbf{単位接線ベクトル}$\bm{t}$という.
\begin{equation}
	\bm{t} = \bunsuu{\bunsuu{d\bm{r}}{dt}}{\left|\bunsuu{d\bm{r}}{dt}\right|} \label{equ2-16}
\end{equation}

曲線上のある定点$\mathrm{A}$を基準として,そこからの長さ$s$によって位置ベクトル$\bm{r}(s)$を定義する方法をとる.$\mathrm{A,\ P}$の位置ベクトルをそれぞれ$\bm{r}(\alpha),\ \bm{r}(t)$とすると,$\mathrm{AP}$の長さ$s$が次のようになるので,$\bunsuu{ds}{dt}$が求められる.
\begin{gather}
	s = s(t) = \int_{\alpha}^{t} \sqrt{\left(\bunsuu{dx}{dt}\right)^2 + \Bigl(\bunsuu{dy}{dt}\Bigr)^2 + \left(\bunsuu{dz}{dt}\right)^2}\,dt\\
	\bunsuu{ds}{dt} = \sqrt{\left(\bunsuu{dx}{dt}\right)^2 + \Bigl(\bunsuu{dy}{dt}\Bigr)^2 + \left(\bunsuu{dz}{dt}\right)^2} = \sqrt{\bunsuu{d\bm{r}}{dt} \cdot \bunsuu{d\bm{r}}{dt}} = \left|\bunsuu{d\bm{r}}{dt}\right|
\end{gather}

よって,$\bunsuu{d\bm{r}}{ds}$は合成関数の微分法より
\begin{equation}
	\bunsuu{d\bm{r}}{ds} = \bunsuu{d\bm{r}}{dt}\bunsuu{dt}{ds} = \bunsuu{\bunsuu{d\bm{r}}{dt}}{\bunsuu{ds}{dt}} = \bunsuu{\bunsuu{d\bm{r}}{dt}}{\left|\bunsuu{d\bm{r}}{dt}\right|}
\end{equation}
これは(\ref{equ2-16})と同じ式である.よって次が成り立つ.

\begin{kousiki}{単位接線ベクトル}
	\begin{equation}
		\bm{t} = \bunsuu{\bunsuu{d\bm{r}}{dt}}{\left|\bunsuu{d\bm{r}}{dt}\right|} = \bunsuu{d\bm{r}}{ds}
	\end{equation}
\end{kousiki}

一方,$\bm{t} \cdot \bm{t} = |\bm{t}|^2 = 1$の両辺を$t$で微分すると,内積の微分公式より,$2\bunsuu{d\bm{t}}{dt} \cdot \bm{t} = 0$なので,$\bunsuu{d\bm{t}}{dt} \cdot \bm{t} = 0$である.つまり,$\bunsuu{d\bm{t}}{dt} \perp \bm{t}$が成立する$\left(\text{$\bunsuu{d\bm{t}}{dt} \ne \bm{0}$ならば}\right)$.よって,次を\textbf{単位主法線ベクトル}とし,$\bm{n}$で表すと
\begin{equation}
	\bm{n} = \bunsuu{\bunsuu{d\bm{t}}{dt}}{\left|\bunsuu{d\bm{t}}{dt}\right|}
\end{equation}
となる.

\begin{kousiki}{単位主法線ベクトル}
	\begin{equation}
		\bm{n} = \bunsuu{\bunsuu{d\bm{t}}{dt}}{\left|\bunsuu{d\bm{t}}{dt}\right|} = \bunsuu{\bunsuu{d\bm{t}}{ds}}{\left|\bunsuu{d\bm{t}}{ds}\right|}
	\end{equation}
\end{kousiki}



\section{2変数ベクトル関数の微分法}
\subsection{2変数ベクトル関数の極限と連続}

$u_0,\ v_0$を定数,$\bm{c} = 
\begin{bmatrix}
	c_1\\ c_2\\ c_3
\end{bmatrix}
$を定ベクトルとする.$u,\ v$をスカラー変数とするベクトル関数$\bm{f}(u,\ v) = 
\begin{bmatrix}
	f_1(u,\ v)\\ f_2(u,\ v)\\ f_3(u,\ v)
\end{bmatrix}
$について,$(u,\ v)$が$(u_0,\ v_0)$に限りなく近づくときの$\bm{f}(u,\ v)$の\textbf{極限}は次で定義される.
\begin{align}
	&\lim_{(u,\ v) \to (u_0,\ v_0)} |\bm{f}(u,\ v) - \bm{c}| = 0 \notag\\
	&\iff \lim_{(u,\ v) \to (u_0,\ v_0)} \bm{f}(u,\ v) = \bm{c}\\
	&\iff \lim_{(u,\ v) \to (u_0,\ v_0)} f_1(u,\ v) = c_1,\quad \lim_{(u,\ v) \to (u_0,\ v_0)} f_2(u,\ v) = c_2,\quad \lim_{(u,\ v) \to (u_0,\ v_0)} f_3(u,\ v) = c_3
\end{align}

また,$\dlim_{(u,\ v) \to (u_0,\ v_0)} \bm{f}(u,\ v) = \bm{f}(u_0,\ v_0)$が成立するとき,$\bm{f}(u,\ v)$は$(u,\ v) = (u_0,\ v_0)$で\textbf{連続である}という.



\subsection{ベクトル関数の偏微分法}

領域$D$で定義されたベクトル関数$\bm{f}(u,\ v)$に於いて
\begin{equation}
	\bunsuu{\partial \bm{f}}{\partial u} = \lim_{\varDelta u \to 0} \bunsuu{\bm{f}(u + \varDelta u,\ v) - \bm{f}(u,\ v)}{\varDelta u} =
	{
	\begin{bmatrix}
		\bunsuu{\partial f_1(u,\ v)}{\partial u} &
		\bunsuu{\partial f_2(u,\ v)}{\partial u} &
		\bunsuu{\partial f_3(u,\ v)}{\partial u}
	\end{bmatrix}
	}^\top
\end{equation}
を$\bm{f}(u,\ v)$の\textbf{$u$についての偏導関数}といい,$\bm{f}_u(u,\ v)$とも表す.また,
\begin{equation}
	\bunsuu{\partial \bm{f}}{\partial v} = \lim_{\varDelta v \to 0} \bunsuu{\bm{f}(u,\ v + \varDelta v) - \bm{f}(u,\ v)}{\varDelta v} =
	{
	\begin{bmatrix}
		\bunsuu{\partial f_1(u,\ v)}{\partial v} &
		\bunsuu{\partial f_2(u,\ v)}{\partial v} &
		\bunsuu{\partial f_3(u,\ v)}{\partial v}
	\end{bmatrix}
	}^\top
\end{equation}
を$\bm{f}(u,\ v)$の\textbf{$v$についての偏導関数}といい,$\bm{f}_v(u,\ v)$とも表す.

\begin{enumerate}[leftmargin=18pt, labelsep=10pt, labelsep=10pt, itemindent=9pt]
	\item[\f{例}] $\bm{f}(u,\ v) =
		\begin{bmatrix}
			u\\ v\\ u^2 + v^2
		\end{bmatrix}
		$の偏導関数は
		\begin{equation}
			\bunsuu{\partial \bm{f}}{\partial u} =
			\begin{bmatrix}
				1\\ 0\\ 2u
			\end{bmatrix}
			,\quad \bunsuu{\partial \bm{f}}{\partial v} =
			\begin{bmatrix}
				0\\ 1\\ 2v
			\end{bmatrix}
		\end{equation}
\end{enumerate}

また,次のチェーン・ルールが成り立つ.
\begin{kousiki}{チェーン・ルール}
	\begin{enumerate}[label=\textbf{[\arabic*]}, labelsep=10pt, leftmargin=23pt]
		\item $u,\ v$がともにスカラー変数$t$の関数のとき
			\begin{equation}
				\bunsuu{d\bm{f}}{dt} = \bunsuu{\partial \bm{f}}{\partial u}\bunsuu{du}{dt} + \bunsuu{\partial \bm{f}}{\partial v}\bunsuu{dv}{dt}
			\end{equation}
		\item $u,\ v$がともにスカラー変数$s,\ t$の変数のとき
			\begin{equation}
				\bunsuu{\partial \bm{f}}{\partial s} =
				\bunsuu{\partial \bm{f}}{\partial u}\bunsuu{\partial u}{\partial s} + \bunsuu{\partial \bm{f}}{\partial v}\bunsuu{\partial v}{\partial s}, \qquad
				\bunsuu{\partial \bm{f}}{\partial t} =
				\bunsuu{\partial \bm{f}}{\partial u}\bunsuu{\partial u}{\partial t} + \bunsuu{\partial \bm{f}}{\partial v}\bunsuu{\partial v}{\partial t}
			\end{equation}
	\end{enumerate}
\end{kousiki}



\subsection{接平面}

空間内の点$\mathrm{P}$の位置ベクトルが$\bm{r} =
\begin{bmatrix}
	x(u,\ v)\\ y(u,\ v)\\ z(u,\ v)
\end{bmatrix}
$のようにベクトル関数であるとき,点$(u,\ v)$が領域$D$を動くと,$\mathrm{P}$は1つの曲面$S$を描く.この$S$を$\bm{r} = \bm{r}(u,\ v)$の表す曲面という.

$\bm{r}(u,\ v)$が$D$で連続な偏導関数をもつとする.$v$を一定にして$u$を変化させると$\bm{r}$は$S$上で1つの曲線(\textbf{$u$曲線})を描く.よって,$\bunsuu{\partial \bm{r}(u,\ v)}{\partial u}$は$u$曲線上の$\mathrm{P}$における接線ベクトルを与える.

同様に,$u$を一定にして$v$を変化させると$\bm{r}$は$S$上で1つの曲線(\textbf{$v$曲線})を描く.よって,$\bunsuu{\partial \bm{r}(u,\ v)}{\partial v}$は$v$曲線上の$\mathrm{P}$における接線ベクトルを与える.

このとき,外積の性質より$\bunsuu{\partial \bm{r}}{\partial u}$と$\bunsuu{\partial \bm{r}}{\partial v}$が平行ならば$\bunsuu{\partial \bm{r}}{\partial u} \times \bunsuu{\partial \bm{r}}{\partial v} = \bm{0}$なので,$\bunsuu{\partial \bm{r}}{\partial u} \times \bunsuu{\partial \bm{r}}{\partial v} \ne \bm{0}$ならば,この2つの接線ベクトルを含む平面$H$が存在する.この$H$を$S$の点$\mathrm{P}$における\textbf{接平面}という.その法線ベクトルは
\begin{equation}
	\bunsuu{\partial \bm{r}}{\partial u} \times \bunsuu{\partial \bm{r}}{\partial v} =
	\begin{vmatrix}
		\bm{i} & x_u & x_v\\
		\bm{j} & y_u & y_v\\
		\bm{k} & z_u & z_v
	\end{vmatrix}
\end{equation}
である.

\begin{enumerate}[leftmargin=18pt, labelsep=10pt, labelsep=10pt, itemindent=9pt]
	\item[\f{例}] ベクトル関数$\bm{r} = \bm{r}(u,\ v) =
		\begin{bmatrix}
			u\cos v\\ u\sin v\\ u^2
		\end{bmatrix}
		$を例にとって,詳しく見ていく.
		\begin{enumerate}[label=\textbf{[\arabic*]}, labelsep=10pt, leftmargin=23pt, itemsep=12pt]
			\item \underline{与式の表す曲面の,$x,\ y,\ z$に関する方程式}\\
				$x = u\cos v,\ y = u\sin v$より$x^2 + y^2$を計算すると$u^2$なので,$z = u^2$となる.よって,$z = x^2 + y^2$.
			\item \underline{$v = \bunsuu{\pi}{2}$のときの$u$曲線}\\
				$x = u\cos\bunsuu{\pi}{2} = 0,\ y = u\sin\bunsuu{\pi}{2} = u$なので,$z = 0^2 + y^2 = y^2$.これは平面$x = 0$上の放物線である.
			\item \underline{$u = 1$のときの$v$曲線}\\
				$x = \cos v,\ y = \sin v$より,$z = 1$.これは平面$z = 1$上の半径$1$の円である.
			\item \underline{$(u,\ v) = \left(1,\ \bunsuu{\pi}{2}\right)$における接線ベクトル}\\
				$\bunsuu{\partial \bm{r}}{\partial u} =
				\begin{bmatrix}
					\cos v\\ \sin v\\ 2u
				\end{bmatrix}
				=
				\begin{bmatrix}
					0\\ 1\\ 2
				\end{bmatrix}
				$,\qquad
				$\bunsuu{\partial \bm{r}}{\partial v} =
				\begin{bmatrix}
					-u\sin v\\ u\cos v\\ 0
				\end{bmatrix}
				=
				\begin{bmatrix}
					-1\\ 0\\ 0
				\end{bmatrix}
				$
			\item \underline{$(u,\ v) = \left(1,\ \bunsuu{\pi}{2}\right)$における接平面の法線ベクトル}\\
				$\bunsuu{\partial \bm{r}}{\partial u} \times \bunsuu{\partial \bm{r}}{\partial v} =
				\begin{vmatrix}
					\bm{i} & x_u & x_v\\
					\bm{j} & y_u & y_v\\
					\bm{k} & z_u & z_v
				\end{vmatrix}
				=
				\begin{vmatrix}
					\bm{i} & 0 & -1\\
					\bm{j} & 1 & 0\\
					\bm{k} & 2 & 0
				\end{vmatrix}
				=
				\begin{bmatrix}
					0\\ -2\\ 1
				\end{bmatrix}
				$
		\end{enumerate}
\end{enumerate}



\section{空間曲線}
\subsection{曲率}

曲線$\bm{r} = \bm{r}(s)$の単位接線ベクトル$\bm{t} = \bunsuu{d\bm{r}}{ds}$について次の$\kappa \in \mathbb{R}$を曲線の\textbf{曲率}という.
\begin{equation}
	\kappa = \left|\bunsuu{d\bm{t}}{ds}\right| = \left|\bunsuu{d^2\bm{r}}{ds^2}\right|
\end{equation}

$\kappa$は曲線の局所的な曲がり具合である.この値が大きいほどカーブが急になる.$\varDelta \bm{t}$が十分に小さいとき,$|\varDelta \bm{t}| \approx \varDelta \theta$と見做すことができ,$\kappa = \left|\bunsuu{d\theta}{ds}\right|$と表すこともできるので,$\kappa = $は回転角の変化率であるとも言える.

\begin{enumerate}[label=\textbf{[\arabic*]}, labelsep=10pt, leftmargin=23pt]
	\item 半径$a$の円の曲率は$\kappa = \bunsuu{1}{a}$で,半径が大きくなるほど曲線の曲がり具合が小さくなる.
	\item 直線の曲率は$\kappa = 0$である.
\end{enumerate}

曲率の逆数を\textbf{曲率半径}といい,$\sigma$で表す.
\begin{equation}
	\sigma = \bunsuu{1}{\kappa}
\end{equation}

\begin{enumerate}[label=\textbf{[\arabic*]}, labelsep=10pt, leftmargin=23pt]
	\item 半径$a$の曲率半径は$\sigma = a$である.$\kappa = 0$のときは$\sigma = \infty$と定義する.
	\item 直線の曲率半径は$\sigma = \infty$である.
\end{enumerate}



\subsection{捩率}



\subsection{速度・加速度}


\section{スカラー場の勾配}
\subsection{勾配}

$D$で定義されたスカラー場$f(x,\ y,\ z)$について,その偏導関数$\bunsuu{\partial f}{\partial x},\ \bunsuu{\partial f}{\partial y},\ \bunsuu{\partial f}{\partial z}$を係数とするベクトル$\bunsuu{\partial f}{\partial x}\bm{i} + \bunsuu{\partial f}{\partial y}\bm{j} + \bunsuu{\partial f}{\partial z}\bm{k}$を考えると,領域$D$にあるベクトル場が定義される.これをスカラー場$f$の\textbf{勾配}といい,$\grad f$で表す.形式的なベクトル$\bm{\nabla} =
\begin{bmatrix}
	\scriptstyle\myfrac{\partial}{\partial x}\\[5pt]
	\scriptstyle\myfrac{\partial}{\partial y}\\[5pt]
	\scriptstyle\myfrac{\partial}{\partial z}
\end{bmatrix}
$を使って,$\bm{\nabla}f$で表すこともある.$\bm{\nabla}$を\textbf{Hamiltonの演算子}と呼ばれ,\textbf{ナブラ}と読む.

\begin{kousiki}{スカラー場の勾配}
	スカラー場$f(x,\ y,\ z)$の勾配$\grad f$とは次のベクトル場である.
	\begin{equation}
		\grad f = \bm{\nabla}f =
		{
		\begin{bmatrix}
			\bunsuu{\partial f}{\partial x} &
			\bunsuu{\partial f}{\partial y} &
			\bunsuu{\partial f}{\partial z}
		\end{bmatrix}
		}^\top
	\end{equation}
\end{kousiki}

$c$を定数とすると,方程式$f(x,\ y,\ z) = c$は一般に1つの曲面を表す.これを$f$の\textbf{等位面}という.$c$をパラメータとすると方程式は等位面の群を表す.点$\mathrm{P}$を通る等位面上に$\mathrm{P}$を通る任意の曲線$\bm{r}(t) =
\begin{bmatrix}
	x(t)\\ y(t)\\ z(t)
\end{bmatrix}
$を描くと \vskip-\baselineskip
\begin{equation*}
	f\bigl(x(t),\ y(t),\ z(t)\bigr) = c
\end{equation*}
を満たす.両辺を$t$で微分すると,チェーン・ルールより
\begin{align*}
	&\bunsuu{\partial f}{\partial x}\bunsuu{dx}{dt} + \bunsuu{\partial f}{\partial y}\bunsuu{dy}{dt} + \bunsuu{\partial f}{\partial z}\bunsuu{dz}{dt} = 0\\
	&\iff \bm{\nabla}f \cdot \bunsuu{d\bm{r}}{dt} = 0\\
	&\iff \bm{\nabla}f \perp \bunsuu{d\bm{r}}{dt}
\end{align*}
となる.

これは点$\mathrm{P}$に対応するベクトル場$\bm{\nabla}f$は,$\mathrm{P}$を通る等位面上の曲線の接線ベクトル$\bunsuu{d\bm{r}}{dt}$に垂直になる.また,曲線は任意であるから次のことが言える.
\begin{kousiki}{勾配$\bm{\nabla}f$の意味}
	点$\mathrm{P}$における勾配$\bm{\nabla}f$は,$\mathrm{P}$における等位面の法線ベクトルになる.
\end{kousiki}

\vskip\baselineskip

任意の単位ベクトル$\bm{e} =
\begin{bmatrix}
	e_x\\ e_y\\ e_z
\end{bmatrix}
$について,点$\mathrm{P}$から$\bm{e}$の方向に$\varDelta s$だけ動いた時の$f$の微分係数(接線の傾き)を調べる.これを\textbf{方向微分係数}$\bunsuu{df}{ds}$という.
\begin{kousiki}{方向微分係数}
	\begin{align}
		\bunsuu{df}{ds} &=
		\lim_{\varDelta s \to 0}
			\bunsuu{f(x + e_x \varDelta s,\ y + e_y \varDelta s,\ z + e_z \varDelta s) - f(x,\ y,\ z)}{\varDelta s}\\
			&= \bm{\nabla}f \cdot \bm{e}
	\end{align}
\end{kousiki}

\f{証明} $\mathrm{P}$から$\bm{e}$の方向に$\varDelta s$だけ離れた点の座標は$(x + e_x \varDelta s,\ y + e_y \varDelta s,\ z + e_z \varDelta s)$なので,$f$の変化率はチェーン・ルールより
\begin{align*}
	\bunsuu{d}{ds}&f(x + e_x \varDelta s,\ y + e_y \varDelta s,\ z + e_z \varDelta s)\\
	&= \bunsuu{\partial f}{\partial x}\bunsuu{d}{ds}(x + e_x \varDelta s) + \bunsuu{\partial f}{\partial y}\bunsuu{d}{ds}(y + e_y \varDelta s) + \bunsuu{\partial f}{\partial z}\bunsuu{d}{ds}(z + e_z \varDelta s)\\
	&= \bunsuu{\partial f}{\partial x}e_x + \bunsuu{\partial f}{\partial y}e_y + \bunsuu{\partial f}{\partial z}e_z = \bm{\nabla}f \cdot \bm{e}
\end{align*}
となる.

このとき,$\bm{\nabla}f$と$\bm{e}$の成す角を$\theta$とすると
\begin{equation*}
	\bm{\nabla}f \cdot \bm{e} = |\bm{\nabla} f||\bm{e}|\cos\theta = |\bm{\nabla}f|\cos\theta
\end{equation*}
となり,$\theta = 0$のとき正の最大値をとる.すなわち,勾配=法線ベクトルの方向への方向微分係数が最大となる.方程式
\begin{equation*}
	f(x,\ y,\ z) = c_i \quad (c = 1,\ 2,\ \cdots)
\end{equation*}
で,$c_1$での方向微分係数より,$c_2$での方向微分係数の値の方が大きいとき,
% \chapter{ラプラス変換}
\setcounter{page}{1}
\section{定義と基本的性質}
\subsection{ラプラス変換の定義}

関数$f(t)$は,$t \in \mathbb{R}_{> 0}$で定義され,$s$を$t$と無関係な実数とする.このとき次のような関数$F(s)$を考える.
\begin{equation}
	F(s) = \int_{0}^{\infty} e^{-st}f(t)\,dt = \lim_{\substack{T \to \infty \\ \varepsilon \to +0}} \int_{\varepsilon}^{T} e^{-st}f(t)\,dt
\end{equation}

これは,関数$f(t)$に関数$F(s)$を対応させる規則を与えている.その対応を$\mathcal{L}$で表す.$\mathcal{L}$を適用させた$F(s)$を$f(t)$の\textbf{ラプラス変換}といい,
\begin{equation}
	F(s) = \mathcal{L}[f(t)] \text{\quad または \quad} f(t) \stackrel{\mathcal{L}}{\longrightarrow} F(s)
\end{equation}
と表す.$f(t)$を\textbf{原関数},$F(s)$を\textbf{像関数}という.一般に$s \in \mathbb{C}$であるが,ここでは$s \in \mathbb{R}$とする.

\begin{enumerate}[leftmargin=18pt, labelsep=10pt, labelsep=10pt, itemindent=9pt]
	\item[\f{例1}] $f(t) = 1$のラプラス変換$\mathcal{L}[1]$
		\begin{align*}
			F(s) = \mathcal{1}[1] = \lim_{T \to \infty} \int_{0}^{T} e^{-st}\,dt = \lim_{T \to \infty} \teisekibun{{-\bunsuu{1}{s}e^{-st}}}{0}{T} = \lim_{T \to \infty} \bunsuu{1}{s}(1 - e^{-sT}) = \bunsuu{1}{s} \quad (s > 0)
		\end{align*}
		$s \le 0$のとき,$\dlim_{T \to \infty} \bunsuu{1}{s}(1 - e^{-sT}) = \infty$なので存在しない.
	\item[\f{例2}] $f(t) = t$のラプラス変換$\mathcal{L}[t]$
		\begin{align*}
			F(s) &= \mathcal{L}[t] = \lim_{T \to \infty} \int_{0}^{T} e^{-st}t\,dt = \lim_{T \to \infty} \left\{ \teisekibun{{-\bunsuu{e^{-st}}{s}t}}{0}{T} + \int_{0}^{T} \bunsuu{e^{-st}}{s}\,dt \right\}
			= \lim_{T \to \infty} \left\{ -\bunsuu{e^{-sT}}{s}T - \teisekibun{\bunsuu{e^{-st}}{s^2}}{0}{T} \right\}\\
			&= \lim_{T \to \infty} \left\{
				-\bunsuu{e^{-sT}T}{s} - \bunsuu{e^{-sT}}{s^2} + \bunsuu{1}{s^2}
			\right\}
			= \bunsuu{1}{s^2}
		\end{align*}
		ここで,$s > 0$のとき
		\begin{gather*}
			\lim_{T \to \infty} \left(-\bunsuu{e^{-sT}T}{s}\right) = -\bunsuu{1}{s}\lim_{T \to \infty} \bunsuu{T}{e^{sT}} = -\bunsuu{1}{s}\lim_{T \to \infty} \bunsuu{1}{se^{sT}} = 0\\
			\lim_{T \to \infty} \left(-\bunsuu{e^{-sT}}{s^2}\right) = -\bunsuu{1}{s^2}\lim_{T \to \infty} \bunsuu{1}{e^{sT}} = 0
		\end{gather*}
	\item[\f{例3}] $f(t) = e^{\alpha t}$($\alpha$は定数)のラプラス変換$\mathcal{L}[e^{\alpha t}]$
		\begin{align*}
			F(s) = \mathcal{L}[e^{\alpha t}] &= \lim_{T \to \infty} \int_{0}^{T} e^{-st}e^{\alpha t}\,dt
			= \lim_{T \to \infty} \int_{0}^{T} e^{-(s - \alpha)t}\,dt
			= \lim_{T \to \infty} \teisekibun{{-\bunsuu{1}{s - \alpha}e^{-(s - \alpha)t}}}{0}{T}\\
			&= \lim_{T \to \infty} \left(-\bunsuu{1}{s - \alpha}\{e^{-(s - \alpha)T} - 1\}\right) = \bunsuu{1}{s - \alpha} \quad (s > \alpha)
		\end{align*}
		$s - \alpha \le 0$,即ち$s \le \alpha$のとき,極限値は存在しない.
	\item[\f{例4}] $f(t) = \sin \omega t$のラプラス変換$\mathcal{L}[\sin \omega t]$
		\begin{align*}
			F(s) &= \mathcal{L}[\sin \omega t] = \lim_{T \to \infty} \int_{0}^{T} e^{-st}\sin \omega t\,dt
			\intertext{$I_1 = \dint_{0}^{T} e^{-st}\sin \omega t\,dt$とする.}
			I_1 &= \int_{0}^{T} e^{-st}\sin \omega t\,dt = \teisekibun{ \bunsuu{1}{-s}e^{-st}\sin \omega t }{0}{T} - \int_{0}^{T} \bunsuu{1}{-s}e^{-st}\omega\cos \omega t\,dt\\
			&= \bunsuu{1}{-s}e^{-sT}\sin \omega T + \bunsuu{\omega}{s}\int_{0}^{T} e^{-st}\cos \omega t\,dt\\
			&= \bunsuu{1}{-s}e^{-sT}\sin \omega T + \bunsuu{\omega}{s}\left\{
				\teisekibun{
					\bunsuu{1}{-s}e^{-st}\cos \omega t
				}{0}{T} - \int_{0}^{T} \bunsuu{1}{-s}e^{-st}(-\omega\sin \omega t)\,dt
			\right\}\\
			&= \bunsuu{1}{-s}e^{-sT}\sin \omega T + \bunsuu{\omega}{s}\left\{
				\left(
					\bunsuu{1}{-s}e^{-sT}\cos \omega T + \bunsuu{1}{s}
				\right)
				- \bunsuu{\omega}{s}I_1
			\right\}
			\intertext{よって,$I_1$について解くと}
			I_1 &= \bunsuu{s^2}{s^2 + \omega^2}\left\{
				-\bunsuu{1}{s}e^{-sT}\sin \omega T - \bunsuu{\omega}{s^2}e^{-sT}\cos \omega T + \bunsuu{\omega}{s^2}
			\right\}
		\end{align*}
		$s > 0$のとき$\dlim_{T \to \infty} e^{-sT}\sin \omega T = \dlim_{T \to \infty} e^{-sT}\cos \omega T = 0$なので,
		\begin{equation*}
			\mathcal{L}[\sin \omega t] = \lim_{T \to \infty} I_1 = \bunsuu{\omega}{s^2 + \omega^2}\quad(s > 0)
		\end{equation*}
		同様にすると,$\mathcal{L}[\cos \omega t] = \bunsuu{s}{s^2 + \omega^2}\quad(s > 0)$
\end{enumerate}



\subsection{単位ステップ関数}

次のように定義される関数$H(t)$を\textbf{Heavisideの階段関数}という.
\begin{equation}
	H(t) =
	\left\{
		\begin{array}{ll}
			0 & (t < 0)\\ 1 & (t > 0)
		\end{array}
	\right.
\end{equation}

《注》$t = 0$に於ける値は任意に決めることができる.

また,次のように定義される関数$U(t)$を\textbf{単位ステップ関数}という.
\begin{equation}
	U(t) =
	\left\{
		\begin{array}{ll}
			0 & (t \le 0)\\ 1 & (t > 0)
		\end{array}
	\right.
\end{equation}

関数$U(t - a)$は$U(t)$を$t$軸方向に$a$平行移動して得られる.

$a \ge 0$のときの$\mathcal{L}[U(t - a)]$を求める.$t - a \le 0$,即ち$t \le a$のときは$U(t - a) = 0$,$t > a$のときは$U(t - a) = 1$なので
\begin{align*}
	F(s) = \mathcal{L}[U(t - a)] &= \int_{0}^{\infty} e^{-st}U(t - a)\,dt = \int_{0}^{a} e^{-st}U(t - a)\,dt + \int_{a}^{\infty} e^{-st}U(t - a)\,dt = \int_{a}^{\infty} e^{-st}\,dt\\
	&= \teisekibun{{
		-\bunsuu{1}{s}e^{-st}
	}}{a}{\infty}
	= -\bunsuu{1}{s}\lim_{t \to \infty} \left(
		e^{-st} - e^{-as}
	\right)
\end{align*}
ここで,$s > 0$のとき$\dlim_{t \to \infty} e^{-st} = 0$なので
\begin{equation*}
	F(s) = -\bunsuu{1}{s}\cdot(-e^{-as}) = \bunsuu{e^{-as}}{s} \quad (s > 0)
\end{equation*}
である.



\section{ラプラス変換の基本的性質}
\subsection{ラプラス変換の線形性}

関数$f(t),\ g(t)$,定数$a,\ b$に対し
\begin{equation}
	\mathcal{L}[af(t) + bg(t)] = a\mathcal{L}[f(t)] + b\mathcal{L}[g(t)]
\end{equation}
が成り立つ.これによって,項別にラプラス変換をすることで求めることができる.



\subsection{ラプラス変換の相似性}

関数$f(t)$,定数$\lambda > 0$に対し,$F(s) = \mathcal{L}[f(t)]$とする.このとき
\begin{equation}
	\mathcal{L}[f(\lambda t)] = \bunsuu{1}{\lambda}F\left(\bunsuu{s}{\lambda}\right)
\end{equation}
が成り立つ.即ち,$t$軸方向に$\bunsuu{1}{\lambda}$倍してからラプラス変換すると,$F(s)$を$s$軸方向に$\lambda$倍したものを$\bunsuu{1}{\lambda}$倍したものになる.

\begin{equation*}
	\mspace{100mu}
	\begin{array}{llll}
		& x = f(t)	& \xlongrightarrow[\text{ラプラス変換}]{\mathcal{L}} & X = F(s)\\
		\text{\small \fbox{$t$軸方向に$\bunsuu{1}{\lambda}$倍}}
			& \text{\LARGE \raisebox{-4pt}{☟}} &
			& \text{\LARGE \raisebox{-4pt}{☟}}\quad
			\text{\small \fbox{$s$軸方向に$\lambda$倍して$X$軸方向に$\bunsuu{1}{\lambda}$倍}}\\[10pt]
		& x = f(\lambda t)	& \xlongrightarrow[\text{ラプラス変換}]{\mathcal{L}} & X = \bunsuu{1}{\lambda}F\left(\bunsuu{s}{\lambda}\right)
	\end{array}
\end{equation*}



\subsection{第1移動定理(像関数の移動法則)}

関数$f(t)$,定数$\alpha$に対し,$F(s) = \mathcal{L}[f(t)]$とする.このとき
\begin{equation}
	\mathcal{L}[e^{\alpha t}f(t)] = F(s - \alpha)
\end{equation}
が成り立つ.即ち,原関数に$e^{\alpha t}$をかけたものをラプラス変換すると,像関数は$F(s)$を$s$軸方向に$\alpha$平行移動したものになる.

\begin{equation*}
	\mspace{150mu}
	\begin{array}{llll}
		& x = f(t)	& \xlongrightarrow[\text{ラプラス変換}]{\mathcal{L}} & X = F(s)\\
		\text{\small \fbox{$e^{\alpha t}$をかける}}
			& \text{\LARGE \raisebox{-4pt}{☟}} &
			& \text{\LARGE \raisebox{-4pt}{☟}}\quad
			\text{\small \fbox{$s$軸方向に$\alpha$平行移動}}\\[10pt]
		& x = e^{\alpha t}f(t)	& \xlongrightarrow[\text{ラプラス変換}]{\mathcal{L}} & X = F(s - \alpha)
	\end{array}
\end{equation*}



\subsection{第2移動定理(原関数の移動法則)}

ラプラス変換は$t > 0$の範囲で行うので,単位ステップ関数$U(t)$をかけても変わらない.このとき,関数$f(t)$,定数$\mu > 0$に対し,$F(s) = \mathcal{L}[f(t)]$とすると
\begin{equation}
	\mathcal{L}[f(t - \mu)U(t - \mu)] = e^{-\mu s}F(s)
\end{equation}
が成り立つ.即ち,原関数を$t$軸方向に$\mu$移動したものをラプラス変換すると,像関数は$F(s)$に$e^{-\mu s}$をかけたものになる.

\begin{equation*}
	\mspace{120mu}
	\begin{array}{llll}
		& x = f(t)	& \xlongrightarrow[\text{ラプラス変換}]{\mathcal{L}} & X = F(s)\\
		\text{\small \fbox{$t$軸方向に$\mu$移動}}
			& \text{\LARGE \raisebox{-4pt}{☟}} &
			& \text{\LARGE \raisebox{-4pt}{☟}}\quad
			\text{\small \fbox{$e^{-\mu s}$をかける}}\\[10pt]
		& x = f(t - \mu)U(t - \mu)	& \xlongrightarrow[\text{ラプラス変換}]{\mathcal{L}} & X = e^{-\mu s}F(s)
	\end{array}
\end{equation*}



\subsection{微分法則}

原関数$f(t)$のラプラス変換を$F(s)$とすると次が成り立つ.
\begin{kousiki}{1階の微分法則}
	\begin{enumerate}[label=\textbf{[\arabic*]}, labelsep=10pt, leftmargin=23pt]
		\item $\mathcal{L}[f'(t)] = sF(s) - f(+0)$ \hfill (原関数の微分法則)
		\item $\mathcal{L}[tf(t)] = -F'(s)$ \hfill (像関数の微分法則)
	\end{enumerate}
\end{kousiki}

$f(+0)$は,$t \to +0$のときの$f(t)$の極限値を表す.原関数の微分法則は微分方程式に使われることがある.

原関数や像関数の微分法則を繰り返し使うと以下を得る.
\begin{kousiki}{高次微分法則}
	\begin{enumerate}[label=\textbf{[\arabic*]}, labelsep=10pt, leftmargin=23pt]
		\item $\mathcal{L}[f^{(n)}(t)] = s^n F(s) - s^{n - 1}f(+0) - s^{n - 2}f'(+0) - s^{n - 3}f''(+0) - \cdots - f^{(n - 1)}(+0)$ \hfill (原関数の高次微分法則)
		\item $\mathcal{L}[t^n f(t)] = (-1)^n F^{(n)}(s)$ \hfill (像関数の高次微分法則)
	\end{enumerate}
\end{kousiki}



\subsection{積分法則}

原関数$f(t)$のラプラス変換を$F(s)$とすると積分についての次が成り立つ.

\begin{kousiki}{積分法則}
	\begin{enumerate}[label=\textbf{[\arabic*]}, labelsep=10pt, leftmargin=23pt]
		\item $\mathcal{L}\left[\dint_{0}^{t} f(\tau)\,d\tau\right] = \bunsuu{F(s)}{s}$ \hfill (原関数の積分法則)
		\item $\mathcal{L}\left[\bunsuu{f(t)}{t}\right] = \dint_{s}^{\infty} F(\sigma)\,d\sigma$ \hfill (像関数の積分法則)
	\end{enumerate}
\end{kousiki}

\begin{enumerate}[leftmargin=18pt, labelsep=10pt, itemindent=9pt]
	\item[\f{例}] \underline{$\dint_{0}^{t} e^{-2\tau}\,d\tau$のラプラス変換}
		\begin{equation*}
			\mathcal{L}[e^{-2t}] = \bunsuu{1}{s + 2}
		\end{equation*}
		であるので,
		\begin{equation*}
			\mathcal{L}\left[\int_{0}^{t} e^{-2\tau}\,d\tau\right] = \bunsuu{1}{s}\mathcal{L}[e^{-2t}] = \bunsuu{1}{s(s + 2)}
		\end{equation*}
\end{enumerate}



\subsection{たたみこみ}

区間$[0,\ \infty)$で定義された関数$f(t),\ g(t)$に対し
\begin{equation}
	(f * g)(t) = \int_{0}^{t} f(\tau)g(t - \tau)\,d\tau
\end{equation}
を$f(t)$と$g(t)$の\textbf{たたみこみ}または\textbf{合成積}という.たたみこみのラプラス変換について,次の関係が成り立つ.
\begin{kousiki}{たたみこみのラプラス変換}
	\begin{equation}
		\mathcal{L}[(f * g)(t)] = \mathcal{L}[f(t)]\mathcal{L}[g(t)]
	\end{equation}
\end{kousiki}

\begin{enumerate}[leftmargin=18pt, labelsep=10pt, itemindent=9pt]
	\item[\f{例}] \underline{$f(t) = \sin t$,$g(t) = \cos t$について,たたみこみ$(f * g)(t)$を求める.}
		\begin{align*}
			(f * g)(t) &= \int_{0}^{t} \sin \tau \cos(t - \mu)\,d\tau\\
			&= \bunsuu{1}{2}\int_{0}^{t} \{\sin t + \sin(2\tau - t)\}\,d\tau\\
			&= \bunsuu{1}{2}\teisekibun{\tau\sin t - \bunsuu{1}{2}\cos(2\tau - t)}{0}{t}\\
			&= \bunsuu{1}{2}t\sin t
		\end{align*}
		\underline{たたみこみのラプラス変換}
		\begin{align*}
			\mathcal{L}\left[\bunsuu{1}{2}t\sin t\right] &= \mathcal{L}[\sin t * \cos t] = \mathcal{L}[\sin t]\mathcal{L}[\cos t]\\
			&= \bunsuu{1}{s^2 + 1}\bunsuu{s}{s^2 + 1} = \bunsuu{s}{(s^2 + 1)^2}
		\end{align*}
\end{enumerate}



\section{ラプラス変換の表}
\subsection{主な性質}

\begin{table}[H]
	\centering
	\begin{tabular}{c|c}
		\hline
		\textsf{原関数} & \textsf{像関数}\\
		\hline
		$\alpha f(t) + \beta g(t)$ & $\alpha F(s) + \beta G(s)$\\[3mm]
		$f(at)$ & $\bunsuu{1}{a}F\left(\bunsuu{s}{a}\right) \quad (a > 0)$\\[3mm]
		$e^{\alpha t}f(t)$ & $F(s - \alpha)$\\[3mm]
		$f(t - \mu)U(t - \mu)$ & $e^{-\mu s} F(s) \qquad (\mu > 0)$\\[3mm]
		$f'(t)$ & $s F(s) - f(+0)$\\[3mm]
		$f^{(n)}(t)$ & $s^n F(s) - s^{n - 1}f(+0) - s^{n - 2}f'(+0) - \cdots - f^{(n - 1)}(+0)$\\[3mm]
		$tf(t)$ & $-F'(s)$\\[3mm]
		$t^n f(t)$ & $(-1)^n F^{(n)}(s)$\\[3mm]
		$\dint_{0}^{t} f(\tau)\,d\tau$ & $\bunsuu{F(s)}{s}$\\[3mm]
		$\bunsuu{f(t)}{t}$ & $\dint_{s}^{\infty} F(\sigma)\,d\sigma$\\[3mm]
		\hline
	\end{tabular}
\end{table}



\subsection{いろいろな関数のラプラス変換}

\begin{table}[H]
	\centering
	\begin{tabular}{c|c}
		\hline
		\hspace*{6\zw}\textsf{原関数}\hspace*{6\zw} & \hspace*{6\zw}\textsf{像関数}\hspace*{6\zw}\\
		\hline
		\raisebox{-1mm}{$1$} & \raisebox{-1mm}{$\bunsuu{1}{s}$}\\[4mm]
		$t$ & $\bunsuu{1}{s^2}$\\[3mm]
		$t^n$ & $\bunsuu{n!}{s^{n + 1}}$\\[3mm]
		$e^{\alpha t}$ & $\bunsuu{1}{s - \alpha}$\\[3mm]
		$te^{\alpha t}$ & $\bunsuu{1}{(s - \alpha)^2}$\\[3mm]
		$t^n e^{\alpha t}$ & $\bunsuu{n!}{(s - \alpha)^{n + 1}}$\\[3mm]
		$\sin \omega t$ & $\bunsuu{\omega}{s^2 + \omega^2}$\\[3mm]
		$\cos \omega t$ & $\bunsuu{s}{s^2 + \omega^2}$\\[3mm]
		$t\sin \omega t$ & $\bunsuu{2\omega s}{(s^2 + \omega^2)^2}$\\[3mm]
		$t\cos \omega t$ & $\bunsuu{s^2 - \omega^2}{(s^2 + \omega^2)^2}$\\[3mm]
		$\sinh \omega t$ & $\bunsuu{\omega}{s^2 - \omega^2}$\\[3mm]
		$\cosh \omega t$ & $\bunsuu{s}{s^2 - \omega^2}$\\[3mm]
		$U(t - \alpha)$ & $\bunsuu{e^{-as}}{s} \quad (a \ge 0)$\\[3mm]
		\hline
	\end{tabular}
\end{table}
% \chapter{フーリエ解析}
\setcounter{page}{1}
% \chapter{複素解析}
\setcounter{page}{1}
% \chapter{力学}
\setcounter{page}{1}
\section{ベクトル,速度,加速度}
\subsection{点の位置の表し方}

無限に広い平面にある点$\mathrm{P}$の位置を表すのには,基準となる物体(基準体)が必要.基準体を1つの点$\mathrm{O}$とすれば,$\mathrm{P}$の位置を表すものに,距離はあるが方向はない.よって,基準体は大きさを持ったものでなくてはならない.時が経っても形の変わらないものを\textbf{剛体}という.この剛体上に2つの定点$\mathrm{A,\ B}$をとれば,$\mathrm{AP,\ BP}$の長さによって$\mathrm{P}$の位置は決まる.

点$\mathrm{P}$の位置を表すのには$\mathrm{OP}$の長さ$r$を使って,$\vecrm{OP} = \bm{r}$のようにベクトルで書き表す.これを\textbf{位置ベクトル}という.$x$軸と$\vecrm{OP}$の成す角を$\varphi$とすると$(x,\ y)$と$(r,\ \varphi)$の関係は
\begin{equation}
	x = r\cos\varphi, \qquad y = r\sin\varphi
\end{equation}

座標系のとり方はいろいろある.

\begin{enumerate}[leftmargin=18pt, labelsep=10pt, labelsep=10pt, itemindent=9pt]
	\item[\f{例}] 原点を共通に持つ2つの座標系の軸が$\bunsuu{\pi}{4}$の角をつくっている.
		\begin{inparaenum}[(1)]
			\item 任意の点$\mathrm{P}$の座標$(x,\ y),\ (x',\ y')$の間にはどんな関係があるか;
			\item $x'^2 + y'^2 = x^2 + y^2$を示せ;
			\item $ax^2 + 2hxy + ay^2 = 1$で示される曲線の方程式を$x',\ y'$を使って表せ.
		\end{inparaenum}
		\begin{enumerate}[label=(\arabic*), labelsep=10pt, leftmargin=23pt]
			\item 図で,$\mathrm{P}$から$x$軸と$x'$軸にそれぞれ垂線$\mathrm{PA,\ PB}$を下す.$\mathrm{A}$から$x'$軸に垂線$\mathrm{AA'}$を下すと
			\begin{equation}
				x' = \mathrm{OB} + \mathrm{OA'} + \mathrm{A'B} = \mathrm{OA}\cos\bunsuu{\pi}{4} + \mathrm{AP}\sin\bunsuu{\pi}{4} = \bunsuu{1}{\sqrt{2}}(x + y) \label{equ:NR1-1}
			\end{equation}
			また
			\begin{equation}
				y' = \mathrm{AP}\cos\bunsuu{\pi}{4} - \mathrm{OA}\sin\bunsuu{\pi}{4} = \bunsuu{1}{\sqrt{2}}(-x + y) \label{equ:NR1-2}
			\end{equation}
			式(\ref{equ:NR1-1}),式(\ref{equ:NR1-2})が$x',\ y'$を$x,\ y$で表す式である.$x,\ y$について解けば
			\begin{gather}
				x = \bunsuu{1}{\sqrt{2}}(x' - y')\\
				y = \bunsuu{1}{\sqrt{2}}(x' + y')
			\end{gather}
			\item (1)より
			\begin{equation}
				x'^2 + y'^2 = x^2 + y^2
			\end{equation}
			\item 与式に代入すると
			\begin{equation*}
				(a + h)x'^2 + (a - h)y'^2 = 1
			\end{equation*}
		\end{enumerate}
\end{enumerate}

空間の直交座標系は,右手の親指,人差し指,中指の順に$x,\ y,\ z$軸をとる(\textbf{右手座標系}).\textbf{極座標}では,$x$軸と$\bm{r}$の正射影の成す角を$\varphi$,$z$軸と$\bm{r}$の成す角を$\theta$とする.$\varphi$は経度,$\theta$は緯度にあたる.$(x,\ y,\ z)$と$(r,\ \varphi,\ \theta)$の関係は
\begin{equation}
	x = r\sin\theta\cos\varphi,\qquad y = r\sin\theta\sin\varphi,\qquad z = r\cos\theta
\end{equation}
となる.



\subsection{速度ベクトル}

\textbf{速度}または\textbf{速度ベクトル}$\bm{v}$は
\begin{equation}
	\bm{v} = \lim_{\varDelta t \to 0} \bunsuu{\varDelta \bm{r}}{\varDelta t} = \bunsuu{d\bm{r}}{dt}
\end{equation}
で求める.

$\mathrm{P}$点の位置を辿ると曲線を描く(\textbf{軌道}または\textbf{径路}).時間の差$\varDelta t$が小さいほど,$|\varDelta\bm{r}|$と軌道に沿っての長さ$\varDelta s$の比が$1$に近づくので$v = |\bm{v}|$は
\begin{equation}
	v = \lim_{\varDelta t \to 0} \bunsuu{|\varDelta \bm{r}|}{\varDelta t} = \lim_{\varDelta t \to 0} \bunsuu{\varDelta s}{\varDelta t} = \bunsuu{ds}{dt}
\end{equation}
となる.これを\textbf{速さ}という.



\subsection{加速度ベクトル}

\textbf{加速度}または\textbf{加速度ベクトル}$\bm{a}$は
\begin{equation}
	\bm{a} = \lim_{\varDelta t \to 0} \bunsuu{\varDelta \bm{v}}{\varDelta t} = \bunsuu{d\bm{v}}{dt}
\end{equation}
で求める.

\begin{equation}
	x = a\cos(\omega t + \alpha) \qquad \text{($a,\ \alpha$は定数)}
\end{equation}
で表される運動は
\begin{gather}
	v = \bunsuu{dx}{dt} = -\omega a \sin(\omega t + \alpha)\\
	a = \bunsuu{d^2x}{dt^2} = -\omega^2 a \cos(\omega t + \alpha) = -\omega^2 x
\end{gather}
となる.加速度はいつも原点の方を向いており,その大きさは原点からの距離に比例している.この運動を\textbf{単振動}という.$x$は$\pm a$の間を往復する.$\omega t + \alpha$の値によって$x$の値が決まるので\textbf{位相}という.$\alpha$を\textbf{初期位相}という.



\subsection{1節 問題}

\begin{enumerate}[label=\textbf{[\arabic*]}, labelsep=10pt, leftmargin=23pt]
	\item 空間の1つの点の位置の極座標を$r,\ \theta,\ \varphi$とする.$r,\ \theta,\ \varphi$方向($r$方向は$\theta$,$\varphi$を一定にして$r$だけが増すような方向,他も同様)の方向余弦を求めよ.
	\item 3つのベクトル$\bm{A},\ \bm{B},\ \bm{C}$を1つの点$\mathrm{O}$から引くときこれらが一平面内にあるための条件を求めよ.
	\item 2つの点$\mathrm{A,\ B}$の位置ベクトルを$\bm{A},\ \bm{B}$とする.$\mathrm{A,\ B}$両方の点を通る直線の方程式は
	\begin{equation*}
		\bm{r} = (1 - \lambda)\bm{A} + \lambda\bm{B}
	\end{equation*}
	であることを証明せよ.
	\item 1つの平面($xy$平面)内にあるベクトル$\bm{A}$の成分が$A_x = A\cos\omega t,\ A_y = A\sin \omega t$($A,\ \omega$は定数)で与えられるとき$\bm{A}$と$\bunsuu{d\bm{A}}{dt}$とは互いに直角になっていることを証明せよ.
\end{enumerate}



\section{運動の法則}
\subsection{慣性の法則(運動の第1法則)}

\begin{tcolorbox}[colback=white]
	すべての物体は,加えられた力によってその状態が変化させられない限り,静止或いは等速直線運動の状態を続ける(\textbf{慣性系の存在}).
\end{tcolorbox}

2つの座標系$\mathrm{S}$系:$\mathrm{O}\text{-}xyz$と$\mathrm{S'}$系:$\mathrm{O}\text{-}x'y'z'$を考える.$\mathrm{S'}$系は$\mathrm{S}$系に対して並進運動(平行移動)をしていると考える.

このとき,空間内に質点$m$があり,力$\bm{F}$が作用しているとする.$\mathrm{S}$系での位置ベクトルは$\bm{r}$,$\mathrm{S'}$系での位置ベクトルは$\bm{r}'$である.また,$\mathrm{O}$から見た$\mathrm{O'}$の位置ベクトルを$\bm{R}$とする.すると
\begin{equation}
	\bm{r} = \bm{R} + \bm{r}'
\end{equation}
の関係がある.これを用いると速度,加速度はそれぞれ
\begin{gather}
	\bunsuu{d\bm{r}}{dt} = \bunsuu{d\bm{R}}{dt} + \bunsuu{d\bm{r}'}{dt}\\
	\bunsuu{d^2\bm{r}}{dt^2} = \bunsuu{d^2\bm{R}}{dt^2} + \bunsuu{d^2\bm{r}'}{dt^2}
\end{gather}
となる.従って,運動方程式(\ref{sec:NR1-2-3}節を参照)より
\begin{equation}
	m\bunsuu{d^2\bm{r}}{dt^2} = m\bunsuu{d^2\bm{R}}{dt^2} + m\bunsuu{s^2\bm{r}'}{dt^2} = \bm{F}
\end{equation}

\subsubsection*{$\mathrm{S}$系に対して$\mathrm{S}'$系が等速直線運動をしている場合}

このとき$\bm{R}$の加速度は$\bm{0}$なので
\begin{equation*}
	\bunsuu{d^2\bm{R}}{dt^2} = \bm{0}
\end{equation*}
よって
\begin{gather}
	\text{$\mathrm{S}$系\qquad} m\bunsuu{d^2\bm{r}}{dt^2} = \bm{F}\\
	\text{$\mathrm{S}'$系\qquad} m\bunsuu{d^2\bm{r}'}{dt^2} = \bm{F}
\end{gather}

従って,どちらの系でも同様に運動を記述できる.


\subsubsection*{$\mathrm{S}$系に対して$\mathrm{S}'$系が加速度運動をしている場合}

このとき
\begin{equation*}
	\bunsuu{d^2\bm{R}}{dt^2} \ne \bm{0}
\end{equation*}
なので
\begin{gather}
	\text{$\mathrm{S}$系\qquad} m\bunsuu{d^2\bm{r}}{dt^2} = \bm{F}\\
	\text{$\mathrm{S}'$系\qquad} m\bunsuu{d^2\bm{r}'}{dt^2} = \bm{F} - m\bunsuu{d^2\bm{R}}{dt^2}
\end{gather}

力が働かない場合($\bm{F} = \bm{0}$)を考えると
\begin{gather}
	\text{$\mathrm{S}$系\qquad} m\bunsuu{d^2\bm{r}}{dt^2} = \bm{0}\\
	\text{$\mathrm{S}'$系\qquad} m\bunsuu{d^2\bm{r}'}{dt^2} =- m\bunsuu{d^2\bm{R}}{dt^2}
\end{gather}
$\mathrm{S}$系では等速直線運動,$\mathrm{S'}$系では加速度運動が観測される.従って$- m\bunsuu{d^2\bm{R}}{dt^2}$を見かけの力(\textbf{慣性力})とする.第1法則が成り立つ系を\textbf{慣性系},成り立たない系を\textbf{非慣性系}という.

非慣性系で運動方程式を記述するには
\begin{equation}
	m\bunsuu{d^2\bm{r}'}{dt^2} = \bm{F}' = \bm{F} - m\bunsuu{d^2\bm{R}}{dt^2}
\end{equation}
と置き換える.

慣性系の問題の解き方
\begin{enumerate}[label=\textbf{[\arabic*]}, labelsep=10pt, leftmargin=23pt]
	\item 慣性系から見た動体の加速度$\alpha$を書き入れ,全ての力を書き込んで運動方程式を立てる.
	\item 非慣性系から見た,動体の中にある物体に働く慣性力を書き込む.
	\item 物体に働く慣性力以外の力を書き込む.
	\item 慣性力を含む物体の運動方程式を立てる(非慣性系の運動方程式).静止している場合はつり合いの式を書く.
	\item 運動方程式(つり合いの式)を解く.
\end{enumerate}



\subsection{ガリレイ変換}

2つの慣性系
$\mathrm{S}(\mathrm{O},\ x,\ y,\ z)$と
$\mathrm{S}'(\mathrm{O}',\ x',\ y',\ z')$
で,$x \heikou x',\ y \heikou y',\ z \heikou z'$とし,$\mathrm{O}'$は$\mathrm{S}$の座標系で$(x_0,\ y_0,\ z_0)$にあり,一定の速度$\bm{v}_0 = (u,\ v,\ w)$であるとする.

任意の点$\mathrm{P}$の座標を$(x,\ y,\ z),\ (x',\ y',\ z')$とすれば
\begin{align}
	x &= x_0 + x' & y &= y_0 + y' & z &= z_0 + z' \label{equ:NR1-3}\\
	x' &= x - x_0 & y' &= y - y_0 & z' &= z - z_0 \label{equ:NR1-4}
\end{align}
である.これらを$t$で微分すると
\begin{align}
	u &= u_0 + u' & v &= v_0 + v' & w &= w_0 + w' \label{equ:NR1-5}\\
	u' &= u - u_0 & v' &= v - v_0 & w' &= w - w_0 \label{equ:NR1-6}
\end{align}
となる.式(\ref{equ:NR1-5})を更に$t$で微分すると$\bunsuu{du_0}{dt} = 0,\ \bunsuu{dv_0}{dt} = 0,\ \bunsuu{dw_0}{dt} = 0$なので
\begin{align}
	\bunsuu{du}{dt} &= \bunsuu{du'}{dt} &
	\bunsuu{dv}{dt} &= \bunsuu{dv'}{dt} &
	\bunsuu{dw}{dt} &= \bunsuu{dw'}{dt} \label{equ4-7}
\end{align}
となる.式(\ref{equ:NR1-3})~式(\ref{equ:NR1-7})を\textbf{ガリレイ変換}という.例えば,式(\ref{equ:NR1-6})は速度$u$で飛んでいる鳥を同方向に速度$u_0$で走っている列車から見ると相対的に$u - u_0$の速度で飛んでいるように見えるということ.

$\bm{v}$が一定であるとき,$\mathrm{S}$が慣性系ならば$\mathrm{S}'$も慣性系である.



\subsection{力と加速度(運動の第2法則)}
\label{sec:NR1-2-3}

\begin{tcolorbox}[colback=white]
	質点に他の物体から力が働いた結果,加速度が生じる.このとき
	\begin{equation}
		m\bunsuu{d^2\bm{r}}{dt} = \bm{F}
	\end{equation}
が成り立つ(\textbf{運動方程式}).
\end{tcolorbox}

《注》運動方程式は$\text{[結果]} = \text{[原因]}$というように書くことが多い.

運動の変化は,\text{運動量}$\bm{p} = m\bm{v}$を用いて
\begin{equation*}
	\varDelta \bm{p} = \bm{p}(t_2) - \bm{p}(t_1)
\end{equation*}
と表される.運動の変化は加えられた駆動力(=力×力を加えた時間)によって起こるので,
\begin{equation}
	\label{equ:NR1-8}
	\varDelta \bm{p} = \bm{F}\varDelta t
\end{equation}
と書くことができる.しかし,実際$\bm{F}$は変化するので積分を用いることで一般化ができる.力を時間で積分したものを\textbf{力積}$\bm{I}$といい
\begin{equation*}
	\bm{I} = \int_{t_1}^{t_2} \bm{F}\,dt
\end{equation*}
で定義される.つまり運動の変化$\varDelta\bm{p}$は力積$\bm{I}$に等しい.

運動の変化率は式(\ref{equ:NR1-8})から
\begin{equation*}
	\bunsuu{\varDelta \bm{p}}{\varDelta t} = \bm{F}
\end{equation*}
と書ける.$\varDelta t \to 0$のとき
\begin{equation}
	\bunsuu{d\bm{p}}{dt} = \bm{F}
\end{equation}
で,質点の運動量の時間微分は,その瞬間に加えられた力に等しいことを意味する.運動量の定義式よりこれは
\begin{equation}
	m\bunsuu{d\bm{v}}{dt} = \bm{F}
\end{equation}
と書くこともできる.

ここで,$\bm{F}$はその物体に加えられた力の\textbf{合力}を指し,$m$はその質点の\textbf{慣性質量}とする.

運動方程式は
\begin{equation*}
	\bunsuu{d^2\bm{r}}{dt} = \bunsuu{\bm{F}}{m}
\end{equation*}
より,$\bm{F}$が一定のとき質量が大きいほど加速度の変化が小さい.物体が運動の状態を続けようとする性質を\textbf{慣性}ということから$m$は慣性質量と呼ばれる.



\subsection{作用・反作用の法則(運動の第3法則)}

\begin{tcolorbox}[colback=white]
	2個の質点1,\ 2があり,互いに力を及ぼしているとき,質点1が質点2から受ける力$\bm{F}_{12}$は,質点2が質点1から受ける力$\bm{F}_{21}$と大きさが同じで向きが反対である.つまり
	\begin{equation}
		\bm{F}_{12} = -\bm{F}_{21}
	\end{equation}
	である(\textbf{作用・反作用の法則}).
\end{tcolorbox}

林檎が落下しているとき,林檎が地球から受ける力(重力)と地球が林檎から受ける力は作用・反作用の関係にある.また,林檎が机の上で静止しているとき,林檎が机から受ける力(垂直抗力)と机が林檎から受ける力も作用・反作用の関係にある.しかし,林檎が地球から受ける力(重力)と林檎が机から受ける力(垂直抗力)は作用・反作用の関係ではなく,つり合いの関係である.



\subsection{2節 問題}

\begin{enumerate}[label=\textbf{[\arabic*]}, labelsep=10pt, leftmargin=23pt]
	\item 滑らかな水平面上にある板(質量$M$)の上を人(質量$m$)が板に対して加速度$a$で歩くとき,板は水平面上に対してどのような加速度を持つか.また,人と板とが互いに水平に及ぼしあう力はどれだけか.
	\item 水平な滑らかな床の上に一様な鎖(質量$M$,長さ$l$)を一直線に置いてその一端を一定の力$F$で引っ張る.鎖の各点での張力を求めよ.
	\item 惑星が太陽から惑星の質量に比例し,太陽からの距離の2乗に反比例する引力を受けて太陽のまわりを円運動を行うものとする.いろいろな惑星が太陽の周りを回る周期$T$と,円運動の半径$a$との間には
	\begin{equation*}
		\bunsuu{T^2}{a^3} = \text{惑星によらない定数}
	\end{equation*}
	の関係があることを示せ.この関係はケプラーの第3法則に相当する.
	\item 太陽系は銀河系の中心から30000光年の距離で,およそ$250\,\mathrm{km\,s^{-1}}$の速さで銀河系の中心を中心として等速円運動をしている.銀河系の形は図のようになっており,太陽系は銀河系の各恒星からの万有引力を受けている.銀河系の恒星は空間に散らばっているが,大雑把にいって太陽系に働く力は,銀河系全体の質量がその中心に集中していると考えても大体の程度のことはいえるであろう.太陽のまわりの地球の運動の速度は$30.0\,\mathrm{km\,s^{-1}}$として,銀河系の総質量と太陽の質量との比を求めよ.
	\item 中性子星と呼ばれる星は中性子が万有引力によって結び付けられてたもので,原子核と同様な密度(およそ$10^{12}\,\mathrm{g\,cm^{-3}}$)を持つ.中性子星は球形で,自転しているとして,赤道で中性子星が飛び去らないための回転の周期の最小値を求めよ.
\end{enumerate}



\section{簡単な運動}
\subsection{落体の運動}

鉛直上方に$y$軸をとり,適当な高さの点を原点とする.質量$m$の質点を$y$軸上で運動させると下向きに加速度を持っているので,下向きに力が働く.よって運動方程式は
\begin{equation*}
	m\bunsuu{d^2y}{dt^2} = -F
\end{equation*}

加速度は物体によらず一定である.元々物体が慣性質量を持つことと,地球が物体を引っ張る(万有引力)ことは独立なことである.よって,「加速度が物体によらず一定であること」は,慣性質量$m$と重力($F$)が比例していなければならない.これは歴史の中で確かめられたので
\begin{equation*}
	F = mg
\end{equation*}
とすれば運動方程式は
\begin{equation}
	\bunsuu{d^2y}{dt^2} = -g
\end{equation}
となる.これを積分して
\begin{equation*}
	\bunsuu{dy}{dt} = -gt + C
\end{equation*}

ここで,初速度を$v_0$とすれば,$t = 0$を代入して
\begin{equation*}
	\bunsuu{dy}{dt} = C \iff v_0 = C
\end{equation*}
なので
\begin{equation*}
	\bunsuu{dy}{dt} = -gt + v_0
\end{equation*}
となる.これを更に積分して
\begin{equation*}
	y = -\bunsuu{1}{2}gt^2 + v_0t + C'
\end{equation*}

投げ出した時の位置を原点とすれば,$t = 0$を代入して
\begin{equation*}
	0 = C'
\end{equation*}
よって
\begin{equation*}
	y = -\bunsuu{1}{2}gt^2 + v_0t
\end{equation*}
となる.



\subsection{粘性抵抗力が働く場合の落体運動}

物体の運動が遅いとき,\textbf{粘性抵抗力}がはたらき
\begin{equation}
	\bm{F} = -\alpha\bm{v}
\end{equation}
の形で与えられる.また,物体の運動が速いときは\textbf{慣性抵抗力}がはたらき
\begin{equation}
	F =
	\left\{
		\begin{array}{ll}
			-\beta v^2 & \text{$v > 0$のとき}\\
			+\beta v^2 & \text{$v < 0$のとき}
		\end{array}
	\right.
\end{equation}
の形で与えられる.

自由落下で,空気抵抗がある場合を考える.鉛直下向きに$y$軸をとると,運動方程式は
\begin{equation}
	m\bunsuu{dv}{dt} = mg - \alpha v
\end{equation}
である.変数分離して
\begin{align*}
	\bunsuu{dv}{dt} = g - \bunsuu{\alpha v}{m} &
	\iff dv = \left(g - \bunsuu{\alpha v}{m}\right)dt\\
	& \iff \bunsuu{dv}{g - \bunsuu{\alpha}{m}v} = dt
\end{align*}
両辺積分すると
\begin{align*}
	\int \bunsuu{dv}{g - \bunsuu{\alpha}{m}v}\ = \int dt
	&\iff -\bunsuu{m}{\alpha}\log\left|g - \bunsuu{\alpha}{m}v\right| = t + C\\
	&\iff \log\left|g - \bunsuu{\alpha}{m}v\right| = -\bunsuu{\alpha}{m}t + C\\
	&\iff g - \bunsuu{\alpha}{m}v = \exp\left(-\bunsuu{\alpha}{m}t + C\right)\\
	&\iff \bunsuu{\alpha}{m} v = g - \exp\left(-\bunsuu{\alpha}{m}t + C\right)\\
	&\iff v = \bunsuu{m}{\alpha}\left\{g - \exp\left(-\bunsuu{\alpha}{m}t + C\right)\right\} = \bunsuu{m}{\alpha}\left\{g - \exp\left(-\bunsuu{\alpha}{m}t\right)\exp(C)\right\}
\end{align*}
ここで,初期条件より$t = 0,\ v = 0$なので
\begin{equation*}
	0 = \bunsuu{m}{\alpha}\{g - \exp(C)\} \iff \exp(C) = g
\end{equation*}
よって,
\begin{equation}
	v = \bunsuu{mg}{\alpha}\left\{1 - \exp\left(-\bunsuu{\alpha}{m}t\right)\right\} \label{equ:NR1-9}
\end{equation}
となる.

式(\ref{equ:NR1-9})で,$t \to \infty$とすると
\begin{equation}
	v_{\infty} = \bunsuu{mg}{\alpha}
\end{equation}
となる.$v_{\infty}$を\textbf{終端速度}といい,これより大きくなることはない.



\subsection{慣性抵抗力が働く場合の落体運動}

半径が大きくなると粘性抵抗力より慣性抵抗力の方が支配的になる.よって,運動方程式は
\begin{equation}
	m\bunsuu{dv}{dt} = mg - \beta v^2
\end{equation}
となる.先程と同じように変数分離して
\begin{equation*}
	\bunsuu{dv}{g - \bunsuu{\beta}{m}v^2} = dt
\end{equation*}
両辺積分すると
\begin{equation*}
	\int \bunsuu{dv}{g - \bunsuu{\beta}{m}v^2} = \int dt
\end{equation*}
\vskip-1.5\baselineskip
\begin{align*}
	\text{左辺} &= \int \bunsuu{dv}{\Bigl(\sqrt{g} + \sqrt{\myfrac{\beta}{m}}\,v\Bigr)\Bigl(\sqrt{g} - \sqrt{\myfrac{\beta}{m}}\,v \Bigr)}
	% 部分分数分解
	= \bunsuu{1}{2\sqrt{g}} \int \left(
		\bunsuu{1}{
			\sqrt{g} + \sqrt{\myfrac{\beta}{m}}\,v
		} +
		\bunsuu{1}{
			\sqrt{g} - \sqrt{\myfrac{\beta}{m}}\,v
		}
	\right)\,dv\text{\hspace*{2\zw} (部分分数分解)}\\
	&= \bunsuu{1}{2\sqrt{g}}\left(
		\sqrt{\bunsuu{m}{\beta}}\log\left|\sqrt{g} + \sqrt{\bunsuu{\beta}{m}}\,v\right| - \sqrt{\bunsuu{m}{\beta}}\log\left|\sqrt{g} - \sqrt{\bunsuu{\beta}{m}}\,v\right|
	\right)\\
	&= \bunsuu{1}{2}\sqrt{\bunsuu{m}{g\beta}}\biggl(
		\log\biggl|\sqrt{g} + \sqrt{\myfrac{\beta}{m}}\,v\biggr| - \log\biggl|\sqrt{g} - \sqrt{\myfrac{\beta}{m}}\,v\biggr|
	\biggr) =
	\bunsuu{1}{2}\sqrt{\bunsuu{m}{g\beta}}\log\left|
		\bunsuu{\sqrt{g} + \sqrt{\myfrac{\beta}{m}}\,v}{\sqrt{g} - \sqrt{\myfrac{\beta}{m}}\,v}
	\right|\\
	\text{右辺} &= t + C
\end{align*}
よって
\begin{align*}
	\bunsuu{1}{2}\sqrt{\bunsuu{m}{g\beta}}\log\left|
		\bunsuu{\sqrt{g} + \sqrt{\myfrac{\beta}{m}}\,v}{\sqrt{g} - \sqrt{\myfrac{\beta}{m}}\,v}
	\right| = t + C
	& \quad\stackrel{2C\sqrt{\myfrac{g\beta}{m}} = C'}{\iff}\quad \log\left|
		\bunsuu{\sqrt{g} + \sqrt{\myfrac{\beta}{m}}\,v}{\sqrt{g} - \sqrt{\myfrac{\beta}{m}}\,v}
	\right| = 2t\sqrt{\bunsuu{g\beta}{m}} + C'\\
	&\qquad\iff \bunsuu{\sqrt{g} + \sqrt{\myfrac{\beta}{m}}\,v}{\sqrt{g} - \sqrt{\myfrac{\beta}{m}}\,v} = e^{2t\sqrt{\myfrac{g\beta}{m}} + C'}\\
	&\qquad\iff \sqrt{g} + \sqrt{\bunsuu{\beta}{m}}\,v = \biggl(\sqrt{g} - \sqrt{\bunsuu{\beta}{m}}\,v\biggr)e^{2t\sqrt{\myfrac{g\beta}{m}} + C'}
\end{align*}
\begin{align*}
	\text{(前頁の続き)}
	&\iff \sqrt{\bunsuu{\beta}{m}}\,v + \sqrt{\bunsuu{\beta}{m}}\,v e^{2t\sqrt{\myfrac{g\beta}{m}} + C'} = \sqrt{g}\,e^{2t\sqrt{\myfrac{g\beta}{m}} + C'} - \sqrt{g}\\
	&\iff \sqrt{\bunsuu{\beta}{m}}\,v \left(1 + e^{2t\sqrt{\myfrac{g\beta}{m}} + C'}\right) = \sqrt{g}\left(e^{2t\sqrt{\myfrac{g\beta}{m}} + C'} - 1\right)\\
	&\iff v = -\sqrt{\bunsuu{mg}{\beta}}
	\bunsuu{
		1 - e^{2t\sqrt{\myfrac{g\beta}{m}} + C'}
	}{
		1 + e^{2t\sqrt{\myfrac{g\beta}{m}} + C'}
	} 
\end{align*}
となる.初期条件より$t = 0,\ v = 0$なので
\begin{align*}
	0 = 1 - e^{C'} \iff C' = 0
\end{align*}
よって
\begin{equation}
	v = -\sqrt{\bunsuu{mg}{\beta}}
	\bunsuu{
		1 - e^{2t\sqrt{\myfrac{g\beta}{m}}}
	}{
		1 + e^{2t\sqrt{\myfrac{g\beta}{m}}}
	} 
\end{equation}
となる.ここで,分子分母に$e^{-2t\sqrt{\myfrac{g\beta}{m}}}$をかけると
\begin{equation*}
	v = -\sqrt{\bunsuu{mg}{\beta}}
	\bunsuu{
		e^{-2t\sqrt{\myfrac{g\beta}{m}}} - 1
	}{
		e^{-2t\sqrt{\myfrac{g\beta}{m}}} + 1
	} = \sqrt{\bunsuu{mg}{\beta}}
	\bunsuu{
		1 - e^{-2t\sqrt{\myfrac{g\beta}{m}}}
	}{
		1 + e^{-2t\sqrt{\myfrac{g\beta}{m}}}
	}
\end{equation*}
なので,$t \to \infty$とすると
\begin{equation}
	v_{\infty} = \sqrt{\bunsuu{mg}{\beta}}
\end{equation}
と,終端速度が求められる.



\subsection{放物運動}

質量$m$の物体を,仰角(地表と成す角)$\theta\ \left(0 < \theta < \bunsuu{\pi}{2}\right)$,速さ$v_0$で放り投げた場合の運動を考える.

空気抵抗を無視すると,任意の点での物体に加わる力は重力だけなので運動方程式は次のようになる.
\begin{align}
	&m\bunsuu{d^2 x}{dt^2} = 0 & &m\bunsuu{d^2 y}{dt^2} = 0 & &m\bunsuu{d^2 z}{dt^2} = -mg
\end{align}
これらの微分方程式を解くと次のようになる.
\begin{align*}
	\bunsuu{d^2 x}{dt^2} &= 0 & \bunsuu{d^2 y}{dt^2} &= 0 & \bunsuu{d^2 z}{dt^2} &= -g\\
	\bunsuu{dx}{dt} &= C_x & \bunsuu{dy}{dt} &= C_y & \bunsuu{dz}{dt} &= -gt + C_z\\
	x &= C_x\,t + D_x & y &= C_y\,t + D_y & z &= -\bunsuu{1}{2}gt^2 + C_z\,t + D_z
\end{align*}
放物運動は2次元平面内の運動なので,$xz$平面内での運動と考えると,$t = 0$のとき$\bm{r} = (0,\ 0,\ 0)$,$\bm{v}_0 = (v_0\cos\theta,\ 0,\ v_0\sin\theta)$なので,代入すると
\begin{align*}
	\bunsuu{d}{dt}x(0) &= v_0\cos\theta = C_x &
	\bunsuu{d}{dt}y(0) &= 0 = C_y &
	\bunsuu{d}{dt}z(0) &= v_o\sin\theta = C_z\\
	x(0) &= 0 = D_x &
	y(0) &= 0 = D_y &
	z(0) &= 0 = D_z
\end{align*}
なので,特殊解は
\begin{align}
	x &= v_0t\cos\theta & y &= 0 & z &= v_0 t\sin\theta
\end{align}
となる.



\subsection{粘性抵抗力が働く場合の放物運動}

図を描くと,任意の点での物体に加わる力は重力$m\bm{g}$と粘性抵抗力$-\alpha\bm{v}$である.水平方向を$x$軸,鉛直方向を$z$軸とすると
\begin{align*}
	m\bm{g} &= -mg\bm{k} & -\alpha\bm{v} = -\alpha v_x \bm{i} - \alpha v_z\,\bm{k}
\end{align*}
となる.また,初期条件は先程と同じとする.運動方程式は
\begin{align}
	m\bunsuu{d v_x}{dt} &= -\alpha v_x & m\bunsuu{d v_z}{dt} &= -mg - \alpha v_z
\end{align}
となる.










\subsection{3節問題}

\begin{enumerate}[label=\textbf{[\arabic*]}, labelsep=10pt, leftmargin=23pt]
	\item 全質量$M$の風船が$\alpha$の加速度で落ちている.逆に上向きに加速度$\alpha$の運動をするためにはどれだけの質量の砂袋を捨てなければならないか.
	\item 軽い定滑車に糸をかけてその両端に質量$m_1,\ m_2$の質量をつるして放す.両質点の加速度を求めよ.また,糸の張力を求めよ.(この装置をアトウッドの装置とよぶ).
	\item 前の問題で滑車を$\beta$の加速度で引き上げるとき,両質点の滑車に対する加速度と糸の張力はどうなるか.
	\item 地上から一定の速さで石を投げるとき地面の達することのできる区域の面積は$S_0$である.地上から上方$h$のところから同じ速さで投げると区域は$S_h = S_0 + 2h\sqrt{\pi s_0}$で与えられることを証明せよ.
	\item 物体を投げるときの初速を知りたいがこれを直接に測ることが難しい.それで投射距離と時間を測定してこれを求めたいと考える.公式を求めよ.
	\item 図に示すように,正,負に帯電した平行金属板(偏向板)の間に電子(質量$m$)を両板に走らせる.電子には一方の力$eE$($e$:電子の荷電,$E$:電場の強さ)が負の方から正の方に働く.電子が偏向板の間を$l$だけ走ってその端に来たとき,はじめ目指していた位置からどれだけずれるか.またそのときはじめの方向とどれだけの角をつくる方向に運動するか.
	\item 空気抵抗が速さに比例する大きさ($kmV$)を持つときの放物運動で,抵抗が小さいとして放物距離の近似式を求めよ.
	\item 放物運動を行う物体に及ぼす空気の抵抗が$m\phi(V)$(ただし,$\phi$は任意の関数)であるとき,速さ$V$,鉛直線と軌道の接線のつくる角$\psi$の関係は
		\begin{equation*}
			\bunsuu{1}{V}\bunsuu{dV}{d\psi} = -\bunsuu{\psi(V)}{g\sin \psi} - \cot \psi
		\end{equation*}
		を積分することによって求められることを示せ.
	\item 前の問題で$\phi(V) = mkV^2$の場合どうなるかを論ぜよ.
	\item 角振動数$\omega_0$で単振動を行っている質点に,角振動数$\omega_1,\ \omega_2$の周期的な2つの力が作用するときこの質点はどのような運動を行うか.
	\item 上の問題で質点に$T$を周期とする周期的な力$f(t)$が働く時を考えよ.$f(t)$の平均値は$0$とする.
\end{enumerate}



\section{運動方程式の変換}
\subsection{4節問題}

\begin{enumerate}[label=\textbf{[\arabic*]}, labelsep=10pt, leftmargin=23pt]
	\item 螺旋
		\begin{equation*}
			x = a\cos \phi,\quad y = a\sin\phi,\quad z = k\phi \quad \text{($a,\ k$は定数)}
		\end{equation*}
		の接線,主法線,陪法線の方向を求めよ.\\
		この螺旋に沿って上向きに一定の速さ$V$で昇る点の加速度を求めよ.
	\item 環面
		\begin{equation*}
			x = (c + a\sin\theta)\cos\phi,\quad y = (c + a\sin\theta)\sin\phi,\quad z = a\cos\theta
		\end{equation*}
		の上を運動する点の子午線方向($\phi = \text{一定}$)で$\theta$だけが増す方向),法線方向,方位角方向($\phi$だけが増す方向)の加速度成分を求めよ.
	\item $(x,\ y)$面を運動する点の描く軌道が$r = a\sin n\phi$($a,\ n$は定数)で与えられ,加速度$\dot{\phi}$が$r^2$に反比例するとき,この点の加速度を求めよ.
\end{enumerate}



\section{力学的エネルギー 面積の原理}
\subsection{5節問題}

\begin{enumerate}[label=\textbf{[\arabic*]}, labelsep=10pt, leftmargin=23pt]
	\item 平面内を運動する質点に働く力の成分が質点の座標を$x,\ y$として
		\begin{equation*}
			X = axy,\quad Y = \bunsuu{1}{2}ax^2
		\end{equation*}
		で与えられるとき,保存力かどうか調べよ.保存力ならば位置エネルギーはどうなるか.
	\item 一平面内を運動する質点に働く力の成分が,質点の座標を$x,\ y$として
		\begin{equation*}
			X = axy,\quad Y = by^2
		\end{equation*}
		で与えられるとき,保存力かどうか調べよ.また,$x$軸上の$(r,\ 0)$で与えられる点$\mathrm{A}$から,$y$軸上の$(0,\ r)$で与えられる点$\mathrm{C}$まで,円周$\mathrm{ABC}$に沿ってゆく場合と,弦$\mathrm{AB'C}$に沿ってゆく場合とで,この力の行う仕事を比較せよ.
	\item 次の諸式を証明せよ.
		\begin{enumerate}[label={(\alph*)}, labelsep=10pt]
			\item $\bm{A} \times (\bm{B} \times \bm{C}) = \bm{B}(\bm{A} \cdot \bm{C}) - \bm{C}(\bm{A} \cdot \bm{B})$
			\item $(\bm{A} \times \bm{B}) \cdot (\bm{C} \times \bm{D}) = (\bm{A} \cdot \bm{C})(\bm{B} \cdot \bm{D}) - (\bm{B} \cdot \bm{C})(\bm{A} \cdot \bm{D})$
			\item $(\bm{B} \times \bm{C})\cdot(\bm{A} \times \bm{D}) + (\bm{C} \times \bm{A}) \cdot (\bm{B} \times \bm{D}) + (\bm{A} \times \bm{B}) \cdot (\bm{C} \times \bm{D}) = 0$
		\end{enumerate}
	\item $\bm{A},\ \bm{B},\ \bm{C}$がこの順に右手系(一般に互いに直角でなくでよい)をつくっているとすれば
		\begin{equation*}
			\bm{A} \cdot (\bm{B} \times \bm{C}) = \bm{B} \cdot (\bm{C} \times \bm{A}) = \bm{C} \cdot (\bm{A} \times \bm{B})
		\end{equation*}
		は$\bm{A},\ \bm{B},\ \bm{C}$を稜(かど)とする六面体の体積であることを証明せよ.
	\item 1つの単位ベクトルを$\bm{n}$とすれば,任意のベクトル$\bm{A}$は
		\begin{equation*}
			\bm{A} = (\bm{A} \cdot \bm{n})\bm{n} + \bm{n} \times (\bm{A} \times \bm{n})
		\end{equation*}
		と書くことができることを示せ.
	\item 滑らかな水平板の上においてある質点に糸を結び付け,その糸を板にあけた穴$\mathrm{O}$に通しておく.質点を,はじめOのまわりにある角速度で運動させ,糸を引っ張ってOと質点との距離を変えるとき,質点の角速度はどう変わっていくか.
	\item 一平面内で
		\begin{equation*}
			r = a(1 + c\cos \varphi)\qquad (0 < c < 1)
		\end{equation*}
		で与えられる軌道を描く質点に働く中心力はどんな力か.
\end{enumerate}
% \chapter{力学II}
\setcounter{page}{1}
\section{力学的エネルギー 面積の原理}
\subsection{5節問題}

\begin{enumerate}[label=\textbf{[\arabic*]}, labelsep=10pt, leftmargin=23pt]
	\item 平面内を運動する質点に働く力の成分が質点の座標を$x,\ y$として
		\begin{equation*}
			X = axy,\quad Y = \bunsuu{1}{2}ax^2
		\end{equation*}
		で与えられるとき,保存力かどうか調べよ.保存力ならば位置エネルギーはどうなるか.
	\item 一平面内を運動する質点に働く力の成分が,質点の座標を$x,\ y$として
		\begin{equation*}
			X = axy,\quad Y = by^2
		\end{equation*}
		で与えられるとき,保存力かどうか調べよ.また,$x$軸上の$(r,\ 0)$で与えられる点$\mathrm{A}$から,$y$軸上の$(0,\ r)$で与えられる点$\mathrm{C}$まで,円周$\mathrm{ABC}$に沿ってゆく場合と,弦$\mathrm{AB'C}$に沿ってゆく場合とで,この力の行う仕事を比較せよ.
	\item 次の諸式を証明せよ.
		\begin{enumerate}[label={(\alph*)}, labelsep=10pt]
			\item $\bm{A} \times (\bm{B} \times \bm{C}) = \bm{B}(\bm{A} \cdot \bm{C}) - \bm{C}(\bm{A} \cdot \bm{B})$
			\item $(\bm{A} \times \bm{B}) \cdot (\bm{C} \times \bm{D}) = (\bm{A} \cdot \bm{C})(\bm{B} \cdot \bm{D}) - (\bm{B} \cdot \bm{C})(\bm{A} \cdot \bm{D})$
			\item $(\bm{B} \times \bm{C})\cdot(\bm{A} \times \bm{D}) + (\bm{C} \times \bm{A}) \cdot (\bm{B} \times \bm{D}) + (\bm{A} \times \bm{B}) \cdot (\bm{C} \times \bm{D}) = 0$
		\end{enumerate}
	\item $\bm{A},\ \bm{B},\ \bm{C}$がこの順に右手系(一般に互いに直角でなくでよい)をつくっているとすれば
		\begin{equation*}
			\bm{A} \cdot (\bm{B} \times \bm{C}) = \bm{B} \cdot (\bm{C} \times \bm{A}) = \bm{C} \cdot (\bm{A} \times \bm{B})
		\end{equation*}
		は$\bm{A},\ \bm{B},\ \bm{C}$を稜(かど)とする六面体の体積であることを証明せよ.
	\item 1つの単位ベクトルを$\bm{n}$とすれば,任意のベクトル$\bm{A}$は
		\begin{equation*}
			\bm{A} = (\bm{A} \cdot \bm{n})\bm{n} + \bm{n} \times (\bm{A} \times \bm{n})
		\end{equation*}
		と書くことができることを示せ.
	\item 滑らかな水平板の上においてある質点に糸を結び付け,その糸を板にあけた穴$\mathrm{O}$に通しておく.質点を,はじめOのまわりにある角速度で運動させ,糸を引っ張ってOと質点との距離を変えるとき,質点の角速度はどう変わっていくか.
	\item 一平面内で
		\begin{equation*}
			r = a(1 + c\cos \varphi)\qquad (0 < c < 1)
		\end{equation*}
		で与えられる軌道を描く質点に働く中心力はどんな力か.
\end{enumerate}



\section{単振り子の運動と惑星の運動}



\section{非慣性系に相対的な運動}



\section{質点系の運動量と角運動量}



\section{剛体のつりあいと運動}
% \chapter{電磁気学I}
\setcounter{page}{1}


電磁気学に入る前に,この章で必要なベクトル解析の知識を確認する.詳しくはベクトル解析の章を見ていただきたい.



\section{線積分,面積分,体積分}

電磁気学を学ぶには,線積分,面積分,体積分の3つの積分の知識が必要不可欠である.



\subsection{スカラー場の線積分}

\textbf{線積分}とは,ある線分に沿って物理量を足していくことである.

空間上に曲線$C$がある.この$C$を\textbf{経路}という.線積分は,その経路上の各点に存在する物理量(今から説明するのはスカラー場$f$)を足して合わせていくことを考える.経路上の位置$\bm{r}$に存在するスカラー場$f(\bm{r})$が与えられているとする.

まず,経路$C$を$n$個の小区間に分割する.$C$の始点を点$\mathrm{A}$,終点を点$\mathrm{B}$として$\mathrm{A}$から順に,分点
\begin{equation*}
	\mathrm{A} = \mathrm{P}_0,\ \mathrm{P}_1,\ \mathrm{P}_2,\ \cdots,\ \mathrm{P}_n = \mathrm{B}
\end{equation*}
をとる.点$\mathrm{P}_k$の位置ベクトルを$\bm{r}_k$とし,$\varDelta s_k = \mathrm{P}_k - \mathrm{P}_{k - 1}$とする.

この小区間$\varDelta s_k$とスカラー場$f(\bm{r}_k)$をかけたものを足し合わせていく:
\begin{equation*}
	\sum_{k = 1}^{n} f(\bm{r}_k)\varDelta s_k
\end{equation*}
ここで,$n$を無限大までに増やして小区間の長さを限りなく0に近づかせると次のように積分の形に表すことができる.これを経路$C$に沿うスカラー場の線積分という.
\begin{equation}
	\sum_{k = 1}^{n} f(\bm{r}_k)\varDelta s_k \quad\xrightarrow{n \to \infty}\quad \int_{C} f(\bm{r})\,ds
\end{equation}



\subsection{ベクトル場の線積分}

経路$C$上にベクトル場$\bm{F}(\bm{r})$が与えられているとする.

ベクトルの線積分も基本的な考え方はスカラーのときと同じである.ただし,ベクトル場の線積分では,\textcolor{teal}{ベクトル場の経路に沿った方向の成分}を積分する.この成分は,ベクトル$\bm{F}$と単位接線ベクトル$\bm{t}$の内積で求めることができる.よって,ベクトル場$\bm{F}(\bm{r})$の線積分は以下の式である.
\begin{equation}
	\int_{C} \bm{F}(\bm{r}) \cdot \bm{t}\,ds = \int_{C} \bm{F}(\bm{r}) \cdot d\bm{r}
\end{equation}

経路の始点Aと終点Bが一致している曲線(経路)を\textbf{閉曲線}という.例えば$C: x^2 + y^2 = 1$は円なので閉曲線である.経路が閉曲線の場合の線積分を\textbf{周回積分}といい,次のように表す.
\begin{equation}
	\oint_{C} \bm{F}(\bm{r}) \cdot \bm{t}\,ds
\end{equation}



\subsection{スカラー場の面積分}

\textbf{面積分}とは,ある面全体について物理量を足し合わせていくことである.

空間上に曲面$S$があり,$S$上でスカラー場$f(\bm{r})$が定義されているとする.曲面を網目状の微小な面積$\varDelta S$に分け,それにスカラー場$f(\bm{r})$を掛けたものを足し合わせることを考える.そして,分割数を増やして$\varDelta S \to 0$としていけば面積分が得られる.
\begin{equation}
	\int_{\color{teal}S} f(\bm{r})\,d{\color{cyan}S} \label{equ5-1-1}
\end{equation}
ここで注意なのが,\textcolor{teal}{青緑色}の$S$は,曲面$S$上の面積分という意味での$S$であり,\textcolor{cyan}{水色}の$S$は,面積の$S$という意味である.

$dS$は微小な長方形の面積なので$dS = dxdy$である.よって,式(\ref{equ5-1-1})は
\begin{equation}
	\iint_{D} f(\bm{r})\,dxdy
\end{equation}
のように表すこともできる.$D$は曲面$S$が定義されている領域である.



\subsection{ベクトル場の面積分}

曲面$S$上にベクトル場$\bm{F}(\bm{r})$が与えられているとする.

ベクトル場の面積分では,\textcolor{teal}{ベクトル場の単位法線ベクトル方向の成分}を積分する.この成分は,ベクトル$\bm{F}$と単位法線ベクトル$\bm{n}$の内積で求めることができる.よって,ベクトル場$\bm{F}(\bm{r})$の面積分は以下の式である.
\begin{equation}
	\int_{S} \bm{F}(\bm{r}) \cdot \bm{n} \, dS
\end{equation}

球面のような閉じた面を\textbf{閉曲面}という.曲面が閉曲面の場合の面積分は次のように表す.
\begin{equation}
	\oint_{S} \bm{F}(\bm{r}) \cdot \bm{n}\,dS
\end{equation}



\subsection{体積分}

\textbf{体積分}とは,ある体積領域全体について物理量を足し合わせていくことである.

空間上に体積領域$V$があり,$V$上でスカラー場$f(\bm{r})$が定義されているとする.この領域を微小な体積領域$\varDelta V$に分け,それにスカラー場$f(\bm{r})$を掛けたものを足し合わせることを考える.そして,分割数を増やして$\varDelta V \to 0$としていけば体積分が得られる.
\begin{equation}
	\int_{V} f(\bm{r})\,dV
\end{equation}



\section{ベクトル場の微分:勾配,発散,回転}

\subsection{ハミルトンの演算子}

\textbf{ハミルトンの演算子}は,以下のような形式的なベクトルであり,\textbf{ナブラ}と読む.
\begin{equation}
	\bm{\nabla} =
	\begin{bmatrix}
		\bunsuu{\partial}{\partial x} &
		\bunsuu{\partial}{\partial y} &
		\bunsuu{\partial}{\partial z}
	\end{bmatrix}^\top
\end{equation}

ナブラの定義や逆三角形の2辺の太さが太いことから,$\bm{\nabla}$はベクトルの形をしていることは明らかである.
勿論,$\bunsuu{\partial}{\partial x}$だけでは成り立たず,$\bunsuu{\partial f}{\partial x}$のように微分される関数がないといけない.では何故こんなものがあるのかというと,後の勾配,発散,回転の式を表すのに便利だからだ.これらは電磁気学で出てくる大切な概念である.

例えば,$\bm{\nabla} f$と表せば,ベクトルのスカラー倍と見做すことができ
\begin{equation*}
	\bm{\nabla} f =
	\begin{bmatrix}
		\bunsuu{\partial}{\partial x} &
		\bunsuu{\partial}{\partial y} &
		\bunsuu{\partial}{\partial z}
	\end{bmatrix}^\top
	f
	=
	\begin{bmatrix}
		\bunsuu{\partial f}{\partial x} &
		\bunsuu{\partial f}{\partial y} &
		\bunsuu{\partial f}{\partial z}
	\end{bmatrix}^\top
\end{equation*}
と,スカラー関数$f$を$x,\ y,\ z$で偏微分したものをそれぞれの成分とするベクトルを表すことができる.尚,これは勾配の定義そのものである.

また,ベクトル関数$\bm{F}$を内積の形でかければ,
\begin{equation*}
	\bm{\nabla} \cdot \bm{F} =
	\begin{bmatrix}
		\bunsuu{\partial}{\partial x} &
		\bunsuu{\partial}{\partial y} &
		\bunsuu{\partial}{\partial z}
	\end{bmatrix}^\top
	\cdot
	\begin{bmatrix}
		F_x & F_y & F_z
	\end{bmatrix}^\top
	= \bunsuu{\partial F_x}{\partial x} + \bunsuu{\partial F_y}{\partial y} + \bunsuu{\partial F_z}{\partial z}
\end{equation*}
となる.これは,発散の定義である.

外積の形でかければ
\begin{align*}
	\bm{\nabla} \times \bm{F} &=
	\begin{bmatrix}
		\bunsuu{\partial}{\partial x} &
		\bunsuu{\partial}{\partial y} &
		\bunsuu{\partial}{\partial z}
	\end{bmatrix}^\top
	\times
	\begin{bmatrix}
		F_x & F_y & F_z
	\end{bmatrix}^\top
	=
	\begin{vmatrix}
		\bm{i} & \bunsuu{\partial}{\partial x} & F_x\\[5mm]
		\bm{j} & \bunsuu{\partial}{\partial y} & F_y\\[5mm]
		\bm{k} & \bunsuu{\partial}{\partial z} & F_z\\
	\end{vmatrix}\\
	&=
	\bunsuu{\partial F_z}{\partial y}\bm{i} +
	\bunsuu{\partial F_x}{\partial z}\bm{j} +
	\bunsuu{\partial F_y}{\partial x}\bm{k} -
	\bunsuu{\partial F_y}{\partial z}\bm{i} -
	\bunsuu{\partial F_z}{\partial x}\bm{j} -
	\bunsuu{\partial F_x}{\partial y}\bm{k}\\
	&=
	\left(
		\bunsuu{\partial F_z}{\partial y} -
		\bunsuu{\partial F_y}{\partial z}
	\right)\bm{i}
	+
	\left(
		\bunsuu{\partial F_x}{\partial z} -
		\bunsuu{\partial F_z}{\partial x}
	\right)\bm{j}
	+
	\left(
		\bunsuu{\partial F_y}{\partial x} -
		\bunsuu{\partial F_x}{\partial y}
	\right)\bm{k}
	=
	\begin{bmatrix}
		\bunsuu{\partial F_z}{\partial y} -
		\bunsuu{\partial F_y}{\partial z}\\[5mm]
		\bunsuu{\partial F_x}{\partial z} -
		\bunsuu{\partial F_z}{\partial x}\\[5mm]
		\bunsuu{\partial F_y}{\partial x} -
		\bunsuu{\partial F_x}{\partial y}
	\end{bmatrix}
\end{align*}
となる.これは,回転の定義である.



\subsection{勾配(グラディエント)}

スカラー場$\varphi(x,\ y,\ z)$が直交座標で定義されているとする.このとき,$x$についての偏微分を$x$成分,$y$についての偏微分を$y$成分,$z$についての偏微分を$z$成分とするベクトル場を$\varphi$の\textbf{勾配}または\textbf{グラディエント}といい,$\grad \varphi$や$\bm{\nabla} \varphi$で表す.
\begin{equation}
	\bm{\nabla} \varphi = \grad \varphi =
	\begin{bmatrix}
		\bunsuu{\partial \varphi}{\partial x} &
		\bunsuu{\partial \varphi}{\partial y} &
		\bunsuu{\partial \varphi}{\partial z}
	\end{bmatrix}^\top
\end{equation}

勾配はそれぞれの点において$\varphi$が最も増大する方向を指し示す.勾配の向きは等位面に垂直で,大きさはどれくらい増大しているかを表す.等位面の間隔が狭いとき,勾配が急峻になっているので,$\bm{\nabla} \varphi$の大きさ(矢印の長さ)が大きく,間隔が広いとき,勾配が緩やかなので,$\bm{\nabla} \varphi$の大きさは小さい.

《注》勾配は,入力をスカラー場として,ベクトル場を出力する.



\subsection{発散(ダイヴァージェンス)}

ベクトル場$\bm{F}(x,\ y,\ z)$が直交座標で定義されているとする.このとき,$\bm{F}$の$x$成分$F_x$を$x$で,$y$成分$F_y$を$y$で,$z$成分$F_z$を$z$で偏微分したあと,それらを足し合わせたものを$\bm{F}$の\textbf{発散}または\textbf{ダイヴァージェンス}といい,$\dive \bm{F}$や$\bm{\nabla} \cdot \bm{F}$で表す.
\begin{equation}
	\bm{\nabla} \cdot \bm{F} = \dive \bm{F} = \bunsuu{\partial F_x}{\partial x} + \bunsuu{\partial F_y}{\partial y} + \bunsuu{\partial F_z}{\partial z}
\end{equation}

発散は,それぞれの点においてベクトル場$\bm{F}$の流入出を評価する.例えば,$\bm{v}(x,\ y,\ z)$は点$(x,\ y,\ z)$での流体の速度を示すとする.ある空間に3辺の長さが$\varDelta x,\ \varDelta y,\ \varDelta z$の微小な直方体を考える.この直方体に流入する流体の量を$V_{\mathrm{in}}$,流出する流体の量を$V_{\mathrm{out}}$とするとき,$V_{\mathrm{out}} - V_{\mathrm{in}} > 0$なら,流出量の方が多いので,この直方体の中に蛇口のようなものがあってそこから流体が「湧き出て」いることを表す.逆に$V_{\mathrm{out}} - V_{\mathrm{in}} < 0$なら,流出量の方が少ないので,この直方体の中に排水溝のようなものがあってそこに流体が「吸い込まれて」いることを表す.証明は略すが,この直方体の\textcolor{cyan}{中}から流れ出る流体は
\begin{equation*}
	V_{\mathrm{out}} - V_{\mathrm{in}} = 
	\left(
		\bunsuu{\partial v_x}{\partial x} + \bunsuu{\partial v_y}{\partial y} + \bunsuu{\partial v_z}{\partial z}
	\right)
	\varDelta x \varDelta y \varDelta z
	= (\dive \bm{v}) \varDelta x \varDelta y \varDelta z
\end{equation*}
で表すことができる.直方体の体積は$\varDelta x \varDelta y \varDelta z$なので,これで割ると$\dive \bm{v}$は\textcolor{teal}{単位時間における単位体積あたりでの直方体の中から流出した流体の体積}を表す.$\dive\bm{v} > 0$のとき「湧き出し」,$\dive\bm{v} < 0$のとき「吸い込む」.$\dive\bm{v} = 0$のときは流れ入った流体はそのまま流れ出る.

《注》発散は,入力をベクトル場として,スカラー場を出力する.



\subsection{回転(ローテーション)}

ベクトル場$\bm{F}(x,\ y,\ z)$が直交座標で定義されているとする.このとき,次のように定義されるものを$\bm{F}$の\textbf{回転}または\textbf{ローテーション}といい,$\rot \bm{F}$や$\bm{\nabla} \times \bm{F}$で表す.
\begin{equation}
	\bm{\nabla} \times \bm{F} =
	\begin{vmatrix}
		\bm{i} & \bunsuu{\partial}{\partial x} & F_x\\[5mm]
		\bm{j} & \bunsuu{\partial}{\partial y} & F_y\\[5mm]
		\bm{k} & \bunsuu{\partial}{\partial z} & F_z\\
	\end{vmatrix}
	=
	\begin{bmatrix}
		\bunsuu{\partial F_z}{\partial y} -
		\bunsuu{\partial F_y}{\partial z}\\[5mm]
		\bunsuu{\partial F_x}{\partial z} -
		\bunsuu{\partial F_z}{\partial x}\\[5mm]
		\bunsuu{\partial F_y}{\partial x} -
		\bunsuu{\partial F_x}{\partial y}
	\end{bmatrix}
\end{equation}

回転は,それぞれの点においてベクトル場$\bm{F}$の回転の向きや強さを表している.$\rot \bm{F} \ne \bm{0}$のとき$\bm{F}$が回転軸に対して渦巻いている,つまり回転する要素があるということになる.

《注》回転は,入力をベクトル場として,ベクトル場を出力する.


\section{クーロン力と電場}
% \subsection{万有引力の法則}

% クーロン力(静電気力)は,重力と対比すると分かりやすい.理由はクーロンの法則と万有引力の法則が非常に似ているからである.ここでは万有引力の法則を振り返る.

% \begin{kousiki}{万有引力の法則}
% 	2つの物体1,2の位置ベクトルをそれぞれ$\bm{r}_1,\ \bm{r}_2$とする.物体1が物体2から受ける\textbf{万有引力}$\bm{F}_{12}$,物体2が物体1から受ける万有引力$\bm{F}_{21}$はそれぞれ
% 	\begin{gather}
% 		\bm{F}_{12} = {\color{blue}
% 			G\bunsuu{m_1 m_2}{|\bm{r}_1 - \bm{r}_2|^2}
% 		}
% 		\cdot 
% 		{\color{red}
% 			\left(-\bunsuu{\bm{r}_1 - \bm{r}_2}{|\bm{r}_1 - \bm{r}_2|}\right)
% 		}\\
% 		\bm{F}_{21} = {\color{blue}
% 			G\bunsuu{m_1 m_2}{|\bm{r}_2 - \bm{r}_1|^2}
% 		}
% 		\cdot 
% 		{\color{red}
% 			\left(-\bunsuu{\bm{r}_2 - \bm{r}_1}{|\bm{r}_2 - \bm{r}_1|}\right)
% 		}
% 	\end{gather}
% 	\textcolor{blue}{青字}は万有引力の大きさ,\textcolor{red}{赤字}は万有引力の向きを表す.
% \end{kousiki}



% \subsubsection*{大きさ}

% $m$は\textbf{重力質量}という.万有引力から定義されるのでこう呼ばれる.$1\,\mathrm{N}$とは,$1\,\mathrm{kg}$の質量をもつ物体に$1\,\mathrm{m/s^2}$の加速度を与える力の大きさである.$G$は\textbf{万有引力定数}といい,$G \approx 6.674 \times 10^{-11}\,\mathrm{N \cdot m^2 / kg^2}$と測定されている.

% $|\bm{r}_1 - \bm{r}_2|$は2点間の距離を表す.万有引力の法則は距離の2乗に反比例する.



% \subsubsection*{向き($\bm{F}_{12}$の場合)}

% $\bm{r}_1 - \bm{r}_2$は物体2から物体1へ向かう向きである.$\bm{F}_{12}$は物体1に働く\textbf{引力}なので,マイナスが付く.すると物体1は物体2に引きつけられる.また,向きのみを表すため,大きさが変わってはいけないので,絶対値で割って大きさを$1$にする.



\subsection{クーロン力(静電気力)}

クーロン力が起こる原因になるものを\textbf{電荷}という.電荷には正負があり,正の電荷をもつ代表的なものに陽子,負の電荷をもつ代表的なものに電子がある.電荷の量を\textbf{電荷量}といい,$q$と表す.単位は$\mathrm{C}$である.

$1\,\mathrm{C}$は$1\,\mathrm{A}$の電流を1秒間流した時に流れる電荷量なので
\begin{equation}
	1\,\mathrm{C} = 1\,\mathrm{A \cdot s}
\end{equation}
である.

陽子と電子の電荷量は大きさが同じで向きが反対である.陽子と電子の電荷の大きさを\textbf{電気素量}といい,$e$で表す.
\begin{equation}
	e = 1.602 \times 10^{-19}\,\mathrm{C}
\end{equation}

つまり,陽子の電荷量は$+e$,電子の電荷量は$-e$である.

電荷は陽子と陽子,電子と電子のように同符号であるとき,2つの電荷の間に反発する力(斥力)が働く.また,陽子と電子のように異符号であるとき,2つの電荷の間に引き合う力(引力)が働く.このような力を\textbf{クーロン力}といい,以下の法則が成り立つ.

\begin{kousiki}{クーロンの法則}
	2つの点電荷があり,それぞれの電荷量を$Q,\ q$とする.$Q$から$q$へ向かうベクトルを$\bm{r}$とするとき,点電荷$q$に働く\textbf{クーロン力}は
	\begin{equation}
		\bm{F} = {\color{teal}
			\bunsuu{1}{4\pi \varepsilon_0}\bunsuu{qQ}{|\bm{r}|^2}
		}
		\cdot 
		{\color{cyan}
			\bunsuu{\bm{r}}{|\bm{r}|}
		}
	\end{equation}
	である.$\varepsilon_0$は真空の誘電率である.また,\textcolor{teal}{青緑色}はクーロン力の大きさ,\textcolor{cyan}{水色}はクーロン力の向きを表す.
\end{kousiki}



\subsubsection*{クーロン力の大きさ}

$\bunsuu{1}{4\pi\varepsilon_0}$は定数である.
$\varepsilon_0$は真空の誘電率\footnote{真空の誘電率についてはここでは触れない.そういうものだと認識してもよい.なお,真空の誘電率は,$c$を光の速さ,$\mu_0$を真空の透磁率とすると$\varepsilon_0 = \bunsuu{1}{c^2\mu_0}$で求められる.} で,
$\bunsuu{1}{4\pi\varepsilon_0} \approx 8.987552 \times 10^9\, \mathrm{V^2 / N}$である.この定数値はクーロンの法則が成り立つように辻褄合わせで決められた値であるのでそこまで深く考える必要はない.クーロン力は電荷量に比例し,2点間の距離$|\bm{r}|$の2乗に反比例する.



\subsubsection*{クーロン力の向き}

$Q$と$q$が同符号の場合,$q$には引力が働くので向きは$\bm{r}$と同じ向きになる.また,\textcolor{cyan}{水色}は向きのみを表すため,大きさが変わってはいけないので絶対値で割って大きさを$1$にする.



\subsubsection*{重ね合わせの原理}

電荷が$q,\ Q_1,\ Q_2,\ \cdots,\ Q_n$と複数個あったとする.また,$Q_1$が$q$に及ぼすクーロン力を$\bm{F}_1$,…,$Q_n$が$q$に及ぼすクーロン力を$\bm{F}_n$とするとき,$q$が受けるクーロン力はそれぞれのクーロン力の和で表すことができる(\textbf{重ね合わせの原理}).

\begin{kousiki}{クーロン力の重ね合わせの原理}
	$Q_i$から$q$へ向かうベクトルを$\bm{r}_i$とするとき
	,点電荷$q$が受けるクーロン力は,
	\begin{align}
		\bm{F} &= \sum_{i = 0}^{n} \bm{F}_i\\
			&= \bm{F}_1 + \bm{F}_2 + \cdots + \bm{F}_n \notag\\
			&= \bunsuu{q}{4\pi\varepsilon_0} \sum_{i = 0}^{n} \bunsuu{Q_i}{|\bm{r}_i|^2} \cdot \bunsuu{\bm{r}_i}{|\bm{r}_i|}
	\end{align}
\end{kousiki}

\begin{enumerate}[leftmargin=18pt, labelsep=10pt, itemindent=9pt]
	\item[\f{例}] 点$\mathrm{A}_1(a,\ 0)$に点電荷$Q$,点$\mathrm{A}_2(-a,\ 0)$に点電荷$Q$,点$\mathrm{D}(0,\ d)$に点電荷$q$がある.このとき,$q$にかかるクーロン力を求めよ.\\
		$\vecrm{A_1 D} = \bm{r}_1 = \begin{bmatrix*}[r] -a\\ d \end{bmatrix*},\ \vecrm{A_2 D} = \bm{r}_2 = \begin{bmatrix*}[r] a\\ d \end{bmatrix*}$とする.よって
		\begin{align*}
			\bm{F} &= \bm{F}_1 + \bm{F}_2\\
				&= \bunsuu{1}{4\pi\varepsilon_0}\bunsuu{qQ}{|\bm{r}_1|^2}\bunsuu{\bm{r}_1}{|\bm{r}_1|} + \bunsuu{1}{4\pi\varepsilon_0}\bunsuu{qQ}{|\bm{r}_2|^2}\bunsuu{\bm{r}_2}{|\bm{r}_2|}\\
				&= \bunsuu{qQ}{4\pi\varepsilon_0}\bunsuu{\bm{r}_1}{|\bm{r}_1|^3} + \bunsuu{qQ}{4\pi\varepsilon_0}\bunsuu{\bm{r}_2}{|\bm{r}_2|^3}\\
				&= \bunsuu{qQ}{4\pi\varepsilon_0}\bunsuu{1}{(\sqrt{a^2 + d^2})^3}
				\begin{bmatrix*}[r]
					-a\\ d
				\end{bmatrix*}
				+ \bunsuu{qQ}{4\pi\varepsilon_0}\bunsuu{1}{(\sqrt{a^2 + d^2})^3}
				\begin{bmatrix*}[r]
					a\\ d
				\end{bmatrix*}
				\\
				&= \bunsuu{qQ}{4\pi\varepsilon_0}\bunsuu{1}{(\sqrt{a^2 + d^2})^3}\left(
					\begin{bmatrix*}[r]
						-a\\ d
					\end{bmatrix*}
					+
					\begin{bmatrix*}[r]
						a\\ d
					\end{bmatrix*}
				\right) = \bunsuu{qQ}{4\pi\varepsilon_0}\bunsuu{1}{(\sqrt{a^2 + d^2})^3}
					\begin{bmatrix*}[r]
						0\\ d
					\end{bmatrix*}
		\end{align*}
\end{enumerate}



\subsection{電場(1)}

クーロン力は,何かの物体に接触していなくても発生する力である.これからある点電荷Aがあったとき,その点電荷Aが周りの空間に影響を及ぼしていて,その空間の中に点電荷Bが置かれたときにAによる影響を受けてクーロン力が働くのではないかと考えることにする.そのような影響を表す量を\textbf{電場}といい,$\bm{E}$で表す.

ここで,クーロンの法則による表し方を変えてみる(一先ず,クーロン力の大きさ(スカラー)だけを考える).
\begin{align}
	F &= \bunsuu{1}{4\pi\varepsilon_0}\bunsuu{Q_1 Q_2}{r^2} \notag\\
	&= Q_2 {\color{teal}\left(\bunsuu{1}{4\pi\varepsilon_0}\bunsuu{Q_1}{r^2}\right)} \label{equ5-1}
\end{align}
式(\ref{equ5-1})をこういう見方で考える.
\begin{enumerate}[label=\textbf{[\arabic*]}, labelsep=10pt, leftmargin=23pt]
	\item $Q_2$の存在は置いといて,点電荷$Q_1$があることによって$Q_1$が周りの空間にある影響$E$を作り出す(本来はベクトル$\bm{E}$).これは式(\ref{equ5-1})の\textcolor{teal}{青緑色}の部分に該当する.
	\item $Q_2$は,$E$を感じてクーロン力を受ける.
\end{enumerate}

\begin{kousiki}{電場(1)}
	電荷に力を作用させる電気的な空間(このような空間のことを\textbf{場}と呼ぶ.特に,場の値がベクトルであるとき\textbf{ベクトル場}という).
\end{kousiki}

では,このことをベクトルでまとめる.

\begin{kousiki}{電場(2)}
	原点にある電荷$Q_1$があって,この電荷が周りの空間のある1点$\bm{r} = (x,\ y,\ z)$に
	\begin{equation}
		\bm{E}(x,\ y,\ z) = \bunsuu{1}{4\pi\varepsilon_0}\bunsuu{Q_1}{|\bm{r}|^2}\bunsuu{\bm{r}}{|\bm{r}|}
	\end{equation}
	だけの電場を作っていると考える.そして,その電場に入った電荷$Q_2$は,電場から
	\begin{equation}
		\bm{F} = Q_2 \bm{E}
	\end{equation}
	と表せる力を受ける.電場$\bm{E}$の単位は$\mathrm{N/C}$である.よって,電場は「$1\,\mathrm{C}$の電荷に働くクーロン力」ということもできる.
\end{kousiki}

電場の向きは$\bm{r}$に平行である.$Q_2 < 0$であれば,$\bm{r}$と反対向きになる.



\subsection{電荷の分布}

今までは点電荷について進めてきたが,実際電荷は空間にどのように分布しているか考える.

\textbf{点電荷}は,空間のある1点に存在する電荷で,大きさをもたない.これでは現実味に欠けるので,空間的な広がりを与えていく.

まず,点電荷がある1方向に広がって分布する\textbf{線電荷}を考える.このとき,単位長さ(MKS単位系では$1\,\mathrm{m}$)あたりの電荷量を\textbf{線電荷密度}といい,$\lambda$で表す.単位は$\mathrm{C / m}$である.経路$C$上に存在する電荷の総量,つまり経路$C$にある総電荷$Q$は,以下のように線積分で求めることができる.
\begin{equation}
	Q = \int_{C} \lambda\,dl
\end{equation}

しかし,この線電荷は1方向には広がりを持つが,太さを持たない線であることが条件である.これではまだ非現実的なので,もう1方向に広がって分布する\textbf{面電荷}を考える.面$H$上の面電荷について,単位面積(MKS単位系では$1\,\mathrm{m^2}$)あたりの電荷量を\textbf{面電荷密度}といい,$\sigma$で表す.単位は$\mathrm{C / m^2}$である.面$H$上に存在する電荷の総量(総電荷)$Q$は,以下のように面積分で求めることができる.
\begin{equation}
	Q = \int_{H} \sigma\,dS
\end{equation}

しかし,これも面に厚みがないものとするので,これでも現実的な電荷分布とは言えない.更にもう1方向に広がって分布する電荷を考える.これは風船のような空間の体積領域$V$の内部に電荷が分布していると考える.このとき,単位体積(MKS単位系では$1\,\mathrm{m^3}$)あたりの電荷量を\textbf{電荷密度}といい,$\rho$で表す.単位は$\mathrm{C / m^3}$である.体積領域$V$に存在する電荷の総量(総電荷)$Q$は,以下のように体積分で求めることができる.
\begin{equation}
	Q = \int_{V} \rho\,dV
\end{equation}



\subsection{電場(2)}
% \chapter{電磁気学II}
\setcounter{page}{1}

% \chapter{電磁気学III}
\setcounter{page}{1}

\chapter{コンピュータアーキテクチャ}
\setcounter{page}{1}

\section{基本アーキテクチャ}
\subsection{基本ハードウェア構成}

コンピュータは,

\begin{enumerate}[label=\textbf{[\arabic*]}, labelsep=10pt, leftmargin=23pt]
	\item \textbf{プロセッサ}(\textbf{CPU})
	\item \textbf{メインメモリ}
	\item \textbf{入出力装置}(外部装置で,キーボードとディスプレイ)
\end{enumerate}
で構成させる.\textbf{[1]},\textbf{[2]}を纏めて\textbf{内部装置}という.



\section{内部装置のアーキテクチャ}\label{sec14-2}
\subsection{内部装置のハードウェア構成}

コンピュータの内部装置は,\textbf{プロセッサ}と\textbf{メインメモリ}の2つの基本ハードウェア装置を\textbf{内部バス}で接続して構成している.このうち,プロセッサは以下の3つの主要なハードウェア装置やハードウェア機構で構成する.

\begin{enumerate}[label=\textbf{[\arabic*]}, labelsep=10pt, leftmargin=23pt]
	\item \textbf{制御機構}
	\item \textbf{演算機構}
	\item \textbf{レジスタ}
\end{enumerate}

\textbf{[2]}演算機構と\textbf{[3]}レジスタは\textbf{データバス}でデータを送受する.



\subsection{プロセッサアーキテクチャ① 制御機構}\label{sec14-2-B}
\vskip-1\baselineskip
\subsubsection{制御機構}

\textbf{制御機構}は,次の2つの機能を実現するハードウェア機構である.

\begin{enumerate}[label=\textbf{[\alph*]}, labelsep=10pt, leftmargin=23pt]
	\item \textbf{順序制御機構}\quad 実行するマシン命令のメモリアドレスを決定する.即ちマシン命令の実行順序を決める.
	\item \textbf{制御信号}の生成\quad プロセッサ更には内部装置全体の各所に,それぞれを制御するために必要なハードウェア信号を供給する.
	\begin{align*}
		&\text{命令コード(OPコード)} \qlongright \text{各装置に動作させる.}\\
		&\text{\f{例} 1001} \hspace*{6.5\zw}\qlongright
		\left\{
			\begin{array}{l}
				\text{格納したい} \qlongright \text{レジスタに信号を送る.}\\
				\text{加算したい} \qlongright \text{演算機構に信号を送る.}
			\end{array}
		\right.
	\end{align*}
\end{enumerate}



\subsubsection{制御方式}

命令コードから制御信号を生成することを\textbf{(命令)デコード}という.制御信号の生成方法は2通りある.

\begin{enumerate}[label=\textbf{[\arabic*]}, labelsep=10pt, leftmargin=23pt]
	\item \textbf{配線論理制御方式(ワイヤードロジック)}\quad 論理回路でデコーダをつくる.ハードウェア的制御方式なので,動作速度は速いが,変更はできない.\\
	単純な機能で簡潔な制御で済むマシン命令に対して適用される.
	\item \textbf{マイクロプログラム制御方式}\quad 命令デコーダをプログラムで表現してデコーダをつくる.ソフトウェア的制御方式なので,動作速度は遅いが,変更ができる.このマイクロプログラムは,\textbf{制御メモリ}という専用メモリに格納してある.\\
	高機能で複雑な制御が必要な場合に用いられる.
\end{enumerate}

現代のコンピュータはこれらを併用している.



\subsubsection{割り込み}

不測の事態や事象(\textbf{割り込み要因})が生じたときに割り込み機構が発動すると,実行中のプログラム(マシン命令列)を一時中断して,\textbf{割り込み処理}プログラムへ制御フローが分岐する.



\subsubsection{割り込みの必要性}

\begin{enumerate}[label=\textbf{[\arabic*]}, labelsep=10pt, leftmargin=23pt]
	\item 不測の事態や事象の対処
	\item 異常や例外の検知・対処
	\item 「ユーザプログラムからOSへの通信」機能の実現
	\item 「ハードウェア装置からOSへの通信」機能の実現
	\item ハードウェア同士の同期の実現
	\item 通信の競合の実現
\end{enumerate}



\subsubsection{割り込み要因}\label{sec14-2-B-5}

割り込みを引き起こす具体的な原因や事象を「割り込み要因の発生場所」で分類する.

\begin{enumerate}[label=\textbf{[\Alph*]}, labelsep=10pt, leftmargin=23pt]
	\item \textbf{内部割り込み}\quad 要因の発生場所が内部装置特にプロセッサにある割り込み.マシン命令の実行するタイミングに合わせて発生し,マシン命令の実行というソフトウェア的要因に依るので,\textbf{ソフトウェア割り込み}ともいう.割り込みの必要性の\textbf{[3]}の手段.
		\begin{enumerate}[label=\textbf{(\arabic*)}, labelsep=10pt, leftmargin=23pt]
			\item \textbf{命令実行例外}
				\begin{enumerate}[label={\color{gray}●}, labelsep=10pt, leftmargin=23pt]
					\item \textbf{メモリアクセス例外}
						\begin{enumerate}[label=\textbf{(\roman*)}, labelsep=10pt, leftmargin=23pt]
							\item 指定したメモリアドレスにマシン命令やデータがないとき
							\item \textbf{ページフォールト}…プログラムを仮想メモリ\footnote{メインメモリ+ファイル装置で作った仮装的なメモリ.}に割り当てていてメインメモリにはまだ読み込んでいない状態で,プログラムにアクセスしようとしたとき
							\item メモリ保護違反…アクセスする権限がないメインメモリへアクセスしようとしたとき
						\end{enumerate}
					\item \textbf{不正命令}
						\begin{enumerate}[label=\textbf{(\roman*)}, labelsep=10pt, leftmargin=23pt]
							\item 命令セットにない(=定義されていない)命令コードであるとき
							\item データなのに命令として実行しようとしたとき
						\end{enumerate}
					\item \textbf{不正オペランド}\quad オペランドで指定するアドレスにデータがないとき
					\item \textbf{演算例外}
						\begin{enumerate}[label=\textbf{(\roman*)}, labelsep=10pt, leftmargin=23pt]
							\item \textbf{オーバーフロー}\footnote{演算結果が演算器そのものやレジスタの最上位ビットから溢れる場合.}を起こしたとき
							\item \textbf{0除算}…0を除数とする除算を行ったとき
						\end{enumerate}
				\end{enumerate}
			\item \textbf{SVC(スーパーバイザコール)}\quad OSを呼び出すSVC命令を実行したとき.SVC命令は以下の通り:
				\begin{enumerate}[label={\color{gray}●}, labelsep=10pt, leftmargin=23pt]
					\item \textbf{入出力命令}…ディスプレイやプリンタなど
					\item \textbf{ブレークポイント命令}…プログラム実行の中断点(ブレークポイント\footnote{デバグ(プログラムの誤りを発見・削除する操作)時にプログラム実行を一時中断する起点,プログラムをトレース(実行履歴を取る)する起点.})をOSに通知する.
				\end{enumerate}
		\end{enumerate}
	\item 外部割り込み\quad 要因の発生場所が外部装置特に入出力装置にある割り込み.マシン命令の実行とは無関係に発生し,外部装置というハードウェア的要因に依るので,\textbf{ハードウェア割り込み}ともいう.割り込みの必要性の\textbf{[4]}の手段.
		\begin{enumerate}[label=\textbf{(\arabic*)}, labelsep=10pt, leftmargin=23pt]
			\item \textbf{入出力割り込み}\quad 入出力装置・通信装置からOSに次の状態を知らせる割り込み
				\begin{enumerate}[label=\textbf{(\roman*)}, labelsep=10pt, leftmargin=23pt]
					\item ユーザが入力装置によってデータを正常に入力した
					\item 出力装置の操作が正常に完了した
					\item 入出力装置に異常があった
					\item 通信装置を使って他のコンピュータが通信を要求した
				\end{enumerate}
			\item \textbf{ハードウェア障害}\quad ハードウェア装置から以下のような「障害発生」の通知
				\begin{enumerate}[label=\textbf{(\roman*)}, labelsep=10pt, leftmargin=23pt]
					\item 電源異常
					\item メモリからの読み出しエラー
					\item 温度異常
				\end{enumerate}
			\item \textbf{リセット}\quad ユーザが電源ボタンを押したとき
		\end{enumerate}
\end{enumerate}

複数の割り込みが発生した時,優先度を付与する.

\noindent\textbf{【優先度】}(高→低)
\begin{enumerate}[label={\color{gray}●}, labelsep=10pt, leftmargin=23pt]
	\item ハードウェア障害
	\item リセット
	\item 命令実行例外
		\begin{enumerate}[label={\color{gray}○}, labelsep=10pt, leftmargin=23pt]
			\item ページフォールト
			\item メモリ保護違反
			\item 演算例外
		\end{enumerate}
	\item 入出力割り込み
		\begin{enumerate}[label={\color{gray}○}, labelsep=10pt, leftmargin=23pt]
			\item ファイル装置(高速)
			\item キーボード・プリンタ
		\end{enumerate}
	\item SVC
	\item ブレークポイント命令
\end{enumerate}



\subsubsection{割り込み処理}

順序制御機構とOS機能とが機能分担して処理する.

\begin{figure}[H]
	\begin{center}
		\framebox{
		\includegraphics[width=15cm]{C:/Users/User/Documents/PowerPoint/コンピュータアーキテクチャ/割り込み.pdf}
		}
		\caption{割り込み処理の流れ}
		\label{fig14-1}
	\end{center}
\end{figure}

\begin{enumerate}[label=\arabic*., labelsep=10pt, leftmargin=23pt]
	\item 割り込みの発生…\ref{sec14-2}.\ref{sec14-2-B}節の\ref{sec14-2-B-5}で紹介した要因で割り込みが発生.
	\item 割り込みの受付…ハードウェア機構が割り込みを受付→割り込み処理開始.
	\item 割り込み禁止状態…他の割り込みを受け付けないようにする.
	\item ハードウェア状態を退避…ハードウェア機構によって今までのプログラムが生成していたハードウェア状態をプロセッサ内の退避領域へ退避.
	\item 割り込み要因の識別…割り込み要因は何かを識別.
	\item 割り込みハンドラへ分岐…割り込み要因ごとの割り込み処理プログラムへ分岐.
	\item ハードウェア状態の回復…退避領域から今までのプログラムが生成していたハードウェア状態を回復.
	\item 割り込み可能にする…他の割り込みを許可する.
\end{enumerate}



\subsubsection{命令パイプライン処理}

\begin{figure}[H]
	\begin{center}
		\framebox{
		\includegraphics[width=10cm]{C:/Users/User/Documents/PowerPoint/コンピュータアーキテクチャ/命令実行サイクル.pdf}
		}
		\caption{命令実行サイクルの流れ}
		\label{fig14-2}
	\end{center}
\end{figure}

命令実行サイクルは以下の通り.

\begin{enumerate}[label=\textbf{[\arabic*]}, labelsep=10pt, leftmargin=23pt]
	\item 命令取り出し(I)
	\item 命令デコード(D)
	\item オペランド取り出し(O)
	\item 実行(E)
	\item 結果格納(W)
	\item 次アドレス決定
\end{enumerate}

上の命令実行サイクルの1つ1つを\textbf{ステージ}という.この1連のマシン命令をプロセッサに投入して処理するとき,高速化するため\textbf{命令パイプライン処理}をする.命令パイプライン処理の具体的な流れは以下の通り.

\begin{figure}[H]
	\begin{center}
		\framebox{
		\includegraphics[width=8cm]{C:/Users/User/Documents/PowerPoint/コンピュータアーキテクチャ/命令パイプライン処理.pdf}
		}
		\caption{命令パイプライン処理の流れ}
		\label{fig14-3}
	\end{center}
\end{figure}

\begin{table}[H]
	\caption{命令パイプライン処理の流れ}
	\label{tab14-1}
	\centering
	\scriptsize
	\begin{tabular}{c|p{10cm}}
		\hline
		時間(クロック) & 実行内容\\
		\hline
		0 & 5つ(赤,青,黄,緑,紫)の命令が実行されるのを待っている.\\
		\hline
		1 & 紫色の命令を取り出す.\\
		\hline
		2 & 紫色の命令をデコードする.\par 緑色の命令を取り出す.\\
		\hline
		3 & 紫色の命令のオペランドを取り出す.\par 緑色の命令をデコードする.\par 黄色の命令を取り出す.\\
		\hline
		4 & 紫色の命令を実行する(実際の命令処理を行う).\par 緑色の命令のオペランドを取り出す.\par 黄色の命令をデコードする.\par 青色の命令を取り出す.\\
		\hline
		5 & 紫色の命令の結果を(レジスタなどに)格納する.\par 緑色の命令を実行する.\par 黄色の命令のオペランドを取り出す.\par 青色の命令をデコードする.\par 赤色の命令を取り出す.\\
		\hline
		6 & 紫色の命令は完了した.\par 緑色の命令の結果を格納する.\par 黄色の命令を実行する.\par 青色の命令のオペランドを取り出す.\par 赤色の命令をデコードする.\\
		\hline
		7 & 緑色の命令は完了した.\par 黄色の命令の結果を格納する.\par 青色の命令を実行する.\par 赤色の命令のオペランドを取り出す.\\
		\hline
		8 & 黄色の命令は完了した.\par 青色の命令の結果を格納する.\par 赤色の命令を実行する.\\
		\hline
		9 & 青色の命令は完了した.\par 赤色の命令の結果を格納する.\\
		\hline
		10 & 赤色の命令は完了した=\textbf{全命令を実行した.}\\
		\hline
	\end{tabular}
\end{table}

このように,5つのマシン命令が完了するまでに10サイクルの時間が必要になる.

\begin{tip}{例題}
	\textsf{上の図のように5ステージで構成する命令実行サイクルを持つ命令パイプライン処理機構がある.ただし,1ステージ時間は10ナノ秒とする.即ち,1命令実行サイクルは50ナノ秒要する.このパイプライン機構に100,000個のマシン命令を投入すると,全部のマシン命令が完了するまでにはどれだけ時間がかかるか求めなさい.また,命令パイプライン処理機構がない場合と比較しなさい.}

	\tcblower

	パイプライン処理をしない場合,1つの命令につき50ナノ秒なので,100,000個の命令では$50 \times 100000 = 5000000$ナノ秒,つまり,5ミリ秒かかる.

	パイプライン処理をする場合,まず,1個目の命令がStage1.に入るまでから100,000個目の命令がStage1.に入るまで10ナノ秒ずつずれていくわけなので,$\text{$10$ナノ秒$\times 100000 = 1000000$ナノ秒}$掛かる.そのあと,100,000個目の命令は残り4つのステージが残っているわけだから,$\text{$10$ナノ秒$\times 4 = 40$ナノ秒}$.
	
	つまり,合計$\text{$1000000$ナノ秒$ + 40$ナノ秒$ = 1000040$ナノ秒$ = 1.00004$ミリ秒$ \approx 1$ミリ秒}$掛かることになる.つまり,しない場合の約5分の1で完了できる.
\end{tip}

この例題は上の図で,「時間(クロック)」が「時間(ナノ秒)」となり,「$0,\ 10,\ 20,\ \cdots,\ 1000040$」と置き換え,100,000色の四角(命令)があると考えた場合である.

このように,命令実行サイクルを$s$個のステージで構成する命令パイプライン処理機構に$I$個のマシン命令を投入すると,処理時間は$I$の値にかかわらず,パイプライン処理をしない場合の\textbf{約$\bm{s}$分の1}に短縮できる.

\noindent
※例題の5ステージに対し100,000個の命令のように,ステージ数に対し命令数が十分大きくではならない($I \gg s$).でないと処理時間はそんなに変わらない.



\subsection{プロセッサアーキテクチャ② 演算機構}
\vskip-1\baselineskip
\subsubsection{数の表現}

\begin{table}[H]
	\caption{数の表現}
	\label{tab27-2}
	\centering
	\begin{tabular}{c|p{10cm}}
		\hline
		\textbf{整数} & \textbf{固定小数点数表現}\par 小数点は最下位ビットの右に固定.\\
		\hline
		\textbf{実数} & \textbf{浮動小数点数表現}\par 小数点の位置を変えて任意の精度で表現する.小数点も“0”か“1”で表現する.問題として,ハードウェアは有限なので\textbf{循環小数}や\textbf{無理数}の\textbf{丸め誤差}が生れる.\\
		\hline
	\end{tabular}
\end{table}



\subsubsection{演算機構とデータバス}

マシン命令の転送とデータの転送の両方で共用する内部バスに対し,データ専転送だけに使用する演算機構 -- レジスタ間の転送路を\textbf{データバス}という.

演算器(ALU)は,データバスに並列接続する.



\subsubsection{加算器}

四則演算は\textbf{加算}を基にしてできる.

\begin{enumerate}[label={\color{gray}●}, labelsep=10pt, leftmargin=23pt]
	\item \textbf{減算}\qquad 加える数を負の数にする.
	\item \textbf{乗算}\qquad 加算の繰り返し:$a \times b$の場合,$0$を最初の加えられる数にして,$a$を$b$回繰り返し加算する.
	\item \textbf{除算}\qquad 減算の繰り返し:$a \div b$の場合,$a$を最初の引かれる数にして,$b$を繰り返し引いたとき,$b$未満になるまでに何回引いたかが商(計算結果)となる.
\end{enumerate}

加算器で,1個下のビットからの繰り上げも考慮したものを\textbf{全加算器}という.足される数を$X$,足す数を$Y$,1個下のビットからの繰り上げを$C_{\mathrm{in}}$,\kenten{当該ビット}の和を$S$\footnote{例えば,2進数の$(1)_{2} + (1)_{2}$をすると,$(10)_{2}$となる.このとき,$S$は$0$,$C_{\mathrm{out}}$は$1$である.},1個上のビットへの繰り上げを$C_{\mathrm{out}}$とすると,真理値表は以下の通り.

1個下のビットからの繰り上げ出力を当該ビットの$C_{\mathrm{in}}$入力として,32個の全加算器を直列に繫げば「32ビット加算器」が構成出来る.このとき,最下位ビットの繰り上げ$C_{\mathrm{out}_0}$が32個の全加算器を最上位ビットからの繰り上げ$C_{\mathrm{out}_{31}}$まで,次々に伝播することによって累積する遅延が加算時間を決める.

よって,加算時間を減らすため以下の2つの加算器を現代のコンピュータは装備している.

\begin{enumerate}[label=\textbf{[\arabic*]}, labelsep=10pt, leftmargin=23pt]
	\item \textbf{桁上げ先見加算器}\qquad 各ビットでの加算数・被加算数を使って予め和を出しておく.桁上げ先見回路という専用の装置を使う.
	\item \textbf{桁上げ保存加算器}\qquad 部分加算を行い,出た部分和の桁上げ情報を保存する.
\end{enumerate}

\begin{table}[H]
	\caption{全加算器の真理値表}
	\label{tab14-3}
	\centering
	\begin{tabular}{c|c|c||c|c}
		\hline
		$X$ & $Y$ & $C_{\mathrm{in}}$ & $S$ & $C_{\mathrm{out}}$\\
        \hline\hline
        $0$ & $0$ & $0$ & $0$ & $0$\\
        $0$ & $0$ & $1$ & $1$ & $0$\\
        $0$ & $1$ & $0$ & $1$ & $0$\\
        $0$ & $1$ & $1$ & $0$ & $1$\\
        $1$ & $0$ & $0$ & $1$ & $0$\\
        $1$ & $0$ & $1$ & $0$ & $1$\\
        $1$ & $1$ & $0$ & $0$ & $1$\\
        $1$ & $1$ & $1$ & $1$ & $1$\\
		\hline
	\end{tabular}
\end{table}

\begin{figure}[H]
	\begin{center}
		\framebox{
		\includegraphics[width=8cm]{C:/Users/User/Documents/PowerPoint/コンピュータアーキテクチャ/全加算器ブロック図.pdf}
		}
		\caption{全加算器のブロック図}
		\label{fig14-4}
	\end{center}
\end{figure}



\subsubsection{負数の2進数表現 -- 補数表現}

負数を2進数で表すとき,“$+$”,“$-$”符号も“0”,“1”で表す必要がある.負整数の2進数表現として,以下の2通りがある.

\begin{enumerate}[label=\textbf{[\arabic*]}, labelsep=10pt, leftmargin=23pt]
	\item \textbf{1の補数$\bm{\overline{N}}$}\\
		$N$に加えると和が$1\cdots 11$になる数.\\
		1の補数は2進数の正整数の各ビットを反転させる(“0”⇔“1”)と求められる.$m$ビットの2進数を10進数で表したとき,以下の式になる.
		\begin{equation}
			\overline{N} = 2^m - 1 - N
		\end{equation}
	\item \textbf{2の補数$\bm{\overline{\overline{N}}}$}\\
		$m$ビットの$N$に加えると和が$1\underbrace{0\cdots 00}_{\text{$m$ビット}}$になる数.\\
		2の補数は1の補数に$+1$すると求められる.$m$ビットの2進数を10進数で表したとき,以下の式になる.
		\begin{equation}
			\overline{\overline{N}} = 2^m - N = \overline{N} + 1
		\end{equation}
\end{enumerate}

\begin{tip}{例題}
	\textsf{$N = (0111)_{2}$の1の補数$\overline{N}$と2の補数$\overline{\overline{N}}$を求めなさい.}

	\tcblower

	1の補数:
	\begin{align*}
		&\text{機械的にする場合} & N &= (0111)_{2} & \overline{N} &= (1000)_{2}\\
		&\text{10進数に直して考える場合} & N &= (0111)_{2} = (7)_{10} & \therefore \overline{N} &= 2^4 - 1 - 7 = (8)_{10} = (1000)_{2}
	\end{align*}

	2の補数:
	\begin{align*}
		&\text{機械的にする場合} & N &= (0111)_{2} & \overline{\overline{N}} &= (1001)_{2}\\
		&\text{10進数に直して考える場合} & N &= (0111)_{2} = (7)_{10} & \therefore \overline{\overline{N}} &= 2^4 - 7 = (9)_{10} = (1001)_{2}
	\end{align*}
\end{tip}

\newpage


符号無しでは,4ビット$(0000)_{2} ~ (1111)_{2}$の2進数が表現出来る範囲は$(0)_{10} ~ (15)_{10}$である.しかし,補数表現を使うと最上位ビットは符号ビット(“0”:正数,“1”:負数)となり,表現できる範囲は$(-8)_{10} ~ (+7)_{10}$となる.

\begin{table}[H]
	\caption{4ビット2進数の2の補数表現に於ける10進数との対応表}
	\label{tab27-4}
	\centering
	\begin{tabular}{c|cccccccccccccccc}
		\hline
		\textsf{対応する10進数} &
		$-8$ & $-7$ & $-6$ & $-5$ &
		$-4$ & $-3$ & $-2$ & $-1$ &
		$ 0$ & $+1$ & $+2$ & $+3$ &
		$+4$ & $+5$ & $+6$ & $+7$ \\
		\hline
		\textsf{2進数} &
		1000 & 1001 & 1010 & 1011 &
		1100 & 1101 & 1110 & 1111 &
		0000 & 0001 & 0010 & 0011 &
		0100 & 0101 & 0110 & 0111 \\
		\hline
	\end{tabular}
\end{table}



\subsubsection{補数による加算}

\begin{enumerate}[label=\textbf{[\arabic*]}, labelsep=10pt, leftmargin=23pt]
	\item \textsf{1の補数を使うとき}
		\begin{enumerate}[label=\textbf{\arabic*.}, labelsep=10pt, leftmargin=23pt]
			\item 負数があるときは1の補数にする.
			\item 2つの値を加算する.
			\item \textbf{2.}の結果,最上位ビットで繰り上がり(エンドキャリ)が発生したら,\textbf{2.}に$+1$をした後,最上位ビットの繰り上がりは無視する.この補正を\textbf{エンドアラウンドキャリ}(\textbf{循環桁上げ})という.
			\item その結果,最上位ビットが“0”であれば正数,“1”であれば1の補数表現で示された負数である.
		\end{enumerate}
		\begin{tip}{例題}
			\textsf{
				$(-4)_{10} + (-2)_{10}$を1の補数によって加算せよ.
			}

			\tcblower

			\begin{enumerate}[label=\textbf{\arabic*.}, labelsep=10pt, leftmargin=23pt]
				\item どちらも負数なので,1の補数にする.
					\begin{gather*}
						(-4)_{10} = \overline{(0100)_{2}} = (1011)_{2}\\
						(-2)_{10} = \overline{(0010)_{2}} = (1101)_{2}
					\end{gather*}
				\item $(1011)_{2} + (1101)_{2} = ({\color{red}1}\,1000)_{2}$
				\item \textcolor{red}{エンドキャリ}が発生したので,$+1$すると$({\color{red}1}\,1001)_{2}$.繰り上がりそのものは無視して結果は$(1001)_{2}$.
				\item 最上位ビットが“1”なので,これは1の補数で示された負数である.よって,符号を反転させて$(0110)_{2} = (+6)_{10}$なので,答えは$(-6)_{10}$である.
			\end{enumerate}
		\end{tip}
	\item \textsf{2の補数を使うとき}
		\begin{enumerate}[label=\textbf{\arabic*.}, labelsep=10pt, leftmargin=23pt]
			\item 負数があるときは2の補数にする.
			\item 2つの値を加算する.
			\item \textbf{2.}の結果,最上位ビットで繰り上がり(エンドキャリ)が発生したら無視する.
			\item その結果,最上位ビットが“0”であれば正数,“1”であれば1の補数表現で示された負数である.
		\end{enumerate}
		\begin{tip}{例題}
			\textsf{
				$(-4)_{10} + (-2)_{10}$を2の補数によって加算せよ.
			}

			\tcblower

			\begin{enumerate}[label=\textbf{\arabic*.}, labelsep=10pt, leftmargin=23pt]
				\item どちらも負数なので,2の補数にする.
					\begin{gather*}
						(-4)_{10} = \overline{\overline{(0100)_{2}}} = (1100)_{2}\\
						(-2)_{10} = \overline{\overline{(0010)_{2}}} = (1110)_{2}
					\end{gather*}
				\item $(1100)_{2} + (1110)_{2} = ({\color{red}1}\,1010)_{2}$
				\item \textcolor{red}{エンドキャリ}が発生したので無視して結果は$(1010)_{2}$.
				\item 最上位ビットが“1”なので,これは2の補数で示された負数である.よって,符号を反転後$+1$して$(0110)_{2} = (+6)_{10}$なので,答えは$(-6)_{10}$である.
			\end{enumerate}
		\end{tip}
\end{enumerate}

2の補数による加算では1の補数と違って補正は必要無くエンドキャリは無視すればいいだけなので都合がよい.よって,現代のコンピュータでは減算・負数の加算は2の補数で行うのが一般的.



\subsubsection{乗算器と除算器}

\begin{enumerate}[label={\color{gray}●}, labelsep=10pt, leftmargin=23pt]
	\item \textbf{乗算器}を実現させる仕組み
		\begin{enumerate}[label=\textbf{(\arabic*)}, labelsep=10pt, leftmargin=23pt]
			\item ブース法…負の数を2の補数で表現し,部分積を足して実現.
			\item 配列型乗算器(並列乗算器)…部分積を2次元配列状に並べ,全加算器で同時に加算.
			\item ウォリス木…tree構造(根付き木)に部分積を格納.
			\item 早見表…基本乗算の結果を表にして部分積を求める.
		\end{enumerate}
	\item \textbf{除算器}を実現させる仕組み
		\begin{enumerate}[label=\textbf{(\arabic*)}, labelsep=10pt, leftmargin=23pt]
			\item 引き戻し法…部分商を求め,結果$0$なら減算\footnote{除算は減算の繰り返しである.}を無かったことにする.
			\item 引き放し法…部分商が$0$のとき,当該ビットでの減算はそのまま次のビットの操作に移行.
			\item 乗算収束型除算法
			\item 部分商の同時生成
		\end{enumerate}
\end{enumerate}



\subsubsection{実数の表現}

実数$R$は\textbf{浮動小数点数表現}を用いる(表\ref{tab27-2}).小数点の左側$m$を\textbf{仮数部},右側$e$を\textbf{指数部}といい,式では以下のように表す.
\begin{equation}
	R = m \times 2^e \qquad \text{($m$も$e$もコンピュータ内部では2進数)}
\end{equation}
$m$や$e$が負数であれば補数表現で表される.このとき,$m$が負であれば$R$は負となる.

$m$は\textbf{有効数字}ともいい,$e$は小数点の位置決めをしている.よって,$e$は$m$が整数となるように調整している.言い換えれば,実数を2つの整数$m,\ e$によって表現する.

\begin{tip}{例題}
	\textsf{
		2進実数$(10100)_{2},\ (11.011)_{2},\ (0.1111)_{2}$を浮動小数点数表現せよ.
	}
	
	\tcblower

	\begin{gather*}
		(10100)_{2} = (101)_{2} \times 2^{(10)_{2}}\\
		(11.011)_{2} = (11011)_{2} \times 2^{(-11)_{2}}\\
		(0.1111)_{2} = (1111)_{2} \times 2^{(-100)_{2}}
	\end{gather*}
	※ここでは,負数を絶対値に“$-$”符号を付けることによって表しているが,実際のコンピュータ内部では補数表現されている.
\end{tip}

$R_1 = m_1 \times 2^{e_1}$と$R_2 = m_2 \times 2^{e_2}$の積$R_1 \times R_2$は次のように計算できる.
\begin{equation}
	R_1 \times R_2 = (m_1 \times m_2) \times 2^{(e_1 + e_2)}
\end{equation}

よって,浮動小数点数用の演算器では,乗算器と加算器を同時に行う工夫をしている.



\subsubsection{論理演算器とシフト演算器}

\textbf{論理演算器}では,ビットの1つ1つが論理値であるので並列で論理演算できればよい.対して,\textbf{シフト演算器}では,上下位のビットを隣り合うビットに移動する操作を行うので論理回路が必要である.



\subsubsection{演算機構のアーキテクチャ}

演算器(ALU)は
\begin{enumerate}[label=\textbf{[\arabic*]}, labelsep=10pt, leftmargin=23pt]
	\item 補数器
	\item 整数の加算器
	\item 整数の乗算器
	\item 整数の除算器
	\item 浮動小数点数の加算器
	\item 浮動小数点数の乗算器
	\item 浮動小数点数の除算器
	\item 論理演算器とシフト演算器シフト演算器
	\item 2進数→10進数変換機構
	\item 10進数→2進数変換機構
\end{enumerate}
で構成される.



\subsection{メモリアーキテクチャ}
\vskip-\baselineskip
\subsubsection{メインメモリ(内部メモリ)}

\textbf{メインメモリ(内部メモリ)}は,プロセッサと共にコンピュータの内部装置を構成するものである.メインメモリとプロセッサで,「プロセッサで処理するマシン命令とデータを予めメインメモリに用意しておく」という\textbf{プログラム内蔵}の実装を実現している.

メインメモリは以下の機能を備え,その機能に求められる評価がある.

\begin{table}[H]
	\caption{メインメモリに求められる機能と評価}
	\label{tab27-5}
	\centering
	\begin{tabular}{c|c}
		\hline
		\textsf{機能} & \textsf{評価}\\
		\hline
		格納 & 容量が大きいほど嬉しい\\
		アクセス & アクセスが速いほど嬉しい\\
		\hline
	\end{tabular}
\end{table}

現代のコンピュータのメインメモリは
\begin{enumerate}[label=\textbf{[\arabic*]}, labelsep=10pt, leftmargin=23pt]
	\item \textbf{ランダムアクセス}\qquad アドレスと格納場所が1対1であり,一定時間でアクセスできる.謂わば,アドレスを指定すれば計算量$O(1)$で即アクセスできる(配列に近いイメージ).
	\item \textbf{線形アドレス}\qquad メモリ空間を決まった空間で区切り,連番のアドレスを付ける.つまり,アドレスを$+1$ずつ増加させるだけで順番にアクセスできる.
\end{enumerate}
の2つの要件を満たさなければならない.



\subsubsection{メインメモリへのアクセス}

プロセッサがメインメモリにアクセスするときは,アドレスを指定し,読み出しと書き込みのどちらのアクセスかを指定する必要がある.その為に次の2つの機構をプロセッサ側に備える.
\begin{enumerate}[label=\textbf{[\arabic*]}, labelsep=10pt, leftmargin=23pt]
	\item \textbf{メモリアドレスレジスタ(MAR)}…メインメモリアドレスをアクセスする間保持しておく.
	\item \textbf{メモリデータレジスタ(MDR)}……アクセスするマシン命令やデータをアクセスする間保持しておく.
\end{enumerate}

プロセッサ内のMAR,MDRを備える機構を\textbf{メインメモリ管理機構(MMU)}という.

\begin{figure}[H]
	\begin{center}
		\framebox{
		\includegraphics[width=8cm]{C:/Users/User/Documents/PowerPoint/コンピュータアーキテクチャ/MMU.pdf}
		}
		\caption{プロセッサによるメインメモリへのアクセス}
		\label{fig14-5}
	\end{center}
\end{figure}



\subsubsection{メモリの階層構造}


メモリ素子とは,“0”か“1”を格納出来るメモリの最小単位(1ビット)である.メモリ素子は以下のハードウェアで構成される.
\begin{enumerate}[label=\textbf{[\arabic*]}, labelsep=10pt, leftmargin=23pt]
	\item \textbf{半導体}\hspace*{4\zw} 電流のON/OFFを切り替える.
	\item \textbf{磁性体}\hspace*{4\zw} 磁場の向きで“0”と“1”を記録する.
	\item \textbf{コンデンサ}\hspace*{2\zw} 電荷を蓄える.
\end{enumerate}

メモリ素子によって表\ref{tab27-5}の評価が決まるが,容量が大きいこととアクセス時間が短いことは\textbf{両立しない}.つまり,どちらの性能も高いメモリは実在しない.結果として,適材適所を図ることが必要.

\begin{figure}[H]
	\begin{center}
		\framebox{
		\includegraphics[width=5cm]{C:/Users/User/Documents/PowerPoint/コンピュータアーキテクチャ/メモリ階層.pdf}
		}
		\caption{主要なメモリ階層}
		\label{fig14-6}
	\end{center}
\end{figure}



\subsubsection{参照局所性}

実行中のプログラムがアクセス又は参照するマシン命令やデータの格納場所(アドレス)が一部或いは特定の場所に集中することを「\textbf{参照局所性}が高い」という.空間的と時間的の2種類がある.
\begin{enumerate}[label=\textbf{[\arabic*]}, labelsep=10pt, leftmargin=23pt]
	\item 「空間的参照局所性が高い」とは,一度アクセスしたアドレスの“格納場所に近いアドレス”が近いうちにアクセスする可能性が高いということ.
	\item 「時間的参照局所性が高い」とは,一度アクセスしたアドレス“そのもの”が近いうちにアクセスする可能性が高いということ.
\end{enumerate}

従って,参照局所性が高いプログラムを上の階層におけば,よりプロセッサに近くなり時間的・空間的に効率よく処理出来るようになる.



\subsubsection{仮想メモリ}

メインメモリには,プロセッサが実行しようとするプログラムやデータを格納しておく,つまりプログラム内蔵である必要がある.しかし,プログラムが巨大で一時的にメインメモリに入らないとき,取り敢えずそれ以外のメモリに格納しておいて必要なときに必要なプログラムをメインメモリに持ってこなくてはいけない.もしくは,複数のプログラムを同時に行うとき,個々のサイズは小さくても数が多過ぎてメインメモリに入りきらないときがある.

このとき,ユーザが作成したプログラムをファイルとして\textbf{ファイル装置}に格納しておき,使用する可能性があるときにメインメモリに転送,そしてプロセッサが実行するときにアクセスするようにする.

このように,使用する可能性が高い=参照局所性が高いファイルだけを置いて残りはファイル装置に置くことで「プロセッサから見えるメインメモリの容量を見かけ上大きくする機能」を\textbf{仮想メモリ}方式という.





\end{document}