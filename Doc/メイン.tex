\documentclass[
	report, paper=a4, head_space=23mm, foot_space=13mm,
	fontsize=9pt, jafontscale=0.9,
	gutter=20mm, line_length=170mm,
	fleqn, twoside
]{jlreq}
\def\pdfpagewidth{\paperwidth}
\def\pdfpageheight{\paperheight}
%%%%%%%%%%%%%%%%%%%%%%%%%%%%%%%%%%%%%%%%%%%%%%%%%%
% report  トップレベル=\chapter
% paper a4に設定
% 余白の設定
%   head_space  天の空き量
%   foot_space  地の空き量
%   gutter   小口(外側)の空き量
% fleqn 別行立て数式を左側に
%%%%%%%%%%%%%%%%%%%%%%%%%%%%%%%%%%%%%%%%%%%%%%%%%%
% todayの再設定
\makeatletter
\newcommand*{\themonth}{\two@digits\month}
\newcommand*{\theday}{\two@digits\day}
\makeatother
\renewcommand{\today}{{\the\year}年{\themonth}月{\theday}日}




%%%%%%%%%%%%%%%%%%%%%%%%%%%%%%%%%%%%%%%%%%%%%%%%%%
%%%%%%%%%%                              %%%%%%%%%%
%%%%%%%%%%           文章関係           %%%%%%%%%%
%%%%%%%%%%                              %%%%%%%%%%
%%%%%%%%%%%%%%%%%%%%%%%%%%%%%%%%%%%%%%%%%%%%%%%%%%
% フォント関係
\usepackage{luatexja-fontspec}
\setmainjfont{ChikuPori-Regular2.ttf}[BoldFont=KozGoPro-Medium.otf]
\setsansjfont{KozGoPro-Medium.otf}[BoldFont=KozGoPro-Bold.otf]
\usepackage{fontspec}
\setsansfont{QTEurotype}
% \setsansfont{Arial}
%%%%%%%%%%%%%%%%%%%%%%%%%%%%%%%%%%%%%%%%%%%%%%%%%%
% ノンブル、柱の設定
\ModifyPageStyle{headings}{
	% 柱
	running_head_position=top-center,
	% 奇数ページの柱内容
	odd_running_head={_chapter},
	% 偶数ページの柱内容
	even_running_head={_section},
	% 柱,ノンブル
	nombre={\thechapter-\thepage},
	nombre_position=top-right,
	nombre_font=\sffamily\bfseries,
}
\pagestyle{headings}
%%%%%%%%%%%%%%%%%%%%%%%%%%%%%%%%%%%%%%%%%%%%%%%%%%
% plain ヘッダーには何も表示せず,フッターにノンブル
%%%%%%%%%%%%%%%%%%%%%%%%%%%%%%%%%%%%%%%%%%%%%%%%%%
% 色関係
\usepackage{graphicx}
\usepackage[dvipsnames]{xcolor}
%%%%%%%%%%%%%%%%%%%%%%%%%%%%%%%%%%%%%%%%%%%%%%%%%%
% xcolor    dvipsnamesをつけるといろいろな名前が使える:
% https://mathlandscape.com/latex-color/#toc3
%%%%%%%%%%%%%%%%%%%%%%%%%%%%%%%%%%%%%%%%%%%%%%%%%%
% 目次
\usepackage[colorlinks,linkcolor=black,urlcolor=black]{hyperref}
\setcounter{tocdepth}{2}
%%%%%%%%%%%%%%%%%%%%%%%%%%%%%%%%%%%%%%%%%%%%%%%%%%
% hyperref  目次リンク
%%%%%%%%%%%%%%%%%%%%%%%%%%%%%%%%%%%%%%%%%%%%%%%%%%
\usepackage{multicol, here, multirow, fancybox, array, incgraph, wrapfig, caption, paralist, ascmac}
%%%%%%%%%%%%%%%%%%%%%%%%%%%%%%%%%%%%%%%%%%%%%%%%%%
% 段組み
% here  図や表のH
% multirow  縦結合
% fancybox いい感じのbox
% array 表の罫線を微妙に改善
% incgraph	表紙
% wrapfig	図の回り込み
% paralist	文中のリスト
% ascmac	別行box
%%%%%%%%%%%%%%%%%%%%%%%%%%%%%%%%%%%%%%%%%%%%%%%%%%
% 四角番号
\newcommand{\f}[1]{\fbox{\textbf{#1}}}
% \renewcommand{\labelenumi}{\f{\arabic{enumi}}}
% \renewcommand{\labelenumii}{(\arabic{enumii})}
%%%%%%%%%%%%%%%%%%%%%%%%%%%%%%%%%%%%%%%%%%%%%%%%%%
% enumerateでのデフォルト \f{1},\f{2},...
% enumerateでの第2階層のデフォルト (1),(2),...
%%%%%%%%%%%%%%%%%%%%%%%%%%%%%%%%%%%%%%%%%%%%%%%%%%
\usepackage{tasks}
\settasks{
	label=(\arabic*), label-align=right, 
	item-indent=6pt, label-offset=6pt,
	column-sep=15pt
}
%%%%%%%%%%%%%%%%%%%%%%%%%%%%%%%%%%%%%%%%%%%%%%%%%%
% enumitem  いろいろな設定ができる(emathを先に読み込むので97行目に移動).
% task  横箇条書き
%%%%%%%%%%%%%%%%%%%%%%%%%%%%%%%%%%%%%%%%%%%%%%%%%%
% カウンタの定義
\newcounter{daimoncounter}
% \toiと打つと,カウンタが1つ増えて,その数が表示される.
\newcommand{\daimon}{\refstepcounter{daimoncounter}\thedaimoncounter}
\newcommand{\toi}{\f{\daimon}}
%%%%%%%%%%%%%%%%%%%%%%%%%%%%%%%%%%%%%%%%%%%%%%%%%%
% 選択問題の四角
% 枠の大きさ設定
\newcommand{\senntakuanaume}[1]{\setlength{\fboxsep}{1pt}\setlength{\fboxrule}{0.6pt} \doublebox{\mbox{\textbf{\phantom{ア}#1\phantom{ア}}}} \setlength{\fboxrule}{0.4pt}} %\senntakuanaumeで選択肢問題の穴埋め
%%%%%%%%%%%%%%%%%%%%%%%%%%%%%%%%%%%%%%%%%%%%%%%%%%
%%%%%%%%%%%%%%%%%%%%%%%%%%%%%%%%%%%%%%%%%%%%%%%%%%





%%%%%%%%%%%%%%%%%%%%%%%%%%%%%%%%%%%%%%%%%%%%%%%%%%
%%%%%%%%%%                              %%%%%%%%%%
%%%%%%%%%%           数式関係           %%%%%%%%%%
%%%%%%%%%%                              %%%%%%%%%%
%%%%%%%%%%%%%%%%%%%%%%%%%%%%%%%%%%%%%%%%%%%%%%%%%%
\usepackage{amsmath, emath, mathtools, extarrows, cancel, ulem, enumitem, mleftright, emathMw, siunitx, bm}
\mleftright
%%%%%%%%%%%%%%%%%%%%%%%%%%%%%%%%%%%%%%%%%%%%%%%%%%
% amsmath   いろいろな機能を追加.
% emath     日本に合わせたいろいろな機能を追加.
% mathtools いろいろな機能を追加.
% extarrows 
% cancel    数式に斜線
% mleftright 括弧の空き
% emathMw	enumerate環境下での図の回り込み
% siunitx	角度\ang{}
%%%%%%%%%%%%%%%%%%%%%%%%%%%%%%%%%%%%%%%%%%%%%%%%%%
\renewcommand{\tagform}[1]{(#1)}%
\preEqlabel{}%数式番号デフォルト
%%%%%%%%%%%%%%%%%%%%%%%%%%%%%%%%%%%%%%%%%%%%%%%%%%
% emathを入れたことにより,デフォルトが①②になったので
%%%%%%%%%%%%%%%%%%%%%%%%%%%%%%%%%%%%%%%%%%%%%%%%%%
\renewcommand{\dint}{\displaystyle\int}
\newcommand{\diint}{\displaystyle\iint}
\newcommand{\dsum}{\displaystyle\sum}
\renewcommand{\dlim}{\lim\limits}
%%%%%%%%%%%%%%%%%%%%%%%%%%%%%%%%%%%%%%%%%%%%%%%%%%
% 積分記号,和,極限(常にdisplaystyle)
%%%%%%%%%%%%%%%%%%%%%%%%%%%%%%%%%%%%%%%%%%%%%%%%%%
\usepackage[e]{esvect}
\renewcommand{\vec}[1]{\hspace*{-0.5pt}\vv{\mathstrut #1}\hspace*{-0.5pt}}
\newcommand{\vecrm}[1]{\vv{\mathrm{\mathstrut #1}}}
%%%%%%%%%%%%%%%%%%%%%%%%%%%%%%%%%%%%%%%%%%%%%%%%%%
% ベクトルの定義しなおし
% 始点と終点を示すベクトルは \vecrm を使用.
%%%%%%%%%%%%%%%%%%%%%%%%%%%%%%%%%%%%%%%%%%%%%%%%%%
\makeatletter

\def\@myfrac@d#1#2{%
	\displaystyle\frac{%
		\raisebox{-.44ex}{$\,#1\,$}%分子
	}{%
		\raisebox{.1ex}{$\,#2\,$}%分母
	}
}
\def\@myfrac@t#1#2{%
	\textstyle\frac{%
		\raisebox{-.04ex}{\scalebox{0.9}{$\,#1\,$}}%分子
	}{%
		\raisebox{-.3ex}{\scalebox{0.9}{$\,#2\,$}}%分母
	}
}
\def\@myfrac@s#1#2{\hspace{-.5pt}%
	\scriptstyle\frac{%
		\raisebox{0.1ex}{\scalebox{0.6}{$\,#1\,$}}%分子
	}{%
		\raisebox{0.1ex}{\scalebox{0.6}{$\,#2\,$}}%分母
	}
\hspace{-.5pt}}
\def\@myfrac@ss#1#2{%
	\scriptscriptstyle#1/#2
}
\def\myfrac#1#2{
	\mathchoice{\@myfrac@d{#1}{#2}}{\@myfrac@t{#1}{#2}}{\@myfrac@s{#1}{#2}}{\@myfrac@ss{#1}{#2}}
}
\makeatother
%%%%%%%%%%%%%%%%%%%%%%%%%%%%%%%%%%%%%%%%%%%%%%%%%%
% \myfrac{分子}{分母}   textstyleとscriptsizeの分数を定義.
% displaystyle ---> displaystyle
% textstyle ---> displaystyleの0.9倍
% scriptstyle ---> displaystyleの0.6倍
% scriptscriptstyle ---> 分子/分母の形
%%%%%%%%%%%%%%%%%%%%%%%%%%%%%%%%%%%%%%%%%%%%%%%%%%
\DeclareMathOperator{\cosec}{cosec}
%%%%%%%%%%%%%%%%%%%%%%%%%%%%%%%%%%%%%%%%%%%%%%%%%%
% 三角関数
%%%%%%%%%%%%%%%%%%%%%%%%%%%%%%%%%%%%%%%%%%%%%%%%%%
\DeclareMathOperator{\csin}{Sin}
\DeclareMathOperator{\Arcsin}{Arcsin}
\DeclareMathOperator{\ccos}{Cos}
\DeclareMathOperator{\Arccos}{Arccos}
\DeclareMathOperator{\ctan}{Tan}
\DeclareMathOperator{\Arctan}{Arctan}
\DeclareMathOperator{\arcsec}{arcsec}
\DeclareMathOperator{\arccsc}{arccsc}
\DeclareMathOperator{\arccot}{arccot}
%%%%%%%%%%%%%%%%%%%%%%%%%%%%%%%%%%%%%%%%%%%%%%%%%%
% 逆三角関数
% cはCapitalの略
%%%%%%%%%%%%%%%%%%%%%%%%%%%%%%%%%%%%%%%%%%%%%%%%%%
\DeclareMathOperator{\sech}{sech}
\DeclareMathOperator{\csch}{csch}
%%%%%%%%%%%%%%%%%%%%%%%%%%%%%%%%%%%%%%%%%%%%%%%%%%
% 双曲線関数
%%%%%%%%%%%%%%%%%%%%%%%%%%%%%%%%%%%%%%%%%%%%%%%%%%
\DeclareMathOperator{\arsinh}{arsinh}
\DeclareMathOperator{\arcosh}{arcosh}
\DeclareMathOperator{\artanh}{artanh}
\DeclareMathOperator{\arsech}{arsech}
\DeclareMathOperator{\arcsch}{arcsch}
\DeclareMathOperator{\arcoth}{arcoth}
%%%%%%%%%%%%%%%%%%%%%%%%%%%%%%%%%%%%%%%%%%%%%%%%%%
% 逆双曲線関数
%%%%%%%%%%%%%%%%%%%%%%%%%%%%%%%%%%%%%%%%%%%%%%%%%%
\DeclareMathOperator{\grad}{grad}
\DeclareMathOperator{\dive}{div}
\DeclareMathOperator{\rot}{rot}
\DeclareMathOperator{\curl}{curl}
%%%%%%%%%%%%%%%%%%%%%%%%%%%%%%%%%%%%%%%%%%%%%%%%%%
% 勾配
%%%%%%%%%%%%%%%%%%%%%%%%%%%%%%%%%%%%%%%%%%%%%%%%%%




%%%%%%%%%%%%%%%%%%%%%%%%%%%%%%%%%%%%%%%%%%%%%%%%%%
%%%%%%%%%%                              %%%%%%%%%%
%%%%%%%%%%           TikZ           %%%%%%%%%%
%%%%%%%%%%                              %%%%%%%%%%
%%%%%%%%%%%%%%%%%%%%%%%%%%%%%%%%%%%%%%%%%%%%%%%%%%
\usepackage{pgfplots, tikz, tikz-3dplot}
\usetikzlibrary{positioning, intersections, calc, arrows.meta, fadings, patterns, lindenmayersystems} %tikzのlibrary
\usepackage{tcolorbox}
\tcbuselibrary{theorems, breakable, raster, skins}

\newtcolorbox{kousiki}[2][]{
	% tikzを用いた記法の処理
	enhanced, %
	% box内左右の余白
	left = 12pt, right = 12pt, %
	% タイトルのフォント指定
	fonttitle = \bfseries\large, %
	% タイトルの文字の色
	coltitle = white, %
	% タイトルの背景の色
	colbacktitle = black, %
	% タイトルを左寄せに、少し微調整
	attach boxed title to top left={}, %
	% タイトルボックスの装飾
	boxed title style = {skin = enhancedfirst jigsaw, arc = 1mm, bottom = 0mm, boxrule = 0mm}, %
	% 枠線の太さ
	boxrule = 0.5pt, %
	% 本文の背景色
	colback = black!5!, %
	% 本文の枠の色
	colframe = black, %
	% 左上の角の調整
	sharp corners = northwest, % 
	% 影をつける
	drop fuzzy shadow, %
	% ページマタギOK
	breakable, %
	% タイトルは直接入力
	title = \vspace{3mm}#2, 
	% 弧
	arc = 1mm, %
	#1
}%


%%%%%%%%%%%%%%%%%%%%%%%%%%%%%%%%%%%%%%%%%%%%%%%%%%
%%%%%%%%%%                              %%%%%%%%%%
%%%%%%%%%%           デザイン           %%%%%%%%%%
%%%%%%%%%%                              %%%%%%%%%%
%%%%%%%%%%%%%%%%%%%%%%%%%%%%%%%%%%%%%%%%%%%%%%%%%%
% 章
\ModifyHeading{chapter}{
	font={\huge\jfontspec{KozGoPro-Bold.otf}\bfseries},
	format={%
		{\color[gray]{0.5}\rule[-0.5\zw]{2\zw}{1.8\zw}}%
		\if@mainmatter
			\hspace{-2\zw}\raisebox{0.1\zw}{%
				\makebox[2\zw]{\color[gray]{1}\thechapter}}%
		\fi
		\hspace{0.5\zw}#1#2
	}
}
% 節
\ModifyHeading{section}{%
	format={%
		\hrule\par\vspace*{0.25ex}\parbox[b]{\linewidth - 1\zw}{\hspace*{-1\zw}\colorbox{black}{\hspace*{1\zw}\textcolor{white}{#1}}\quad#2}\vspace*{0.25ex}\par\hrule}
}
%
% 小節
\renewcommand{\thesubsection}{\Alph{subsection}}
\ModifyHeading{subsection}{%
	format={%
		{\color{gray}◆}#1{\color{gray}\hspace*{-0.7\zw}◆}\quad#2
	}
}
%%%%%%%%%%%%%%%%%%%%%%%%%%%%%%%%%%%%%%%%%%%%%%%%%%
% 章,節のデザインを変更 \MakeUppercase{#1}ですべて大文字   
%%%%%%%%%%%%%%%%%%%%%%%%%%%%%%%%%%%%%%%%%%%%%%%%%%



\begin{document}
\title{\fontsize{40pt}{1\zh} \jfontspec{DelaGothicOne-Regular.ttf}数学・物理 回憶錄}
\date{\LARGE 最終更新日: \today}
\maketitle



% \chapter{偏微分法}
\setcounter{page}{1}
\section{関数の極限}

関数$f(x,\ y)$に於いて,点$(x,\ y)$が点$(a,\ b)$以外の点を取りながら$(a,\ b)$に限りなく近づくとき,関数の値が$C$に限りなく近づくならば,$f(x,\ y)$は$C$に\textbf{収束する}といい,
\begin{equation}
	\lim_{(x,\ y) \to (a,\ b)} f(x,\ y) = C
\end{equation}
と表す.$C$を\textbf{極限値}という.

このとき,$(x,\ y)$がどんな近づき方で$(a,\ b)$に近づいても極限値がある一定の値$C$になることが必要である.

例えば,$f(x,\ y) = \bunsuu{xy}{x^2 + y^2}$について,$\dlim_{(x,\ y) \to (0,\ 0)} f(x,\ y)$を考える.
\begin{enumerate}[label=(\roman*), labelsep=10pt, leftmargin=23pt]
	\item 点を直線$y = x$上で近づけると$f(x,\ y) = \bunsuu{x \cdot x}{x^2 + x^2} = \bunsuu{1}{2}$であるから$\bunsuu{1}{2}$に収束する.
	\item 点を直線$y = 2x$上で近づけると$f(x,\ y) = \bunsuu{x \cdot 2x}{x^2 + (2x)^2} = \bunsuu{2}{5}$であるから$\bunsuu{2}{5}$に収束する.
\end{enumerate}
よって,極限値はない.

関数$f(x,\ y)$の定義域内の点$\mathrm{P}(a,\ b)$について,
\begin{equation}
	\lim_{(x,\ y) \to (a,\ b)} f(x,\ y) = f(a,\ b)
\end{equation}
が成り立つとき,$f(x,\ y)$は点$\mathrm{P}$で\textbf{連続である}という.


\section{偏導関数}

関数$z = f(x,\ y)$に於いて
\begin{equation}
	\bunsuu{\partial f}{\partial x} = \lim_{\varDelta x \to 0} \bunsuu{f(x + \varDelta x,\ y) - f(x,\ y)}{\varDelta x}
\end{equation}
を$f(x,\ y)$の\textbf{$x$についての偏導関数}といい,$f_x(x,\ y)$とも表す.また,
\begin{equation}
	\bunsuu{\partial f}{\partial y} = \lim_{\varDelta y \to 0} \bunsuu{f(x,\ y + \varDelta y) - f(x,\ y)}{\varDelta y}
\end{equation}
を$f(x,\ y)$の\textbf{$y$についての偏導関数}といい,$f_y(x,\ y)$とも表す.

偏微分係数$f_x(a,\ b),\ f_y(a,\ b)$はそれぞれ点$(a,\ b)$の$x$軸方向の傾き,$y$軸方向の傾きを表す.

% \chapter{ベクトル解析}
\setcounter{page}{1}



\section{ベクトル関数の微分法}
\subsection{ベクトル関数の極限と連続}

$t_0$を定数,$\bm{c} = 
\begin{bmatrix}
	c_1\\ c_2\\ c_3
\end{bmatrix}
$を定ベクトルとする.$t$をスカラー変数とするベクトル関数$\bm{f}(t) = 
\begin{bmatrix}
	f_1(t)\\ f_2(t)\\ f_3(t)
\end{bmatrix}
$について,$t$が$t_0$に限りなく近づくときの$\bm{f}(t)$の\textbf{極限}は次で定義される.
\begin{align}
	\lim_{t \to t_0} |\bm{f}(t) - \bm{c}| = 0 &\iff \lim_{t \to t_0} \bm{f}(t) = \bm{c}\\
	&\iff \lim_{t \to t_0} f_1(t) = c_1,\quad \lim_{t \to t_0} f_2(t) = c_2,\quad \lim_{t \to t_0} f_3(t) = c_3
\end{align}

また,$\dlim_{t \to t_0} \bm{f}(t) = \bm{f}(t_0)$が成立するとき,$\bm{f}(t)$は$t = t_0$で\textbf{連続である}という.



\subsection{ベクトル関数の微分法}

ベクトル関数$\bm{f}(t)$に於いて,次式を$\bm{f}(t)$の\textbf{導関数}という.
\begin{align}
	\bunsuu{d\bm{f}(t)}{dt} &= \lim_{\varDelta t \to 0} \bunsuu{\bm{f}(t + \varDelta t) - \bm{f}(t)}{\varDelta t} =
	{
	\begin{bmatrix}
		\bunsuu{df_1(t)}{dt} &
		\bunsuu{df_2(t)}{dt} &
		\bunsuu{df_3(t)}{dt}
	\end{bmatrix}
	}^\top
\end{align}

\begin{kousiki}{微分法の公式}
	$\bm{f}(t),\ \bm{g}(t)$をベクトル関数,$\varphi(t)$をスカラー関数,$\bm{c}$を定ベクトルとする.
	\begin{enumerate}[label=\textbf{[\arabic*]}, labelsep=10pt, leftmargin=23pt, itemsep=6pt]
		\item $\bunsuu{d\bm{c}}{dt} = \bm{0}$
		\item $\bunsuu{d(\bm{f} \pm \bm{g})}{dt} = \bunsuu{d\bm{f}}{dt} \pm \bunsuu{d\bm{g}}{dt}$\hfill(複号同順)
		\item $\bunsuu{d(\varphi\bm{f})}{dt} = \bunsuu{d\varphi}{dt}\bm{f} + \varphi\bunsuu{d\bm{f}}{dt}$
		\item $\bunsuu{d(\bm{f} \cdot \bm{g})}{dt} = \bunsuu{d\bm{f}}{dt} \cdot \bm{g} + \bm{f} \cdot \bunsuu{d\bm{g}}{dt}$
		\item $\bunsuu{d(\bm{f} \times \bm{g})}{dt} = \bunsuu{d\bm{f}}{dt} \times \bm{g} + \bm{f} \times \bunsuu{d\bm{g}}{dt}$
		\item $\bunsuu{d}{dt}\left(\bunsuu{\bm{f}}{\varphi}\right) = \bunsuu{\bunsuu{d\bm{f}}{dt}\varphi - \bm{f}\bunsuu{d\varphi}{dt}}{\varphi^2}$\qquad$(\varphi \ne 0)$
		\item $t = \psi(u)$をスカラー関数とすると \qquad $\bunsuu{d\bm{f}}{du} = \bunsuu{d\bm{f}}{dt}\bunsuu{d\psi}{du}$\hfill(合成関数の微分法)
	\end{enumerate}
\end{kousiki}



\subsection{接線ベクトル}

点$\mathrm{P}$の位置ベクトルが$\bm{r} =
\begin{bmatrix}
	x(t)\\ y(t)\\ z(t)
\end{bmatrix}
$のようにベクトル関数であるとき,$t$が変化するにつれて$\mathrm{P}$はある曲線$C$を描く.この$C$を$\bm{r} = \bm{r}(t)$の表す曲線という.

以下,特に断りがない限り,$\bm{r}(t)$は何度でも微分可能で$\bunsuu{d\bm{r}}{dt} \ne \bm{0}$とする.

$t$の微小変化$\varDelta t$に対応する$\bm{r}(t)$の変化を$\varDelta \bm{r}$とすると$\varDelta \bm{r} = \bm{r}(t + \varDelta t) - \bm{r}(t)$である.
\begin{equation}
	\bunsuu{d\bm{r}}{dt} = \lim_{\varDelta t \to 0} \bunsuu{\varDelta \bm{r}}{\varDelta t} = \lim_{\varDelta t \to 0} \bunsuu{\bm{r}(t + \varDelta t) - \bm{r}(t)}{\varDelta t}
\end{equation}
を曲線$C$の点$\mathrm{P}$における\textbf{接線ベクトル}という.

\begin{enumerate}[leftmargin=18pt, labelsep=10pt, labelsep=10pt, itemindent=9pt]
	\item[\f{例}] $\bm{r}(t) =
		\begin{bmatrix}
			t^2\\ t^3\\ t^4
		\end{bmatrix}
		$ならば,接線ベクトルは$\bunsuu{d}{dt}\bm{r}(t) =
		\begin{bmatrix}
			2t\\ 3t^2\\ 4t^3
		\end{bmatrix}
		$である.特に,$t = 1$の位置$\bm{r}(1) =
		\begin{bmatrix}
			1\\ 1\\ 1
		\end{bmatrix}
		$における接線ベクトルは$\bunsuu{d}{dt}\bm{r}(1) =
		\begin{bmatrix}
			2\\ 3\\ 4
		\end{bmatrix}
		$である.
\end{enumerate}



\subsection{単位接線ベクトル}

接線ベクトルを自身の大きさで割った,大きさ$1$の接線ベクトルを\textbf{単位接線ベクトル}$\bm{t}$という.
\begin{equation}
	\bm{t} = \bunsuu{\bunsuu{d\bm{r}}{dt}}{\left|\bunsuu{d\bm{r}}{dt}\right|} \label{equ:vec_cal-1}
\end{equation}

曲線上のある定点$\mathrm{A}$を基準として,そこからの長さ$s$によって位置ベクトル$\bm{r}(s)$を定義する方法をとる.$\mathrm{A,\ P}$の位置ベクトルをそれぞれ$\bm{r}(\alpha),\ \bm{r}(t)$とすると,$\mathrm{AP}$の長さ$s$が次のようになるので,$\bunsuu{ds}{dt}$が求められる.
\begin{gather}
	s = s(t) = \int_{\alpha}^{t} \sqrt{\left(\bunsuu{dx}{dt}\right)^2 + \Bigl(\bunsuu{dy}{dt}\Bigr)^2 + \left(\bunsuu{dz}{dt}\right)^2}\,dt\\
	\bunsuu{ds}{dt} = \sqrt{\left(\bunsuu{dx}{dt}\right)^2 + \Bigl(\bunsuu{dy}{dt}\Bigr)^2 + \left(\bunsuu{dz}{dt}\right)^2} = \sqrt{\bunsuu{d\bm{r}}{dt} \cdot \bunsuu{d\bm{r}}{dt}} = \left|\bunsuu{d\bm{r}}{dt}\right|
\end{gather}

よって,$\bunsuu{d\bm{r}}{ds}$は合成関数の微分法より
\begin{equation}
	\bunsuu{d\bm{r}}{ds} = \bunsuu{d\bm{r}}{dt}\bunsuu{dt}{ds} = \bunsuu{\bunsuu{d\bm{r}}{dt}}{\bunsuu{ds}{dt}} = \bunsuu{\bunsuu{d\bm{r}}{dt}}{\left|\bunsuu{d\bm{r}}{dt}\right|}
\end{equation}
これは式(\ref{equ:vec_cal-1})と同じ式である.よって次が成り立つ.

\begin{kousiki}{単位接線ベクトル}
	\begin{equation}
		\bm{t} = \bunsuu{\bunsuu{d\bm{r}}{dt}}{\left|\bunsuu{d\bm{r}}{dt}\right|} = \bunsuu{d\bm{r}}{ds}
	\end{equation}
\end{kousiki}

一方,$\bm{t} \cdot \bm{t} = |\bm{t}|^2 = 1$の両辺を$t$で微分すると,内積の微分公式より,$2\bunsuu{d\bm{t}}{dt} \cdot \bm{t} = 0$なので,$\bunsuu{d\bm{t}}{dt} \cdot \bm{t} = 0$である.つまり,$\bunsuu{d\bm{t}}{dt} \perp \bm{t}$が成立する$\left(\text{$\bunsuu{d\bm{t}}{dt} \ne \bm{0}$ならば}\right)$.よって,次を\textbf{単位主法線ベクトル}とし,$\bm{n}$で表すと
\begin{equation}
	\bm{n} = \bunsuu{\bunsuu{d\bm{t}}{dt}}{\left|\bunsuu{d\bm{t}}{dt}\right|}
\end{equation}
となる.

\begin{kousiki}{単位主法線ベクトル}
	\begin{equation}
		\bm{n} = \bunsuu{\bunsuu{d\bm{t}}{dt}}{\left|\bunsuu{d\bm{t}}{dt}\right|} = \bunsuu{\bunsuu{d\bm{t}}{ds}}{\left|\bunsuu{d\bm{t}}{ds}\right|}
	\end{equation}
\end{kousiki}



\section{2変数ベクトル関数の微分法}
\subsection{2変数ベクトル関数の極限と連続}

$u_0,\ v_0$を定数,$\bm{c} = 
\begin{bmatrix}
	c_1\\ c_2\\ c_3
\end{bmatrix}
$を定ベクトルとする.$u,\ v$をスカラー変数とするベクトル関数$\bm{f}(u,\ v) = 
\begin{bmatrix}
	f_1(u,\ v)\\ f_2(u,\ v)\\ f_3(u,\ v)
\end{bmatrix}
$について,$(u,\ v)$が$(u_0,\ v_0)$に限りなく近づくときの$\bm{f}(u,\ v)$の\textbf{極限}は次で定義される.
\begin{align}
	&\lim_{(u,\ v) \to (u_0,\ v_0)} |\bm{f}(u,\ v) - \bm{c}| = 0 \notag\\
	&\iff \lim_{(u,\ v) \to (u_0,\ v_0)} \bm{f}(u,\ v) = \bm{c}\\
	&\iff \lim_{(u,\ v) \to (u_0,\ v_0)} f_1(u,\ v) = c_1,\quad \lim_{(u,\ v) \to (u_0,\ v_0)} f_2(u,\ v) = c_2,\quad \lim_{(u,\ v) \to (u_0,\ v_0)} f_3(u,\ v) = c_3
\end{align}

また,$\dlim_{(u,\ v) \to (u_0,\ v_0)} \bm{f}(u,\ v) = \bm{f}(u_0,\ v_0)$が成立するとき,$\bm{f}(u,\ v)$は$(u,\ v) = (u_0,\ v_0)$で\textbf{連続である}という.



\subsection{ベクトル関数の偏微分法}

領域$D$で定義されたベクトル関数$\bm{f}(u,\ v)$に於いて
\begin{equation}
	\bunsuu{\partial \bm{f}}{\partial u} = \lim_{\varDelta u \to 0} \bunsuu{\bm{f}(u + \varDelta u,\ v) - \bm{f}(u,\ v)}{\varDelta u} =
	{
	\begin{bmatrix}
		\bunsuu{\partial f_1(u,\ v)}{\partial u} &
		\bunsuu{\partial f_2(u,\ v)}{\partial u} &
		\bunsuu{\partial f_3(u,\ v)}{\partial u}
	\end{bmatrix}
	}^\top
\end{equation}
を$\bm{f}(u,\ v)$の\textbf{$u$についての偏導関数}といい,$\bm{f}_u(u,\ v)$とも表す.また,
\begin{equation}
	\bunsuu{\partial \bm{f}}{\partial v} = \lim_{\varDelta v \to 0} \bunsuu{\bm{f}(u,\ v + \varDelta v) - \bm{f}(u,\ v)}{\varDelta v} =
	{
	\begin{bmatrix}
		\bunsuu{\partial f_1(u,\ v)}{\partial v} &
		\bunsuu{\partial f_2(u,\ v)}{\partial v} &
		\bunsuu{\partial f_3(u,\ v)}{\partial v}
	\end{bmatrix}
	}^\top
\end{equation}
を$\bm{f}(u,\ v)$の\textbf{$v$についての偏導関数}といい,$\bm{f}_v(u,\ v)$とも表す.

\begin{enumerate}[leftmargin=18pt, labelsep=10pt, labelsep=10pt, itemindent=9pt]
	\item[\f{例}] $\bm{f}(u,\ v) =
		\begin{bmatrix}
			u\\ v\\ u^2 + v^2
		\end{bmatrix}
		$の偏導関数は
		\begin{equation}
			\bunsuu{\partial \bm{f}}{\partial u} =
			\begin{bmatrix}
				1\\ 0\\ 2u
			\end{bmatrix}
			,\quad \bunsuu{\partial \bm{f}}{\partial v} =
			\begin{bmatrix}
				0\\ 1\\ 2v
			\end{bmatrix}
		\end{equation}
\end{enumerate}

また,次のチェーン・ルールが成り立つ.
\begin{kousiki}{チェーン・ルール}
	\begin{enumerate}[label=\textbf{[\arabic*]}, labelsep=10pt, leftmargin=23pt]
		\item $u,\ v$がともにスカラー変数$t$の関数のとき
			\begin{equation}
				\bunsuu{d\bm{f}}{dt} = \bunsuu{\partial \bm{f}}{\partial u}\bunsuu{du}{dt} + \bunsuu{\partial \bm{f}}{\partial v}\bunsuu{dv}{dt}
			\end{equation}
		\item $u,\ v$がともにスカラー変数$s,\ t$の変数のとき
			\begin{equation}
				\bunsuu{\partial \bm{f}}{\partial s} =
				\bunsuu{\partial \bm{f}}{\partial u}\bunsuu{\partial u}{\partial s} + \bunsuu{\partial \bm{f}}{\partial v}\bunsuu{\partial v}{\partial s}, \qquad
				\bunsuu{\partial \bm{f}}{\partial t} =
				\bunsuu{\partial \bm{f}}{\partial u}\bunsuu{\partial u}{\partial t} + \bunsuu{\partial \bm{f}}{\partial v}\bunsuu{\partial v}{\partial t}
			\end{equation}
	\end{enumerate}
\end{kousiki}



\subsection{接平面}

空間内の点$\mathrm{P}$の位置ベクトルが$\bm{r} =
\begin{bmatrix}
	x(u,\ v)\\ y(u,\ v)\\ z(u,\ v)
\end{bmatrix}
$のようにベクトル関数であるとき,点$(u,\ v)$が領域$D$を動くと,$\mathrm{P}$は1つの曲面$S$を描く.この$S$を$\bm{r} = \bm{r}(u,\ v)$の表す曲面という.

$\bm{r}(u,\ v)$が$D$で連続な偏導関数をもつとする.$v$を一定にして$u$を変化させると$\bm{r}$は$S$上で1つの曲線(\textbf{$u$曲線})を描く.よって,$\bunsuu{\partial \bm{r}(u,\ v)}{\partial u}$は$u$曲線上の$\mathrm{P}$における接線ベクトルを与える.

同様に,$u$を一定にして$v$を変化させると$\bm{r}$は$S$上で1つの曲線(\textbf{$v$曲線})を描く.よって,$\bunsuu{\partial \bm{r}(u,\ v)}{\partial v}$は$v$曲線上の$\mathrm{P}$における接線ベクトルを与える.

このとき,外積の性質より$\bunsuu{\partial \bm{r}}{\partial u}$と$\bunsuu{\partial \bm{r}}{\partial v}$が平行ならば$\bunsuu{\partial \bm{r}}{\partial u} \times \bunsuu{\partial \bm{r}}{\partial v} = \bm{0}$なので,$\bunsuu{\partial \bm{r}}{\partial u} \times \bunsuu{\partial \bm{r}}{\partial v} \ne \bm{0}$ならば,この2つの接線ベクトルを含む平面$H$が存在する.この$H$を$S$の点$\mathrm{P}$における\textbf{接平面}という.その法線ベクトルは
\begin{equation}
	\bunsuu{\partial \bm{r}}{\partial u} \times \bunsuu{\partial \bm{r}}{\partial v} =
	\begin{vmatrix}
		\bm{i} & x_u & x_v\\
		\bm{j} & y_u & y_v\\
		\bm{k} & z_u & z_v
	\end{vmatrix}
\end{equation}
である.

\begin{enumerate}[leftmargin=18pt, labelsep=10pt, labelsep=10pt, itemindent=9pt]
	\item[\f{例}] ベクトル関数$\bm{r} = \bm{r}(u,\ v) =
		\begin{bmatrix}
			u\cos v\\ u\sin v\\ u^2
		\end{bmatrix}
		$を例にとって,詳しく見ていく.
		\begin{enumerate}[label=\textbf{[\arabic*]}, labelsep=10pt, leftmargin=23pt, itemsep=12pt]
			\item \underline{与式の表す曲面の,$x,\ y,\ z$に関する方程式}\\
				$x = u\cos v,\ y = u\sin v$より$x^2 + y^2$を計算すると$u^2$なので,$z = u^2$となる.よって,$z = x^2 + y^2$.
			\item \underline{$v = \bunsuu{\pi}{2}$のときの$u$曲線}\\
				$x = u\cos\bunsuu{\pi}{2} = 0,\ y = u\sin\bunsuu{\pi}{2} = u$なので,$z = 0^2 + y^2 = y^2$.これは平面$x = 0$上の放物線である.
			\item \underline{$u = 1$のときの$v$曲線}\\
				$x = \cos v,\ y = \sin v$より,$z = 1$.これは平面$z = 1$上の半径$1$の円である.
			\item \underline{$(u,\ v) = \left(1,\ \bunsuu{\pi}{2}\right)$における接線ベクトル}\\
				$\bunsuu{\partial \bm{r}}{\partial u} =
				\begin{bmatrix}
					\cos v\\ \sin v\\ 2u
				\end{bmatrix}
				=
				\begin{bmatrix}
					0\\ 1\\ 2
				\end{bmatrix}
				$,\qquad
				$\bunsuu{\partial \bm{r}}{\partial v} =
				\begin{bmatrix}
					-u\sin v\\ u\cos v\\ 0
				\end{bmatrix}
				=
				\begin{bmatrix}
					-1\\ 0\\ 0
				\end{bmatrix}
				$
			\item \underline{$(u,\ v) = \left(1,\ \bunsuu{\pi}{2}\right)$における接平面の法線ベクトル}\\
				$\bunsuu{\partial \bm{r}}{\partial u} \times \bunsuu{\partial \bm{r}}{\partial v} =
				\begin{vmatrix}
					\bm{i} & x_u & x_v\\
					\bm{j} & y_u & y_v\\
					\bm{k} & z_u & z_v
				\end{vmatrix}
				=
				\begin{vmatrix}
					\bm{i} & 0 & -1\\
					\bm{j} & 1 & 0\\
					\bm{k} & 2 & 0
				\end{vmatrix}
				=
				\begin{bmatrix}
					0\\ -2\\ 1
				\end{bmatrix}
				$
		\end{enumerate}
\end{enumerate}



\section{空間曲線}
\subsection{曲率}

曲線$\bm{r} = \bm{r}(s)$の単位接線ベクトル$\bm{t} = \bunsuu{d\bm{r}}{ds}$について次の$\kappa \in \mathbb{R}$を曲線の\textbf{曲率}という.
\begin{equation}
	\kappa = \left|\bunsuu{d\bm{t}}{ds}\right| = \left|\bunsuu{d^2\bm{r}}{ds^2}\right|
\end{equation}

$\kappa$は曲線の局所的な曲がり具合である.この値が大きいほどカーブが急になる.$\varDelta \bm{t}$が十分に小さいとき,$|\varDelta \bm{t}| \approx \varDelta \theta$と見做すことができ,$\kappa = \left|\bunsuu{d\theta}{ds}\right|$と表すこともできるので,$\kappa = $は回転角の変化率であるとも言える.

\begin{enumerate}[label=\textbf{[\arabic*]}, labelsep=10pt, leftmargin=23pt]
	\item 半径$a$の円の曲率は$\kappa = \bunsuu{1}{a}$で,半径が大きくなるほど曲線の曲がり具合が小さくなる.
	\item 直線の曲率は$\kappa = 0$である.
\end{enumerate}

曲率の逆数を\textbf{曲率半径}といい,$\sigma$で表す.
\begin{equation}
	\sigma = \bunsuu{1}{\kappa}
\end{equation}

\begin{enumerate}[label=\textbf{[\arabic*]}, labelsep=10pt, leftmargin=23pt]
	\item 半径$a$の曲率半径は$\sigma = a$である.$\kappa = 0$のときは$\sigma = \infty$と定義する.
	\item 直線の曲率半径は$\sigma = \infty$である.
\end{enumerate}



\subsection{捩率}



\subsection{速度・加速度}


\section{スカラー場の勾配}
\subsection{勾配}

$D$で定義されたスカラー場$f(x,\ y,\ z)$について,その偏導関数$\bunsuu{\partial f}{\partial x},\ \bunsuu{\partial f}{\partial y},\ \bunsuu{\partial f}{\partial z}$を係数とするベクトル$\bunsuu{\partial f}{\partial x}\bm{i} + \bunsuu{\partial f}{\partial y}\bm{j} + \bunsuu{\partial f}{\partial z}\bm{k}$を考えると,領域$D$にあるベクトル場が定義される.これをスカラー場$f$の\textbf{勾配}といい,$\grad f$で表す.形式的なベクトル$\bm{\nabla} =
\begin{bmatrix}
	\bunsuu{\partial}{\partial x} &
	\bunsuu{\partial}{\partial y} &
	\bunsuu{\partial}{\partial z}
\end{bmatrix}^\top
$を使って,$\bm{\nabla}f$で表すこともある.$\bm{\nabla}$を\textbf{Hamiltonの演算子}と呼ばれ,\textbf{ナブラ}と読む.

\begin{kousiki}{スカラー場の勾配}
	スカラー場$f(x,\ y,\ z)$の勾配$\grad f$とは次のベクトル場である.
	\begin{equation}
		\grad f = \bm{\nabla}f =
		{
		\begin{bmatrix}
			\bunsuu{\partial f}{\partial x} &
			\bunsuu{\partial f}{\partial y} &
			\bunsuu{\partial f}{\partial z}
		\end{bmatrix}
		}^\top
	\end{equation}
\end{kousiki}

$c$を定数とすると,方程式$f(x,\ y,\ z) = c$は一般に1つの曲面を表す.これを$f$の\textbf{等位面}という.$c$をパラメータとすると方程式は等位面の群を表す.点$\mathrm{P}$を通る等位面上に$\mathrm{P}$を通る任意の曲線$\bm{r}(t) =
\begin{bmatrix}
	x(t)\\ y(t)\\ z(t)
\end{bmatrix}
$を描くと \vskip-\baselineskip
\begin{equation*}
	f\bigl(x(t),\ y(t),\ z(t)\bigr) = c
\end{equation*}
を満たす.両辺を$t$で微分すると,チェーン・ルールより
\begin{align*}
	&\bunsuu{\partial f}{\partial x}\bunsuu{dx}{dt} + \bunsuu{\partial f}{\partial y}\bunsuu{dy}{dt} + \bunsuu{\partial f}{\partial z}\bunsuu{dz}{dt} = 0\\
	&\iff \bm{\nabla}f \cdot \bunsuu{d\bm{r}}{dt} = 0\\
	&\iff \bm{\nabla}f \perp \bunsuu{d\bm{r}}{dt}
\end{align*}
となる.

これは点$\mathrm{P}$に対応するベクトル場$\bm{\nabla}f$は,$\mathrm{P}$を通る等位面上の曲線の接線ベクトル$\bunsuu{d\bm{r}}{dt}$に垂直になる.また,曲線は任意であるから次のことが言える.
\begin{kousiki}{勾配$\bm{\nabla}f$の意味}
	点$\mathrm{P}$における勾配$\bm{\nabla}f$は,$\mathrm{P}$における等位面の法線ベクトルになる.
\end{kousiki}

\vskip\baselineskip

任意の単位ベクトル$\bm{e} =
\begin{bmatrix}
	e_x\\ e_y\\ e_z
\end{bmatrix}
$について,点$\mathrm{P}$から$\bm{e}$の方向に$\varDelta s$だけ動いた時の$f$の微分係数(接線の傾き)を調べる.これを\textbf{方向微分係数}$\bunsuu{df}{ds}$という.
\begin{kousiki}{方向微分係数}
	\begin{align}
		\bunsuu{df}{ds} &=
		\lim_{\varDelta s \to 0}
			\bunsuu{f(x + e_x \varDelta s,\ y + e_y \varDelta s,\ z + e_z \varDelta s) - f(x,\ y,\ z)}{\varDelta s}\\
			&= \bm{\nabla}f \cdot \bm{e}
	\end{align}
\end{kousiki}

\f{証明} $\mathrm{P}$から$\bm{e}$の方向に$\varDelta s$だけ離れた点の座標は$(x + e_x \varDelta s,\ y + e_y \varDelta s,\ z + e_z \varDelta s)$なので,$f$の変化率はチェーン・ルールより
\begin{align*}
	\bunsuu{d}{ds}&f(x + e_x \varDelta s,\ y + e_y \varDelta s,\ z + e_z \varDelta s)\\
	&= \bunsuu{\partial f}{\partial x}\bunsuu{d}{ds}(x + e_x \varDelta s) + \bunsuu{\partial f}{\partial y}\bunsuu{d}{ds}(y + e_y \varDelta s) + \bunsuu{\partial f}{\partial z}\bunsuu{d}{ds}(z + e_z \varDelta s)\\
	&= \bunsuu{\partial f}{\partial x}e_x + \bunsuu{\partial f}{\partial y}e_y + \bunsuu{\partial f}{\partial z}e_z = \bm{\nabla}f \cdot \bm{e}
\end{align*}
となる.

このとき,$\bm{\nabla}f$と$\bm{e}$の成す角を$\theta$とすると
\begin{equation*}
	\bm{\nabla}f \cdot \bm{e} = |\bm{\nabla} f||\bm{e}|\cos\theta = |\bm{\nabla}f|\cos\theta
\end{equation*}
となり,$\theta = 0$のとき正の最大値をとる.すなわち,勾配=法線ベクトルの方向への方向微分係数が最大となる.方程式
\begin{equation*}
	f(x,\ y,\ z) = c_i \quad (c = 1,\ 2,\ \cdots)
\end{equation*}
で,$c_1$での方向微分係数より,$c_2$での方向微分係数の値の方が大きいとき,
% \chapter{ラプラス変換}
\setcounter{page}{1}
\section{定義と基本的性質}
\subsection{ラプラス変換の定義}

関数$f(t)$は,$t \in \mathbb{R}_{> 0}$で定義され,$s$を$t$と無関係な実数とする.このとき次のような関数$F(s)$を考える.
\begin{equation}
	F(s) = \int_{0}^{\infty} e^{-st}f(t)\,dt = \lim_{\substack{T \to \infty \\ \varepsilon \to +0}} \int_{\varepsilon}^{T} e^{-st}f(t)\,dt
\end{equation}

これは,関数$f(t)$に関数$F(s)$を対応させる規則を与えている.その対応を$\mathcal{L}$で表す.$\mathcal{L}$を適用させた$F(s)$を$f(t)$の\textbf{ラプラス変換}といい,
\begin{equation}
	F(s) = \mathcal{L}[f(t)] \text{\quad または \quad} f(t) \stackrel{\mathcal{L}}{\longrightarrow} F(s)
\end{equation}
と表す.$f(t)$を\textbf{原関数},$F(s)$を\textbf{像関数}という.一般に$s \in \mathbb{C}$であるが,ここでは$s \in \mathbb{R}$とする.

\begin{enumerate}[leftmargin=18pt, labelsep=10pt, labelsep=10pt, itemindent=9pt]
	\item[\f{例1}] $f(t) = 1$のラプラス変換$\mathcal{L}[1]$
		\begin{align*}
			F(s) = \mathcal{1}[1] = \lim_{T \to \infty} \int_{0}^{T} e^{-st}\,dt = \lim_{T \to \infty} \teisekibun{{-\bunsuu{1}{s}e^{-st}}}{0}{T} = \lim_{T \to \infty} \bunsuu{1}{s}(1 - e^{-sT}) = \bunsuu{1}{s} \quad (s > 0)
		\end{align*}
		$s \le 0$のとき,$\dlim_{T \to \infty} \bunsuu{1}{s}(1 - e^{-sT}) = \infty$なので存在しない.
	\item[\f{例2}] $f(t) = t$のラプラス変換$\mathcal{L}[t]$
		\begin{align*}
			F(s) &= \mathcal{L}[t] = \lim_{T \to \infty} \int_{0}^{T} e^{-st}t\,dt = \lim_{T \to \infty} \left\{ \teisekibun{{-\bunsuu{e^{-st}}{s}t}}{0}{T} + \int_{0}^{T} \bunsuu{e^{-st}}{s}\,dt \right\}
			= \lim_{T \to \infty} \left\{ -\bunsuu{e^{-sT}}{s}T - \teisekibun{\bunsuu{e^{-st}}{s^2}}{0}{T} \right\}\\
			&= \lim_{T \to \infty} \left\{
				-\bunsuu{e^{-sT}T}{s} - \bunsuu{e^{-sT}}{s^2} + \bunsuu{1}{s^2}
			\right\}
			= \bunsuu{1}{s^2}
		\end{align*}
		ここで,$s > 0$のとき
		\begin{gather*}
			\lim_{T \to \infty} \left(-\bunsuu{e^{-sT}T}{s}\right) = -\bunsuu{1}{s}\lim_{T \to \infty} \bunsuu{T}{e^{sT}} = -\bunsuu{1}{s}\lim_{T \to \infty} \bunsuu{1}{se^{sT}} = 0\\
			\lim_{T \to \infty} \left(-\bunsuu{e^{-sT}}{s^2}\right) = -\bunsuu{1}{s^2}\lim_{T \to \infty} \bunsuu{1}{e^{sT}} = 0
		\end{gather*}
	\item[\f{例3}] $f(t) = e^{\alpha t}$($\alpha$は定数)のラプラス変換$\mathcal{L}[e^{\alpha t}]$
		\begin{align*}
			F(s) = \mathcal{L}[e^{\alpha t}] &= \lim_{T \to \infty} \int_{0}^{T} e^{-st}e^{\alpha t}\,dt
			= \lim_{T \to \infty} \int_{0}^{T} e^{-(s - \alpha)t}\,dt
			= \lim_{T \to \infty} \teisekibun{{-\bunsuu{1}{s - \alpha}e^{-(s - \alpha)t}}}{0}{T}\\
			&= \lim_{T \to \infty} \left(-\bunsuu{1}{s - \alpha}\{e^{-(s - \alpha)T} - 1\}\right) = \bunsuu{1}{s - \alpha} \quad (s > \alpha)
		\end{align*}
		$s - \alpha \le 0$,即ち$s \le \alpha$のとき,極限値は存在しない.
	\item[\f{例4}] $f(t) = \sin \omega t$のラプラス変換$\mathcal{L}[\sin \omega t]$
		\begin{align*}
			F(s) &= \mathcal{L}[\sin \omega t] = \lim_{T \to \infty} \int_{0}^{T} e^{-st}\sin \omega t\,dt
			\intertext{$I_1 = \dint_{0}^{T} e^{-st}\sin \omega t\,dt$とする.}
			I_1 &= \int_{0}^{T} e^{-st}\sin \omega t\,dt = \teisekibun{ \bunsuu{1}{-s}e^{-st}\sin \omega t }{0}{T} - \int_{0}^{T} \bunsuu{1}{-s}e^{-st}\omega\cos \omega t\,dt\\
			&= \bunsuu{1}{-s}e^{-sT}\sin \omega T + \bunsuu{\omega}{s}\int_{0}^{T} e^{-st}\cos \omega t\,dt\\
			&= \bunsuu{1}{-s}e^{-sT}\sin \omega T + \bunsuu{\omega}{s}\left\{
				\teisekibun{
					\bunsuu{1}{-s}e^{-st}\cos \omega t
				}{0}{T} - \int_{0}^{T} \bunsuu{1}{-s}e^{-st}(-\omega\sin \omega t)\,dt
			\right\}\\
			&= \bunsuu{1}{-s}e^{-sT}\sin \omega T + \bunsuu{\omega}{s}\left\{
				\left(
					\bunsuu{1}{-s}e^{-sT}\cos \omega T + \bunsuu{1}{s}
				\right)
				- \bunsuu{\omega}{s}I_1
			\right\}
			\intertext{よって,$I_1$について解くと}
			I_1 &= \bunsuu{s^2}{s^2 + \omega^2}\left\{
				-\bunsuu{1}{s}e^{-sT}\sin \omega T - \bunsuu{\omega}{s^2}e^{-sT}\cos \omega T + \bunsuu{\omega}{s^2}
			\right\}
		\end{align*}
		$s > 0$のとき$\dlim_{T \to \infty} e^{-sT}\sin \omega T = \dlim_{T \to \infty} e^{-sT}\cos \omega T = 0$なので,
		\begin{equation*}
			\mathcal{L}[\sin \omega t] = \lim_{T \to \infty} I_1 = \bunsuu{\omega}{s^2 + \omega^2}\quad(s > 0)
		\end{equation*}
		同様にすると,$\mathcal{L}[\cos \omega t] = \bunsuu{s}{s^2 + \omega^2}\quad(s > 0)$
\end{enumerate}



\subsection{単位ステップ関数}

次のように定義される関数$H(t)$を\textbf{Heavisideの階段関数}という.
\begin{equation}
	H(t) =
	\left\{
		\begin{array}{ll}
			0 & (t < 0)\\ 1 & (t > 0)
		\end{array}
	\right.
\end{equation}

《注》$t = 0$に於ける値は任意に決めることができる.

また,次のように定義される関数$U(t)$を\textbf{単位ステップ関数}という.
\begin{equation}
	U(t) =
	\left\{
		\begin{array}{ll}
			0 & (t \le 0)\\ 1 & (t > 0)
		\end{array}
	\right.
\end{equation}

関数$U(t - a)$は$U(t)$を$t$軸方向に$a$平行移動して得られる.

$a \ge 0$のときの$\mathcal{L}[U(t - a)]$を求める.$t - a \le 0$,即ち$t \le a$のときは$U(t - a) = 0$,$t > a$のときは$U(t - a) = 1$なので
\begin{align*}
	F(s) = \mathcal{L}[U(t - a)] &= \int_{0}^{\infty} e^{-st}U(t - a)\,dt = \int_{0}^{a} e^{-st}U(t - a)\,dt + \int_{a}^{\infty} e^{-st}U(t - a)\,dt = \int_{a}^{\infty} e^{-st}\,dt\\
	&= \teisekibun{{
		-\bunsuu{1}{s}e^{-st}
	}}{a}{\infty}
	= -\bunsuu{1}{s}\lim_{t \to \infty} \left(
		e^{-st} - e^{-as}
	\right)
\end{align*}
ここで,$s > 0$のとき$\dlim_{t \to \infty} e^{-st} = 0$なので
\begin{equation*}
	F(s) = -\bunsuu{1}{s}\cdot(-e^{-as}) = \bunsuu{e^{-as}}{s} \quad (s > 0)
\end{equation*}
である.



\section{ラプラス変換の基本的性質}
\subsection{ラプラス変換の線形性}

関数$f(t),\ g(t)$,定数$a,\ b$に対し
\begin{equation}
	\mathcal{L}[af(t) + bg(t)] = a\mathcal{L}[f(t)] + b\mathcal{L}[g(t)]
\end{equation}
が成り立つ.これによって,項別にラプラス変換をすることで求めることができる.



\subsection{ラプラス変換の相似性}

関数$f(t)$,定数$\lambda > 0$に対し,$F(s) = \mathcal{L}[f(t)]$とする.このとき
\begin{equation}
	\mathcal{L}[f(\lambda t)] = \bunsuu{1}{\lambda}F\left(\bunsuu{s}{\lambda}\right)
\end{equation}
が成り立つ.即ち,$t$軸方向に$\bunsuu{1}{\lambda}$倍してからラプラス変換すると,$F(s)$を$s$軸方向に$\lambda$倍したものを$\bunsuu{1}{\lambda}$倍したものになる.

\begin{equation*}
	\mspace{100mu}
	\begin{array}{llll}
		& x = f(t)	& \xlongrightarrow[\text{ラプラス変換}]{\mathcal{L}} & X = F(s)\\
		\text{\small \fbox{$t$軸方向に$\bunsuu{1}{\lambda}$倍}}
			& \text{\LARGE \raisebox{-4pt}{☟}} &
			& \text{\LARGE \raisebox{-4pt}{☟}}\quad
			\text{\small \fbox{$s$軸方向に$\lambda$倍して$X$軸方向に$\bunsuu{1}{\lambda}$倍}}\\[10pt]
		& x = f(\lambda t)	& \xlongrightarrow[\text{ラプラス変換}]{\mathcal{L}} & X = \bunsuu{1}{\lambda}F\left(\bunsuu{s}{\lambda}\right)
	\end{array}
\end{equation*}



\subsection{第1移動定理(像関数の移動法則)}

関数$f(t)$,定数$\alpha$に対し,$F(s) = \mathcal{L}[f(t)]$とする.このとき
\begin{equation}
	\mathcal{L}[e^{\alpha t}f(t)] = F(s - \alpha)
\end{equation}
が成り立つ.即ち,原関数に$e^{\alpha t}$をかけたものをラプラス変換すると,像関数は$F(s)$を$s$軸方向に$\alpha$平行移動したものになる.

\begin{equation*}
	\mspace{150mu}
	\begin{array}{llll}
		& x = f(t)	& \xlongrightarrow[\text{ラプラス変換}]{\mathcal{L}} & X = F(s)\\
		\text{\small \fbox{$e^{\alpha t}$をかける}}
			& \text{\LARGE \raisebox{-4pt}{☟}} &
			& \text{\LARGE \raisebox{-4pt}{☟}}\quad
			\text{\small \fbox{$s$軸方向に$\alpha$平行移動}}\\[10pt]
		& x = e^{\alpha t}f(t)	& \xlongrightarrow[\text{ラプラス変換}]{\mathcal{L}} & X = F(s - \alpha)
	\end{array}
\end{equation*}



\subsection{第2移動定理(原関数の移動法則)}

ラプラス変換は$t > 0$の範囲で行うので,単位ステップ関数$U(t)$をかけても変わらない.このとき,関数$f(t)$,定数$\mu > 0$に対し,$F(s) = \mathcal{L}[f(t)]$とすると
\begin{equation}
	\mathcal{L}[f(t - \mu)U(t - \mu)] = e^{-\mu s}F(s)
\end{equation}
が成り立つ.即ち,原関数を$t$軸方向に$\mu$移動したものをラプラス変換すると,像関数は$F(s)$に$e^{-\mu s}$をかけたものになる.

\begin{equation*}
	\mspace{120mu}
	\begin{array}{llll}
		& x = f(t)	& \xlongrightarrow[\text{ラプラス変換}]{\mathcal{L}} & X = F(s)\\
		\text{\small \fbox{$t$軸方向に$\mu$移動}}
			& \text{\LARGE \raisebox{-4pt}{☟}} &
			& \text{\LARGE \raisebox{-4pt}{☟}}\quad
			\text{\small \fbox{$e^{-\mu s}$をかける}}\\[10pt]
		& x = f(t - \mu)U(t - \mu)	& \xlongrightarrow[\text{ラプラス変換}]{\mathcal{L}} & X = e^{-\mu s}F(s)
	\end{array}
\end{equation*}



\subsection{微分法則}

原関数$f(t)$のラプラス変換を$F(s)$とすると次が成り立つ.
\begin{kousiki}{1階の微分法則}
	\begin{enumerate}[label=\textbf{[\arabic*]}, labelsep=10pt, leftmargin=23pt]
		\item $\mathcal{L}[f'(t)] = sF(s) - f(+0)$ \hfill (原関数の微分法則)
		\item $\mathcal{L}[tf(t)] = -F'(s)$ \hfill (像関数の微分法則)
	\end{enumerate}
\end{kousiki}

$f(+0)$は,$t \to +0$のときの$f(t)$の極限値を表す.原関数の微分法則は微分方程式に使われることがある.

原関数や像関数の微分法則を繰り返し使うと以下を得る.
\begin{kousiki}{高次微分法則}
	\begin{enumerate}[label=\textbf{[\arabic*]}, labelsep=10pt, leftmargin=23pt]
		\item $\mathcal{L}[f^{(n)}(t)] = s^n F(s) - s^{n - 1}f(+0) - s^{n - 2}f'(+0) - s^{n - 3}f''(+0) - \cdots - f^{(n - 1)}(+0)$ \hfill (原関数の高次微分法則)
		\item $\mathcal{L}[t^n f(t)] = (-1)^n F^{(n)}(s)$ \hfill (像関数の高次微分法則)
	\end{enumerate}
\end{kousiki}



\subsection{積分法則}

原関数$f(t)$のラプラス変換を$F(s)$とすると積分についての次が成り立つ.

\begin{kousiki}{積分法則}
	\begin{enumerate}[label=\textbf{[\arabic*]}, labelsep=10pt, leftmargin=23pt]
		\item $\mathcal{L}\left[\dint_{0}^{t} f(\tau)\,d\tau\right] = \bunsuu{F(s)}{s}$ \hfill (原関数の積分法則)
		\item $\mathcal{L}\left[\bunsuu{f(t)}{t}\right] = \dint_{s}^{\infty} F(\sigma)\,d\sigma$ \hfill (像関数の積分法則)
	\end{enumerate}
\end{kousiki}



\subsection{たたみこみ}

区間$[0,\ \infty)$で定義された関数$f(t),\ g(t)$に対し
\begin{equation}
	(f * g)(t) = \int_{0}^{t} f(\tau)g(t - \tau)\,d\tau
\end{equation}
を$f(t)$と$g(t)$の\textbf{たたみこみ}または\textbf{合成積}という.たたみこみのラプラス変換について,次の関係が成り立つ.
\begin{kousiki}{たたみこみのラプラス変換}
	\begin{equation}
		\mathcal{L}[(f * g)(t)] = \mathcal{L}[f(t)]\mathcal{L}[g(t)]
	\end{equation}
\end{kousiki}
% \chapter{力学I}
\setcounter{page}{1}
\section{ベクトル,速度,加速度}
\subsection{点の位置の表し方}

無限に広い平面にある点$\mathrm{P}$の位置を表すのには,基準となる物体(基準体)が必要.基準体を1つの点$\mathrm{O}$とすれば,$\mathrm{P}$の位置を表すものに,距離はあるが方向はない.よって,基準体は大きさを持ったものでなくてはならない.時が経っても形の変わらないものを\textbf{剛体}という.この剛体上に2つの定点$\mathrm{A,\ B}$をとれば,$\mathrm{AP,\ BP}$の長さによって$\mathrm{P}$の位置は決まる.

点$\mathrm{P}$の位置を表すのには$\mathrm{OP}$の長さ$r$を使って,$\vecrm{OP} = \bm{r}$のようにベクトルで書き表す.これを\textbf{位置ベクトル}という.$x$軸と$\vecrm{OP}$の成す角を$\varphi$とすると$(x,\ y)$と$(r,\ \varphi)$の関係は
\begin{equation}
	x = r\cos\varphi, \qquad y = r\sin\varphi
\end{equation}

座標系のとり方はいろいろある.

\begin{enumerate}[leftmargin=18pt, labelsep=10pt, labelsep=10pt, itemindent=9pt]
	\item[\f{例}] 原点を共通に持つ2つの座標系の軸が$\bunsuu{\pi}{4}$の角をつくっている.
		\begin{inparaenum}[(1)]
			\item 任意の点$\mathrm{P}$の座標$(x,\ y),\ (x',\ y')$の間にはどんな関係があるか;
			\item $x'^2 + y'^2 = x^2 + y^2$を示せ;
			\item $ax^2 + 2hxy + ay^2 = 1$で示される曲線の方程式を$x',\ y'$を使って表せ.
		\end{inparaenum}
		\begin{enumerate}[label=(\arabic*), labelsep=10pt, leftmargin=23pt]
			\item 図で,$\mathrm{P}$から$x$軸と$x'$軸にそれぞれ垂線$\mathrm{PA,\ PB}$を下す.$\mathrm{A}$から$x'$軸に垂線$\mathrm{AA'}$を下すと
			\begin{equation}
				x' = \mathrm{OB} + \mathrm{OA'} + \mathrm{A'B} = \mathrm{OA}\cos\bunsuu{\pi}{4} + \mathrm{AP}\sin\bunsuu{\pi}{4} = \bunsuu{1}{\sqrt{2}}(x + y) \label{equ:NR1-1}
			\end{equation}
			また
			\begin{equation}
				y' = \mathrm{AP}\cos\bunsuu{\pi}{4} - \mathrm{OA}\sin\bunsuu{\pi}{4} = \bunsuu{1}{\sqrt{2}}(-x + y) \label{equ:NR1-2}
			\end{equation}
			式(\ref{equ:NR1-1}),式(\ref{equ:NR1-2})が$x',\ y'$を$x,\ y$で表す式である.$x,\ y$について解けば
			\begin{gather}
				x = \bunsuu{1}{\sqrt{2}}(x' - y')\\
				y = \bunsuu{1}{\sqrt{2}}(x' + y')
			\end{gather}
			\item (1)より
			\begin{equation}
				x'^2 + y'^2 = x^2 + y^2
			\end{equation}
			\item 与式に代入すると
			\begin{equation*}
				(a + h)x'^2 + (a - h)y'^2 = 1
			\end{equation*}
		\end{enumerate}
\end{enumerate}

空間の直交座標系は,右手の親指,人差し指,中指の順に$x,\ y,\ z$軸をとる(\textbf{右手座標系}).\textbf{極座標}では,$x$軸と$\bm{r}$の正射影の成す角を$\varphi$,$z$軸と$\bm{r}$の成す角を$\theta$とする.$\varphi$は経度,$\theta$は緯度にあたる.$(x,\ y,\ z)$と$(r,\ \varphi,\ \theta)$の関係は
\begin{equation}
	x = r\sin\theta\cos\varphi,\qquad y = r\sin\theta\sin\varphi,\qquad z = r\cos\theta
\end{equation}
となる.



\subsection{速度ベクトル}

\textbf{速度}または\textbf{速度ベクトル}$\bm{v}$は
\begin{equation}
	\bm{v} = \lim_{\varDelta t \to 0} \bunsuu{\varDelta \bm{r}}{\varDelta t} = \bunsuu{d\bm{r}}{dt}
\end{equation}
で求める.

$\mathrm{P}$点の位置を辿ると曲線を描く(\textbf{軌道}または\textbf{径路}).時間の差$\varDelta t$が小さいほど,$|\varDelta\bm{r}|$と軌道に沿っての長さ$\varDelta s$の比が$1$に近づくので$v = |\bm{v}|$は
\begin{equation}
	v = \lim_{\varDelta t \to 0} \bunsuu{|\varDelta \bm{r}|}{\varDelta t} = \lim_{\varDelta t \to 0} \bunsuu{\varDelta s}{\varDelta t} = \bunsuu{ds}{dt}
\end{equation}
となる.これを\textbf{速さ}という.



\subsection{加速度ベクトル}

\textbf{加速度}または\textbf{加速度ベクトル}$\bm{a}$は
\begin{equation}
	\bm{a} = \lim_{\varDelta t \to 0} \bunsuu{\varDelta \bm{v}}{\varDelta t} = \bunsuu{d\bm{v}}{dt}
\end{equation}
で求める.

\begin{equation}
	x = a\cos(\omega t + \alpha) \qquad \text{($a,\ \alpha$は定数)}
\end{equation}
で表される運動は
\begin{gather}
	v = \bunsuu{dx}{dt} = -\omega a \sin(\omega t + \alpha)\\
	a = \bunsuu{d^2x}{dt^2} = -\omega^2 a \cos(\omega t + \alpha) = -\omega^2 x
\end{gather}
となる.加速度はいつも原点の方を向いており,その大きさは原点からの距離に比例している.この運動を\textbf{単振動}という.$x$は$\pm a$の間を往復する.$\omega t + \alpha$の値によって$x$の値が決まるので\textbf{位相}という.$\alpha$を\textbf{初期位相}という.



\subsection{1節 問題}

\begin{enumerate}[label=\textbf{[\arabic*]}, labelsep=10pt, leftmargin=23pt]
	\item 空間の1つの点の位置の極座標を$r,\ \theta,\ \varphi$とする.$r,\ \theta,\ \varphi$方向($r$方向は$\theta$,$\varphi$を一定にして$r$だけが増すような方向,他も同様)の方向余弦を求めよ.
	\item 3つのベクトル$\bm{A},\ \bm{B},\ \bm{C}$を1つの点$\mathrm{O}$から引くときこれらが一平面内にあるための条件を求めよ.
	\item 2つの点$\mathrm{A,\ B}$の位置ベクトルを$\bm{A},\ \bm{B}$とする.$\mathrm{A,\ B}$両方の点を通る直線の方程式は
	\begin{equation*}
		\bm{r} = (1 - \lambda)\bm{A} + \lambda\bm{B}
	\end{equation*}
	であることを証明せよ.
	\item 1つの平面($xy$平面)内にあるベクトル$\bm{A}$の成分が$A_x = A\cos\omega t,\ A_y = A\sin \omega t$($A,\ \omega$は定数)で与えられるとき$\bm{A}$と$\bunsuu{d\bm{A}}{dt}$とは互いに直角になっていることを証明せよ.
\end{enumerate}



\section{運動の法則}
\subsection{慣性の法則(運動の第1法則)}

\begin{tcolorbox}[colback=white]
	すべての物体は,加えられた力によってその状態が変化させられない限り,静止或いは等速直線運動の状態を続ける(\textbf{慣性系の存在}).
\end{tcolorbox}

2つの座標系$\mathrm{S}$系:$\mathrm{O}\text{-}xyz$と$\mathrm{S'}$系:$\mathrm{O}\text{-}x'y'z'$を考える.$\mathrm{S'}$系は$\mathrm{S}$系に対して並進運動(平行移動)をしていると考える.

このとき,空間内に質点$m$があり,力$\bm{F}$が作用しているとする.$\mathrm{S}$系での位置ベクトルは$\bm{r}$,$\mathrm{S'}$系での位置ベクトルは$\bm{r}'$である.また,$\mathrm{O}$から見た$\mathrm{O'}$の位置ベクトルを$\bm{R}$とする.すると
\begin{equation}
	\bm{r} = \bm{R} + \bm{r}'
\end{equation}
の関係がある.これを用いると速度,加速度はそれぞれ
\begin{gather}
	\bunsuu{d\bm{r}}{dt} = \bunsuu{d\bm{R}}{dt} + \bunsuu{d\bm{r}'}{dt}\\
	\bunsuu{d^2\bm{r}}{dt^2} = \bunsuu{d^2\bm{R}}{dt^2} + \bunsuu{d^2\bm{r}'}{dt^2}
\end{gather}
となる.従って,運動方程式(\ref{sec:NR1-2-3}節を参照)より
\begin{equation}
	m\bunsuu{d^2\bm{r}}{dt^2} = m\bunsuu{d^2\bm{R}}{dt^2} + m\bunsuu{s^2\bm{r}'}{dt^2} = \bm{F}
\end{equation}

\subsubsection*{$\mathrm{S}$系に対して$\mathrm{S}'$系が等速直線運動をしている場合}

このとき$\bm{R}$の加速度は$\bm{0}$なので
\begin{equation*}
	\bunsuu{d^2\bm{R}}{dt^2} = \bm{0}
\end{equation*}
よって
\begin{gather}
	\text{$\mathrm{S}$系\qquad} m\bunsuu{d^2\bm{r}}{dt^2} = \bm{F}\\
	\text{$\mathrm{S}'$系\qquad} m\bunsuu{d^2\bm{r}'}{dt^2} = \bm{F}
\end{gather}

従って,どちらの系でも同様に運動を記述できる.


\subsubsection*{$\mathrm{S}$系に対して$\mathrm{S}'$系が加速度運動をしている場合}

このとき
\begin{equation*}
	\bunsuu{d^2\bm{R}}{dt^2} \ne \bm{0}
\end{equation*}
なので
\begin{gather}
	\text{$\mathrm{S}$系\qquad} m\bunsuu{d^2\bm{r}}{dt^2} = \bm{F}\\
	\text{$\mathrm{S}'$系\qquad} m\bunsuu{d^2\bm{r}'}{dt^2} = \bm{F} - m\bunsuu{d^2\bm{R}}{dt^2}
\end{gather}

力が働かない場合($\bm{F} = \bm{0}$)を考えると
\begin{gather}
	\text{$\mathrm{S}$系\qquad} m\bunsuu{d^2\bm{r}}{dt^2} = \bm{0}\\
	\text{$\mathrm{S}'$系\qquad} m\bunsuu{d^2\bm{r}'}{dt^2} =- m\bunsuu{d^2\bm{R}}{dt^2}
\end{gather}
$\mathrm{S}$系では等速直線運動,$\mathrm{S'}$系では加速度運動が観測される.従って$- m\bunsuu{d^2\bm{R}}{dt^2}$を見かけの力(\textbf{慣性力})とする.第1法則が成り立つ系を\textbf{慣性系},成り立たない系を\textbf{非慣性系}という.

非慣性系で運動方程式を記述するには
\begin{equation}
	m\bunsuu{d^2\bm{r}'}{dt^2} = \bm{F}' = \bm{F} - m\bunsuu{d^2\bm{R}}{dt^2}
\end{equation}
と置き換える.

慣性系の問題の解き方
\begin{enumerate}[label=\textbf{[\arabic*]}, labelsep=10pt, leftmargin=23pt]
	\item 慣性系から見た動体の加速度$\alpha$を書き入れ,全ての力を書き込んで運動方程式を立てる.
	\item 非慣性系から見た,動体の中にある物体に働く慣性力を書き込む.
	\item 物体に働く慣性力以外の力を書き込む.
	\item 慣性力を含む物体の運動方程式を立てる(非慣性系の運動方程式).静止している場合はつり合いの式を書く.
	\item 運動方程式(つり合いの式)を解く.
\end{enumerate}



\subsection{ガリレイ変換}

2つの慣性系
$\mathrm{S}(\mathrm{O},\ x,\ y,\ z)$と
$\mathrm{S}'(\mathrm{O}',\ x',\ y',\ z')$
で,$x \heikou x',\ y \heikou y',\ z \heikou z'$とし,$\mathrm{O}'$は$\mathrm{S}$の座標系で$(x_0,\ y_0,\ z_0)$にあり,一定の速度$\bm{v}_0 = (u,\ v,\ w)$であるとする.

任意の点$\mathrm{P}$の座標を$(x,\ y,\ z),\ (x',\ y',\ z')$とすれば
\begin{align}
	x &= x_0 + x' & y &= y_0 + y' & z &= z_0 + z' \label{equ:NR1-3}\\
	x' &= x - x_0 & y' &= y - y_0 & z' &= z - z_0 \label{equ:NR1-4}
\end{align}
である.これらを$t$で微分すると
\begin{align}
	u &= u_0 + u' & v &= v_0 + v' & w &= w_0 + w' \label{equ:NR1-5}\\
	u' &= u - u_0 & v' &= v - v_0 & w' &= w - w_0 \label{equ:NR1-6}
\end{align}
となる.式(\ref{equ:NR1-5})を更に$t$で微分すると$\bunsuu{du_0}{dt} = 0,\ \bunsuu{dv_0}{dt} = 0,\ \bunsuu{dw_0}{dt} = 0$なので
\begin{align}
	\bunsuu{du}{dt} &= \bunsuu{du'}{dt} &
	\bunsuu{dv}{dt} &= \bunsuu{dv'}{dt} &
	\bunsuu{dw}{dt} &= \bunsuu{dw'}{dt} \label{equ4-7}
\end{align}
となる.式(\ref{equ:NR1-3})~式(\ref{equ:NR1-7})を\textbf{ガリレイ変換}という.例えば,式(\ref{equ:NR1-6})は速度$u$で飛んでいる鳥を同方向に速度$u_0$で走っている列車から見ると相対的に$u - u_0$の速度で飛んでいるように見えるということ.

$\bm{v}$が一定であるとき,$\mathrm{S}$が慣性系ならば$\mathrm{S}'$も慣性系である.



\subsection{力と加速度(運動の第2法則)}
\label{sec:NR1-2-3}

\begin{tcolorbox}[colback=white]
	質点に他の物体から力が働いた結果,加速度が生じる.このとき
	\begin{equation}
		m\bunsuu{d^2\bm{r}}{dt} = \bm{F}
	\end{equation}
が成り立つ(\textbf{運動方程式}).
\end{tcolorbox}

《注》運動方程式は$\text{[結果]} = \text{[原因]}$というように書くことが多い.

運動の変化は,\text{運動量}$\bm{p} = m\bm{v}$を用いて
\begin{equation*}
	\varDelta \bm{p} = \bm{p}(t_2) - \bm{p}(t_1)
\end{equation*}
と表される.運動の変化は加えられた駆動力(=力×力を加えた時間)によって起こるので,
\begin{equation}
	\label{equ:NR1-8}
	\varDelta \bm{p} = \bm{F}\varDelta t
\end{equation}
と書くことができる.しかし,実際$\bm{F}$は変化するので積分を用いることで一般化ができる.力を時間で積分したものを\textbf{力積}$\bm{I}$といい
\begin{equation*}
	\bm{I} = \int_{t_1}^{t_2} \bm{F}\,dt
\end{equation*}
で定義される.つまり運動の変化$\varDelta\bm{p}$は力積$\bm{I}$に等しい.

運動の変化率は式(\ref{equ:NR1-8})から
\begin{equation*}
	\bunsuu{\varDelta \bm{p}}{\varDelta t} = \bm{F}
\end{equation*}
と書ける.$\varDelta t \to 0$のとき
\begin{equation}
	\bunsuu{d\bm{p}}{dt} = \bm{F}
\end{equation}
で,質点の運動量の時間微分は,その瞬間に加えられた力に等しいことを意味する.運動量の定義式よりこれは
\begin{equation}
	m\bunsuu{d\bm{v}}{dt} = \bm{F}
\end{equation}
と書くこともできる.

ここで,$\bm{F}$はその物体に加えられた力の\textbf{合力}を指し,$m$はその質点の\textbf{慣性質量}とする.

運動方程式は
\begin{equation*}
	\bunsuu{d^2\bm{r}}{dt} = \bunsuu{\bm{F}}{m}
\end{equation*}
より,$\bm{F}$が一定のとき質量が大きいほど加速度の変化が小さい.物体が運動の状態を続けようとする性質を\textbf{慣性}ということから$m$は慣性質量と呼ばれる.



\subsection{作用・反作用の法則(運動の第3法則)}

\begin{tcolorbox}[colback=white]
	2個の質点1,\ 2があり,互いに力を及ぼしているとき,質点1が質点2から受ける力$\bm{F}_{12}$は,質点2が質点1から受ける力$\bm{F}_{21}$と大きさが同じで向きが反対である.つまり
	\begin{equation}
		\bm{F}_{12} = -\bm{F}_{21}
	\end{equation}
	である(\textbf{作用・反作用の法則}).
\end{tcolorbox}

林檎が落下しているとき,林檎が地球から受ける力(重力)と地球が林檎から受ける力は作用・反作用の関係にある.また,林檎が机の上で静止しているとき,林檎が机から受ける力(垂直抗力)と机が林檎から受ける力も作用・反作用の関係にある.しかし,林檎が地球から受ける力(重力)と林檎が机から受ける力(垂直抗力)は作用・反作用の関係ではなく,つり合いの関係である.



\subsection{2節 問題}

\begin{enumerate}[label=\textbf{[\arabic*]}, labelsep=10pt, leftmargin=23pt]
	\item 滑らかな水平面上にある板(質量$M$)の上を人(質量$m$)が板に対して加速度$a$で歩くとき,板は水平面上に対してどのような加速度を持つか.また,人と板とが互いに水平に及ぼしあう力はどれだけか.
	\item 水平な滑らかな床の上に一様な鎖(質量$M$,長さ$l$)を一直線に置いてその一端を一定の力$F$で引っ張る.鎖の各点での張力を求めよ.
	\item 惑星が太陽から惑星の質量に比例し,太陽からの距離の2乗に反比例する引力を受けて太陽のまわりを円運動を行うものとする.いろいろな惑星が太陽の周りを回る周期$T$と,円運動の半径$a$との間には
	\begin{equation*}
		\bunsuu{T^2}{a^3} = \text{惑星によらない定数}
	\end{equation*}
	の関係があることを示せ.この関係はケプラーの第3法則に相当する.
	\item 太陽系は銀河系の中心から30000光年の距離で,およそ$250\,\mathrm{km\,s^{-1}}$の速さで銀河系の中心を中心として等速円運動をしている.銀河系の形は図のようになっており,太陽系は銀河系の各恒星からの万有引力を受けている.銀河系の恒星は空間に散らばっているが,大雑把にいって太陽系に働く力は,銀河系全体の質量がその中心に集中していると考えても大体の程度のことはいえるであろう.太陽のまわりの地球の運動の速度は$30.0\,\mathrm{km\,s^{-1}}$として,銀河系の総質量と太陽の質量との比を求めよ.
	\item 中性子星と呼ばれる星は中性子が万有引力によって結び付けられてたもので,原子核と同様な密度(およそ$10^{12}\,\mathrm{g\,cm^{-3}}$)を持つ.中性子星は球形で,自転しているとして,赤道で中性子星が飛び去らないための回転の周期の最小値を求めよ.
\end{enumerate}



\section{簡単な運動}
\subsection{落体の運動}

鉛直上方に$y$軸をとり,適当な高さの点を原点とする.質量$m$の質点を$y$軸上で運動させると下向きに加速度を持っているので,下向きに力が働く.よって運動方程式は
\begin{equation*}
	m\bunsuu{d^2y}{dt^2} = -F
\end{equation*}

加速度は物体によらず一定である.元々物体が慣性質量を持つことと,地球が物体を引っ張る(万有引力)ことは独立なことである.よって,「加速度が物体によらず一定であること」は,慣性質量$m$と重力($F$)が比例していなければならない.これは歴史の中で確かめられたので
\begin{equation*}
	F = mg
\end{equation*}
とすれば運動方程式は
\begin{equation}
	\bunsuu{d^2y}{dt^2} = -g
\end{equation}
となる.これを積分して
\begin{equation*}
	\bunsuu{dy}{dt} = -gt + C
\end{equation*}

ここで,初速度を$v_0$とすれば,$t = 0$を代入して
\begin{equation*}
	\bunsuu{dy}{dt} = C \iff v_0 = C
\end{equation*}
なので
\begin{equation*}
	\bunsuu{dy}{dt} = -gt + v_0
\end{equation*}
となる.これを更に積分して
\begin{equation*}
	y = -\bunsuu{1}{2}gt^2 + v_0t + C'
\end{equation*}

投げ出した時の位置を原点とすれば,$t = 0$を代入して
\begin{equation*}
	0 = C'
\end{equation*}
よって
\begin{equation*}
	y = -\bunsuu{1}{2}gt^2 + v_0t
\end{equation*}
となる.



\subsection{粘性抵抗力が働く場合の落体運動}

物体の運動が遅いとき,\textbf{粘性抵抗力}がはたらき
\begin{equation}
	\bm{F} = -\alpha\bm{v}
\end{equation}
の形で与えられる.また,物体の運動が速いときは\textbf{慣性抵抗力}がはたらき
\begin{equation}
	F =
	\left\{
		\begin{array}{ll}
			-\beta v^2 & \text{$v > 0$のとき}\\
			+\beta v^2 & \text{$v < 0$のとき}
		\end{array}
	\right.
\end{equation}
の形で与えられる.

自由落下で,空気抵抗がある場合を考える.鉛直下向きに$y$軸をとると,運動方程式は
\begin{equation}
	m\bunsuu{dv}{dt} = mg - \alpha v
\end{equation}
である.変数分離して
\begin{align*}
	\bunsuu{dv}{dt} = g - \bunsuu{\alpha v}{m} &
	\iff dv = \left(g - \bunsuu{\alpha v}{m}\right)dt\\
	& \iff \bunsuu{dv}{g - \bunsuu{\alpha}{m}v} = dt
\end{align*}
両辺積分すると
\begin{align*}
	\int \bunsuu{dv}{g - \bunsuu{\alpha}{m}v}\ = \int dt
	&\iff -\bunsuu{m}{\alpha}\log\left|g - \bunsuu{\alpha}{m}v\right| = t + C\\
	&\iff \log\left|g - \bunsuu{\alpha}{m}v\right| = -\bunsuu{\alpha}{m}t + C\\
	&\iff g - \bunsuu{\alpha}{m}v = \exp\left(-\bunsuu{\alpha}{m}t + C\right)\\
	&\iff \bunsuu{\alpha}{m} v = g - \exp\left(-\bunsuu{\alpha}{m}t + C\right)\\
	&\iff v = \bunsuu{m}{\alpha}\left\{g - \exp\left(-\bunsuu{\alpha}{m}t + C\right)\right\} = \bunsuu{m}{\alpha}\left\{g - \exp\left(-\bunsuu{\alpha}{m}t\right)\exp(C)\right\}
\end{align*}
ここで,初期条件より$t = 0,\ v = 0$なので
\begin{equation*}
	0 = \bunsuu{m}{\alpha}\{g - \exp(C)\} \iff \exp(C) = g
\end{equation*}
よって,
\begin{equation}
	v = \bunsuu{mg}{\alpha}\left\{1 - \exp\left(-\bunsuu{\alpha}{m}t\right)\right\} \label{equ:NR1-9}
\end{equation}
となる.

式(\ref{equ:NR1-9})で,$t \to \infty$とすると
\begin{equation}
	v_{\infty} = \bunsuu{mg}{\alpha}
\end{equation}
となる.$v_{\infty}$を\textbf{終端速度}といい,これより大きくなることはない.



\subsection{慣性抵抗力が働く場合の落体運動}

半径が大きくなると粘性抵抗力より慣性抵抗力の方が支配的になる.よって,運動方程式は
\begin{equation}
	m\bunsuu{dv}{dt} = mg - \beta v^2
\end{equation}
となる.先程と同じように変数分離して
\begin{equation*}
	\bunsuu{dv}{g - \bunsuu{\beta}{m}v^2} = dt
\end{equation*}
両辺積分すると
\begin{equation*}
	\int \bunsuu{dv}{g - \bunsuu{\beta}{m}v^2} = \int dt
\end{equation*}
\vskip-1.5\baselineskip
\begin{align*}
	\text{左辺} &= \int \bunsuu{dv}{\Bigl(\sqrt{g} + \sqrt{\myfrac{\beta}{m}}\,v\Bigr)\Bigl(\sqrt{g} - \sqrt{\myfrac{\beta}{m}}\,v \Bigr)}
	% 部分分数分解
	= \bunsuu{1}{2\sqrt{g}} \int \left(
		\bunsuu{1}{
			\sqrt{g} + \sqrt{\myfrac{\beta}{m}}\,v
		} +
		\bunsuu{1}{
			\sqrt{g} - \sqrt{\myfrac{\beta}{m}}\,v
		}
	\right)\,dv\text{\hspace*{2\zw} (部分分数分解)}\\
	&= \bunsuu{1}{2\sqrt{g}}\left(
		\sqrt{\bunsuu{m}{\beta}}\log\left|\sqrt{g} + \sqrt{\bunsuu{\beta}{m}}\,v\right| - \sqrt{\bunsuu{m}{\beta}}\log\left|\sqrt{g} - \sqrt{\bunsuu{\beta}{m}}\,v\right|
	\right)\\
	&= \bunsuu{1}{2}\sqrt{\bunsuu{m}{g\beta}}\biggl(
		\log\biggl|\sqrt{g} + \sqrt{\myfrac{\beta}{m}}\,v\biggr| - \log\biggl|\sqrt{g} - \sqrt{\myfrac{\beta}{m}}\,v\biggr|
	\biggr) =
	\bunsuu{1}{2}\sqrt{\bunsuu{m}{g\beta}}\log\left|
		\bunsuu{\sqrt{g} + \sqrt{\myfrac{\beta}{m}}\,v}{\sqrt{g} - \sqrt{\myfrac{\beta}{m}}\,v}
	\right|\\
	\text{右辺} &= t + C
\end{align*}
よって
\begin{align*}
	\bunsuu{1}{2}\sqrt{\bunsuu{m}{g\beta}}\log\left|
		\bunsuu{\sqrt{g} + \sqrt{\myfrac{\beta}{m}}\,v}{\sqrt{g} - \sqrt{\myfrac{\beta}{m}}\,v}
	\right| = t + C
	& \quad\stackrel{2C\sqrt{\myfrac{g\beta}{m}} = C'}{\iff}\quad \log\left|
		\bunsuu{\sqrt{g} + \sqrt{\myfrac{\beta}{m}}\,v}{\sqrt{g} - \sqrt{\myfrac{\beta}{m}}\,v}
	\right| = 2t\sqrt{\bunsuu{g\beta}{m}} + C'\\
	&\qquad\iff \bunsuu{\sqrt{g} + \sqrt{\myfrac{\beta}{m}}\,v}{\sqrt{g} - \sqrt{\myfrac{\beta}{m}}\,v} = e^{2t\sqrt{\myfrac{g\beta}{m}} + C'}\\
	&\qquad\iff \sqrt{g} + \sqrt{\bunsuu{\beta}{m}}\,v = \biggl(\sqrt{g} - \sqrt{\bunsuu{\beta}{m}}\,v\biggr)e^{2t\sqrt{\myfrac{g\beta}{m}} + C'}
\end{align*}
\begin{align*}
	\text{(前頁の続き)}
	&\iff \sqrt{\bunsuu{\beta}{m}}\,v + \sqrt{\bunsuu{\beta}{m}}\,v e^{2t\sqrt{\myfrac{g\beta}{m}} + C'} = \sqrt{g}\,e^{2t\sqrt{\myfrac{g\beta}{m}} + C'} - \sqrt{g}\\
	&\iff \sqrt{\bunsuu{\beta}{m}}\,v \left(1 + e^{2t\sqrt{\myfrac{g\beta}{m}} + C'}\right) = \sqrt{g}\left(e^{2t\sqrt{\myfrac{g\beta}{m}} + C'} - 1\right)\\
	&\iff v = -\sqrt{\bunsuu{mg}{\beta}}
	\bunsuu{
		1 - e^{2t\sqrt{\myfrac{g\beta}{m}} + C'}
	}{
		1 + e^{2t\sqrt{\myfrac{g\beta}{m}} + C'}
	} 
\end{align*}
となる.初期条件より$t = 0,\ v = 0$なので
\begin{align*}
	0 = 1 - e^{C'} \iff C' = 0
\end{align*}
よって
\begin{equation}
	v = -\sqrt{\bunsuu{mg}{\beta}}
	\bunsuu{
		1 - e^{2t\sqrt{\myfrac{g\beta}{m}}}
	}{
		1 + e^{2t\sqrt{\myfrac{g\beta}{m}}}
	} 
\end{equation}
となる.ここで,分子分母に$e^{-2t\sqrt{\myfrac{g\beta}{m}}}$をかけると
\begin{equation*}
	v = -\sqrt{\bunsuu{mg}{\beta}}
	\bunsuu{
		e^{-2t\sqrt{\myfrac{g\beta}{m}}} - 1
	}{
		e^{-2t\sqrt{\myfrac{g\beta}{m}}} + 1
	} = \sqrt{\bunsuu{mg}{\beta}}
	\bunsuu{
		1 - e^{-2t\sqrt{\myfrac{g\beta}{m}}}
	}{
		1 + e^{-2t\sqrt{\myfrac{g\beta}{m}}}
	}
\end{equation*}
なので,$t \to \infty$とすると
\begin{equation}
	v_{\infty} = \sqrt{\bunsuu{mg}{\beta}}
\end{equation}
と,終端速度が求められる.



\subsection{放物運動}

質量$m$の物体を,仰角(地表と成す角)$\theta\ \left(0 < \theta < \bunsuu{\pi}{2}\right)$,速さ$v_0$で放り投げた場合の運動を考える.

空気抵抗を無視すると,任意の点での物体に加わる力は重力だけなので運動方程式は次のようになる.
\begin{align}
	&m\bunsuu{d^2 x}{dt^2} = 0 & &m\bunsuu{d^2 y}{dt^2} = 0 & &m\bunsuu{d^2 z}{dt^2} = -mg
\end{align}
これらの微分方程式を解くと次のようになる.
\begin{align*}
	\bunsuu{d^2 x}{dt^2} &= 0 & \bunsuu{d^2 y}{dt^2} &= 0 & \bunsuu{d^2 z}{dt^2} &= -g\\
	\bunsuu{dx}{dt} &= C_x & \bunsuu{dy}{dt} &= C_y & \bunsuu{dz}{dt} &= -gt + C_z\\
	x &= C_x\,t + D_x & y &= C_y\,t + D_y & z &= -\bunsuu{1}{2}gt^2 + C_z\,t + D_z
\end{align*}
放物運動は2次元平面内の運動なので,$xz$平面内での運動と考えると,$t = 0$のとき$\bm{r} = (0,\ 0,\ 0)$,$\bm{v}_0 = (v_0\cos\theta,\ 0,\ v_0\sin\theta)$なので,代入すると
\begin{align*}
	\bunsuu{d}{dt}x(0) &= v_0\cos\theta = C_x &
	\bunsuu{d}{dt}y(0) &= 0 = C_y &
	\bunsuu{d}{dt}z(0) &= v_o\sin\theta = C_z\\
	x(0) &= 0 = D_x &
	y(0) &= 0 = D_y &
	z(0) &= 0 = D_z
\end{align*}
なので,特殊解は
\begin{align}
	x &= v_0t\cos\theta & y &= 0 & z &= v_0 t\sin\theta
\end{align}
となる.



\subsection{粘性抵抗力が働く場合の放物運動}

図を描くと,任意の点での物体に加わる力は重力$m\bm{g}$と粘性抵抗力$-\alpha\bm{v}$である.水平方向を$x$軸,鉛直方向を$z$軸とすると
\begin{align*}
	m\bm{g} &= -mg\bm{k} & -\alpha\bm{v} = -\alpha v_x \bm{i} - \alpha v_z\,\bm{k}
\end{align*}
となる.また,初期条件は先程と同じとする.運動方程式は
\begin{align}
	m\bunsuu{d v_x}{dt} &= -\alpha v_x & m\bunsuu{d v_z}{dt} &= -mg - \alpha v_z
\end{align}
となる.










\subsection{3節問題}

\begin{enumerate}[label=\textbf{[\arabic*]}, labelsep=10pt, leftmargin=23pt]
	\item 全質量$M$の風船が$\alpha$の加速度で落ちている.逆に上向きに加速度$\alpha$の運動をするためにはどれだけの質量の砂袋を捨てなければならないか.
	\item 軽い定滑車に糸をかけてその両端に質量$m_1,\ m_2$の質量をつるして放す.両質点の加速度を求めよ.また,糸の張力を求めよ.(この装置をアトウッドの装置とよぶ).
	\item 前の問題で滑車を$\beta$の加速度で引き上げるとき,両質点の滑車に対する加速度と糸の張力はどうなるか.
	\item 地上から一定の速さで石を投げるとき地面の達することのできる区域の面積は$S_0$である.地上から上方$h$のところから同じ速さで投げると区域は$S_h = S_0 + 2h\sqrt{\pi s_0}$で与えられることを証明せよ.
	\item 物体を投げるときの初速を知りたいがこれを直接に測ることが難しい.それで投射距離と時間を測定してこれを求めたいと考える.公式を求めよ.
	\item 図に示すように,正,負に帯電した平行金属板(偏向板)の間に電子(質量$m$)を両板に走らせる.電子には一方の力$eE$($e$:電子の荷電,$E$:電場の強さ)が負の方から正の方に働く.電子が偏向板の間を$l$だけ走ってその端に来たとき,はじめ目指していた位置からどれだけずれるか.またそのときはじめの方向とどれだけの角をつくる方向に運動するか.
	\item 空気抵抗が速さに比例する大きさ($kmV$)を持つときの放物運動で,抵抗が小さいとして放物距離の近似式を求めよ.
	\item 放物運動を行う物体に及ぼす空気の抵抗が$m\phi(V)$(ただし,$\phi$は任意の関数)であるとき,速さ$V$,鉛直線と軌道の接線のつくる角$\psi$の関係は
		\begin{equation*}
			\bunsuu{1}{V}\bunsuu{dV}{d\psi} = -\bunsuu{\psi(V)}{g\sin \psi} - \cot \psi
		\end{equation*}
		を積分することによって求められることを示せ.
	\item 前の問題で$\phi(V) = mkV^2$の場合どうなるかを論ぜよ.
	\item 角振動数$\omega_0$で単振動を行っている質点に,角振動数$\omega_1,\ \omega_2$の周期的な2つの力が作用するときこの質点はどのような運動を行うか.
	\item 上の問題で質点に$T$を周期とする周期的な力$f(t)$が働く時を考えよ.$f(t)$の平均値は$0$とする.
\end{enumerate}



\section{運動方程式の変換}
\subsection{4節問題}

\begin{enumerate}[label=\textbf{[\arabic*]}, labelsep=10pt, leftmargin=23pt]
	\item 螺旋
		\begin{equation*}
			x = a\cos \phi,\quad y = a\sin\phi,\quad z = k\phi \quad \text{($a,\ k$は定数)}
		\end{equation*}
		の接線,主法線,陪法線の方向を求めよ.\\
		この螺旋に沿って上向きに一定の速さ$V$で昇る点の加速度を求めよ.
	\item 環面
		\begin{equation*}
			x = (c + a\sin\theta)\cos\phi,\quad y = (c + a\sin\theta)\sin\phi,\quad z = a\cos\theta
		\end{equation*}
		の上を運動する点の子午線方向($\phi = \text{一定}$)で$\theta$だけが増す方向),法線方向,方位角方向($\phi$だけが増す方向)の加速度成分を求めよ.
	\item $(x,\ y)$面を運動する点の描く軌道が$r = a\sin n\phi$($a,\ n$は定数)で与えられ,加速度$\dot{\phi}$が$r^2$に反比例するとき,この点の加速度を求めよ.
\end{enumerate}




\chapter{電磁気学I}
\setcounter{page}{1}


電磁気学に入る前に,この章で必要なベクトル解析の知識を確認する.詳しくはベクトル解析の章を見ていただきたい.



\section{諸定義}
\subsection{ハミルトンの演算子}

位置ベクトルを$\bm{r} =
\begin{bmatrix}
	x\\ y\\ z
\end{bmatrix}$とする.以下で定義される微分演算子のベクトルを\textbf{ハミルトンの演算子}といい,\textbf{ナブラ}と読む.

\begin{equation}
	\bm{\nabla} =
	\begin{bmatrix}
		\bunsuu{\partial}{\partial x} &
		\bunsuu{\partial}{\partial y} &
		\bunsuu{\partial}{\partial z}
	\end{bmatrix}^\top
\end{equation}

ナブラの定義や逆三角形の2辺の太さが太いことから,$\bm{\nabla}$はベクトルの形をしていることは明らかである.
勿論,$\bunsuu{\partial}{\partial x}$だけでは成り立たず,$\bunsuu{\partial f}{\partial x}$のように微分される関数がないといけない.では何故こんなものがあるのかというと,後の勾配,発散,回転の式を表すのに便利だからだ.これらは電磁気学で出てくる大切な概念である.

例えば,$\bm{\nabla} f$と表せば,ベクトルのスカラー倍と見做すことができ
\begin{equation*}
	\bm{\nabla} f =
	\begin{bmatrix}
		\bunsuu{\partial}{\partial x} &
		\bunsuu{\partial}{\partial y} &
		\bunsuu{\partial}{\partial z}
	\end{bmatrix}^\top
	f
	=
	\begin{bmatrix}
		\bunsuu{\partial f}{\partial x} &
		\bunsuu{\partial f}{\partial y} &
		\bunsuu{\partial f}{\partial z}
	\end{bmatrix}^\top
\end{equation*}
と,スカラー関数$f$を$x,\ y,\ z$で偏微分したものをそれぞれの成分とするベクトルを表すことができる.尚,これは勾配の定義そのものである.

また,ベクトル関数$\bm{F}$を内積の形でかければ,
\begin{equation*}
	\bm{\nabla} \cdot \bm{F} =
	\begin{bmatrix}
		\bunsuu{\partial}{\partial x} &
		\bunsuu{\partial}{\partial y} &
		\bunsuu{\partial}{\partial z}
	\end{bmatrix}^\top
	\cdot
	\begin{bmatrix}
		F_x & F_y & F_z
	\end{bmatrix}^\top
	= \bunsuu{\partial F_x}{\partial x} + \bunsuu{\partial F_y}{\partial y} + \bunsuu{\partial F_z}{\partial z}
\end{equation*}
となる.これは,発散の定義である.

外積の形でかければ
\begin{align*}
	\bm{\nabla} \times \bm{F} &=
	\begin{bmatrix}
		\bunsuu{\partial}{\partial x} &
		\bunsuu{\partial}{\partial y} &
		\bunsuu{\partial}{\partial z}
	\end{bmatrix}^\top
	\times
	\begin{bmatrix}
		F_x & F_y & F_z
	\end{bmatrix}^\top
	=
	\begin{vmatrix}
		\bm{i} & \bunsuu{\partial}{\partial x} & F_x\\[5mm]
		\bm{j} & \bunsuu{\partial}{\partial y} & F_y\\[5mm]
		\bm{k} & \bunsuu{\partial}{\partial z} & F_z\\
	\end{vmatrix}\\
	&=
	\bunsuu{\partial F_z}{\partial y}\bm{i} +
	\bunsuu{\partial F_x}{\partial z}\bm{j} +
	\bunsuu{\partial F_y}{\partial x}\bm{k} -
	\bunsuu{\partial F_y}{\partial z}\bm{i} -
	\bunsuu{\partial F_z}{\partial x}\bm{j} -
	\bunsuu{\partial F_x}{\partial y}\bm{k}\\
	&=
	\left(
		\bunsuu{\partial F_z}{\partial y} -
		\bunsuu{\partial F_y}{\partial z}
	\right)\bm{i}
	+
	\left(
		\bunsuu{\partial F_x}{\partial z} -
		\bunsuu{\partial F_z}{\partial x}
	\right)\bm{j}
	+
	\left(
		\bunsuu{\partial F_y}{\partial x} -
		\bunsuu{\partial F_x}{\partial y}
	\right)\bm{k}
	=
	\begin{bmatrix}
		\bunsuu{\partial F_z}{\partial y} -
		\bunsuu{\partial F_y}{\partial z}\\[5mm]
		\bunsuu{\partial F_x}{\partial z} -
		\bunsuu{\partial F_z}{\partial x}\\[5mm]
		\bunsuu{\partial F_y}{\partial x} -
		\bunsuu{\partial F_x}{\partial y}
	\end{bmatrix}
\end{align*}
となる.これは,回転の定義である.



\subsection{ラプラシアン(Laplacian)}

以下で定義される微分演算子を\textbf{ラプラシアン}という.
\begin{equation}
	\Delta = \bm{\nabla} \cdot \bm{\nabla} = \bunsuu{\partial^2}{\partial x} + \bunsuu{\partial^2}{\partial y} + \bunsuu{\partial^2}{\partial z}
\end{equation}



\section{ベクトル場の微分:勾配,発散,回転}
\subsection{勾配(グラディエント)}

スカラー場$\varphi(x,\ y,\ z)$が直交座標で定義されているとする.このとき,$x$についての偏微分を$x$成分,$y$についての偏微分を$y$成分,$z$についての偏微分を$z$成分とするベクトル場を$\varphi$の\textbf{勾配}または\textbf{グラディエント}といい,$\grad \varphi$や$\bm{\nabla} \varphi$で表す.
\begin{equation}
	\bm{\nabla} \varphi = \grad \varphi =
	\begin{bmatrix}
		\bunsuu{\partial \varphi}{\partial x} &
		\bunsuu{\partial \varphi}{\partial y} &
		\bunsuu{\partial \varphi}{\partial z}
	\end{bmatrix}^\top
\end{equation}

勾配はそれぞれの点において$\varphi$が最も増大する方向を指し示す.勾配の向きは等位面に垂直で,大きさはどれくらい増大しているかを表す.等位面の間隔が狭いとき,勾配が急峻になっているので,$\bm{\nabla} \varphi$の大きさ(矢印の長さ)が大きく,間隔が広いとき,勾配が緩やかなので,$\bm{\nabla} \varphi$の大きさは小さい.

《注》勾配は,入力をスカラー場として,ベクトル場を出力する.



\subsection{発散(ダイヴァージェンス)}

ベクトル場$\bm{F}(x,\ y,\ z)$が直交座標で定義されているとする.このとき,$\bm{F}$の$x$成分$F_x$を$x$で,$y$成分$F_y$を$y$で,$z$成分$F_z$を$z$で偏微分したあと,それらを足し合わせたものを$\bm{F}$の\textbf{発散}または\textbf{ダイヴァージェンス}といい,$\dive \bm{F}$や$\bm{\nabla} \cdot \bm{F}$で表す.
\begin{equation}
	\bm{\nabla} \cdot \bm{F} = \dive \bm{F} = \bunsuu{\partial F_x}{\partial x} + \bunsuu{\partial F_y}{\partial y} + \bunsuu{\partial F_z}{\partial z}
\end{equation}

発散は,それぞれの点においてベクトル場$\bm{F}$の流入出を評価する.例えば,$\bm{v}(x,\ y,\ z)$は点$(x,\ y,\ z)$での流体の速度を示すとする.ある空間に3辺の長さが$\varDelta x,\ \varDelta y,\ \varDelta z$の微小な直方体を考える.この直方体に流入する流体の量を$V_{\mathrm{in}}$,流出する流体の量を$V_{\mathrm{out}}$とするとき,$V_{\mathrm{out}} - V_{\mathrm{in}} > 0$なら,流出量の方が多いので,この直方体の中に蛇口のようなものがあってそこから流体が「湧き出て」いることを表す.逆に$V_{\mathrm{out}} - V_{\mathrm{in}} < 0$なら,流出量の方が少ないので,この直方体の中に排水溝のようなものがあってそこに流体が「吸い込まれて」いることを表す.証明は略すが,この直方体の\textcolor{cyan}{中}から流れ出る流体は
\begin{equation*}
	V_{\mathrm{out}} - V_{\mathrm{in}} = 
	\left(
		\bunsuu{\partial v_x}{\partial x} + \bunsuu{\partial v_y}{\partial y} + \bunsuu{\partial v_z}{\partial z}
	\right)
	\varDelta x \varDelta y \varDelta z
	= (\dive \bm{v}) \varDelta x \varDelta y \varDelta z
\end{equation*}
で表すことができる.直方体の体積は$\varDelta x \varDelta y \varDelta z$なので,これで割ると$\dive \bm{v}$は\textcolor{teal}{単位時間における単位体積あたりでの直方体の中から流出した流体の体積}を表す.$\dive\bm{v} > 0$のとき「湧き出し」,$\dive\bm{v} < 0$のとき「吸い込む」.$\dive\bm{v} = 0$のときは流れ入った流体はそのまま流れ出る.

《注》発散は,入力をベクトル場として,スカラー場を出力する.



\subsection{回転(ローテーション)}

ベクトル場$\bm{F}(x,\ y,\ z)$が直交座標で定義されているとする.このとき,次のように定義されるものを$\bm{F}$の\textbf{回転}または\textbf{ローテーション}といい,$\rot \bm{F}$や$\bm{\nabla} \times \bm{F}$で表す.
\begin{equation}
	\bm{\nabla} \times \bm{F} =
	\begin{vmatrix}
		\bm{i} & \bunsuu{\partial}{\partial x} & F_x\\[5mm]
		\bm{j} & \bunsuu{\partial}{\partial y} & F_y\\[5mm]
		\bm{k} & \bunsuu{\partial}{\partial z} & F_z\\
	\end{vmatrix}
	=
	\begin{bmatrix}
		\bunsuu{\partial F_z}{\partial y} -
		\bunsuu{\partial F_y}{\partial z}\\[5mm]
		\bunsuu{\partial F_x}{\partial z} -
		\bunsuu{\partial F_z}{\partial x}\\[5mm]
		\bunsuu{\partial F_y}{\partial x} -
		\bunsuu{\partial F_x}{\partial y}
	\end{bmatrix}
\end{equation}

回転は,それぞれの点においてベクトル場$\bm{F}$の回転の向きや強さを表している.$\rot \bm{F} \ne \bm{0}$のとき$\bm{F}$が回転軸に対して渦巻いている,つまり回転する要素があるということになる.

《注》回転は,入力をベクトル場として,ベクトル場を出力する.



\section{線積分,面積分,体積分}
\subsection{線積分}

曲線$C: \bm{r}(t) =
\begin{bmatrix}
	x(t)\\ y(t)\\ z(t)
\end{bmatrix}\ (\alpha \le t \le \beta)$に沿うスカラー場$\varphi(\bm{r})$の\textbf{線積分}は,$s$を曲線の長さとすると
\begin{equation}
	\int_{C} \varphi\,\md s = \int_{\alpha}^{\beta} \varphi\,\bunsuu{\md s}{\md t}dt = \int_{\alpha}^{\beta} \varphi\,\left|\bunsuu{\md\bm{r}}{\md t}\right|\md t
\end{equation}

また,ベクトル場$\bm{A}(\bm{r})$の線積分は,$\md\bm{r} =
\begin{bmatrix}
	\md x\\ \md y\\ \md z
\end{bmatrix}
$とすると
\begin{equation}
	\int_{C} \bm{A} \cdot \md\bm{r} = \int_{\alpha}^{\beta} \bm{A} \cdot \bunsuu{\md\bm{r}}{\md t}\,\md t
\end{equation}



\section{クーロン力と電場}
\subsection{クーロン力(静電気力)}

クーロン力が起こる原因になるものを\textbf{電荷}という.電荷には正負があり,正の電荷をもつ代表的なものに陽子,負の電荷をもつ代表的なものに電子がある.電荷の量を\textbf{電荷量}といい,$q$と表す.単位は$\mathrm{C}$である.

$1\,\mathrm{C}$は$1\,\mathrm{A}$の電流を1秒間流した時に流れる電荷量なので
\begin{equation}
	1\,\mathrm{C} = 1\,\mathrm{A \cdot s}
\end{equation}
である.

陽子と電子の電荷量は大きさが同じで向きが反対である.陽子と電子の電荷の大きさを\textbf{電気素量}といい,$e$で表す.
\begin{equation}
	e = 1.602 \times 10^{-19}\,\mathrm{C}
\end{equation}

つまり,陽子の電荷量は$+e$,電子の電荷量は$-e$である.

電荷は陽子と陽子,電子と電子のように同符号であるとき,2つの電荷の間に反発する力(斥力)が働く.また,陽子と電子のように異符号であるとき,2つの電荷の間に引き合う力(引力)が働く.このような力を\textbf{クーロン力}といい,以下の法則が成り立つ.

\begin{kousiki}{クーロンの法則}
	2つの点電荷があり,それぞれの電荷量を$Q,\ q$とする.$Q$から$q$へ向かうベクトルを$\bm{r}$とするとき,点電荷$q$に働く\textbf{クーロン力}は
	\begin{equation}
		\bm{F} = {\color{teal}
			\bunsuu{1}{4\pi \varepsilon_0}\bunsuu{qQ}{|\bm{r}|^2}
		}
		\cdot 
		{\color{cyan}
			\bunsuu{\bm{r}}{|\bm{r}|}
		}
	\end{equation}
	である.$\varepsilon_0$は真空の誘電率である.また,\textcolor{teal}{青緑色}はクーロン力の大きさ,\textcolor{cyan}{水色}はクーロン力の向きを表す.
\end{kousiki}



\subsubsection*{クーロン力の大きさ}

$\bunsuu{1}{4\pi\varepsilon_0}$は定数である.
$\varepsilon_0$は真空の誘電率\footnote{真空の誘電率についてはここでは触れない.そういうものだと認識してもよい.なお,真空の誘電率は,$c$を光の速さ,$\mu_0$を真空の透磁率とすると$\varepsilon_0 = \bunsuu{1}{c^2\mu_0}$で求められる.} で,
$\bunsuu{1}{4\pi\varepsilon_0} \approx 8.987552 \times 10^9\, \mathrm{V^2 / N}$である.この定数値はクーロンの法則が成り立つように辻褄合わせで決められた値であるのでそこまで深く考える必要はない.クーロン力は電荷量に比例し,2点間の距離$|\bm{r}|$の2乗に反比例する.



\subsubsection*{クーロン力の向き}

$Q$と$q$が同符号の場合,$q$には引力が働くので向きは$\bm{r}$と同じ向きになる.また,\textcolor{cyan}{水色}は向きのみを表すため,大きさが変わってはいけないので絶対値で割って大きさを$1$にする.



\subsubsection*{重ね合わせの原理}

電荷が$q,\ Q_1,\ Q_2,\ \cdots,\ Q_n$と複数個あったとする.また,$Q_1$が$q$に及ぼすクーロン力を$\bm{F}_1$,…,$Q_n$が$q$に及ぼすクーロン力を$\bm{F}_n$とするとき,$q$が受けるクーロン力はそれぞれのクーロン力の和で表すことができる(\textbf{重ね合わせの原理}).

\begin{kousiki}{クーロン力の重ね合わせの原理}
	$Q_i$から$q$へ向かうベクトルを$\bm{r}_i$とするとき
	,点電荷$q$が受けるクーロン力は,
	\begin{align}
		\bm{F} &= \sum_{i = 0}^{n} \bm{F}_i\\
			&= \bm{F}_1 + \bm{F}_2 + \cdots + \bm{F}_n \notag\\
			&= \bunsuu{q}{4\pi\varepsilon_0} \sum_{i = 0}^{n} \bunsuu{Q_i}{|\bm{r}_i|^2} \cdot \bunsuu{\bm{r}_i}{|\bm{r}_i|}
	\end{align}
\end{kousiki}

\begin{enumerate}[leftmargin=18pt, labelsep=10pt, itemindent=9pt]
	\item[\f{例}] 点$\mathrm{A}_1(a,\ 0)$に点電荷$Q$,点$\mathrm{A}_2(-a,\ 0)$に点電荷$Q$,点$\mathrm{D}(0,\ d)$に点電荷$q$がある.このとき,$q$にかかるクーロン力を求めよ.\\
		$\vecrm{A_1 D} = \bm{r}_1 = \begin{bmatrix*}[r] -a\\ d \end{bmatrix*},\ \vecrm{A_2 D} = \bm{r}_2 = \begin{bmatrix*}[r] a\\ d \end{bmatrix*}$とする.よって
		\begin{align*}
			\bm{F} &= \bm{F}_1 + \bm{F}_2\\
				&= \bunsuu{1}{4\pi\varepsilon_0}\bunsuu{qQ}{|\bm{r}_1|^2}\bunsuu{\bm{r}_1}{|\bm{r}_1|} + \bunsuu{1}{4\pi\varepsilon_0}\bunsuu{qQ}{|\bm{r}_2|^2}\bunsuu{\bm{r}_2}{|\bm{r}_2|}\\
				&= \bunsuu{qQ}{4\pi\varepsilon_0}\bunsuu{\bm{r}_1}{|\bm{r}_1|^3} + \bunsuu{qQ}{4\pi\varepsilon_0}\bunsuu{\bm{r}_2}{|\bm{r}_2|^3}\\
				&= \bunsuu{qQ}{4\pi\varepsilon_0}\bunsuu{1}{(\sqrt{a^2 + d^2})^3}
				\begin{bmatrix*}[r]
					-a\\ d
				\end{bmatrix*}
				+ \bunsuu{qQ}{4\pi\varepsilon_0}\bunsuu{1}{(\sqrt{a^2 + d^2})^3}
				\begin{bmatrix*}[r]
					a\\ d
				\end{bmatrix*}
				\\
				&= \bunsuu{qQ}{4\pi\varepsilon_0}\bunsuu{1}{(\sqrt{a^2 + d^2})^3}\left(
					\begin{bmatrix*}[r]
						-a\\ d
					\end{bmatrix*}
					+
					\begin{bmatrix*}[r]
						a\\ d
					\end{bmatrix*}
				\right) = \bunsuu{qQ}{4\pi\varepsilon_0}\bunsuu{1}{(\sqrt{a^2 + d^2})^3}
					\begin{bmatrix*}[r]
						0\\ d
					\end{bmatrix*}
		\end{align*}
\end{enumerate}



\subsection{電場(1)}

クーロン力は,何かの物体に接触していなくても発生する力である.これからある点電荷Aがあったとき,その点電荷Aが周りの空間に影響を及ぼしていて,その空間の中に点電荷Bが置かれたときにAによる影響を受けてクーロン力が働くのではないかと考えることにする.そのような影響を表す量を\textbf{電場}といい,$\bm{E}$で表す.

ここで,クーロンの法則による表し方を変えてみる(一先ず,クーロン力の大きさ(スカラー)だけを考える).
\begin{align}
	F &= \bunsuu{1}{4\pi\varepsilon_0}\bunsuu{Q_1 Q_2}{r^2} \notag\\
	&= Q_2 {\color{teal}\left(\bunsuu{1}{4\pi\varepsilon_0}\bunsuu{Q_1}{r^2}\right)} \label{equ:DJ1-2}
\end{align}
式(\ref{equ:DJ1-2})をこういう見方で考える.
\begin{enumerate}[label=\textbf{[\arabic*]}, labelsep=10pt, leftmargin=23pt]
	\item $Q_2$の存在は置いといて,点電荷$Q_1$があることによって$Q_1$が周りの空間にある影響$E$を作り出す(本来はベクトル$\bm{E}$).これは式(\ref{equ:DJ1-2})の\textcolor{teal}{青緑色}の部分に該当する.
	\item $Q_2$は,$E$を感じてクーロン力を受ける.
\end{enumerate}

\begin{kousiki}{電場(1)}
	電荷に力を作用させる電気的な空間(このような空間のことを\textbf{場}と呼ぶ.特に,場の値がベクトルであるとき\textbf{ベクトル場}という).
\end{kousiki}

では,このことをベクトルでまとめる.

\begin{kousiki}{電場(2)}
	原点にある電荷$Q_1$があって,この電荷が周りの空間のある1点$\bm{r} = (x,\ y,\ z)$に
	\begin{equation}
		\bm{E}(x,\ y,\ z) = \bunsuu{1}{4\pi\varepsilon_0}\bunsuu{Q_1}{|\bm{r}|^2}\bunsuu{\bm{r}}{|\bm{r}|} \label{equ:DJ1-3}
	\end{equation}
	だけの電場を作っていると考える.そして,その電場に入った電荷$Q_2$は,電場から
	\begin{equation}
		\bm{F} = Q_2 \bm{E}
	\end{equation}
	と表せる力を受ける.電場$\bm{E}$の単位は$\mathrm{N/C}$である.よって,電場は「$1\,\mathrm{C}$の電荷に働くクーロン力」ということもできる.
\end{kousiki}

電場の向きは$\bm{r}$に平行である.$Q_2 < 0$であれば,$\bm{r}$と反対向きになる.



\subsection{電荷の分布}

今までは点電荷について進めてきたが,実際電荷は空間にどのように分布しているか考える.

\textbf{点電荷}は,空間のある1点に存在する電荷で,大きさをもたない.これでは現実味に欠けるので,空間的な広がりを与えていく.

まず,点電荷がある1方向に広がって分布する\textbf{線電荷}を考える.このとき,単位長さ(MKS単位系では$1\,\mathrm{m}$)あたりの電荷量を\textbf{線電荷密度}といい,$\lambda$で表す.単位は$\mathrm{C / m}$である.経路$C$上に存在する電荷の総量,つまり経路$C$にある総電荷$Q$は,以下のように線積分で求めることができる.
\begin{equation}
	Q = \int_{C} \lambda\,dl
\end{equation}

しかし,この線電荷は1方向には広がりを持つが,太さを持たない線であることが条件である.これではまだ非現実的なので,もう1方向に広がって分布する\textbf{面電荷}を考える.面$H$上の面電荷について,単位面積(MKS単位系では$1\,\mathrm{m^2}$)あたりの電荷量を\textbf{面電荷密度}といい,$\sigma$で表す.単位は$\mathrm{C / m^2}$である.面$H$上に存在する電荷の総量(総電荷)$Q$は,以下のように面積分で求めることができる.
\begin{equation}
	Q = \int_{H} \sigma\,dS
\end{equation}

しかし,これも面に厚みがないものとするので,これでも現実的な電荷分布とは言えない.更にもう1方向に広がって分布する電荷を考える.これは風船のような空間の体積領域$V$の内部に電荷が分布していると考える.このとき,単位体積(MKS単位系では$1\,\mathrm{m^3}$)あたりの電荷量を\textbf{電荷密度}といい,$\rho$で表す.単位は$\mathrm{C / m^3}$である.体積領域$V$に存在する電荷の総量(総電荷)$Q$は,以下のように体積分で求めることができる.
\begin{equation}
	Q = \int_{V} \rho\,dV \label{equ:DJ1-4}
\end{equation}



\subsection{電場(2)}

電荷が先程述べた通り3次元的に分布しているとする.この電荷が分布している体積領域$V$の外側に点(点の位置ベクトルを$\bm{r}$とする)があるとして,その点に作られる電場を表現する.重ね合わせの原理から,外側の点で作られる電場は,体積領域$V$の中にあるすべての電荷からの影響の総和である.

まず,体積領域$V$の中の微小体積$dV$を考え,$dV$の中にある電荷が作る微小電場$d\bm{E}$を求める.$dV$の中に存在する電荷の総量は式(\ref{equ:DJ1-4})より,$\rho\,dV$である.さらに,$dV$が存在する場所の位置ベクトルを$\bm{r}'$とすると式(\ref{equ:DJ1-3})より$d\bm{E}$は次のように表される.
\begin{equation}
	d\bm{E} = \bunsuu{1}{4\pi\varepsilon_0}\bunsuu{\rho(\bm{r}')\,dV}{|\bm{r} - \bm{r}'|^2}\bunsuu{\bm{r} - \bm{r}'}{|\bm{r} - \bm{r}'|}
\end{equation}

$\bm{r} - \bm{r}'$は,微小体積(位置ベクトル$\bm{r}'$)から体積領域の外側の点(位置ベクトル$\bm{r}$)へ向かうベクトルである.

体積領域$V$の中の微小体積$dV$の中にある電荷が作る電場を求めたので,$V$の中にある電荷が作る電場は領域$V$で体積分することによって求めることができる.

\begin{kousiki}{電場(3)-電場の一般的な表現}
	あ
\end{kousiki}





\end{document}