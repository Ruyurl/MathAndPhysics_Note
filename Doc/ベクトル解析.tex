\chapter{ベクトル解析}
\setcounter{page}{1}



\section{ベクトル関数の微分法}
\subsection{ベクトル関数の極限と連続}

$t_0$を定数,$\bm{c} = 
\begin{bmatrix}
	c_1\\ c_2\\ c_3
\end{bmatrix}
$を定ベクトルとする.$t$をスカラー変数とするベクトル関数$\bm{f}(t) = 
\begin{bmatrix}
	f_1(t)\\ f_2(t)\\ f_3(t)
\end{bmatrix}
$について,$t$が$t_0$に限りなく近づくときの$\bm{f}(t)$の\textbf{極限}は次で定義される.
\begin{align}
	\lim_{t \to t_0} |\bm{f}(t) - \bm{c}| = 0 &\iff \lim_{t \to t_0} \bm{f}(t) = \bm{c}\\
	&\iff \lim_{t \to t_0} f_1(t) = c_1,\quad \lim_{t \to t_0} f_2(t) = c_2,\quad \lim_{t \to t_0} f_3(t) = c_3
\end{align}

また,$\dlim_{t \to t_0} \bm{f}(t) = \bm{f}(t_0)$が成立するとき,$\bm{f}(t)$は$t = t_0$で\textbf{連続である}という.



\subsection{ベクトル関数の微分法}

ベクトル関数$\bm{f}(t)$に於いて,次式を$\bm{f}(t)$の\textbf{導関数}という.
\begin{align}
	\bunsuu{d\bm{f}(t)}{dt} &= \lim_{\varDelta t \to 0} \bunsuu{\bm{f}(t + \varDelta t) - \bm{f}(t)}{\varDelta t} =
	{
	\begin{bmatrix}
		\bunsuu{df_1(t)}{dt} &
		\bunsuu{df_2(t)}{dt} &
		\bunsuu{df_3(t)}{dt}
	\end{bmatrix}
	}^\top
\end{align}

\begin{kousiki}{微分法の公式}
	$\bm{f}(t),\ \bm{g}(t)$をベクトル関数,$\varphi(t)$をスカラー関数,$\bm{c}$を定ベクトルとする.
	\begin{enumerate}[label=\textbf{[\arabic*]}, labelsep=10pt, leftmargin=23pt, itemsep=6pt]
		\item $\bunsuu{d\bm{c}}{dt} = \bm{0}$
		\item $\bunsuu{d(\bm{f} \pm \bm{g})}{dt} = \bunsuu{d\bm{f}}{dt} \pm \bunsuu{d\bm{g}}{dt}$\hfill(複号同順)
		\item $\bunsuu{d(\varphi\bm{f})}{dt} = \bunsuu{d\varphi}{dt}\bm{f} + \varphi\bunsuu{d\bm{f}}{dt}$
		\item $\bunsuu{d(\bm{f} \cdot \bm{g})}{dt} = \bunsuu{d\bm{f}}{dt} \cdot \bm{g} + \bm{f} \cdot \bunsuu{d\bm{g}}{dt}$
		\item $\bunsuu{d(\bm{f} \times \bm{g})}{dt} = \bunsuu{d\bm{f}}{dt} \times \bm{g} + \bm{f} \times \bunsuu{d\bm{g}}{dt}$
		\item $\bunsuu{d}{dt}\left(\bunsuu{\bm{f}}{\varphi}\right) = \bunsuu{\bunsuu{d\bm{f}}{dt}\varphi - \bm{f}\bunsuu{d\varphi}{dt}}{\varphi^2}$\qquad$(\varphi \ne 0)$
		\item $t = \psi(u)$をスカラー関数とすると \qquad $\bunsuu{d\bm{f}}{du} = \bunsuu{d\bm{f}}{dt}\bunsuu{d\psi}{du}$\hfill(合成関数の微分法)
	\end{enumerate}
\end{kousiki}



\subsection{接線ベクトル}

点$\mathrm{P}$の位置ベクトルが$\bm{r} =
\begin{bmatrix}
	x(t)\\ y(t)\\ z(t)
\end{bmatrix}
$のようにベクトル関数であるとき,$t$が変化するにつれて$\mathrm{P}$はある曲線$C$を描く.この$C$を$\bm{r} = \bm{r}(t)$の表す曲線という.

以下,特に断りがない限り,$\bm{r}(t)$は何度でも微分可能で$\bunsuu{d\bm{r}}{dt} \ne \bm{0}$とする.

$t$の微小変化$\varDelta t$に対応する$\bm{r}(t)$の変化を$\varDelta \bm{r}$とすると$\varDelta \bm{r} = \bm{r}(t + \varDelta t) - \bm{r}(t)$である.
\begin{equation}
	\bunsuu{d\bm{r}}{dt} = \lim_{\varDelta t \to 0} \bunsuu{\varDelta \bm{r}}{\varDelta t} = \lim_{\varDelta t \to 0} \bunsuu{\bm{r}(t + \varDelta t) - \bm{r}(t)}{\varDelta t}
\end{equation}
を曲線$C$の点$\mathrm{P}$における\textbf{接線ベクトル}という.

\begin{enumerate}[leftmargin=18pt, labelsep=10pt, labelsep=10pt, itemindent=9pt]
	\item[\f{例}] $\bm{r}(t) =
		\begin{bmatrix}
			t^2\\ t^3\\ t^4
		\end{bmatrix}
		$ならば,接線ベクトルは$\bunsuu{d}{dt}\bm{r}(t) =
		\begin{bmatrix}
			2t\\ 3t^2\\ 4t^3
		\end{bmatrix}
		$である.特に,$t = 1$の位置$\bm{r}(1) =
		\begin{bmatrix}
			1\\ 1\\ 1
		\end{bmatrix}
		$における接線ベクトルは$\bunsuu{d}{dt}\bm{r}(1) =
		\begin{bmatrix}
			2\\ 3\\ 4
		\end{bmatrix}
		$である.
\end{enumerate}



\subsection{単位接線ベクトル}

接線ベクトルを自身の大きさで割った,大きさ$1$の接線ベクトルを\textbf{単位接線ベクトル}$\bm{t}$という.
\begin{equation}
	\bm{t} = \bunsuu{\bunsuu{d\bm{r}}{dt}}{\left|\bunsuu{d\bm{r}}{dt}\right|} \label{equ:vec_cal-1}
\end{equation}

曲線上のある定点$\mathrm{A}$を基準として,そこからの長さ$s$によって位置ベクトル$\bm{r}(s)$を定義する方法をとる.$\mathrm{A,\ P}$の位置ベクトルをそれぞれ$\bm{r}(\alpha),\ \bm{r}(t)$とすると,$\mathrm{AP}$の長さ$s$が次のようになるので,$\bunsuu{ds}{dt}$が求められる.
\begin{gather}
	s = s(t) = \int_{\alpha}^{t} \sqrt{\left(\bunsuu{dx}{dt}\right)^2 + \Bigl(\bunsuu{dy}{dt}\Bigr)^2 + \left(\bunsuu{dz}{dt}\right)^2}\,dt\\
	\bunsuu{ds}{dt} = \sqrt{\left(\bunsuu{dx}{dt}\right)^2 + \Bigl(\bunsuu{dy}{dt}\Bigr)^2 + \left(\bunsuu{dz}{dt}\right)^2} = \sqrt{\bunsuu{d\bm{r}}{dt} \cdot \bunsuu{d\bm{r}}{dt}} = \left|\bunsuu{d\bm{r}}{dt}\right|
\end{gather}

よって,$\bunsuu{d\bm{r}}{ds}$は合成関数の微分法より
\begin{equation}
	\bunsuu{d\bm{r}}{ds} = \bunsuu{d\bm{r}}{dt}\bunsuu{dt}{ds} = \bunsuu{\bunsuu{d\bm{r}}{dt}}{\bunsuu{ds}{dt}} = \bunsuu{\bunsuu{d\bm{r}}{dt}}{\left|\bunsuu{d\bm{r}}{dt}\right|}
\end{equation}
これは式(\ref{equ:vec_cal-1})と同じ式である.よって次が成り立つ.

\begin{kousiki}{単位接線ベクトル}
	\begin{equation}
		\bm{t} = \bunsuu{\bunsuu{d\bm{r}}{dt}}{\left|\bunsuu{d\bm{r}}{dt}\right|} = \bunsuu{d\bm{r}}{ds}
	\end{equation}
\end{kousiki}

一方,$\bm{t} \cdot \bm{t} = |\bm{t}|^2 = 1$の両辺を$t$で微分すると,内積の微分公式より,$2\bunsuu{d\bm{t}}{dt} \cdot \bm{t} = 0$なので,$\bunsuu{d\bm{t}}{dt} \cdot \bm{t} = 0$である.つまり,$\bunsuu{d\bm{t}}{dt} \perp \bm{t}$が成立する$\left(\text{$\bunsuu{d\bm{t}}{dt} \ne \bm{0}$ならば}\right)$.よって,次を\textbf{単位主法線ベクトル}とし,$\bm{n}$で表すと
\begin{equation}
	\bm{n} = \bunsuu{\bunsuu{d\bm{t}}{dt}}{\left|\bunsuu{d\bm{t}}{dt}\right|}
\end{equation}
となる.

\begin{kousiki}{単位主法線ベクトル}
	\begin{equation}
		\bm{n} = \bunsuu{\bunsuu{d\bm{t}}{dt}}{\left|\bunsuu{d\bm{t}}{dt}\right|} = \bunsuu{\bunsuu{d\bm{t}}{ds}}{\left|\bunsuu{d\bm{t}}{ds}\right|}
	\end{equation}
\end{kousiki}



\section{2変数ベクトル関数の微分法}
\subsection{2変数ベクトル関数の極限と連続}

$u_0,\ v_0$を定数,$\bm{c} = 
\begin{bmatrix}
	c_1\\ c_2\\ c_3
\end{bmatrix}
$を定ベクトルとする.$u,\ v$をスカラー変数とするベクトル関数$\bm{f}(u,\ v) = 
\begin{bmatrix}
	f_1(u,\ v)\\ f_2(u,\ v)\\ f_3(u,\ v)
\end{bmatrix}
$について,$(u,\ v)$が$(u_0,\ v_0)$に限りなく近づくときの$\bm{f}(u,\ v)$の\textbf{極限}は次で定義される.
\begin{align}
	&\lim_{(u,\ v) \to (u_0,\ v_0)} |\bm{f}(u,\ v) - \bm{c}| = 0 \notag\\
	&\iff \lim_{(u,\ v) \to (u_0,\ v_0)} \bm{f}(u,\ v) = \bm{c}\\
	&\iff \lim_{(u,\ v) \to (u_0,\ v_0)} f_1(u,\ v) = c_1,\quad \lim_{(u,\ v) \to (u_0,\ v_0)} f_2(u,\ v) = c_2,\quad \lim_{(u,\ v) \to (u_0,\ v_0)} f_3(u,\ v) = c_3
\end{align}

また,$\dlim_{(u,\ v) \to (u_0,\ v_0)} \bm{f}(u,\ v) = \bm{f}(u_0,\ v_0)$が成立するとき,$\bm{f}(u,\ v)$は$(u,\ v) = (u_0,\ v_0)$で\textbf{連続である}という.



\subsection{ベクトル関数の偏微分法}

領域$D$で定義されたベクトル関数$\bm{f}(u,\ v)$に於いて
\begin{equation}
	\bunsuu{\partial \bm{f}}{\partial u} = \lim_{\varDelta u \to 0} \bunsuu{\bm{f}(u + \varDelta u,\ v) - \bm{f}(u,\ v)}{\varDelta u} =
	{
	\begin{bmatrix}
		\bunsuu{\partial f_1(u,\ v)}{\partial u} &
		\bunsuu{\partial f_2(u,\ v)}{\partial u} &
		\bunsuu{\partial f_3(u,\ v)}{\partial u}
	\end{bmatrix}
	}^\top
\end{equation}
を$\bm{f}(u,\ v)$の\textbf{$u$についての偏導関数}といい,$\bm{f}_u(u,\ v)$とも表す.また,
\begin{equation}
	\bunsuu{\partial \bm{f}}{\partial v} = \lim_{\varDelta v \to 0} \bunsuu{\bm{f}(u,\ v + \varDelta v) - \bm{f}(u,\ v)}{\varDelta v} =
	{
	\begin{bmatrix}
		\bunsuu{\partial f_1(u,\ v)}{\partial v} &
		\bunsuu{\partial f_2(u,\ v)}{\partial v} &
		\bunsuu{\partial f_3(u,\ v)}{\partial v}
	\end{bmatrix}
	}^\top
\end{equation}
を$\bm{f}(u,\ v)$の\textbf{$v$についての偏導関数}といい,$\bm{f}_v(u,\ v)$とも表す.

\begin{enumerate}[leftmargin=18pt, labelsep=10pt, labelsep=10pt, itemindent=9pt]
	\item[\f{例}] $\bm{f}(u,\ v) =
		\begin{bmatrix}
			u\\ v\\ u^2 + v^2
		\end{bmatrix}
		$の偏導関数は
		\begin{equation}
			\bunsuu{\partial \bm{f}}{\partial u} =
			\begin{bmatrix}
				1\\ 0\\ 2u
			\end{bmatrix}
			,\quad \bunsuu{\partial \bm{f}}{\partial v} =
			\begin{bmatrix}
				0\\ 1\\ 2v
			\end{bmatrix}
		\end{equation}
\end{enumerate}

また,次のチェーン・ルールが成り立つ.
\begin{kousiki}{チェーン・ルール}
	\begin{enumerate}[label=\textbf{[\arabic*]}, labelsep=10pt, leftmargin=23pt]
		\item $u,\ v$がともにスカラー変数$t$の関数のとき
			\begin{equation}
				\bunsuu{d\bm{f}}{dt} = \bunsuu{\partial \bm{f}}{\partial u}\bunsuu{du}{dt} + \bunsuu{\partial \bm{f}}{\partial v}\bunsuu{dv}{dt}
			\end{equation}
		\item $u,\ v$がともにスカラー変数$s,\ t$の変数のとき
			\begin{equation}
				\bunsuu{\partial \bm{f}}{\partial s} =
				\bunsuu{\partial \bm{f}}{\partial u}\bunsuu{\partial u}{\partial s} + \bunsuu{\partial \bm{f}}{\partial v}\bunsuu{\partial v}{\partial s}, \qquad
				\bunsuu{\partial \bm{f}}{\partial t} =
				\bunsuu{\partial \bm{f}}{\partial u}\bunsuu{\partial u}{\partial t} + \bunsuu{\partial \bm{f}}{\partial v}\bunsuu{\partial v}{\partial t}
			\end{equation}
	\end{enumerate}
\end{kousiki}



\subsection{接平面}

空間内の点$\mathrm{P}$の位置ベクトルが$\bm{r} =
\begin{bmatrix}
	x(u,\ v)\\ y(u,\ v)\\ z(u,\ v)
\end{bmatrix}
$のようにベクトル関数であるとき,点$(u,\ v)$が領域$D$を動くと,$\mathrm{P}$は1つの曲面$S$を描く.この$S$を$\bm{r} = \bm{r}(u,\ v)$の表す曲面という.

$\bm{r}(u,\ v)$が$D$で連続な偏導関数をもつとする.$v$を一定にして$u$を変化させると$\bm{r}$は$S$上で1つの曲線(\textbf{$u$曲線})を描く.よって,$\bunsuu{\partial \bm{r}(u,\ v)}{\partial u}$は$u$曲線上の$\mathrm{P}$における接線ベクトルを与える.

同様に,$u$を一定にして$v$を変化させると$\bm{r}$は$S$上で1つの曲線(\textbf{$v$曲線})を描く.よって,$\bunsuu{\partial \bm{r}(u,\ v)}{\partial v}$は$v$曲線上の$\mathrm{P}$における接線ベクトルを与える.

このとき,外積の性質より$\bunsuu{\partial \bm{r}}{\partial u}$と$\bunsuu{\partial \bm{r}}{\partial v}$が平行ならば$\bunsuu{\partial \bm{r}}{\partial u} \times \bunsuu{\partial \bm{r}}{\partial v} = \bm{0}$なので,$\bunsuu{\partial \bm{r}}{\partial u} \times \bunsuu{\partial \bm{r}}{\partial v} \ne \bm{0}$ならば,この2つの接線ベクトルを含む平面$H$が存在する.この$H$を$S$の点$\mathrm{P}$における\textbf{接平面}という.その法線ベクトルは
\begin{equation}
	\bunsuu{\partial \bm{r}}{\partial u} \times \bunsuu{\partial \bm{r}}{\partial v} =
	\begin{vmatrix}
		\bm{i} & x_u & x_v\\
		\bm{j} & y_u & y_v\\
		\bm{k} & z_u & z_v
	\end{vmatrix}
\end{equation}
である.

\begin{enumerate}[leftmargin=18pt, labelsep=10pt, labelsep=10pt, itemindent=9pt]
	\item[\f{例}] ベクトル関数$\bm{r} = \bm{r}(u,\ v) =
		\begin{bmatrix}
			u\cos v\\ u\sin v\\ u^2
		\end{bmatrix}
		$を例にとって,詳しく見ていく.
		\begin{enumerate}[label=\textbf{[\arabic*]}, labelsep=10pt, leftmargin=23pt, itemsep=12pt]
			\item \underline{与式の表す曲面の,$x,\ y,\ z$に関する方程式}\\
				$x = u\cos v,\ y = u\sin v$より$x^2 + y^2$を計算すると$u^2$なので,$z = u^2$となる.よって,$z = x^2 + y^2$.
			\item \underline{$v = \bunsuu{\pi}{2}$のときの$u$曲線}\\
				$x = u\cos\bunsuu{\pi}{2} = 0,\ y = u\sin\bunsuu{\pi}{2} = u$なので,$z = 0^2 + y^2 = y^2$.これは平面$x = 0$上の放物線である.
			\item \underline{$u = 1$のときの$v$曲線}\\
				$x = \cos v,\ y = \sin v$より,$z = 1$.これは平面$z = 1$上の半径$1$の円である.
			\item \underline{$(u,\ v) = \left(1,\ \bunsuu{\pi}{2}\right)$における接線ベクトル}\\
				$\bunsuu{\partial \bm{r}}{\partial u} =
				\begin{bmatrix}
					\cos v\\ \sin v\\ 2u
				\end{bmatrix}
				=
				\begin{bmatrix}
					0\\ 1\\ 2
				\end{bmatrix}
				$,\qquad
				$\bunsuu{\partial \bm{r}}{\partial v} =
				\begin{bmatrix}
					-u\sin v\\ u\cos v\\ 0
				\end{bmatrix}
				=
				\begin{bmatrix}
					-1\\ 0\\ 0
				\end{bmatrix}
				$
			\item \underline{$(u,\ v) = \left(1,\ \bunsuu{\pi}{2}\right)$における接平面の法線ベクトル}\\
				$\bunsuu{\partial \bm{r}}{\partial u} \times \bunsuu{\partial \bm{r}}{\partial v} =
				\begin{vmatrix}
					\bm{i} & x_u & x_v\\
					\bm{j} & y_u & y_v\\
					\bm{k} & z_u & z_v
				\end{vmatrix}
				=
				\begin{vmatrix}
					\bm{i} & 0 & -1\\
					\bm{j} & 1 & 0\\
					\bm{k} & 2 & 0
				\end{vmatrix}
				=
				\begin{bmatrix}
					0\\ -2\\ 1
				\end{bmatrix}
				$
		\end{enumerate}
\end{enumerate}



\section{空間曲線}
\subsection{曲率}

曲線$\bm{r} = \bm{r}(s)$の単位接線ベクトル$\bm{t} = \bunsuu{d\bm{r}}{ds}$について次の$\kappa \in \mathbb{R}$を曲線の\textbf{曲率}という.
\begin{equation}
	\kappa = \left|\bunsuu{d\bm{t}}{ds}\right| = \left|\bunsuu{d^2\bm{r}}{ds^2}\right|
\end{equation}

$\kappa$は曲線の局所的な曲がり具合である.この値が大きいほどカーブが急になる.$\varDelta \bm{t}$が十分に小さいとき,$|\varDelta \bm{t}| \approx \varDelta \theta$と見做すことができ,$\kappa = \left|\bunsuu{d\theta}{ds}\right|$と表すこともできるので,$\kappa = $は回転角の変化率であるとも言える.

\begin{enumerate}[label=\textbf{[\arabic*]}, labelsep=10pt, leftmargin=23pt]
	\item 半径$a$の円の曲率は$\kappa = \bunsuu{1}{a}$で,半径が大きくなるほど曲線の曲がり具合が小さくなる.
	\item 直線の曲率は$\kappa = 0$である.
\end{enumerate}

曲率の逆数を\textbf{曲率半径}といい,$\sigma$で表す.
\begin{equation}
	\sigma = \bunsuu{1}{\kappa}
\end{equation}

\begin{enumerate}[label=\textbf{[\arabic*]}, labelsep=10pt, leftmargin=23pt]
	\item 半径$a$の曲率半径は$\sigma = a$である.$\kappa = 0$のときは$\sigma = \infty$と定義する.
	\item 直線の曲率半径は$\sigma = \infty$である.
\end{enumerate}



\subsection{捩率}



\subsection{速度・加速度}


\section{スカラー場の勾配}
\subsection{勾配}

$D$で定義されたスカラー場$f(x,\ y,\ z)$について,その偏導関数$\bunsuu{\partial f}{\partial x},\ \bunsuu{\partial f}{\partial y},\ \bunsuu{\partial f}{\partial z}$を係数とするベクトル$\bunsuu{\partial f}{\partial x}\bm{i} + \bunsuu{\partial f}{\partial y}\bm{j} + \bunsuu{\partial f}{\partial z}\bm{k}$を考えると,領域$D$にあるベクトル場が定義される.これをスカラー場$f$の\textbf{勾配}といい,$\grad f$で表す.形式的なベクトル$\bm{\nabla} =
\begin{bmatrix}
	\bunsuu{\partial}{\partial x} &
	\bunsuu{\partial}{\partial y} &
	\bunsuu{\partial}{\partial z}
\end{bmatrix}^\top
$を使って,$\bm{\nabla}f$で表すこともある.$\bm{\nabla}$を\textbf{Hamiltonの演算子}と呼ばれ,\textbf{ナブラ}と読む.

\begin{kousiki}{スカラー場の勾配}
	スカラー場$f(x,\ y,\ z)$の勾配$\grad f$とは次のベクトル場である.
	\begin{equation}
		\grad f = \bm{\nabla}f =
		{
		\begin{bmatrix}
			\bunsuu{\partial f}{\partial x} &
			\bunsuu{\partial f}{\partial y} &
			\bunsuu{\partial f}{\partial z}
		\end{bmatrix}
		}^\top
	\end{equation}
\end{kousiki}

$c$を定数とすると,方程式$f(x,\ y,\ z) = c$は一般に1つの曲面を表す.これを$f$の\textbf{等位面}という.$c$をパラメータとすると方程式は等位面の群を表す.点$\mathrm{P}$を通る等位面上に$\mathrm{P}$を通る任意の曲線$\bm{r}(t) =
\begin{bmatrix}
	x(t)\\ y(t)\\ z(t)
\end{bmatrix}
$を描くと \vskip-\baselineskip
\begin{equation*}
	f\bigl(x(t),\ y(t),\ z(t)\bigr) = c
\end{equation*}
を満たす.両辺を$t$で微分すると,チェーン・ルールより
\begin{align*}
	&\bunsuu{\partial f}{\partial x}\bunsuu{dx}{dt} + \bunsuu{\partial f}{\partial y}\bunsuu{dy}{dt} + \bunsuu{\partial f}{\partial z}\bunsuu{dz}{dt} = 0\\
	&\iff \bm{\nabla}f \cdot \bunsuu{d\bm{r}}{dt} = 0\\
	&\iff \bm{\nabla}f \perp \bunsuu{d\bm{r}}{dt}
\end{align*}
となる.

これは点$\mathrm{P}$に対応するベクトル場$\bm{\nabla}f$は,$\mathrm{P}$を通る等位面上の曲線の接線ベクトル$\bunsuu{d\bm{r}}{dt}$に垂直になる.また,曲線は任意であるから次のことが言える.
\begin{kousiki}{勾配$\bm{\nabla}f$の意味}
	点$\mathrm{P}$における勾配$\bm{\nabla}f$は,$\mathrm{P}$における等位面の法線ベクトルになる.
\end{kousiki}

\vskip\baselineskip

任意の単位ベクトル$\bm{e} =
\begin{bmatrix}
	e_x\\ e_y\\ e_z
\end{bmatrix}
$について,点$\mathrm{P}$から$\bm{e}$の方向に$\varDelta s$だけ動いた時の$f$の微分係数(接線の傾き)を調べる.これを\textbf{方向微分係数}$\bunsuu{df}{ds}$という.
\begin{kousiki}{方向微分係数}
	\begin{align}
		\bunsuu{df}{ds} &=
		\lim_{\varDelta s \to 0}
			\bunsuu{f(x + e_x \varDelta s,\ y + e_y \varDelta s,\ z + e_z \varDelta s) - f(x,\ y,\ z)}{\varDelta s}\\
			&= \bm{\nabla}f \cdot \bm{e}
	\end{align}
\end{kousiki}

\f{証明} $\mathrm{P}$から$\bm{e}$の方向に$\varDelta s$だけ離れた点の座標は$(x + e_x \varDelta s,\ y + e_y \varDelta s,\ z + e_z \varDelta s)$なので,$f$の変化率はチェーン・ルールより
\begin{align*}
	\bunsuu{d}{ds}&f(x + e_x \varDelta s,\ y + e_y \varDelta s,\ z + e_z \varDelta s)\\
	&= \bunsuu{\partial f}{\partial x}\bunsuu{d}{ds}(x + e_x \varDelta s) + \bunsuu{\partial f}{\partial y}\bunsuu{d}{ds}(y + e_y \varDelta s) + \bunsuu{\partial f}{\partial z}\bunsuu{d}{ds}(z + e_z \varDelta s)\\
	&= \bunsuu{\partial f}{\partial x}e_x + \bunsuu{\partial f}{\partial y}e_y + \bunsuu{\partial f}{\partial z}e_z = \bm{\nabla}f \cdot \bm{e}
\end{align*}
となる.

このとき,$\bm{\nabla}f$と$\bm{e}$の成す角を$\theta$とすると
\begin{equation*}
	\bm{\nabla}f \cdot \bm{e} = |\bm{\nabla} f||\bm{e}|\cos\theta = |\bm{\nabla}f|\cos\theta
\end{equation*}
となり,$\theta = 0$のとき正の最大値をとる.すなわち,勾配=法線ベクトルの方向への方向微分係数が最大となる.方程式
\begin{equation*}
	f(x,\ y,\ z) = c_i \quad (c = 1,\ 2,\ \cdots)
\end{equation*}
で,$c_1$での方向微分係数より,$c_2$での方向微分係数の値の方が大きいとき,