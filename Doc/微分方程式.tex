\chapter{微分方程式}
\setcounter{page}{1}


\section{1階線形常微分方程式}

以下の形の微分方程式を\textbf{1階線形常微分方程式}という.
\begin{equation}
	\bunsuu{dx}{dt} + P(t)x = Q(t)
\end{equation}

このとき,$Q(t) = 0$の場合,即ち
\begin{equation}
	\bunsuu{dx}{dt} + P(t)x = 0
\end{equation}
の場合を\textbf{斉次}といい,$Q(t) \ne 0$の場合,即ち
\begin{equation}
	\bunsuu{dx}{dt} + P(t)x = Q(t)
\end{equation}
の場合を\textbf{非斉次}という.



\subsection{変数分離形}

1階微分方程式で,
\begin{equation}
	\bunsuu{dx}{dt} = F(x)G(t)
\end{equation}
のように,$x$の関数と$t$の関数の積になる形を\textbf{変数分離形}という.

\noindent
\textbf{【解法】}
\begin{equation*}
	\bunsuu{dx}{dt} = 2tx
\end{equation*}
を例にする.

\begin{enumerate}[label=\textbf{[\arabic*]}, labelsep=10pt, leftmargin=23pt]
	\item 左辺を$x$だけの式,右辺を$t$だけの式にする.
		\begin{equation*}
			\bunsuu{dx}{x} = 2t\,dt
		\end{equation*}
	\item $\int$をつけて両辺積分する.
		\begin{gather*}
			\int \bunsuu{dx}{x} = \int 2t\,dt\\
			\log |x| = t^2 + C_1 \quad \text{($C_1$は任意定数)}
		\end{gather*}
	\item $x$について解く.
		\begin{gather*}
			x = \pm e^{t^2 + C_1} = \pm e^{C_1} e^{t^2}\\
		\intertext{$\pm e^{C_1} = C$とおいて}
			x = Ce^{t^2} \quad \text{($C$は任意定数)}
		\end{gather*}
\end{enumerate}


\subsection{定数変化法}

1階線形常微分方程式
\begin{equation}
	\bunsuu{dx}{dt} + P(t)x = Q(t)
\end{equation}
の一般解を求める.まず,$Q(t) = 0$といた斉次方程式を解く.これは変数分離形で解ける.その結果を用いて非斉次方程式の一般解を導く.

\noindent\textbf{【解法】}
\begin{equation}
	\bunsuu{dx}{dt} + \bunsuu{1}{t}x = 4t^2 + 1 \label{equ:JBH-1}
\end{equation}
を例にする.

\begin{enumerate}[label=\textbf{[\arabic*]}, labelsep=10pt, leftmargin=23pt]
	\item まず,斉次方程式
		\begin{equation}
			\bunsuu{dx}{dt} + \bunsuu{1}{t}x = 0
		\end{equation}
		の一般解を求める.
		\begin{gather*}
			\bunsuu{dx}{x} = -\bunsuu{dt}{t}\\
			\int \bunsuu{dx}{x} = -\int \bunsuu{dt}{t}\\
			\log |x| = -\log |t| + C_1 \quad \text{($C_1$は任意定数)}\\
			\log |x| + \log |t| = C_1\\
			\log |xt| = C_1\\
			\pm e^{C_1} = xt\\
			\intertext{$\pm e^{C_1} = C$とおいて}
			x = \bunsuu{C}{t}
		\end{gather*}
	\item これは斉次方程式の一般解である.求めたいのは
\end{enumerate}