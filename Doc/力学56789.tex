\chapter{力学II}
\setcounter{page}{1}
\section{力学的エネルギー 面積の原理}
\subsection{5節問題}

\begin{enumerate}[label=\textbf{[\arabic*]}, labelsep=10pt, leftmargin=23pt]
	\item 平面内を運動する質点に働く力の成分が質点の座標を$x,\ y$として
		\begin{equation*}
			X = axy,\quad Y = \bunsuu{1}{2}ax^2
		\end{equation*}
		で与えられるとき,保存力かどうか調べよ.保存力ならば位置エネルギーはどうなるか.
	\item 一平面内を運動する質点に働く力の成分が,質点の座標を$x,\ y$として
		\begin{equation*}
			X = axy,\quad Y = by^2
		\end{equation*}
		で与えられるとき,保存力かどうか調べよ.また,$x$軸上の$(r,\ 0)$で与えられる点$\mathrm{A}$から,$y$軸上の$(0,\ r)$で与えられる点$\mathrm{C}$まで,円周$\mathrm{ABC}$に沿ってゆく場合と,弦$\mathrm{AB'C}$に沿ってゆく場合とで,この力の行う仕事を比較せよ.
	\item 次の諸式を証明せよ.
		\begin{enumerate}[label={(\alph*)}, labelsep=10pt]
			\item $\bm{A} \times (\bm{B} \times \bm{C}) = \bm{B}(\bm{A} \cdot \bm{C}) - \bm{C}(\bm{A} \cdot \bm{B})$
			\item $(\bm{A} \times \bm{B}) \cdot (\bm{C} \times \bm{D}) = (\bm{A} \cdot \bm{C})(\bm{B} \cdot \bm{D}) - (\bm{B} \cdot \bm{C})(\bm{A} \cdot \bm{D})$
			\item $(\bm{B} \times \bm{C})\cdot(\bm{A} \times \bm{D}) + (\bm{C} \times \bm{A}) \cdot (\bm{B} \times \bm{D}) + (\bm{A} \times \bm{B}) \cdot (\bm{C} \times \bm{D}) = 0$
		\end{enumerate}
	\item $\bm{A},\ \bm{B},\ \bm{C}$がこの順に右手系(一般に互いに直角でなくでよい)をつくっているとすれば
		\begin{equation*}
			\bm{A} \cdot (\bm{B} \times \bm{C}) = \bm{B} \cdot (\bm{C} \times \bm{A}) = \bm{C} \cdot (\bm{A} \times \bm{B})
		\end{equation*}
		は$\bm{A},\ \bm{B},\ \bm{C}$を稜(かど)とする六面体の体積であることを証明せよ.
	\item 1つの単位ベクトルを$\bm{n}$とすれば,任意のベクトル$\bm{A}$は
		\begin{equation*}
			\bm{A} = (\bm{A} \cdot \bm{n})\bm{n} + \bm{n} \times (\bm{A} \times \bm{n})
		\end{equation*}
		と書くことができることを示せ.
	\item 滑らかな水平板の上においてある質点に糸を結び付け,その糸を板にあけた穴$\mathrm{O}$に通しておく.質点を,はじめOのまわりにある角速度で運動させ,糸を引っ張ってOと質点との距離を変えるとき,質点の角速度はどう変わっていくか.
	\item 一平面内で
		\begin{equation*}
			r = a(1 + c\cos \varphi)\qquad (0 < c < 1)
		\end{equation*}
		で与えられる軌道を描く質点に働く中心力はどんな力か.
\end{enumerate}



\section{単振り子の運動と惑星の運動}



\section{非慣性系に相対的な運動}



\section{質点系の運動量と角運動量}



\section{剛体のつりあいと運動}