\chapter{固有値とその応用}
\setcounter{page}{1}

\section{固有値とは?}

1次変換の章での通り,線形変換$f$が行列$A =
\begin{bmatrix}
	3 & -2\\ 2 & -2
\end{bmatrix}
$で表されているとき,ベクトル$\bm{x}_1 =
\begin{bmatrix*}[r]
	2\\ -3
\end{bmatrix*}
$は,
\begin{align*}
	&f(\bm{x}_1) = A\bm{x}_1 =
	\begin{bmatrix*}[r]
		3 & -2\\ 2 & -2
	\end{bmatrix*}
	\begin{bmatrix*}[r]
		2\\ -3
	\end{bmatrix*}
	=
	\begin{bmatrix*}[r]
		0\\ -2
	\end{bmatrix*}
	&
	&\therefore\quad \bm{x}_1 \nheikou f(\bm{x}_1) \quad \left(
		\begin{bmatrix*}[r]
			2\\ -3
		\end{bmatrix*}
		\nheikou
		\begin{bmatrix*}[r]
			0\\ -2
		\end{bmatrix*}
	\right)
\end{align*}
と,大きさも向きも異なるベクトルに移されるのが大抵である.しかし,$\bm{x}$を上手く選ぶと,移した後のベクトル$f(\bm{x})$が元のベクトルを\kenten{スカラー倍したかのような}変換である(=向きが変わらない)ように見えることがある.例えば$\bm{x}_2 =
\begin{bmatrix*}[r]
	2\\ 1
\end{bmatrix*}$というベクトルは
\begin{align*}
	&f(\bm{x}_2) = A\bm{x}_2 =
	\begin{bmatrix*}[r]
		3 & -2\\ 2 & -2
	\end{bmatrix*}
	\begin{bmatrix*}[r]
		2\\ 1
	\end{bmatrix*}
	=
	\begin{bmatrix*}[r]
		4\\ 2
	\end{bmatrix*}
	&
	&\therefore\quad \bm{x}_2 \heikou f(\bm{x}_2) \quad \left(
		\begin{bmatrix*}[r]
			2\\ 1
		\end{bmatrix*}
		\heikou
		\begin{bmatrix*}[r]
			4\\ 2
		\end{bmatrix*}
	\right)
\end{align*}
より,$f(\bm{x}_2)$は$\bm{x}_2$を2倍にしたかのように見える.このようなベクトルは少なく,存在しないときもある.この$\bm{x}_2$のベクトルを$A$の\textbf{固有ベクトル},倍率$2$を$A$の\textbf{固有値}という.

固有値・固有ベクトルの定義を式に表すと次のようになる.
\begin{kousiki}{固有値・固有ベクトル}
	$A$を$n$次正方行列とする.以下を満たす$n$次元列ベクトル$\bm{x}$が存在するとき,$\lambda$を$A$の\textbf{固有値},$\bm{x}$を$\lambda$に対する\textbf{固有ベクトル}という.
	\begin{equation}
		A\bm{x} = \lambda\bm{x} \qquad (\bm{x} \ne \bm{0})
	\end{equation}
\end{kousiki}

※$\lambda\bm{x}$は$\bm{x}$のスカラー$\lambda$倍を表す.



\subsection{固有値と固有ベクトルの計算}

固有値・固有ベクトルの定義より
\begin{equation*}
	A\bm{x} = \lambda\bm{x}
\end{equation*}
左辺は行列とベクトルの積,右辺はベクトルのスカラー倍になっているので,両辺に単位行列$E$を左から掛けると
\begin{equation*}
	EA\bm{x} = \lambda E\bm{x}
\end{equation*}
$EA = A$より
\begin{equation*}
	A\bm{x} = \lambda E\bm{x}
\end{equation*}
これで両辺とも行列とベクトルの積になったので,移項して
\begin{equation*}
	A\bm{x} - \lambda E\bm{x} = \bm{0}
\end{equation*}
$\bm{x}$を括りだして
\begin{equation}
	(A - \lambda E)\bm{x} = \bm{0} \label{equ:eig-1}
\end{equation}
ここで定義より$\bm{x} \ne \bm{0}$なので,(\ref{equ:eig-1})が$\bm{x} = \bm{0}$以外の解を持つには
\begin{equation}
	|A - \lambda E| = 0\label{equ:eig-2}
\end{equation}
でなければならない(教科書p.111).

$|A - \lambda E|$を$A$の\textbf{固有多項式},(\ref{equ:eig-2})を$A$の\textbf{固有方程式}という.固有値は固有方程式を解くことで求められる.つまり,$A$の固有値を求める場合,$A$から$\lambda E =
\begin{bmatrix}
	\lambda & 0 & \cdots & 0\\
	0 & \lambda & \cdots & 0\\
	\vdots & \vdots & \ddots & \vdots\\
	0 & 0 & \cdots & \lambda
\end{bmatrix}$を引いた行列の行列式が$0$になるときの$\lambda$を求める.固有値が$\lambda$のときの固有ベクトル$\bm{x}$は(\ref{equ:eig-1})に代入して求める.

\begin{tip}{例題(1)}
	行列$A = 
	\begin{bmatrix*}[r]
		3 & -2\\ 2 & -2
	\end{bmatrix*}$の固有値と固有ベクトルを求めよ.なお,この行列は冒頭の行列と同じである.

	\tcblower

	固有値を$\lambda$とすると
	\begin{equation*}
		A - \lambda E =
			\begin{bmatrix*}[r]
				3 & -2\\ 2 & -2
			\end{bmatrix*}
			- \lambda
			\begin{bmatrix*}[r]
				1 & 0\\ 0 & 1
			\end{bmatrix*}
		= 
			\begin{bmatrix*}[r]
				3 & -2\\ 2 & -2
			\end{bmatrix*}
			- 
			\begin{bmatrix*}[r]
				\lambda & 0\\ 0 & \lambda
			\end{bmatrix*}
		=
		\begin{bmatrix}
			3 - \lambda & -2\\ 2 & -2 - \lambda
		\end{bmatrix}
	\end{equation*}
	これより
	\begin{equation*}
		\begin{vmatrix}
			3 - \lambda & -2\\ 2 & -2 - \lambda
		\end{vmatrix}
		= (3 - \lambda)(-2 - \lambda) + 4 = \lambda^2 - \lambda - 2 = (\lambda - 2)(\lambda + 1) = 0
	\end{equation*}
	よって,$\lambda = -1,\ 2$が固有値である.

	固有値が$\lambda = -1$のときの固有ベクトル$\bm{x}_1$は
	\begin{equation*}
		(A - \lambda E)\bm{x} = \bm{0} \qLonglr \begin{bmatrix}
			3 - \lambda & -2\\ 2 & -2 - \lambda
		\end{bmatrix}
		\begin{bmatrix}
			x\\ y
		\end{bmatrix} =
		\begin{bmatrix}
			0\\ 0
		\end{bmatrix} \qLonglr
		\begin{bmatrix}
			4 & -2\\ 2 & -1
		\end{bmatrix}
		\begin{bmatrix}
			x\\ y
		\end{bmatrix} =
		\begin{bmatrix}
			0\\ 0
		\end{bmatrix}
	\end{equation*}
	より,
	\begin{equation*}
		\begin{cases*}
			4x - 2y = 0\\ 2x - y = 0
		\end{cases*}
		\longrightarrow 2x - y = 0
	\end{equation*}
	$x = \alpha$とおくと,$y = 2\alpha$.よって,$\bm{x}_1 =
	\begin{bmatrix}
		\alpha\\ 2\alpha
	\end{bmatrix}\ (\alpha \ne 0)$

	固有値が$\lambda = 2$のときの固有ベクトル$\bm{x}_2$は
	\begin{equation*}
		(A - \lambda E)\bm{x} = \bm{0} \qLonglr \begin{bmatrix}
			3 - \lambda & -2\\ 2 & -2 - \lambda
		\end{bmatrix}
		\begin{bmatrix}
			x\\ y
		\end{bmatrix} =
		\begin{bmatrix}
			0\\ 0
		\end{bmatrix} \qLonglr
		\begin{bmatrix}
			1 & -2\\ 2 & -4
		\end{bmatrix}
		\begin{bmatrix}
			x\\ y
		\end{bmatrix} =
		\begin{bmatrix}
			0\\ 0
		\end{bmatrix}
	\end{equation*}
	より,
	\begin{equation*}
		\begin{cases*}
			x - 2y = 0\\ 2x - 4y = 0
		\end{cases*}
		\longrightarrow x - 2y = 0
	\end{equation*}
	$y = \beta$とおくと,$x = 2\beta$.よって,$\bm{x}_2 =
	\begin{bmatrix}
		2\beta\\ \beta
	\end{bmatrix}\ (\beta \ne 0)$
\end{tip}



\subsubsection*{2次正方行列$A$の固有値の簡単な求め方}

\begin{kousiki}{2次正方行列の固有値}
	$A =
	\begin{bmatrix}
		a & b\\ c & d
	\end{bmatrix}$の固有値$\lambda$は,$a$と$d$の平均値を$m$,$|A|$を$p$とすると
	\begin{equation}
		\lambda = m \pm \sqrt{m^2 - p}
	\end{equation}
\end{kousiki}

\begin{tip}{例題(2)}
	行列$A = 
	\begin{bmatrix*}[r]
		3 & -2\\ 2 & -2
	\end{bmatrix*}$の固有値を求めよ(行列は例題(1)と同じ).

	\tcblower

	固有値を$\lambda$とすると
	\begin{align*}
		&m = \bunsuu{3 + (-2)}{2} = \bunsuu{1}{2} & &p = |A| = -6 + 4 = -2
	\end{align*}
	より
	\begin{equation*}
		\lambda = \bunsuu{1}{2} \pm \sqrt{\left(\bunsuu{1}{2}\right)^2 - (-2)} = \bunsuu{1}{2} \pm \sqrt{\bunsuu{9}{4}} = \bunsuu{1}{2} \pm \bunsuu{3}{2} = -1,\ 2 
	\end{equation*}
\end{tip}