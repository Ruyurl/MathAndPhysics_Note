\chapter{偏微分法}
\setcounter{page}{1}
\section{関数の極限}

関数$f(x,\ y)$に於いて,点$(x,\ y)$が点$(a,\ b)$以外の点を取りながら$(a,\ b)$に限りなく近づくとき,関数の値が$C$に限りなく近づくならば,$f(x,\ y)$は$C$に\textbf{収束する}といい,
\begin{equation}
	\lim_{(x,\ y) \to (a,\ b)} f(x,\ y) = C
\end{equation}
と表す.$C$を\textbf{極限値}という.

このとき,$(x,\ y)$がどんな近づき方で$(a,\ b)$に近づいても極限値がある一定の値$C$になることが必要である.

例えば,$f(x,\ y) = \bunsuu{xy}{x^2 + y^2}$について,$\dlim_{(x,\ y) \to (0,\ 0)} f(x,\ y)$を考える.
\begin{enumerate}[label=(\roman*), labelsep=10pt, leftmargin=23pt]
	\item 点を直線$y = x$上で近づけると$f(x,\ y) = \bunsuu{x \cdot x}{x^2 + x^2} = \bunsuu{1}{2}$であるから$\bunsuu{1}{2}$に収束する.
	\item 点を直線$y = 2x$上で近づけると$f(x,\ y) = \bunsuu{x \cdot 2x}{x^2 + (2x)^2} = \bunsuu{2}{5}$であるから$\bunsuu{2}{5}$に収束する.
\end{enumerate}
よって,極限値はない.

関数$f(x,\ y)$の定義域内の点$\mathrm{P}(a,\ b)$について,
\begin{equation}
	\lim_{(x,\ y) \to (a,\ b)} f(x,\ y) = f(a,\ b)
\end{equation}
が成り立つとき,$f(x,\ y)$は点$\mathrm{P}$で\textbf{連続である}という.


\section{偏導関数}

関数$z = f(x,\ y)$に於いて
\begin{equation}
	\bunsuu{\partial z}{\partial x} = \lim_{\varDelta x \to 0} \bunsuu{f(x + \varDelta x,\ y) - f(x,\ y)}{\varDelta x}
\end{equation}
$f(x,\ y)$の\textbf{$x$についての偏導関数}といい,$f_x$とも表す.また,
\begin{equation}
	\bunsuu{\partial z}{\partial y} = \lim_{\varDelta y \to 0} \bunsuu{f(x,\ y + \varDelta y) - f(x,\ y)}{\varDelta y}
\end{equation}
を$f(x,\ y)$の\textbf{$y$についての偏導関数}といい,$f_y$とも表す.

\textbf{$x\ (y)$について偏微分可能}であるとは,点$x = a\ (y = b)$での偏微分係数が存在することである.また,偏微分係数$f_x(a,\ b),\ f_y(a,\ b)$はそれぞれ点$(a,\ b)$の$x$軸方向の傾き,$y$軸方向の傾きを表す.



\subsection{高階偏導関数}

$z = f(x,\ y)$で,2階偏微分可能で全て連続のとき\quad$\bunsuu{\partial^2 z}{\partial y \partial x} = \bunsuu{\partial^2 z}{\partial x \partial y}$

一般に$n$階偏微分可能で全て連続のとき,$n = k + l$とすると$\bunsuu{\partial^n z}{\partial x^k \partial y^l}$は全て等しい.



\subsection{全微分可能性}

\begin{itemize}
	\item $f(x,\ y)$が点$(a,\ b)$で全微分可能 $\qLongright$ $f(x,\ y)$は$(a,\ b)$で連続かつ$(a,\ b)$で偏微分可能
	\item $f(x,\ y)$の$\bunsuu{\partial f}{\partial x},\ \bunsuu{\partial f}{\partial y}$が$(a,\ b)$で存在してそれらが連続である $\qLongright$ $f(x,\ y)$は$(a,\ b)$で全微分可能
	\item $f(x,\ y)$が偏微分可能であっても全微分可能ではない(全微分の方が強い概念).
\end{itemize}
\begin{equation}
	df = \bunsuu{\partial f}{\partial x}dx + \bunsuu{\partial f}{\partial y}
\end{equation}

\newpage
\begin{tip}{導出}
	関数$f(x,\ y)$の$\varDelta x,\ \varDelta y$に対する増加量$\varDelta f$は
	\begin{equation*}
		\varDelta f = 
		\underbrace{
			\underbrace{
				\bunsuu{\partial f}{\partial x}\varDelta x
			}_{\text{$\varDelta x$についての増加量}}
			+
			\underbrace{
				\bunsuu{\partial f}{\partial y}\varDelta y
			}_{\text{$\varDelta y$についての増加量}}
		}_{接平面の公式}
		{} + \varepsilon
	\end{equation*}
	更に,$\dlim_{\scalebox{0.5}{$(\varDelta x,\ \varDelta y) \to (0,\ 0)$}} \bunsuu{\varepsilon}{\sqrt{(\varDelta x)^2 + (\varDelta y)^2}} = 0$であれば,$f(x,\ y)$は全微分可能という.
\end{tip}

ここで,$\dlim_{\scalebox{0.5}{$(\varDelta x,\ \varDelta y) \to (0,\ 0)$}} \bunsuu{\varepsilon}{\sqrt{(\varDelta x)^2 + (\varDelta y)^2}} = 0$を,「$\varepsilon$は$\sqrt{(\varDelta x)^2 + (\varDelta y)^2}$より\textbf{高位の無限小}」という.これは,「$\varepsilon$は$\sqrt{(\varDelta x)^2 + (\varDelta y)^2}$とは比べ物にならないくらい速く$0$に近付く」という意味である.



\subsection{接平面の方程式}

曲面$z = f(x,\ y)$上の点$\bigl(a,\ b,\ f(a,\ b)\bigr)$に於ける接平面の方程式
\begin{equation}
	z - f(a,\ b) = \bunsuu{\partial f(a,\ b)}{\partial x}(x - a) + \bunsuu{\partial f(a,\ b)}{\partial y}(y - b) \label{equ5:1}
\end{equation}

曲面$f(x,\ y,\ z) = 0$上の点$(a,\ b,\ c)$に於ける接平面の方程式
\begin{equation}
	\begin{bmatrix}
		\bunsuu{\partial f}{\partial x} & \bunsuu{\partial f}{\partial y} & \bunsuu{\partial f}{\partial z} 
	\end{bmatrix}
	\begin{bmatrix}
		x - a\\ y - b\\ z - c
	\end{bmatrix}
	= 0
\end{equation}

これは,陰関数の微分法と式(\ref{equ5:1})から分かる.また,これより曲面$f(x,\ y,\ z) = 0$の法線ベクトルは以下である.
\begin{equation}
	{
		\begin{bmatrix}
			\bunsuu{\partial f}{\partial x} & \bunsuu{\partial f}{\partial y} & \bunsuu{\partial f}{\partial z}
		\end{bmatrix}
	}^\top
\end{equation}



\subsection{チェーン・ルール(連鎖律)}

\begin{kousiki}{チェーン・ルール(1)}
	$z = f(x,\ y)$が全微分可能で,$x = x(t),\ y = y(t)$が微分可能であるとき,$z$は$t$の関数である.
	\begin{equation}
		\bunsuu{dz}{dt} = \bunsuu{\partial f}{\partial x}\bunsuu{dx}{dt} + \bunsuu{\partial f}{\partial y}\bunsuu{dy}{dt}
	\end{equation}
\end{kousiki}

\begin{kousiki}{チェーン・ルール(2)}
	$z = f(x,\ y)$が全微分可能で,$x = x(u,\ v),\ y = y(u,\ v)$が偏微分可能であるとき
	\begin{gather}
		\bunsuu{\partial z}{\partial u} = \bunsuu{\partial z}{\partial x}\bunsuu{\partial x}{\partial u} + \bunsuu{\partial z}{\partial y}\bunsuu{\partial y}{\partial u}\\
		\bunsuu{\partial z}{\partial v} = \bunsuu{\partial z}{\partial x}\bunsuu{\partial x}{\partial v} + \bunsuu{\partial z}{\partial y}\bunsuu{\partial y}{\partial v}
	\end{gather}
\end{kousiki}



\subsection{陰関数の微分法}

$f(x,\ y) = 0$によって表された$x$の関数$y$の導関数
\begin{equation}
	\bunsuu{dy}{dx} = -\bunsuu{\partial f}{\partial x} \left/ \bunsuu{\partial f}{\partial y}\right.
\end{equation}

\begin{tip}{導出}
	陰関数$y(x)$より,$f\bigl(x,\ y(x)\bigr) = 0$.両辺微分して
	\begin{equation}
		\bunsuu{\partial f}{\partial x}\bunsuu{dx}{dx} + \bunsuu{\partial f}{\partial y}\bunsuu{dy}{dx} = 0 \iff \bunsuu{\partial f}{\partial x} + \bunsuu{\partial f}{\partial y}\bunsuu{dy}{dx} = 0 \quad \left(\because \bunsuu{dx}{dx} = 1\right)
	\end{equation}
\end{tip}

$f(x,\ y,\ z) = 0$によって表された$x,\ y$の関数$z$の導関数
\begin{align}
	&\bunsuu{\partial z}{\partial x} = -\bunsuu{\partial f}{\partial x} \left/ \bunsuu{\partial f}{\partial z}\right.
	&
	&\bunsuu{\partial z}{\partial y} = -\bunsuu{\partial f}{\partial y} \left/ \bunsuu{\partial f}{\partial z}\right.
\end{align}

\begin{tip}{導出}
	陰関数$z(x,\ y)$より,$f\bigl(x,\ y,\ z(x,\ y)\bigr) = 0$.両辺$x$で偏微分して
	\begin{equation}
		\bunsuu{\partial f}{\partial x}\bunsuu{\partial x}{\partial x} + \bunsuu{\partial f}{\partial y}\bunsuu{\partial y}{\partial x} + \bunsuu{\partial f}{\partial z}\bunsuu{\partial z}{\partial x} = 0
		\iff \bunsuu{\partial f}{\partial x} + \bunsuu{\partial f}{\partial z}\bunsuu{\partial z}{\partial x} = 0 \quad \left(\because \bunsuu{\partial x}{\partial x} = 1,\ \bunsuu{\partial y}{\partial x} = 0\right)
	\end{equation}
	両辺$y$で偏微分すると第2式が得られる.
\end{tip}



\subsection{2変数のテイラーの定理}

$f(x,\ y)$が$n$次までの連続な偏導関数を持つとき($C^n$級関数),
\begin{equation*}
	\bunsuu{d^n f}{dt^n} = D^n f = \Bigl(h\bunsuu{\partial}{\partial x} + k\bunsuu{\partial}{\partial y}\Bigr)^n f
\end{equation*}
とおくと
\begin{equation}
	f(a + h,\ b + k) = f(a,\ b) + \bunsuu{1}{1!}Df(a,\ b) + \bunsuu{1}{2!}D^2f(a,\ b) + \cdots + \bunsuu{1}{(n - 1)!}D^{n - 1}f(a,\ b) + 
	\underbrace{
		\bunsuu{1}{n!}D^n f(a + \theta h,\ b + \theta k)
	}_{R_n}
\end{equation}
を満たす$\theta\ (0 < \theta < 1)$が存在する(\textbf{テイラーの定理}).

また,$R_n \xrightarrow{n \to \infty} 0$であるとき
\begin{equation}
	f(a + h,\ b + k) = f(a,\ b) + \bunsuu{1}{1!}Df(a,\ b) + \bunsuu{1}{2!}D^2f(a,\ b) + \cdots + \bunsuu{1}{(n - 1)!}D^{n - 1}f(a,\ b) + \bunsuu{1}{n!}D^n f(a,\ b) + \cdots
\end{equation}
が成り立つ(\textbf{テイラー展開}).

\begin{tip}{補足}
	$z = f(x,\ y)$で,$x = a + ht$,$y = b + kt$とすると,$z = f(a + ht,\ b + kt)$より,1変数$t$の関数になる.微分して
	\begin{equation}
		\bunsuu{df}{dt} = \bunsuu{dx}{dt}\bunsuu{\partial f}{\partial x} + \bunsuu{dy}{dt}\bunsuu{\partial f}{\partial y} = h\bunsuu{\partial f}{\partial x} + k\bunsuu{\partial f}{\partial y} \label{equ5:4}
	\end{equation}
	ここで,$D = \Bigl(h\bunsuu{\partial}{\partial x} + k\bunsuu{\partial}{\partial y}\Bigr)$とおくと,$\bunsuu{df}{dt}$は$Df = \Bigl(h\bunsuu{\partial}{\partial x} + k\bunsuu{\partial}{\partial y}\Bigr)f$と書ける.
	\vskip\baselineskip
	次に,$\bunsuu{d^2 f}{dt^2}$を考える.式(\ref{equ5:4})より
	\begin{equation*}
		\bunsuu{d^2 f}{dt^2} = h\bunsuu{d}{dt}\Bigl(\bunsuu{\partial f}{\partial x}\Bigr) + k\bunsuu{d}{dt}\Bigl(\bunsuu{\partial f}{\partial y}\Bigr)
	\end{equation*}
	今,$\bunsuu{\partial f}{\partial x},\ \bunsuu{\partial f}{\partial y}$は2変数$x,\ y$の関数なので,チェーン・ルールより
	\begin{gather*}
		\bunsuu{d}{dt}\Bigl(\bunsuu{\partial f}{\partial x}\Bigr)
		= \bunsuu{\partial}{\partial x}\Bigl(\bunsuu{\partial f}{\partial x}\Bigr)\bunsuu{dx}{dt} + \bunsuu{\partial}{\partial y}\Bigl(\bunsuu{\partial f}{\partial x}\Bigr)\bunsuu{dy}{dt}
		= h\bunsuu{\partial^2 f}{\partial x^2} + k\bunsuu{\partial^2 f}{\partial y \partial x}\\
		% 
		\bunsuu{d}{dt}\Bigl(\bunsuu{\partial f}{\partial y}\Bigr)
		= \bunsuu{\partial}{\partial x}\Bigl(\bunsuu{\partial f}{\partial y}\Bigr)\bunsuu{dx}{dt} + \bunsuu{\partial}{\partial y}\Bigl(\bunsuu{\partial f}{\partial y}\Bigr)\bunsuu{dy}{dt}
		= h\bunsuu{\partial^2 f}{\partial x \partial y} + k\bunsuu{\partial^2 f}{\partial y^2}
	\end{gather*}
	よって,代入して $\bunsuu{d^2 f}{dt^2} = h\Bigl(h\bunsuu{\partial^2 f}{\partial x^2} + k\bunsuu{\partial^2 f}{\partial y \partial x}\Bigr) + k\Bigl(h\bunsuu{\partial^2 f}{\partial x \partial y} + k\bunsuu{\partial^2 f}{\partial y^2}\Bigr)$\\
	第2次偏導関数が存在し,ともに連続とすると$\bunsuu{\partial^2 f}{\partial y \partial x} = \bunsuu{\partial^2 f}{\partial x \partial y}$なので,
	\begin{equation}
		\bunsuu{d^2 f}{dt^2} = h^2\bunsuu{\partial^2 f}{\partial x^2} + 2hk\bunsuu{\partial^2 f}{\partial x \partial y} + k^2\bunsuu{\partial^2 f}{\partial y^2} = \Bigl(h\bunsuu{\partial}{\partial x} + k\bunsuu{\partial}{\partial y}\Bigr)^2 f = D^2 f
	\end{equation}
	このように,$D$を用いると
	\begin{equation}
		\bunsuu{d^n f}{dt^n} = D^n f = \Bigl(h\bunsuu{\partial}{\partial x} + k\bunsuu{\partial}{\partial y}\Bigr)^n f
	\end{equation}
	と略記することができる.
\end{tip}

$a = b = 0$としたときのテイラーの定理を\textbf{マクローリンの定理}という.$h,\ k$の代わりに$x,\ y$でよく表す.
\begin{equation}
	f(x,\ y) = f(0,\ 0) + \bunsuu{1}{1!}Df(0,\ 0) + \bunsuu{1}{2!}D^2 f(0,\ 0) + \cdots + \bunsuu{1}{(n - 1)!}D^{n - 1} f(0,\ 0) + \bunsuu{1}{n!}D^n f(\theta x,\ \theta y)
\end{equation}
を満たす$\theta\ (0 < \theta < 1)$が存在する.



\subsection{包絡線}

変数$x,\ y$の他に任意定数$\alpha$を含んでいる方程式
\begin{equation}
	f(x,\ y,\ \alpha) = 0
\end{equation}
は$\alpha$を変化させて得られる全ての曲線の集合(曲線群)を表している.これを\textbf{曲線群の方程式}という.

曲線群の全ての曲線に接する曲線or直線を曲線群の\textbf{包絡線}という.包絡線上の点$(x,\ y)$は
\begin{equation}
	f(x,\ y,\ \alpha) = 0,\ \bunsuu{\partial}{\partial\alpha}f(x,\ y,\ \alpha) = 0
\end{equation}
を満たす.この2式を求めて,$\alpha$を消去すると包絡線の方程式が求まる.



\section{極値問題}
\subsection{2変数関数の極値}

\textbf{ヘッシアン}$H(a,\ b) =
\begin{vmatrix}
	f_{xx}(a,\ b) & f_{xy}(a,\ b)\\
	f_{xy}(a,\ b) & f_{yy}(a,\ b)
\end{vmatrix}
$とおくと,点$(a,\ b)$に於いて
\begin{enumerate}[label=\textbf{[\arabic*]}, labelsep=10pt, leftmargin=23pt]
	\item $H > 0$のとき
		\begin{align}
			&f_{xx} > 0 \qLongright \text{点$(a,\ b)$で極小をとる} & &f_{xx} < 0 \qLongright \text{点$(a,\ b)$で極大をとる}
		\end{align}
	\item $H < 0$のとき
		\begin{equation}
			\text{点$(a,\ b)$では極値を取らない.}
		\end{equation}
	\item $H = 0$のとき
		\begin{equation*}
			\text{極値の判定は出来ない.}
		\end{equation*}
\end{enumerate}

\begin{tip}{証明}
	テイラーの定理より
	\begin{equation*}
		f(a + h,\ b + k) - f(a,\ b) = Df(a,\ b) + \bunsuu{1}{2}D^2 f(a + \theta h,\ b + \theta k) \quad (0 < \theta < 1)
	\end{equation*}
	$(a,\ b)$で極値を取るので,$Df(a,\ b) = h \cdot 0 + k \cdot 0 = 0$より
	\begin{equation*}
		f(a + h,\ b + k) - f(a,\ b) = \bunsuu{1}{2}D^2 f(a + \theta h,\ b + \theta k)
	\end{equation*}
	となる.ここで,簡単のため
	\begin{align*}
		A &= f_{xx}(a + \theta h,\ b + \theta k), &
		B &= f_{xy}(a + \theta h,\ b + \theta k), &
		C &= f_{yy}(a + \theta h,\ b + \theta k)
	\end{align*}
	とおくと,$f(a + h,\ b + k) - f(a,\ b) = \bunsuu{1}{2}(Ah^2 + 2Bhk + Ck^2)$\\
	$h,\ k$が十分$0$に近ければ,$AC - B^2,\ A$の符号はそれぞれ$H,\ f_{xx}$の符号に等しくなる.\\
	変形して
	\begin{equation*}
		f(a + h,\ b + k) - f(a,\ b) = \bunsuu{\color{blue}A}{2}\Bigl\{\Bigl(h + \bunsuu{B}{A}k\Bigr)^2 + \bunsuu{\color{red}AC - B^2}{A^2}k^2\Bigr\}
	\end{equation*}
	\begin{enumerate}[label=\textbf{[\arabic*]}, labelsep=10pt, leftmargin=23pt]
		\item $H > 0$(即ち,${\color{red}AC - B^2} > 0$)のとき\\
			$f_{xx} > 0\ ({\color{blue}A} > 0)$ならば,$f(a + h,\ b + k) > f(a,\ b)$(下に凸)より,極小\\
			$f_{xx} < 0\ ({\color{blue}A} < 0)$ならば,$f(a + h,\ b + k) < f(a,\ b)$(上に凸)より,極大
		\item $H < 0$(即ち,${\color{red}AC - B^2} < 0$)のとき
			\begin{enumerate}[labelsep=10pt, leftmargin=23pt]
				\item[(ア)] $A \ne 0$または$C \ne 0$のとき\\
					${\color{red}AC - B^2} < 0$より,$Ah^2 + 2Bhk + Ck^2$は$h,\ k$によって正にも負にもなる.
				\item[(イ)] $A = C = 0$のとき\\
					${\color{red}AC - B^2} < 0$より$B \ne 0$なので,$Ah^2 + 2Bhk + Ck^2 = 2Bhk$は$h,\ k$によって正にも負にもなる.
			\end{enumerate}
	\end{enumerate}
\end{tip}



\subsection{条件付極値問題}

$xy$平面上の点$(x,\ y)$が条件$\varphi(x,\ y) = 0$で表される曲線上を動くとき,平面$x = f(x,\ y)$が極値を取り得る点
\begin{equation}
	\bunsuu{f_x}{\varphi_x} = \bunsuu{f_y}{\varphi_y} \quad (\varphi_x,\ \varphi_y \ne 0) \label{equ5:5}
\end{equation}

\begin{tip}{導出}
	方程式$\varphi(x,\ y) = 0$の$y$が$x$の関数,即ち$\varphi\bigl(x,\ y(x)\bigr) = 0$とすると,陰関数の微分法より
	\begin{equation}
		\bunsuu{dy}{dx} = -\bunsuu{\partial\varphi}{\partial x} \left/ \bunsuu{\partial \varphi}{\partial y}\right. \label{equ5:2}
	\end{equation}
	このとき,関数$z = f(x,\ y)$は$x$の関数となるのでチェーン・ルールより,極値をとるとき
	\begin{equation}
		\bunsuu{df}{dx} = \bunsuu{\partial f}{\partial x}\bunsuu{dx}{dx} + \bunsuu{\partial f}{\partial y}\bunsuu{dy}{dx} = 0 \label{equ5:3}
	\end{equation}
	式(\ref{equ5:3})に式(\ref{equ5:2})を代入して
	\begin{equation*}
		\bunsuu{\partial f}{\partial x} - \bunsuu{\partial f}{\partial y} \cdot \bunsuu{\bunsuu{\partial\varphi}{\partial x}}{\bunsuu{\partial\varphi}{\partial y}} = 0
	\end{equation*}
	整理すると得られる.
\end{tip}



\subsubsection*{ラグランジュの未定乗数法}

$f(x,\ y)$は条件$\varphi(x,\ y) = 0$のもとで,$(a,\ b)$で極値をとるとする.$\varphi_x(a,\ b) \ne 0$または$\varphi_y(a,\ b) \ne 0$であれば
\begin{equation}
	\left\{
		\begin{aligned}
			f_x(a,\ b) &= \lambda \cdot \varphi_x(a,\ b)\\
			f_y(a,\ b) &= \lambda \cdot \varphi_y(a,\ b) 
		\end{aligned}
	\right.\quad\text{を満たす$\lambda$が存在}
\end{equation}

\begin{tip}{導出}
	式(\ref{equ5:5})で,$\bunsuu{f_x}{\varphi_x} = \bunsuu{f_y}{\varphi_y} = \lambda$とおくことによって得られる.
\end{tip}

\vskip\baselineskip

\textbf{【条件付極値問題で,最大値・最小値を問われたとき】}

$\varphi(x,\ y) = 0$が端点をもたない場合,$z = f(x,\ y)$が連続関数であれば,極値を取り得る点が最大値・最小値になる.




\newpage