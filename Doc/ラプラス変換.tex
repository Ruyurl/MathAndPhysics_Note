\chapter{ラプラス変換}
\setcounter{page}{1}
\section{定義と基本的性質}
\subsection{ラプラス変換の定義}

関数$f(t)$は,$t \in \mathbb{R}_{> 0}$で定義され,$s$を$t$と無関係な実数とする.このとき次のような関数$F(s)$を考える.
\begin{equation}
	F(s) = \int_{0}^{\infty} e^{-st}f(t)\,dt = \lim_{\substack{T \to \infty \\ \varepsilon \to +0}} \int_{\varepsilon}^{T} e^{-st}f(t)\,dt
\end{equation}

これは,関数$f(t)$に関数$F(s)$を対応させる規則を与えている.その対応を$\mathcal{L}$で表す.$\mathcal{L}$を適用させた$F(s)$を$f(t)$の\textbf{ラプラス変換}といい,
\begin{equation}
	F(s) = \mathcal{L}[f(t)] \text{\quad または \quad} f(t) \stackrel{\mathcal{L}}{\longrightarrow} F(s)
\end{equation}
と表す.$f(t)$を\textbf{原関数},$F(s)$を\textbf{像関数}という.一般に$s \in \mathbb{C}$であるが,ここでは$s \in \mathbb{R}$とする.

\begin{enumerate}[leftmargin=18pt, labelsep=10pt, labelsep=10pt, itemindent=9pt]
	\item[\f{例1}] $f(t) = 1$のラプラス変換$\mathcal{L}[1]$
		\begin{align*}
			F(s) = \mathcal{1}[1] = \lim_{T \to \infty} \int_{0}^{T} e^{-st}\,dt = \lim_{T \to \infty} \teisekibun{{-\bunsuu{1}{s}e^{-st}}}{0}{T} = \lim_{T \to \infty} \bunsuu{1}{s}(1 - e^{-sT}) = \bunsuu{1}{s} \quad (s > 0)
		\end{align*}
		$s \le 0$のとき,$\dlim_{T \to \infty} \bunsuu{1}{s}(1 - e^{-sT}) = \infty$なので存在しない.
	\item[\f{例2}] $f(t) = t$のラプラス変換$\mathcal{L}[t]$
		\begin{align*}
			F(s) &= \mathcal{L}[t] = \lim_{T \to \infty} \int_{0}^{T} e^{-st}t\,dt = \lim_{T \to \infty} \left\{ \teisekibun{{-\bunsuu{e^{-st}}{s}t}}{0}{T} + \int_{0}^{T} \bunsuu{e^{-st}}{s}\,dt \right\}
			= \lim_{T \to \infty} \left\{ -\bunsuu{e^{-sT}}{s}T - \teisekibun{\bunsuu{e^{-st}}{s^2}}{0}{T} \right\}\\
			&= \lim_{T \to \infty} \left\{
				-\bunsuu{e^{-sT}T}{s} - \bunsuu{e^{-sT}}{s^2} + \bunsuu{1}{s^2}
			\right\}
			= \bunsuu{1}{s^2}
		\end{align*}
		ここで,$s > 0$のとき
		\begin{gather*}
			\lim_{T \to \infty} \left(-\bunsuu{e^{-sT}T}{s}\right) = -\bunsuu{1}{s}\lim_{T \to \infty} \bunsuu{T}{e^{sT}} = -\bunsuu{1}{s}\lim_{T \to \infty} \bunsuu{1}{se^{sT}} = 0\\
			\lim_{T \to \infty} \left(-\bunsuu{e^{-sT}}{s^2}\right) = -\bunsuu{1}{s^2}\lim_{T \to \infty} \bunsuu{1}{e^{sT}} = 0
		\end{gather*}
	\item[\f{例3}] $f(t) = e^{\alpha t}$($\alpha$は定数)のラプラス変換$\mathcal{L}[e^{\alpha t}]$
		\begin{align*}
			F(s) = \mathcal{L}[e^{\alpha t}] &= \lim_{T \to \infty} \int_{0}^{T} e^{-st}e^{\alpha t}\,dt
			= \lim_{T \to \infty} \int_{0}^{T} e^{-(s - \alpha)t}\,dt
			= \lim_{T \to \infty} \teisekibun{{-\bunsuu{1}{s - \alpha}e^{-(s - \alpha)t}}}{0}{T}\\
			&= \lim_{T \to \infty} \left(-\bunsuu{1}{s - \alpha}\{e^{-(s - \alpha)T} - 1\}\right) = \bunsuu{1}{s - \alpha} \quad (s > \alpha)
		\end{align*}
		$s - \alpha \le 0$,即ち$s \le \alpha$のとき,極限値は存在しない.
	\item[\f{例4}] $f(t) = \sin \omega t$のラプラス変換$\mathcal{L}[\sin \omega t]$
		\begin{align*}
			F(s) &= \mathcal{L}[\sin \omega t] = \lim_{T \to \infty} \int_{0}^{T} e^{-st}\sin \omega t\,dt
			\intertext{$I_1 = \dint_{0}^{T} e^{-st}\sin \omega t\,dt$とする.}
			I_1 &= \int_{0}^{T} e^{-st}\sin \omega t\,dt = \teisekibun{ \bunsuu{1}{-s}e^{-st}\sin \omega t }{0}{T} - \int_{0}^{T} \bunsuu{1}{-s}e^{-st}\omega\cos \omega t\,dt\\
			&= \bunsuu{1}{-s}e^{-sT}\sin \omega T + \bunsuu{\omega}{s}\int_{0}^{T} e^{-st}\cos \omega t\,dt\\
			&= \bunsuu{1}{-s}e^{-sT}\sin \omega T + \bunsuu{\omega}{s}\left\{
				\teisekibun{
					\bunsuu{1}{-s}e^{-st}\cos \omega t
				}{0}{T} - \int_{0}^{T} \bunsuu{1}{-s}e^{-st}(-\omega\sin \omega t)\,dt
			\right\}\\
			&= \bunsuu{1}{-s}e^{-sT}\sin \omega T + \bunsuu{\omega}{s}\left\{
				\left(
					\bunsuu{1}{-s}e^{-sT}\cos \omega T + \bunsuu{1}{s}
				\right)
				- \bunsuu{\omega}{s}I_1
			\right\}
			\intertext{よって,$I_1$について解くと}
			I_1 &= \bunsuu{s^2}{s^2 + \omega^2}\left\{
				-\bunsuu{1}{s}e^{-sT}\sin \omega T - \bunsuu{\omega}{s^2}e^{-sT}\cos \omega T + \bunsuu{\omega}{s^2}
			\right\}
		\end{align*}
		$s > 0$のとき$\dlim_{T \to \infty} e^{-sT}\sin \omega T = \dlim_{T \to \infty} e^{-sT}\cos \omega T = 0$なので,
		\begin{equation*}
			\mathcal{L}[\sin \omega t] = \lim_{T \to \infty} I_1 = \bunsuu{\omega}{s^2 + \omega^2}\quad(s > 0)
		\end{equation*}
		同様にすると,$\mathcal{L}[\cos \omega t] = \bunsuu{s}{s^2 + \omega^2}\quad(s > 0)$
\end{enumerate}



\subsection{単位ステップ関数}

次のように定義される関数$H(t)$を\textbf{Heavisideの階段関数}という.
\begin{equation}
	H(t) =
	\left\{
		\begin{array}{ll}
			0 & (t < 0)\\ 1 & (t > 0)
		\end{array}
	\right.
\end{equation}

《注》$t = 0$に於ける値は任意に決めることができる.

また,次のように定義される関数$U(t)$を\textbf{単位ステップ関数}という.
\begin{equation}
	U(t) =
	\left\{
		\begin{array}{ll}
			0 & (t \le 0)\\ 1 & (t > 0)
		\end{array}
	\right.
\end{equation}

関数$U(t - a)$は$U(t)$を$t$軸方向に$a$平行移動して得られる.

$a \ge 0$のときの$\mathcal{L}[U(t - a)]$を求める.$t - a \le 0$,即ち$t \le a$のときは$U(t - a) = 0$,$t > a$のときは$U(t - a) = 1$なので
\begin{align*}
	F(s) = \mathcal{L}[U(t - a)] &= \int_{0}^{\infty} e^{-st}U(t - a)\,dt = \int_{0}^{a} e^{-st}U(t - a)\,dt + \int_{a}^{\infty} e^{-st}U(t - a)\,dt = \int_{a}^{\infty} e^{-st}\,dt\\
	&= \teisekibun{{
		-\bunsuu{1}{s}e^{-st}
	}}{a}{\infty}
	= -\bunsuu{1}{s}\lim_{t \to \infty} \left(
		e^{-st} - e^{-as}
	\right)
\end{align*}
ここで,$s > 0$のとき$\dlim_{t \to \infty} e^{-st} = 0$なので
\begin{equation*}
	F(s) = -\bunsuu{1}{s}\cdot(-e^{-as}) = \bunsuu{e^{-as}}{s} \quad (s > 0)
\end{equation*}
である.



\section{ラプラス変換の基本的性質}
\subsection{ラプラス変換の線形性}

関数$f(t),\ g(t)$,定数$a,\ b$に対し
\begin{equation}
	\mathcal{L}[af(t) + bg(t)] = a\mathcal{L}[f(t)] + b\mathcal{L}[g(t)]
\end{equation}
が成り立つ.これによって,項別にラプラス変換をすることで求めることができる.



\subsection{ラプラス変換の相似性}

関数$f(t)$,定数$\lambda > 0$に対し,$F(s) = \mathcal{L}[f(t)]$とする.このとき
\begin{equation}
	\mathcal{L}[f(\lambda t)] = \bunsuu{1}{\lambda}F\left(\bunsuu{s}{\lambda}\right)
\end{equation}
が成り立つ.即ち,$t$軸方向に$\bunsuu{1}{\lambda}$倍してからラプラス変換すると,$F(s)$を$s$軸方向に$\lambda$倍したものを$\bunsuu{1}{\lambda}$倍したものになる.

\begin{equation*}
	\mspace{100mu}
	\begin{array}{llll}
		& x = f(t)	& \xlongrightarrow[\text{ラプラス変換}]{\mathcal{L}} & X = F(s)\\
		\text{\small \fbox{$t$軸方向に$\bunsuu{1}{\lambda}$倍}}
			& \text{\LARGE \raisebox{-4pt}{☟}} &
			& \text{\LARGE \raisebox{-4pt}{☟}}\quad
			\text{\small \fbox{$s$軸方向に$\lambda$倍して$X$軸方向に$\bunsuu{1}{\lambda}$倍}}\\[10pt]
		& x = f(\lambda t)	& \xlongrightarrow[\text{ラプラス変換}]{\mathcal{L}} & X = \bunsuu{1}{\lambda}F\left(\bunsuu{s}{\lambda}\right)
	\end{array}
\end{equation*}



\subsection{第1移動定理(像関数の移動法則)}

関数$f(t)$,定数$\alpha$に対し,$F(s) = \mathcal{L}[f(t)]$とする.このとき
\begin{equation}
	\mathcal{L}[e^{\alpha t}f(t)] = F(s - \alpha)
\end{equation}
が成り立つ.即ち,原関数に$e^{\alpha t}$をかけたものをラプラス変換すると,像関数は$F(s)$を$s$軸方向に$\alpha$平行移動したものになる.

\begin{equation*}
	\mspace{150mu}
	\begin{array}{llll}
		& x = f(t)	& \xlongrightarrow[\text{ラプラス変換}]{\mathcal{L}} & X = F(s)\\
		\text{\small \fbox{$e^{\alpha t}$をかける}}
			& \text{\LARGE \raisebox{-4pt}{☟}} &
			& \text{\LARGE \raisebox{-4pt}{☟}}\quad
			\text{\small \fbox{$s$軸方向に$\alpha$平行移動}}\\[10pt]
		& x = e^{\alpha t}f(t)	& \xlongrightarrow[\text{ラプラス変換}]{\mathcal{L}} & X = F(s - \alpha)
	\end{array}
\end{equation*}



\subsection{第2移動定理(原関数の移動法則)}

ラプラス変換は$t > 0$の範囲で行うので,単位ステップ関数$U(t)$をかけても変わらない.このとき,関数$f(t)$,定数$\mu > 0$に対し,$F(s) = \mathcal{L}[f(t)]$とすると
\begin{equation}
	\mathcal{L}[f(t - \mu)U(t - \mu)] = e^{-\mu s}F(s)
\end{equation}
が成り立つ.即ち,原関数を$t$軸方向に$\mu$移動したものをラプラス変換すると,像関数は$F(s)$に$e^{-\mu s}$をかけたものになる.

\begin{equation*}
	\mspace{120mu}
	\begin{array}{llll}
		& x = f(t)	& \xlongrightarrow[\text{ラプラス変換}]{\mathcal{L}} & X = F(s)\\
		\text{\small \fbox{$t$軸方向に$\mu$移動}}
			& \text{\LARGE \raisebox{-4pt}{☟}} &
			& \text{\LARGE \raisebox{-4pt}{☟}}\quad
			\text{\small \fbox{$e^{-\mu s}$をかける}}\\[10pt]
		& x = f(t - \mu)U(t - \mu)	& \xlongrightarrow[\text{ラプラス変換}]{\mathcal{L}} & X = e^{-\mu s}F(s)
	\end{array}
\end{equation*}



\subsection{微分法則}

原関数$f(t)$のラプラス変換を$F(s)$とすると次が成り立つ.
\begin{kousiki}{1階の微分法則}
	\begin{enumerate}[label=\textbf{[\arabic*]}, labelsep=10pt, leftmargin=23pt]
		\item $\mathcal{L}[f'(t)] = sF(s) - f(+0)$ \hfill (原関数の微分法則)
		\item $\mathcal{L}[tf(t)] = -F'(s)$ \hfill (像関数の微分法則)
	\end{enumerate}
\end{kousiki}

$f(+0)$は,$t \to +0$のときの$f(t)$の極限値を表す.原関数の微分法則は微分方程式に使われることがある.

原関数や像関数の微分法則を繰り返し使うと以下を得る.
\begin{kousiki}{高次微分法則}
	\begin{enumerate}[label=\textbf{[\arabic*]}, labelsep=10pt, leftmargin=23pt]
		\item $\mathcal{L}[f^{(n)}(t)] = s^n F(s) - s^{n - 1}f(+0) - s^{n - 2}f'(+0) - s^{n - 3}f''(+0) - \cdots - f^{(n - 1)}(+0)$ \hfill (原関数の高次微分法則)
		\item $\mathcal{L}[t^n f(t)] = (-1)^n F^{(n)}(s)$ \hfill (像関数の高次微分法則)
	\end{enumerate}
\end{kousiki}



\subsection{積分法則}

原関数$f(t)$のラプラス変換を$F(s)$とすると積分についての次が成り立つ.

\begin{kousiki}{積分法則}
	\begin{enumerate}[label=\textbf{[\arabic*]}, labelsep=10pt, leftmargin=23pt]
		\item $\mathcal{L}\left[\dint_{0}^{t} f(\tau)\,d\tau\right] = \bunsuu{F(s)}{s}$ \hfill (原関数の積分法則)
		\item $\mathcal{L}\left[\bunsuu{f(t)}{t}\right] = \dint_{s}^{\infty} F(\sigma)\,d\sigma$ \hfill (像関数の積分法則)
	\end{enumerate}
\end{kousiki}

\begin{enumerate}[leftmargin=18pt, labelsep=10pt, itemindent=9pt]
	\item[\f{例}] \underline{$\dint_{0}^{t} e^{-2\tau}\,d\tau$のラプラス変換}
		\begin{equation*}
			\mathcal{L}[e^{-2t}] = \bunsuu{1}{s + 2}
		\end{equation*}
		であるので,
		\begin{equation*}
			\mathcal{L}\left[\int_{0}^{t} e^{-2\tau}\,d\tau\right] = \bunsuu{1}{s}\mathcal{L}[e^{-2t}] = \bunsuu{1}{s(s + 2)}
		\end{equation*}
\end{enumerate}



\subsection{たたみこみ}

区間$[0,\ \infty)$で定義された関数$f(t),\ g(t)$に対し
\begin{equation}
	(f * g)(t) = \int_{0}^{t} f(\tau)g(t - \tau)\,d\tau
\end{equation}
を$f(t)$と$g(t)$の\textbf{たたみこみ}または\textbf{合成積}という.たたみこみのラプラス変換について,次の関係が成り立つ.
\begin{kousiki}{たたみこみのラプラス変換}
	\begin{equation}
		\mathcal{L}[(f * g)(t)] = \mathcal{L}[f(t)]\mathcal{L}[g(t)]
	\end{equation}
\end{kousiki}

\begin{enumerate}[leftmargin=18pt, labelsep=10pt, itemindent=9pt]
	\item[\f{例}] \underline{$f(t) = \sin t$,$g(t) = \cos t$について,たたみこみ$(f * g)(t)$を求める.}
		\begin{align*}
			(f * g)(t) &= \int_{0}^{t} \sin \tau \cos(t - \mu)\,d\tau\\
			&= \bunsuu{1}{2}\int_{0}^{t} \{\sin t + \sin(2\tau - t)\}\,d\tau\\
			&= \bunsuu{1}{2}\teisekibun{\tau\sin t - \bunsuu{1}{2}\cos(2\tau - t)}{0}{t}\\
			&= \bunsuu{1}{2}t\sin t
		\end{align*}
		\underline{たたみこみのラプラス変換}
		\begin{align*}
			\mathcal{L}\left[\bunsuu{1}{2}t\sin t\right] &= \mathcal{L}[\sin t * \cos t] = \mathcal{L}[\sin t]\mathcal{L}[\cos t]\\
			&= \bunsuu{1}{s^2 + 1}\bunsuu{s}{s^2 + 1} = \bunsuu{s}{(s^2 + 1)^2}
		\end{align*}
\end{enumerate}



\section{ラプラス変換の表}
\subsection{主な性質}

\begin{table}[H]
	\centering
	\begin{tabular}{c|c}
		\hline
		\textsf{原関数} & \textsf{像関数}\\
		\hline
		$\alpha f(t) + \beta g(t)$ & $\alpha F(s) + \beta G(s)$\\[3mm]
		$f(at)$ & $\bunsuu{1}{a}F\left(\bunsuu{s}{a}\right) \quad (a > 0)$\\[3mm]
		$e^{\alpha t}f(t)$ & $F(s - \alpha)$\\[3mm]
		$f(t - \mu)U(t - \mu)$ & $e^{-\mu s} F(s) \qquad (\mu > 0)$\\[3mm]
		$f'(t)$ & $s F(s) - f(+0)$\\[3mm]
		$f^{(n)}(t)$ & $s^n F(s) - s^{n - 1}f(+0) - s^{n - 2}f'(+0) - \cdots - f^{(n - 1)}(+0)$\\[3mm]
		$tf(t)$ & $-F'(s)$\\[3mm]
		$t^n f(t)$ & $(-1)^n F^{(n)}(s)$\\[3mm]
		$\dint_{0}^{t} f(\tau)\,d\tau$ & $\bunsuu{F(s)}{s}$\\[3mm]
		$\bunsuu{f(t)}{t}$ & $\dint_{s}^{\infty} F(\sigma)\,d\sigma$\\[3mm]
		\hline
	\end{tabular}
\end{table}



\subsection{いろいろな関数のラプラス変換}

\begin{table}[H]
	\centering
	\begin{tabular}{c|c}
		\hline
		\hspace*{6\zw}\textsf{原関数}\hspace*{6\zw} & \hspace*{6\zw}\textsf{像関数}\hspace*{6\zw}\\
		\hline
		\raisebox{-1mm}{$1$} & \raisebox{-1mm}{$\bunsuu{1}{s}$}\\[4mm]
		$t$ & $\bunsuu{1}{s^2}$\\[3mm]
		$t^n$ & $\bunsuu{n!}{s^{n + 1}}$\\[3mm]
		$e^{\alpha t}$ & $\bunsuu{1}{s - \alpha}$\\[3mm]
		$te^{\alpha t}$ & $\bunsuu{1}{(s - \alpha)^2}$\\[3mm]
		$t^n e^{\alpha t}$ & $\bunsuu{n!}{(s - \alpha)^{n + 1}}$\\[3mm]
		$\sin \omega t$ & $\bunsuu{\omega}{s^2 + \omega^2}$\\[3mm]
		$\cos \omega t$ & $\bunsuu{s}{s^2 + \omega^2}$\\[3mm]
		$t\sin \omega t$ & $\bunsuu{2\omega s}{(s^2 + \omega^2)^2}$\\[3mm]
		$t\cos \omega t$ & $\bunsuu{s^2 - \omega^2}{(s^2 + \omega^2)^2}$\\[3mm]
		$\sinh \omega t$ & $\bunsuu{\omega}{s^2 - \omega^2}$\\[3mm]
		$\cosh \omega t$ & $\bunsuu{s}{s^2 - \omega^2}$\\[3mm]
		$U(t - \alpha)$ & $\bunsuu{e^{-as}}{s} \quad (a \ge 0)$\\[3mm]
		\hline
	\end{tabular}
\end{table}