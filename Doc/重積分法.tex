\chapter{重積分法}
\setcounter{page}{1}

\section{2重積分}
\subsection{定義}
\vskip-\baselineskip
\begin{equation}
	\iint_{D} f(x,\ y)\,dxdy = \lim_{|\varDelta| \to 0} \sum_{i = 1}^{l}\sum_{j = 1}^{m} f(\xi_i,\ \eta_j)\,\varDelta x_i \varDelta y_j
\end{equation}


\subsection{性質}
\vskip-\baselineskip
\begin{equation}
	\left|\iint_{D} f(x,\ y)\,dxdy\right| \le \iint_{D} |f(x,\ y)|\,dxdy
\end{equation}



\subsection{累次積分(逐次積分)}

\begin{kousiki}{累次積分(1)}
	$D = \{(x,\ y) \mid a \le x \le b,\ c \le y \le d\}$とする.
	\begin{equation}
		\iint_{D} f(x,\ y)\,dxdy = \int_{a}^{b} \left\{\int_{c}^{d} f(x,\ y)\,dy\right\}dx = \int_{c}^{d} \left\{\int_{a}^{b} f(x,\ y)\,dx\right\}\,dy
	\end{equation}
\end{kousiki}

範囲に関数が含まれている場合は変数を含んでいる方を先に計算する.
\begin{kousiki}{累次積分(2)}
	\begin{enumerate}[label=\textbf{[\arabic*]}, labelsep=10pt, leftmargin=23pt]
		\item $D = \{(x,\ y) \mid a \le x \le b,\ {\color{cyan}\varphi_1(x)} \le y \le {\color{cyan}\varphi_2(x)}\}$とする.
			\begin{equation}
				\iint_{D} f(x,\ y)\,dxdy = \int_{a}^{b} \left\{\int_{\textcolor{cyan}{\varphi_1(x)}}^{\textcolor{cyan}{\varphi_2(x)}} f(x,\ y)\,dy\right\}dx
			\end{equation}
		\item $D = \{(x,\ y) \mid {\color{cyan}\psi_1(y)} \le x \le {\color{cyan}\psi_2(y)},\ c \le y \le d\}$とする.
			\begin{equation}
				\iint_{D} f(x,\ y)\,dxdy = \int_{c}^{d} \left\{\int_{\textcolor{cyan}{\psi_1(y)}}^{\textcolor{cyan}{\psi_2(y)}} f(x,\ y)\,dx\right\}dy
			\end{equation}
	\end{enumerate}
\end{kousiki}



\subsection{積分順序の交換}

\begin{tip}{例題(1)}
	$y = \sqrt{x}$と$y = \bunsuu{1}{2}x$に囲まれた領域
	\tcblower
	\begin{align*}
		D &= \left\{(x,\ y) \relmiddle| 0 \le x \le 4,\ \bunsuu{1}{2}x \le y \le \sqrt{x}\right\}\\
		&= \{(x,\ y) \mid y^2 \le x \le 2y,\ 0 \le y \le 2\}
	\end{align*}
\end{tip}

\begin{tip}{例題(2)}
	$y = \log x \qLonglr x = e^y$より
	\begin{equation*}
		\int_{1}^{e} \left\{\int_{0}^{\log x} f(x,\ y)\,dy\right\}dx = \int_{0}^{1} \left\{\int_{e^y}^{e} f(x,\ y)\,dx\right\}\,dy
	\end{equation*}
\end{tip}



\subsection{変数変換}

\begin{kousiki}{変数変換}
	$x = \varphi(u,\ v),\ y = \psi(u,\ v)$について
	\begin{equation}
		\iint_{D} f(x,\ y)\,dxdy = \iint_{D'} f(\varphi,\ \psi)\left|\det
			\begin{bmatrix}
				x_u & x_v\\ y_u & y_v
			\end{bmatrix}
		\right|\,dudv
	\end{equation}
\end{kousiki}

$\det\begin{bmatrix}
	x_u & x_v\\ y_u & y_v
\end{bmatrix}$を\textbf{ヤコビアン}といい,$\bunsuu{\partial(x,\ y)}{\partial(u,\ v)}$や$J(u,\ v)$で表す.また,$D'$は$D$を$u,\ v$で表しなおした領域である.



\subsection{極座標変換}

$x = r\cos\theta,\ y = r\sin\theta$より
\begin{align}
	\iint_{D} f(x,\ y)\,dxdy &= \iint_{D'} f(r\cos\theta,\ r\sin\theta) \left|\det
	\begin{bmatrix*}[r]
		\cos\theta & -r\sin\theta\\ \sin\theta & r\cos\theta
	\end{bmatrix*}
	\right|\,drd\theta \notag\\
	&= \iint_{D'} f(r\cos\theta,\ r\sin\theta)\,rdrd\theta
\end{align}



\subsection{広義積分}
\subsubsection*{領域に関数の定義されない部分が含まれている場合}

\begin{tip}{例題}
	$D$が不等式$x^2 + y^2 \le 1$の表す領域とするとき
	\begin{equation*}
		\iint_{D} (x^2 + y^2)^{-\myfrac{3}{4}}\,dxdy
	\end{equation*}
\end{tip}

\newpage