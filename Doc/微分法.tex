\chapter{微分法}
\setcounter{page}{1}



\section{微分法}

\subsection{関数の連続}

関数$f(x)$の極限値について
\begin{equation}
	\lim_{x \to a} f(x) = \alpha \iff \lim_{x \to a + 0} f(x) = \lim_{x \to a - 0} f(x) = \alpha
\end{equation}
が成り立つ.また,$\dlim_{x \to a} f(x)$が存在して
\begin{equation}
	\lim_{x \to a} f(x) = f(a)
\end{equation}
が成り立つとき,$f(x)$は$x = a$で\textbf{連続}であるという.



\subsection{微分可能性}

\vskip-\baselineskip
\begin{equation}
	f'(a) = \lim_{h \to 0} \bunsuu{f(a + h) - f(a)}{h}
\end{equation}
が存在するとき,関数$f(x)$は$x = a$に於いて\textbf{微分可能}であるという.このとき,次が成り立つ.
\begin{equation}
	\textbf{$f(x)$は$x = a$で微分可能} \qLongright \textbf{$f(x)$は$x = a$で連続}
\end{equation}



\subsection{導関数}

次の式で定義される関数を$f(x)$の\textbf{導関数}という.
\begin{equation}
	\bunsuu{d}{dx}f(x) = \lim_{\varDelta x \to 0} \bunsuu{f(x + \varDelta x) - f(x)}{\varDelta x}
\end{equation}



\begin{kousiki}{導関数の性質と公式}
	$c$を定数とする.
\end{kousiki}

\newpage